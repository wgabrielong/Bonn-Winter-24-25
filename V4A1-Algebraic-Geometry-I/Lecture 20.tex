\section{Lecture 20 -- 16th December 2024}\label{sec: lecture 20}
Recall the construction of the twist of a graded module as in \Cref{def: twist of graded module}. We consider the case of $M_{\bullet}=A_{\bullet}$ considered as a module over itself, which is defined to be Serre's twisting sheaf. 
\begin{definition}[Serre's Twisting Sheaf]\label{def: serre twisting sheaf}
    Let $A_{\bullet}$ be a graded ring that is finitely generated by $A_{1}$ as an $A_{0}$-algebra. Serre's twisting sheaf $\Ocal_{\proj(A_{\bullet})}(d)$ is defined to be $\widetilde{A(d)}$ for $A_{\bullet}$ considered as a module over itself. 
\end{definition} 
We deduce some properties of Serre's twisting sheaf. 
\begin{proposition}\label{prop: serre twisting sheaf properties}
    Let $A_{\bullet}$ be a graded ring that is finitely generated by $A_{1}$ as an $A_{0}$-algebra. Then:
    \begin{enumerate}[label=(\roman*)]
        \item Serre's twisting sheaf $\Ocal_{\proj(A_{\bullet})}(d)$ is an invertible sheaf on $\proj(A_{\bullet})$. 
        \item Let $M_{\bullet}$ be a graded $A_{\bullet}$-module. Then $\widetilde{M}(d)\cong\widetilde{M(d)}$. 
        \item There is an isomorphism $$\Ocal_{\proj(A_{\bullet})}(d_{1})\otimes_{\Ocal_{\proj(A_{\bullet})}}\Ocal_{\proj(A_{\bullet})}(d_{2})\cong\Ocal_{\proj(A_{\bullet})}(d_{1}+d_{2}).$$ 
    \end{enumerate}
\end{proposition}
Recall here that an invertible sheaf is a sheaf such that the sections over any affine open subscheme is an invertible ideal.  
\begin{proof}[Proof of (i)]
    Let $a\in A_{1}$ and consider the restriction $\Ocal_{\proj(A_{\bullet})}(d)|_{D_{+}(a)}$. By \Cref{def: homogeneous localization of modules}, $\Ocal_{\proj(A_{\bullet})}(d)|_{D_{+}(a)}\cong A(d)_{(a)}$ is a free $(A_{\bullet})_{(a)}$-module of rank 1. In particular, the sections of $\Ocal_{\proj(A_{\bullet})}(d)|_{D_{+}(a)}\cong A(d)_{(a)}$ consist of elements of degree $d$ in $(A_{\bullet})_{a}$. There is a morphism $A(d)_{(a)}\to (A_{\bullet})_{(a)}$ by $\frac{b}{a^{n}}\mapsto \frac{b}{a^{n+d}}$ where $b\in A_{d+n}$ with inverse given by multiplication by $a^{d}$ showing it is an isomorphism of modules. 

    As such, $\Ocal_{\proj(A_{\bullet})}(d)|_{D_{+}(a)}\cong A(d)_{(a)}$ is an invertible ideal for all $a\in A_{1}$ and by hypothesis the open sets $D_{+}(a)$ form a basis for the topology on $\proj(A_{\bullet})$ hence the claim. 
\end{proof}
\begin{proof}[Proof of (ii)]
    Once again, we note that $D_{+}(a)$ form a basis for the topology on $\proj(A_{\bullet})$. Denoting $i_{a}$ the inclusion $D_{+}(a)\to\proj(A_{\bullet})$, $i_{a}^{*}$ is symmetric monoidal so computing affine-locally
    $$\widetilde{M}(d)|_{D_{+}(a)}\cong M_{(a)}\otimes\Ocal_{\proj(A_{\bullet})}(d)|_{D_{+}(a)}\cong \widetilde{M(d)}|_{D_{+}(a)}$$
    as claimed. 
\end{proof}
\begin{proof}[Proof of (iii)]
    This is immediate from (ii) above for $M=\Ocal_{\proj(A_{\bullet})}(d_{1})$. 
\end{proof}
We return to a discussion of the correspondence between graded modules and quasicoherent sheaves via the following construction. 
\begin{definition}[Graded Module of a Quasicoherent Sheaf]\label{def: graded module of quasicoherent sheaf}
    Let $A_{\bullet}$ be a graded ring and $\Fcal$ a quasicoherent sheaf on $\proj(A_{\bullet})$. The graded module $\Gamma_{\bullet}(\proj(A_{\bullet}),\Fcal)$ associated to $\Fcal$ is given by $\bigoplus_{d\in\ZZ}\Gamma(\proj(A_{\bullet}),\Fcal(d))$. 
\end{definition}
In general, this does not define an equivalence of categories with inverse $\widetilde{(-)}$ on the $\proj(-)$ of an arbitrary graded ring: any graded module that is zero in sufficiently large degrees has trivial $\widetilde{(-)}$-ification. We can show that this construction is an equivalence in our setting of interest where $A_{\bullet}$ is finitely generated by $A_{1}$ as an $A_{0}$-algebra. We set up the proof with the following preparatory lemmata. 
\begin{lemma}\label{lem: sections of module sheaf over distinguished opens}
    Let $A$ be a ring and $a\in A$. 
    \begin{enumerate}[label=(\alph*)]
        \item If $s\in\Gamma(X,\Fcal)$ such that $s|_{D(a)}=0$ then there exists some $n\in\NN$ such that $a^{n}\cdot s=0$. 
        \item If $t\in\Gamma(D(f),\Fcal)$ then there exists $n\in\NN$ such that $f^{n}t$ is a global section of $\Fcal$. 
    \end{enumerate}
\end{lemma}
\begin{proof}[Proof of (i)]
    Let $\{\spec(A_{f_{i}})\}_{i=1}^{n}$ be an affine open cover of $\spec(A)$ on which $\Fcal|_{\spec(A_{f_{i}})}\cong\widetilde{M_{f_{i}}}$. Observe $\spec(A_{f_{i}})\cap D(a)=\spec(A_{af_{i}})$ on which $(a|_{D(f_{i})})^{n_{i}}\cdot(s|_{D(f_{i})})=0$ for some $n_{i}\in\NN$ large depending on $f_{i}$. But taking $N$ to be larger than $n_{i}$ for all $1\leq i\leq n$, $a^{N}\cdot s=0$. 
\end{proof}
\begin{proof}[Proof of (ii)]
    Denote $t_{i}$ the restriction $t|_{D(f_{i})}$ that further restrict to $a^{n_{i}}t$ on $D(af_{i})$. On intersections $D(f_{i}f_{j})$, the sections $t_{i}$ and $t_{j}$ agree and hence so too they agree on the smaller $D(af_{i}f_{j})=D(a)\cap D(f_{i}f_{j})$ where they are both $a^{n_{i}}t=a^{n_{j}}t$. As such we can take $N$ sufficiently large such that $a^{N}t=a^{N}t$ on all pairs $D(af_{i}f_{j})$ and all pairs $D(f_{i}f_{j})$ allowing us to glue to a global section. 
\end{proof}
This construction globalizes after a twist with a line bundle. 
\begin{lemma}\label{lem: line bundle cohomololgy on qc schemes}
    Let $X$ be a quasicompact quasiseparated scheme, $\Fcal$ a quasicoherent sheaf on $X$, and $\Lcal$ an invertible sheaf on $X$. Let $a\in\Gamma(X,\Lcal)$ and 
    $$X_{a}=\{x\in X:a_{x}\notin\mfrak_{x}\Lcal_{x}\}.$$
    \begin{enumerate}[label=(\roman*)]
        \item Let $s\in\Gamma(X,\Fcal)$ such that $s|_{X_{a}}=0$. Then there exists $n\in\NN$ large such that $a^{n}s=0$ in $\Gamma(X,\Fcal\otimes_{\Ocal_{X}} \Lcal^{\otimes n})$. 
        \item If $t\in\Gamma(X_{a},\Fcal)$, there exists some $n\in\NN$ large such that  $a^{n}t\in\Gamma(X_{a},\Fcal\otimes_{\Ocal_{X}}\Lcal^{\otimes n})$ extends to a global section of $\Fcal\otimes\Lcal^{\otimes n}$. 
    \end{enumerate}
\end{lemma}
\begin{proof}[Proof of (i)]
    Let $\{\spec(A_{i})\}_{i=1}^{n}$ be an affine open cover of $X$ on which $\Lcal$ is trivialized. For such $\spec(A_{i})$, we have $\Gamma(\spec(A_{i}),\Lcal)\cong A_{i}$ with $X_{a}\cap\spec(A_{i})=D(a|_{\spec(A_{i})})$ and $\Gamma(\spec(A_{i}),\Fcal)\cong M$ with $s$ restricting to a section $s|_{\spec(A_{i})}\in M$. By hypothesis, $s|_{\spec(A_{i})}=0$ so $(a|_{\spec(A_{i})})^{n_{i}}\cdot(s|_{\spec(A_{i})})=0$ in $\Gamma(\spec(A_{i}),\Fcal\otimes\Lcal^{\otimes n_{i}})=\Gamma(\spec(A_{i}),\Fcal)$ for $n_{i}$ sufificiently large depending on $\spec(A_{i})$. So taking $N$ sufficiently large such that $(a|_{\spec(A_{i})})^{N}\cdot(s|_{\spec(A_{i})})=0$ for all $1\leq i\leq n$ simultaneously, $a^{N}\cdot s=0$ in $\Gamma(X,\Fcal\otimes_{\Ocal_{X}}\Lcal^{\otimes N})$ as desired. 
\end{proof}
\begin{proof}[Proof of (ii)]
    We can extend affine locally by \Cref{lem: sections of module sheaf over distinguished opens} and globalize the construction using (a). 
\end{proof}
We can now show the desired result. 
\begin{proposition}\label{prop: graded module correspondence for nice graded rings}
    Let $A_{\bullet}$ be a graded ring that is finitely generated by $A_{1}$ as an $A_{0}$-algebra. Then there is an isomorphism of quasicoherent sheaves on $\proj(A_{\bullet})$ $\widetilde{\Gamma_{\bullet}(\proj(A_{\bullet}),\Fcal)}\to\Fcal$. 
\end{proposition}
\begin{proof}
    We define the morphism which we denote $\beta$ on basis sets $D_{+}(a)$ to send a section $\frac{m}{a^{d}}\in\Gamma(X,\Fcal(d))$ to $m\otimes a^{-d}$ as a section of $\Fcal$ over $D_{+}(a)$ but by \Cref{lem: line bundle cohomololgy on qc schemes} (ii) the construction glues to give $\Fcal$. 
\end{proof}
We can use the preceding discussion to understand closed subschemes of projective schemes. 
\begin{definition}[Projective Morphism]\label{def: projective morphism}
    Let $f:X\to Y$ be a  morphsim of schemes. $f$ is projective if it factors as $X\to\PP^{n}_{Y}\to Y$ for some $n\geq0$ where $X\to\PP^{n}_{Y}$ is a closed immersion. 
\end{definition}
\begin{remark}
    This does not agree with the definition of projective morphisms as stated in EGA -- which requires a closed embedding into the relative proj $\underline{\proj}(-)$ of some sheaf of graded algebras. 
\end{remark}
We can define quasiprojective morphisms similarly. 
\begin{definition}[Quasiprojective Morphism]\label{def: quasiprojective morphism}
    Let $f:X\to Y$ be a  morphsim of schemes. $f$ is quasiprojective if it factors as $X\to\overline{X}\to\PP^{n}_{Y}\to Y$ for some $n\geq0$ where $X\to\overline{X}$ is an immersion and $\overline{X}\to\PP^{n}_{Y}$ is a closed immersion. 
\end{definition}
These constructions are closely related to very ample invertible sheaves. 
\begin{definition}[Very Ample Invertible Sheaf]\label{def: very ample invertible sheaf}
    Let $f:X\to Y$ be a morphism of schemes. An invertible sheaf $\Lcal$ on $X$ is $f$-relatively very ample if there exists a diagram 
    $$% https://q.uiver.app/#q=WzAsMyxbMCwwLCJYIl0sWzIsMCwiXFxQUF57bn1fe1l9Il0sWzEsMSwiWSJdLFswLDIsImYiLDJdLFswLDEsImkiXSxbMSwyXV0=
    \begin{tikzcd}
        X && {\PP^{n}_{Y}} \\
        & Y
        \arrow["i", from=1-1, to=1-3]
        \arrow["f"', from=1-1, to=2-2]
        \arrow[from=1-3, to=2-2]
    \end{tikzcd}$$
    such that $\Lcal\cong i^{*}\Ocal_{\PP^{n}_{Y}}(1)$ for some $n\geq 0$. 
\end{definition}
Projective morphisms are in fact proper and separated as defined in \Cref{def: proper morphism,def: separated scheme}. 
\begin{proposition}\label{prop: projective implies proper and separated}
    Let $f:X\to Y$ be a morphism of schemes. If $f$ is projective, then $f$ is proper and separated. 
\end{proposition}
\begin{proof}
    The morphism is universally closed as it is a composition of a closed immersion and a closed map which is stable under base change and separated as closed immersions are separated and $f$ is the composition of separated maps. Finally, $f$ is finite type since on affine opens, the morphism factors over the quotient of a finite type algebra -- namely the graded polynomial ring in $n+1$ variables. 
\end{proof}
On projective schemes, twisting by a very ample line bundle can make coherent sheaves finitely globally generated as we now define using the following proposition. 
\begin{definition}[Finitely Globally Generated]\label{def: finitely globally generated}
    Let $X$ be a scheme and $\Fcal$ a quasicoherent sheaf on $X$. $\Fcal$ is finitely globally generated if there exist sections $s_{0},\dots,s_{n}$ such that the stalks $s_{0,x},\dots,s_{n,x}$ generate $\Fcal_{x}$ as an $\Ocal_{X,x}$-module for all $x\in X$. 
\end{definition}