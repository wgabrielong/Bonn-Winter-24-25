\section{Lecture 19 -- 13th December 2024}\label{sec: lecture 19}
Let $A_{\bullet}$ be a graded ring. We can consider homogeneous localizations which we define as follows. 
\begin{definition}[Homogeneous Localization]\label{def: homogeneous localization}
    Let $A_{\bullet}$ be a graded ring and $S\subseteq A_{\bullet}$ a multiplicative subset. The homogeneous localization $(S^{-1}A)_{\bullet}$ is given by $\bigoplus_{d\geq0}(S^{-1}A)_{d}$ where 
    $$(S^{-1}A)_{d}=\left\{\frac{a}{b}: a\in A_{e}, b\in S\cap A_{e'}, e-e'=d\right\}.$$
\end{definition}
Two special cases of interest will be the localization at a prime ideal and the locaization at a homogeneous element. 
\begin{itemize}
    \item (Homogeneous Element) Let $A_{\bullet}$ be a graded ring and $a\in A_{d}$ a homogeneous element for some $d$. The localization $(A_{\bullet})_{a}$ is the degree 0 piece of the localization of \Cref{def: homogeneous localization} for $S$ the multiplicative set generated by $a$.
    \item (Prime Ideal) Let $A_{\bullet}$ be a graded ring and $\pfrak\subseteq A_{\bullet}$ a prime ideal. The localization $(A_{\bullet})_{\pfrak}$ is the degree 0 piece of the localization of \Cref{def: homogeneous localization} for $S$ the complement of the prime ideal $\pfrak$. 
\end{itemize}
This will allow us to construct $\proj(A_{\bullet})$ and in turn projective space and its variants. We first define $\proj(A_{\bullet})$ as a topological space. 
\begin{definition}[Proj Space]\label{def: proj space}
    Let $A_{\bullet}$ be a graded ring. The topological space $\proj(A_{\bullet})$ consists of the set of homogeneous prime ideals of $A_{\bullet}$ not containing the irrelevant ideal $A_{+}$ and closed sets of the form 
    $$V_{+}(\afrak)=\{\pfrak\in\proj(A_{\bullet}):\afrak\subseteq\pfrak\}$$
    where $\afrak$ is a homogeneous ideal of $A_{\bullet}$. 
\end{definition}
\begin{remark}
    There is an obvious inclusion $\proj(A_{\bullet})\to\spec(A_{\bullet})$, but this is often not a morphism of schemes. 
\end{remark}
Having defined the underlying topological space, we seek to define a sheaf of rings. 
\begin{proposition}\label{prop: covering of proj}
    Let $A_{\bullet}$ be a graded ring such that $(a_{1},\dots,a_{n})=A_{\bullet}$. Then $D_{+}(a_{i})=\proj(A_{\bullet})\setminus V_{+}(a_{i})$ form an open cover of $\proj(A_{\bullet})$ and $D_{+}(a_{i})\cong\spec((A_{\bullet})_{a_{i}})$. 
\end{proposition}
\begin{proof}
    The $D_{+}(a_{i})$ are open by definition and the first statement follows by the argument in the affine case. The second statement follows by definition of homogeneous localization. 
\end{proof}
It is now clear that $\proj(A_{\bullet})$ is a locally ringed space, and in particular a scheme. 
\begin{corollary}\label{corr: proj is a scheme}
    Let $A_{\bullet}$ be a graded ring. Then $\proj(A_{\bullet})$ is a scheme. 
\end{corollary}
\begin{proof}
    The construction of \Cref{prop: covering of proj} gives $\proj(A_{\bullet})$ a cover by affine schemes as a locally ringed space, and hence as a scheme. 
\end{proof}
Moreover, we can determine the stalks of the structure sheaf at stalks $\pfrak$ of $\proj(A_{\bullet})$.
\begin{corollary}\label{corr: stalk of projective scheme}
    Let $A_{\bullet}$ be a graded ring and $\pfrak\in\proj(A_{\bullet})$. $\Ocal_{\proj(A_{\bullet}),\pfrak}=(A_{\bullet})_{\pfrak}$. 
\end{corollary}
\begin{proof}
    This once again follows fromthe definition of homogeneous localization. 
\end{proof}
We have the following condition for $\proj$ being empty. 
\begin{proposition}
    Let $A_{\bullet}$ be a graded ring. $\proj(A_{\bullet})=\emptyset$ if and only if every element of $A_{+}$ is nilpotent. 
\end{proposition}
\begin{proof}
    $(\Longrightarrow)$ Suppose each element of $A_{+}$ is nilpotent. We show each $f\in A_{+}$ lies in $\pfrak\subseteq A_{\bullet}$ prime. Since $f$ is nilpotent, $f^{n}=0$ for some $n$ and thus $f\in \pfrak$ since $\pfrak$ is prime. So $\pfrak$ contains $A_{+}$ and thus $\proj(A_{\bullet})=\emptyset$.
    \\\\
    $(\Longleftarrow)$ Suppose $\proj(A_{\bullet})=\emptyset$ so each prime ideal $\pfrak\subseteq A_{\bullet}$ contains $A_{+}$. For such $\pfrak$ consider the graded prime ideal $\pfrak_{\bullet}=\bigoplus_{d\geq0}\pfrak\cap A_{d}$ giving $A_{+}\subseteq\pfrak_{\bullet}\subseteq\pfrak$. In particular, $A_{+}$ is contained in every prime ideal, and thus every element of $A_{+}$ is nilpotent. 
\end{proof}
We now make the definition of projective space. 
\begin{definition}[Projective Space]\label{def: projective space}
    Let $A$ be a ring. Projective space $\PP^{n}_{A}$ is $\proj(A[x_{0},\dots,x_{n}])$. 
\end{definition}
\begin{remark}\label{rmk: canonical map to A0}
    The construction of $\proj(A_{\bullet})$ induces a canonical morphism of schemes $\proj(A_{\bullet})\to\spec(A_{0})$. 
\end{remark}
More generally we can consider projective space over a scheme. 
\begin{definition}[Projective Space Over Scheme]\label{def: projective space over scheme}
    Let $X$ be a scheme. The projective space $\PP^{n}_{X}$ is the fibered product $\PP^{n}_{\ZZ}\times_{\spec(\ZZ)}X$. 
\end{definition}
Generalizing quasicoherent sheaves on modules, we can also consider the Abelian category of graded modules over a graded ring. 
\begin{definition}[Graded Modules]\label{def: graded modules}
    Let $A_{\bullet}$ be a graded ring. The category $\Mod_{A_{\bullet}}$ is the category of graded $A_{\bullet}$-modules with degree-preserving homomorphisms. 
\end{definition}
The desideratum is to construct a functor from graded modules to quasicoherent sheaves analogous to \Cref{prop: equivalence of categories modules on affine schemes}, however the functor $\Mod_{A_{\bullet}}\to\QCoh(\proj(A_{\bullet}))$ by $M_{\bullet}\mapsto\widetilde{M_{\bullet}}$ will not be an equivalence in general. 

We consider the localization of graded modules. 
\begin{definition}[Homogeneous Localization of Modules]\label{def: homogeneous localization of modules}
    Let $M_{\bullet}$ be a graded $A_{\bullet}$-module and $S\subseteq A_{\bullet}$ a multiplicative subset. The homogeneous localization $(S^{-1}M)_{\bullet}$ is given by $\bigoplus_{d\geq0}(S^{-1}M)_{d}$ where 
    $$(S^{-1}M)_{d}=\left\{\frac{m}{a}:m\in M_{e},a\in A_{e'}, e-e'=d\right\}.$$
\end{definition}
As in the case of \Cref{def: homogeneous localization}, we are most interested in the case of localization at a prime ideal and the localization at a homogeneous element. 
\begin{itemize}
    \item Let $M_{\bullet}$ be a graded $A_{\bullet}$-module and $a\in A_{d}$ a homogeneous element for some $d$. The localization $(M_{\bullet})_{a}$ is the degree 0 piece of the localization of \Cref{def: homogeneous localization of modules} for $S$ the multiplicative set generated by $a$. 
    \item Let $M_{\bullet}$ be a graded $A_{\bullet}$-module and $\pfrak\subseteq A_{\bullet}$ a homogeneous prime ideal. The localization $(M_{\bullet})_{\pfrak}$ is the degree 0 piece of the localization of \Cref{def: homogeneous localization of modules} for $S$ the complement of the prime ideal $\pfrak$. 
\end{itemize}
We conclude with a very important construction: that of the twist of a module $M(d)$ for a graded module $M_{\bullet}$. 
\begin{definition}[Twist of Graded Module]
    Let $M_{\bullet}$ be a graded $A_{\bullet}$-module and $d\in\ZZ$. The $d$th twist $M(d)$ is the shift of the grading by $d$ given by $M_{d\bullet}$. 
\end{definition}