\documentclass{amsart}
\usepackage[margin=1.5in]{geometry} 
\usepackage{amsmath}
\usepackage{tcolorbox}
\usepackage{amssymb}
\usepackage{amsthm}
\usepackage{lastpage}
\usepackage{fancyhdr}
\usepackage{accents}
\usepackage{hyperref}
\usepackage{xcolor}
\usepackage{color}
% Fields
\newcommand{\CC}{\mathbb{C}}
\newcommand{\RR}{\mathbb{R}}
\newcommand{\QQ}{\mathbb{Q}}
\newcommand{\ZZ}{\mathbb{Z}}
\newcommand{\HH}{\mathbb{H}}
\newcommand{\KK}{\mathbb{K}}
\newcommand{\NN}{\mathbb{N}}
\newcommand{\FF}{\mathbb{F}}
\newcommand{\PP}{\mathbb{P}}

% mathcal letters
\newcommand{\Acal}{\mathcal{A}}
\newcommand{\Bcal}{\mathcal{B}}
\newcommand{\Ccal}{\mathcal{C}}
\newcommand{\Dcal}{\mathcal{D}}
\newcommand{\Ecal}{\mathcal{E}}
\newcommand{\Fcal}{\mathcal{F}}
\newcommand{\Gcal}{\mathcal{G}}
\newcommand{\Hcal}{\mathcal{H}}
\newcommand{\Ical}{\mathcal{I}}
\newcommand{\Jcal}{\mathcal{J}}
\newcommand{\Kcal}{\mathcal{K}}
\newcommand{\Lcal}{\mathcal{L}}
\newcommand{\Mcal}{\mathcal{M}}
\newcommand{\Ncal}{\mathcal{N}}
\newcommand{\Ocal}{\mathcal{O}}
\newcommand{\Pcal}{\mathcal{P}}
\newcommand{\Qcal}{\mathcal{Q}}
\newcommand{\Rcal}{\mathcal{R}}
\newcommand{\Scal}{\mathcal{S}}
\newcommand{\Tcal}{\mathcal{T}}
\newcommand{\Ucal}{\mathcal{U}}
\newcommand{\Vcal}{\mathcal{V}}
\newcommand{\Wcal}{\mathcal{W}}
\newcommand{\Xcal}{\mathcal{X}}
\newcommand{\Ycal}{\mathcal{Y}}
\newcommand{\Zcal}{\mathcal{Z}}

% abstract categories
\newcommand{\Asf}{\mathsf{A}}
\newcommand{\Bsf}{\mathsf{B}}
\newcommand{\Csf}{\mathsf{C}}
\newcommand{\Dsf}{\mathsf{D}}
\newcommand{\Esf}{\mathsf{E}}
\newcommand{\Ssf}{\mathsf{S}}

% algebraic geometry
\newcommand{\spec}{\operatorname{Spec}}
\newcommand{\proj}{\operatorname{Proj}}

% categories 
\newcommand{\id}{\mathrm{id}}
\newcommand{\Obj}{\mathrm{Obj}}
\newcommand{\Mor}{\mathrm{Mor}}
\newcommand{\Hom}{\mathrm{Hom}}
\newcommand{\Aut}{\mathrm{Aut}}
\newcommand{\Sets}{\mathsf{Sets}}
\newcommand{\SSets}{\mathsf{SSets}}
\newcommand{\kVect}{\mathsf{Vect}_{k}}
\newcommand{\Vect}{\mathsf{Vect}}
\newcommand{\Alg}{\mathsf{Alg}}
\newcommand{\Ring}{\mathsf{Ring}}
\newcommand{\Mod}{\mathsf{Mod}}
\newcommand{\Grp}{\mathsf{Grp}}
\newcommand{\AbGrp}{\mathsf{AbGrp}}
\newcommand{\PSh}{\mathsf{PSh}}
\newcommand{\Sh}{\mathsf{Sh}}
\newcommand{\PSch}{\mathsf{PSch}}
\newcommand{\Sch}{\mathsf{Sch}}
\newcommand{\Top}{\mathsf{Top}}
\newcommand{\Com}{\mathsf{Com}}
\newcommand{\Coh}{\mathsf{Coh}}
\newcommand{\QCoh}{\mathsf{QCoh}}
\newcommand{\Opens}{\mathsf{Opens}}
\newcommand{\Opp}{\mathsf{Opp}}
\newcommand{\Cat}{\mathsf{Cat}}
\newcommand{\NatTrans}{\mathrm{NatTrans}}
\newcommand{\pr}{\mathrm{pr}}
\newcommand{\Fun}{\mathrm{Fun}}
\newcommand{\colim}{\mathrm{colim}}
\newcommand{\lifts}{\boxslash}
\DeclareMathOperator\squarediv{\lifts}
\newcommand{\Kan}{\mathsf{Kan}}
\newcommand{\Path}{\mathsf{Path}}
\newcommand{\SPSh}{\mathsf{SPSh}}
\newcommand{\SSh}{\mathsf{SSh}}
\newcommand{\Bord}{\mathsf{Bord}}

% simplicial sets
\newcommand{\DDelta}{\Updelta}
\newcommand{\Sing}{\operatorname{Sing}}

% ideal theory
\newcommand{\mfrak}{\mathfrak{m}}
\newcommand{\afrak}{\mathfrak{a}}
\newcommand{\bfrak}{\mathfrak{b}}
\newcommand{\pfrak}{\mathfrak{p}}
\newcommand{\qfrak}{\mathfrak{q}}

% number theory
\newcommand{\Tr}{\mathrm{Tr}}
\newcommand{\Nm}{\mathrm{Nm}}
\newcommand{\Gal}{\mathrm{Gal}}
\newcommand{\Frob}{\mathrm{Frob}}

\newcommand{\SL}{\mathrm{SL}}
\newcommand{\Li}{\mathrm{Li}}
\setlength{\headheight}{40pt}


\newenvironment{solution}
  {\renewcommand\qedsymbol{$\blacksquare$}
  \begin{proof}[Solution]}
  {\end{proof}}
\renewcommand\qedsymbol{$\blacksquare$}

\usepackage{amsmath, amssymb, tikz, amsthm, csquotes, multicol, footnote, tablefootnote, biblatex, wrapfig, float, quiver, mathrsfs, cleveref, enumitem, upgreek, stmaryrd, marginnote, todonotes}
\addbibresource{refs.bib}
\theoremstyle{definition}
\newtheorem{theorem}{Theorem}[section]
\newtheorem{lemma}[theorem]{Lemma}
\newtheorem{corollary}[theorem]{Corollary}
\newtheorem{exercise}[theorem]{Exercise}
\newtheorem{question}[theorem]{Question}
\newtheorem{example}[theorem]{Example}
\newtheorem{proposition}[theorem]{Proposition}
\newtheorem{conjecture}[theorem]{Conjecture}
\newtheorem{remark}[theorem]{Remark}
\newtheorem{definition}[theorem]{Definition}
\numberwithin{equation}{section}
\setuptodonotes{color=blue!20, size=tiny}
\begin{document}
\large
\title[Algebraic Geometry I -- Bonn, Winter 2024/25]{V4A1 -- Algebraic Geometry I \\ Winter Semester 2024/25}
\author{Wern Juin Gabriel Ong}
\address{Universit\"{a}t Bonn, Bonn, D-53113}
\email{wgabrielong@uni-bonn.de}
\urladdr{https://wgabrielong.github.io/}
\maketitle
\section*{Preliminaries}
These notes roughly correspond to the course \textbf{V4A1 -- Algebraic Geometry I} taught by Prof. Daniel Huybrechts at the Universit\"{a}t Bonn in the Winter 2024/25 semester. These notes are \LaTeX-ed after the fact with significant alteration and are subject to misinterpretation and mistranscription. Use with caution. Any errors are undoubtedly my own and any virtues that could be ascribed to these notes ought be attributed to the instructor and not the typist. Knowledge of commutative algebra, topology, and category theory will be assumed. 
\tableofcontents
\section{Lecture 1 -- 8th October 2024}\label{sec: lecture 1}
We first set the following conventions to be used throughout these notes:
\begin{itemize}
    \item Let $X$ be a topological space. A neighborhood of a point $p\in X$ is an open set $U\subseteq X$ containing $p$. 
    \item For $p=(p_{1},\dots,p_{n})\in\RR^{n}$ and $r\in\RR_{\geq0}$, $B_{r}(p)=\{x\in\RR^{n}:|x-p|^{2}<r\}$ is the open ball of radius $r$ centered at $p$. 
\end{itemize}
We begin with a review of point set topology. 

Recall the definition of locally Euclidean spaces. 
\begin{definition}[Locally Euclidean Space]\label{def: locally Euclidean space}
    Let $X$ be a topological space. $X$ is locally euclidean if each point $x\in X$ has a neighborhood homeomorphic to $\RR^{n}$ for some fixed $n$. 
\end{definition}
\begin{remark}
    Note that the definition above does not permit topological spaces with points $x,y\in X$ such that $x$ admits a neighborhood homeomorphic to $\RR^{n}$ and $y$ admits a neighborhood homeomorphic to $\RR^{m}$ for $m\neq n$. 
\end{remark}
On a locally Euclidean topological space, we can take the neighborhoods homeomorphic to $\RR^{n}$ and consider open subsets of such neighborhoods which also possess a map to $\RR^{n}$. 
\begin{definition}[Chart]\label{def: chart}
    Let $X$ be a locally Euclidean topological space. A chart $(U,\phi)$ consists of an open set $U\subseteq X$ and a continuous map $\phi:U\to\RR^{n}$ that is a homeomorphism onto its image. 
\end{definition}
Given a point $x\in X$ and a neighborhood, we can consider charts with a prescribed image $\phi(x)\in\RR^{n}$. An especially nice case is when $\phi(x)=0\in\RR^{n}$.
\begin{definition}[Centered Chart]\label{def: centered chart}
    Let $X$ be a locally Euclidean topological space. A chart $(U,\phi)$ is centered at $x\in U$ if $\phi(x)=0\in\RR^{n}$. 
\end{definition}
In fact, one can show that locally Euclidean topological spaces have charts centered at $x$ for all points $x\in X$. 
\begin{proposition}\label{prop: locally euclidean and centered charts}
    Let $X$ be a topological space. The following are equivalent:
    \begin{enumerate}[label=(\alph*)]
        \item $X$ is locally Euclidean. 
        \item For any point $x\in X$, there is a chart centered at $x$ with image the unit ball of $\RR^{n}$. 
        \item For any point $x\in X$, there is a chart centered at $x$ with image $\RR^{n}$. 
    \end{enumerate}
\end{proposition}
\begin{proof}
    (b)$\Longleftrightarrow$(c) by composing appropriately with the homeomorphism $B_{1}(0)\to\RR^{n}$ by fixing the origin and the map on the complement defined by $x\mapsto\frac{1}{1-\Vert x\Vert}$. Furthermore (c)$\Rightarrow$(a) since (c) is a homeomorphisms of a neighborhood to $\RR^{n}$ are in particular continuous maps to $\RR^{n}$ homeomorphic onto its image. 
    
    It remains to show (a)$\Rightarrow$(b). Consider a chart $(U,\phi)$. For $x\in U$, we can consider the map $U\to\RR^{n}$ by $y\mapsto y-\phi(x)$ yielding a chart centered at $x$. By scaling this map by some $\lambda\in\RR_{>0}$ we can consider a map $\widetilde{\phi}$ by $y\mapsto \lambda y-\lambda\phi(x)$ with image containing $B_{1}(0)$. Restriction to the preimage of $B_{1}(0)$ under $\widetilde{\phi}$ yields a chart centered at $x$ with image the unit ball $(U|_{\widetilde{\phi}^{-1}(B_{1}(0))}, \widetilde{\phi})$. 
\end{proof}
We now introduce the notion of Hausdorff spaces, which include the spaces of concern in this course, as well as a large proportion of spaces one will encounter over the course of one's mathematical life. 
\begin{definition}[Hausdorff]\label{def: Hausdorff}
    Let $X$ be a topological space. $X$ is Hausdorff if for any two distinct points $x,x'\in X$ there exist open neighborhoods $U,U'$ of $x,x'$, respectively, such that $U\cap U'=\emptyset$. 
\end{definition}
\begin{example}
    Euclidean space $\RR^{n}$ is Hausdorff. 
\end{example}
\begin{example}
    CW complexes are Hausdorff. 
\end{example}
\begin{example}\label{ex: R by units is not Hausdorff}
    Let $X$ be the topological space given by the set $\{0,1\}$ and open sets $\emptyset, \{0\}, \{0,1\}$. This space is not Hausdorff since the points 0 and 1 cannot be separated by open sets. This space is in fact the quotient space $\RR/\RR^{\times}$ with $\RR^{\times}$ acting on $\RR$ by multiplication. 
\end{example}
\begin{remark}
    As suggested by \Cref{ex: R by units is not Hausdorff}, quotient spaces are the prototypical example of non-Hausdorff spaces. 
\end{remark}
We can show the following properties of Hausdorff spaces. 
\begin{proposition}\label{prop: properties of Hausdorff spaces}
    Let $X$ be a Hausdorff topological space. Then:
    \begin{enumerate}[label=(\roman*)]
        \item Compact sequences have unique limits. 
        \item Compact subsets are closed. 
        \item One-point subsets are closed. 
    \end{enumerate}
\end{proposition}
\begin{proof}[Proof of (a)]
    Suppose to the contrary that there is a sequence $\{x_{i}\}_{i=1}^{\infty}$ with limit points $x,x'$ distinct. Since $X$ is Hausdorff, we can take open neighborhoods $U,U'$ of $x,x'$, respectively, such that $U\cap U'=\emptyset$. However we can take $N$ large we have both $x_{i}\in U$ and $x_{i}\in V$, a contradiction as $U,U'$ are disjoint. 
\end{proof}
\begin{proof}[Proof of (b)]
    Let $K\subseteq X$ be compact. We want to show that its complement $X\setminus K$ is open. Let $x\in X\setminus K$. Since $X$ is Hausdorff, we can consider a neighborhood $V_{y}$ for each $y\in K$ disjoint from (possibly varying) neighborhoods $U_{y}$ of $x$. Since $K$ is compact, $K$ is covered by finitely many $V_{y}$'s say $V_{y_{1}},\dots,V_{y_{n}}$ and set $U=\bigcap_{i=1}^{n}U_{y_{i}}$. Note that each $U_{y_{i}}$ is an open set of $X$ containing $x$ in the complement of $V_{y_{i}}$ in $X$ and as such their intersection contains $x$ and is in the complement of $K$. As such, any $x\in X\setminus K$ admits an open neighborhood disjoint from $K$ showing $K$ is closed.
\end{proof}
\begin{proof}[Proof of (c)]
    This is immediate from (b), for one-point sets are compact. 
\end{proof}
We now discuss bases and covers of topological spaces. 
\begin{definition}[Basis for a Topological Space]\label{def: basis of topological space}
    Let $X$ be a topological space. A collection $\Bcal$ of arbitrary subsets of $X$ is a basis of $X$ if for any $p\in X$ and any neighborhood $U$ of $p$ there exists an element of $B$ containing $p$ and contained in $U$. 
\end{definition}
It can be shown that any open set of a topological space can be written as a union of basis sets. 
\begin{proposition}\label{lem: basis iff every open is a union of elements}
    Let $X$ be a topological space and $\Bcal$ an arbitrary collection of subsets of $X$. $\Bcal$ is a basis of $X$ if and only if every open set of $X$ can be written as a union of sets of $\Bcal$. 
\end{proposition}
\begin{proof}
    $(\Rightarrow)$ Suppose $\Bcal$ is a basis of $X$ and let $U\subseteq X$ be open. For $x\in U$ consider $V_{x}\in\Bcal$ containing $x$ but contained in $U$ where we have $U=\bigcup_{x\in U}V_{x}$, writing $U$ as a union of basis sets. 

    $(\Leftarrow)$ Suppose for each open $U\subseteq X$ we can write $U=\bigcup_{i\in I}V_{i}$. As such, for each point $x\in U$ there is some $V_{i}$ contained in $U$ containing $X$ thus forming a basis. 
\end{proof}
We want to focus our attention on topological spaces that are appropriately ``small'' by imposing size conditions on the basis. 
\begin{definition}[Second Countable Space]\label{def: second countable space}
    Let $X$ be a topological space. $X$ is a second countable space if $X$ admits a countable basis $\Bcal$.
\end{definition}
The countability property is preserved under the following conditions. 
\begin{proposition}\label{prop: second countability preserved}
    Let $X$ be a topological space. Then:
    \begin{enumerate}[label=(\roman*)]
        \item If $X$ is second countable, then any subspace of $X$ with the subspace topology is second countable. 
        \item If $\{U_{i}\}_{i\in I}$ is a countable open cover of $X$ with each each $U_{i}$ second countable then $X$ is countable. 
        \item If $X$ is locally Euclidean and $\{K_{i}\}_{i=1}^{\infty}$ is a sequence of compact subsets such that $X=\bigcup_{i=1}^{\infty}K_{i}$ then $X$ is second countable. 
    \end{enumerate}
\end{proposition}
\begin{remark}
    The property of being second countable is not preserved under arbitrary quotients, though this holds when the quotient map is open. 
\end{remark}
We can describe the second countability property in terms of covers. 
\begin{proposition}\label{prop: second countability via covers}
    Let $X$ be a topological space. If $X$ is second countable then any open cover of $X$ admits a countable subcover. 
\end{proposition}
\begin{proof}
    Let $\Bcal$ be a countable basis for $X$ and $\{U_{i}\}_{i\in I}$ an open cover of $X$. Consider $\widetilde{\Bcal}$ consisting of those basis elements of $X$ contained in some $U_{i}$. Note that $\widetilde{\Bcal}$ is a cover of $X$ since for any point $x\in U_{i}$ there is an element of $\Bcal$ containing $x$ contained in $U_{i}$. For each $V\in\widetilde{\Bcal}$ of which there are countably many, consider $U_{V}\in\{U_{i}\}_{i\in I}$ such that $V\subseteq U_{V}$. These form a cover of $X$ indexed by a countable set $\widetilde{\Bcal}$ giving the claim. 
\end{proof}
We also introduce the following notion of compact exhaustability. 
\begin{definition}[Compact Exhaustability]\label{def: compact exhaustability}
    Let $X$ be a topological space. $X$ is compactly exhaustible if there exists a sequence of compact subsets $\{K_{i}\}_{i=1}^{\infty}$ of $X$ such that $K_{i}\subseteq K_{i+1}^{\circ}$ and $X=\bigcup_{i=1}^{\infty}K_{i}$.
\end{definition}
The condition of compact exhaustability is satisfied under relatively mild hypotheses. 
\begin{proposition}\label{prop: locally euclidean, Hausdorff, second countable implies compactly exhaustible}
    Let $X$ be a topological space. If $X$ is locally Euclidean, Hausdorff, and second countable, $X$ admits an exhaustion by compact subsets. 
\end{proposition}
\begin{proof}
    We first note that since $X$ is locally Euclidean, it admits a basis $\Bcal$ of open subsets having compact closure: for each chart $(U,\phi)$ we can take some $x\in U$ and set the image of the chart to be centered at $x$ homeomorphic onto the open unit ball by \Cref{prop: locally euclidean and centered charts} and produce a countable basis of the ball by smaller balls wich have compact closure. By taking preimages, we can consider the countable union of countable balls with compact closures inducing the respective property for each open of $X$. 
 
    Furthermore, since $X$ is second countable, it is covered -- up to a choice of bijection of the countable indexing set with the natural numbers -- by countably many sets $\{U_{i}\}_{i=1}^{\infty}$ with compact closure. Suppose $K_{1}=\overline{U_{1}}$. We proceed by induction and suppose that there are compact sets $K_{1},\dots,K_{m}$ such that $U_{i}\subseteq K_{i}$ for each $i$ and $K_{i}\subseteq K_{i+1}^{\circ}$ for $2\leq i\leq m-1$. Since $K_{m}$ is compact, there is some $N_{m}\geq m+1$ large such that $K_{m}\subseteq U_{1}\cup\dots\cup U_{N_{m}}$. If $K_{m+1}=\overline{U_{1}}\cup\dots\cup\overline{U_{N_{m}}}$ then $K_{m+1}$ is closed and thus compact with interior containing $K_{m}$ giving the claim. 
\end{proof}
\section{Lecture 2 -- 18th October 2024}\label{sec: lecture 2}
Let us revisit the $q$-Pochhammer function as an example of Nahm sums, and in particular to consider the asymptotic behavior of these rings as $q$ approaches roots of unity. Indeed, the study of such phenomena is precisely the study of the Habiro ring. 

We consider the theory of $q$-calculus and some of its more modern incarnations. The following table summarizes the analogy.

\begin{table}[h]\label{table: q-calculus comparison}
    \begin{tabular}{c c c}
        \textbf{Classical} & \textbf{$q$-deformed} & \textbf{$q$-deformed (adapted)}\\
        $\nabla:\ZZ[t]\to\ZZ[t]$ & $\nabla_{q}:\ZZ[q][t]\to\ZZ[q][t]$ & $\nabla_{q}':\ZZ[q][t]\to\ZZ[q][t]$ \\
        $f(t)\mapsto\lim_{h\to 0}\frac{f(t)-f(t+h)}{h}$ & $f(t)\mapsto\frac{f(t)-f(qt)}{t-qt}$ & $f(t)\mapsto\frac{f(t)-f(qt)}{t}$\\
        $t^{n}\mapsto nt^{n-1}$ & $t^{n}\mapsto \frac{1-q^{n}}{1-q}\cdot t^{n-1}=[n]_{q}t^{n-1}$ & $t^{n}\mapsto (1-q^{n})t^{n-1}$\\
    \end{tabular}
    \caption{Comparison between classical calculus and two variants of $q$-calculus.}
\end{table}
The $q$-deformed construction recovers the classical case as $q\to1$, but the adapted $q$-deformed variant often works better since the lack of $(1-q)$-factors ``treats all roots of unity the same,'' unlike in the classical $q$-deformed variant which ``singles out the first root of unity.''

Note that $\nabla$ is coordinate independent as a local operator, but $\nabla_{q}$ is not, and multiplication by $q$ is information that needs to be remembered. However, despite these issues, $q$-de Rahm cohomology groups turn out to be coordinate independent after $(q-1)$-adic completion, as implied by the theory of prismatic cohomology as developed in \cite{PrismsPrismatic}. In the $\nabla_{q}'$-variant, however, this coordinate independence does not hold as shown by Wagner in \cite{WagnerMSThesis} (vis. \cite{WagnerQWittQHodge}), but recent work of Meyer-Wagner shows the theory does still hold at some level of generality \cite{MeyerWagner}.

Let us now consider $q$-integration, to the end of considering solutions to $q$-difference equations. Classically, $\nabla f(t)=f(t)$ with initial value $f(0)=1$ yields the power series $f(t)=\sum_{n\geq0}\frac{t^{n}}{n!}=\exp(t)$. In the $q$-deformed setting, we have the following. 
\begin{proposition}\label{prop: q-deformed exponential}
    The $q$-difference equation with $\nabla_{q}f(t)=f(t)$ with initial value $f(0)=1$ has solution $f(t)=\sum_{n\geq0}\frac{t^{n}}{[n]_{q}!}$ where $[n]_{q}!=\frac{(q;q)_{n}}{(1-q)^{n}}$.
\end{proposition}
\begin{proof}
    This can be verified by a direct computation:
    \begin{align*}
        \nabla_{q}f(t) &= \sum_{n\geq0}\frac{[n]_{q}t^{n-1}}{[n]_{q}!} \\
        &=\sum_{n\geq0}\frac{t^{n-1}}{[n-1]_{q}!} = f(t)
    \end{align*}
    which satisfies the initial value condition by inspection. 
\end{proof}
Similarly in the case of adapted $q$-deformations, we have the following. 
\begin{proposition}\label{prop: adapted q-deformed exponential}
    The $q$-difference equation with $\nabla_{q}'f(t)=f(t)$ with initial value $f(0)=1$ has solution $f(t)=\sum_{n\geq0}\frac{t^{n}}{(q;q)_{n}}$. 
\end{proposition}
\begin{proof}
    Computing as above:
    \begin{align*}
        \nabla_{q}'f(t) &= \sum_{n\geq0}\frac{(1-q^{n})t^{n-1}}{(q;q)_{n}} \\
        &=\sum_{n\geq0}\frac{t^{n-1}}{(q;q)_{n-1}}
    \end{align*}
    which once again, by observation, satisfies the initial value condition. 
\end{proof}
\begin{remark}
    We will primarily focus on the adapted variant $\nabla_{q}'$. 
\end{remark}
\begin{remark}
    The above are examples of Nahm sums of \Cref{def: Nahm sum} for the case $N=1$ and $A=0$. 
\end{remark}
In the adapted variant, we can alternatively describe $f(t)$ as follows. 
\begin{proposition}\label{prop: Pochhammer as exponential}
    The $q$-difference equation with $\nabla_{q}'f(t)=f(t)$ with initial value $f(0)=1$ has solution $f(t)=(t;q)_{\infty}^{-1}$.
\end{proposition}
\begin{proof}
    Given $f(t)=\frac{f(t)-f(qt)}{t}$ we have that $(1-t)f(t)=f(qt)$ and thus 
    \begin{equation}\label{eqn: q Pochhammer expansion of exponential}
        f(t) = (1-t)f(qt)
    \end{equation} Applying the same manipulation to $f(qt)=\frac{f(qt)-f(q^{2}t)}{qt}$ we have $f(qt)=(1-qt)f(q^{2}t)$ which by induction and substituting into (\ref{eqn: q Pochhammer expansion of exponential}) we get $f(t)=(t;q)_{\infty}^{-1}$, yielding the claim. 
\end{proof}
As an immediate corollary, we deduce the following. 
\begin{corollary}
    There is an equality
    $$(t;q)_{\infty}^{-1}=\sum_{n\geq0}\frac{t^{n}}{(q;q)_{n}}$$
    in $\ZZ[q^{\pm},\frac{1}{1-q},\frac{1}{1-q^{2}}, \dots][[t]]$.
\end{corollary}
\begin{proof}
    This is immediate from \Cref{prop: adapted q-deformed exponential,prop: Pochhammer as exponential}.
\end{proof}
We are interested in two phenomena:
\begin{itemize}
    \item the asymptotics as $q\to 1$ recovering the classical theory, and 
    \item the asymptotics at roots of unity. 
\end{itemize}
One immediately observes that these functions have poles at roots of unity, but their logarithms converge as power series in $t$ with coefficients in $\QQ(q)$. We show the logarithm of $(t;q)_{\infty}^{-1}$ has at worst simple poles at roots of unity. 
\begin{proposition}\label{prop: logarithm at worst simple poles at roots of unity}
    There is an equality 
    \begin{equation}\label{eqn: expresssion of logarithm of q exponential}
        \log(t;q)_{\infty}^{-1}=\sum_{\ell\geq1}\frac{1}{\ell(1-q^{\ell})}\cdot t^{\ell}
    \end{equation}
    in $\QQ(q)[[t]]$. As such, $\log(t;q)_{\infty}^{-1}$ has at worst simple poles at all roots of unity. 
\end{proposition}
\begin{proof}
    We compute 
    \begin{align*}
        \log(t;q)_{\infty}^{-1} &= \sum_{n\geq0}\log(1-q^{n}t)^{-1} \\
        &= \sum_{n\geq0}\sum_{\ell\geq1}\frac{q^{n\ell}t^{\ell}}{\ell}&& \log(1-x)^{-1}=\sum_{\ell\geq1}\frac{x^{\ell}}{\ell} \\
        &= \sum_{\ell\geq1}\left(\sum_{n\geq0}q^{n\ell}\right)\frac{t^{\ell}}{\ell} \\
        &= \sum_{\ell\geq1}\left(\frac{1}{1-q^{\ell}}\right)\frac{t^{\ell}}{\ell} && \text{sum of geom. series} \\
        &= \sum_{\ell\geq1}\frac{1}{\ell(1-q^{\ell})}t^{\ell}
    \end{align*}
    giving the first claim. 

    For the second claim, observe that the denominator of (\ref{eqn: expresssion of logarithm of q exponential}) vanishes at order at most 1 at roots of unity, yielding the proposition. 
\end{proof}
We now consider the behavior at $q=1$, and to that end we consider $\log(t;q)_{\infty}^{-1}$ as an element of $\frac{1}{q-1}\QQ[[q-1,t]]$. To simplify computations, we make the variable change $q=\exp(h)$ and writing our power series in $\frac{1}{h}\QQ[[h,t]]$ since $\log(q)=\log(1-(q-1))=h$ with $\log(q)$ in $\QQ[[q^{-1}]]$ and understand the asymptotic behavior by writing equations as power series in the variable $h$. To that end, we recall the following definitions. 
\begin{definition}[Bernoulli Number]\label{def: Bernoulli number}
    Let $n\geq0$. The $n$th Bernoulli number $B_{n}$ is the $n$th coefficient in the power series expansion 
    $$-\frac{x}{1-e^{x}}=\sum_{n\geq0}\frac{B_{n}}{n!}x^{n}\in\QQ[[x]].$$
\end{definition}
\begin{definition}[Polylogarithm]\label{def: polylogarithm}
    Let $n\in\ZZ$. The $n$th polylogarithm is the function
    $$\Li_{n}(x)=\sum_{\ell\geq1}\frac{x^{\ell}}{\ell^{n}}\in\QQ[[x]].$$
\end{definition}
Let us consider some elementary properties of the polylogarithm. 
\begin{lemma}\label{lem: polylogarithm differential equation}
    The $n$th polylogarithm satisfies the differential equation $\nabla\Li_{n}(t)=\frac{1}{t}\Li_{n-1}(t)$ with initial condition $\Li_{n}(0)=0$. 
\end{lemma}
\begin{proof}
    We compute
    $$ \nabla\Li_{n}(t)=\sum_{\ell\geq1}\frac{\ell t^{\ell-1}}{\ell^{n}}=\sum_{\ell\geq1}\frac{t^{\ell-1}}{\ell^{n-1}}$$
    so multiplying by $t$ we have 
    $$t\cdot\nabla\Li_{n}(t)=\sum_{\ell\geq 1}\frac{t^{\ell}}{\ell^{n-1}}= \Li_{n-1}(t)$$
    so $\frac{1}{t}\Li_{n-1}(t)=\Li_{n}(t)$ with the intial condition holding since $\sum_{\ell\geq1}\frac{0^{\ell}}{\ell^{n}}=0$. 
\end{proof}
The some small values of the polylogarithm are given below \cite{Polylogarithm}. 

\begin{table}[h]\label{table: polylogarithm values}
    \begin{align*}
        \Li_{-2}(t)&=\frac{t(t+1)}{(1-t)^{3}} && \Li_{-1}(t)=\frac{t}{1-t} \\
        \Li_{0}(t)&=\frac{t}{1-t} && \Li_{1}(t)=-\log(1-t)
    \end{align*}
    \caption{Values of $\Li_{n}(t)$ for $-2\leq n\leq 1$.}
\end{table}

\begin{lemma}\label{lem: form of negative polylogarithms}
    $\Li_{n}(t)\in t\cdot\ZZ[t,\frac{1}{1-t}]$ for $n\leq 0$. 
\end{lemma}
\begin{proof}
    $\Li_{0}(t)$ satisfies this by Table \ref{table: polylogarithm values}. We proceed by induction, supposing that $\Li_{-k}(t)\in t\cdot\ZZ[t,\frac{1}{1-t}]$, we have by \Cref{lem: polylogarithm differential equation} that $\Li_{-k-1}(t)=t\cdot\nabla\Li_{-k}(t)$. The induction hypothesis implies $\Li_{-k}(t)$ is a $\ZZ$-linear combination of elements of the form $\frac{t^{a}}{(1-t)^{b}}$ so by the quotient rule, the derivative lies in $\ZZ[t,\frac{1}{1-t}]$ which suffices by the discussion above. 
\end{proof}
Elements of in the ring $t\cdot\ZZ[t,\frac{1}{1-t}]$ behave especially nicely with respect to exponentiation. 
\begin{lemma}\label{lem: behavior of nonpositive dilogarithms under exponentials}
    Let $A$ be a ring of characteristic 0. If $f(x)\in t\cdot A[t,\frac{1}{1-t}][[x]]$ then $\exp(f)$ admits a power series expansion in $\mathrm{Frac}(A)[t,\frac{1}{1-t}][[x]]$ as $x\to 0$. 
\end{lemma}
\begin{proof}
    Let us write $f(x)=\sum_{n\geq0}c_{n}(t)x^{n}$ with $c_{n}(t)\in t\cdot A[t,\frac{1}{1-t}]$ depending on $n$. We compute
    \begin{align*}
        \exp(f) &= \exp\left(\sum_{n\geq0}c_{n}(t)x^{n}\right) \\
        &= \prod_{n\geq0}\exp(c_{n}(t)x^{n}) \\
        &= \prod_{n\geq0}\sum_{k\geq0}\frac{c_{n}(t)^{k}}{k!}x^{nk}.
    \end{align*}
    However, for any fixed $N$, the coefficient of $x^{N}$ is a polynomial combination of terms $\frac{c_{n}(t)^{k}}{k!}$ where $n\leq N, k\leq N$ of which there are only finitely many, in particular given by some restriction of $\prod_{0\leq n\leq N}\sum_{0\leq k\leq N}\frac{c_{n}(t)^{k}}{k!}$ which lies in $\mathrm{Frac}(A)[t,\frac{1}{1-t}]$ since each term does. 
\end{proof}

With this language in hand, we deduce the following asymptotic result about the Pochhammer symbol $(t;q)_{\infty}$. 
\begin{proposition}\label{prop: asymptotics q t Pochhammer at 1}
    The $q$-Pochhammer symbol $(t;q)_{\infty}$ satisfies the asymptotic formula
    \begin{equation}\label{eqn: asymptotics of q t Pochhammer at 1}
        (t;q)_{\infty}\sim\exp\left(\frac{\Li_{2}(t)}{h}\right)\cdot\sqrt{1-t}\cdot O(h)
    \end{equation}
    as $q\to 1$ with $O(h)\in\QQ[t,\frac{1}{1-t}][[h]]$. 
\end{proposition}
\begin{proof}
    We compute 
    \begin{align*}
        \frac{t^{\ell}}{\ell(1-q^{\ell})} &= \frac{t^{\ell}}{\ell(1-e^{h\ell})} && q^{\ell}=(e^{h})^{\ell}=e^{h\ell}\\
        &= \frac{h\ell}{1-e^{h\ell}}\cdot\frac{t^{\ell}}{h\ell^{2}} \\
        &= -\sum_{k\geq0}\frac{B_{k}}{k!}(h\ell)^{k}\cdot\frac{t^{\ell}}{h\ell^{2}} && \frac{h\ell}{1-e^{h\ell}}=-\sum_{k\geq0}\frac{B_{k}}{k!}(h\ell)^{k} \\
        &= -\sum_{k\geq0}\left(\frac{t^{\ell}}{h\ell^{2}}\cdot\frac{B_{k}}{k!}\cdot(h\ell)^{k}\right)
    \end{align*}
    so applying this to $\log(t;q)_{\infty}^{-1}$, we have by \Cref{prop: logarithm at worst simple poles at roots of unity} that
    \begin{align*}
        -\log(t;q)_{\infty}^{-1} &= -\sum_{\ell\geq1}\frac{t^{\ell}}{\ell(1-q^{\ell})} \\ 
        &= \sum_{\ell\geq1}\left(\sum_{k\geq0}\frac{t^{\ell}}{h\ell^{2}}\cdot\frac{B_{k}}{k!}\cdot(h\ell)^{k}\right)&& \text{as above} \\
        &= \sum_{k\geq0}\left(\sum_{\ell\geq1}\frac{t^{\ell}}{\ell^{2-k}}\right)\frac{B_{k}}{k!}\cdot h^{k-1} && \\
        &= \sum_{k\geq0}\Li_{2-k}(t)\cdot\frac{B_{k}}{k!}\cdot h^{k-1} && \Li_{2-k}(t)=\sum_{\ell\geq1}\frac{t^{\ell}}{\ell^{2-k}}.
    \end{align*}
    We write this as 
    $$\Li_{2}(t)\cdot B_{0}\cdot\frac{1}{h} + \Li_{1}(t)\cdot B_{1}+\sum_{k\geq2}\Li_{2-k}(t)\cdot\frac{B_{k}}{k!}\cdot h^{k-1}.$$
    Note here that the third summand is a power series in $h$ with coefficients in $\QQ[t,\frac{1}{1-t}]$. Exponentiating, we get, up to constants, 
    $$\exp\left(\frac{\Li_{2}(t)}{h}\right)\cdot\sqrt{1-t}\cdot O(h)$$
    where the second factor follows from $B_{0}=\frac{1}{2}$ and $\exp(-\frac{1}{2}\log(1-t))=\sqrt{1-t}$, and the third factor from applying \Cref{lem: behavior of nonpositive dilogarithms under exponentials} to the observation above. 
\end{proof}
\begin{remark}
    Something similar to \Cref{prop: asymptotics q t Pochhammer at 1} is true for all Nahm sums. 
\end{remark}
\begin{remark}
    It is crucial here to do the expansion in terms of $h$ in order to get a simple result. Doing a power series expansion in other variables will necessitate the use of much more complicated functions. 
\end{remark}



The proofs we have encountered thus far have largely centered around explicit computation, yielding qualitative descriptions of the expansions. The qualitative features of the higher order terms $a_{i}(t)$ of (\ref{eqn: asymptotics of q t Pochhammer at 1}) can in fact be defined recursively by integrating lower order terms. The fact that the integrals of these rational functions remain rational without introducing exotic functions hints at the existence of additional underlying structure to these Nahm sums that may allow qualitative behavior to be deduced without explicit computation.

Returning to the broader discussion at hand, \Cref{prop: adapted q-deformed exponential} suggests that $(t;q)_{\infty}^{-1}$ is the $q$-analogue of the exponential function, and recovering the classical exponential as $q\to 1$, but the behavior we have deduced above is indeed much more complicated. This arises as a consequence of working with $\nabla_{q}'$ in place of $\nabla_{q}$.

To show the asymptotics at other roots of unity, we will require Bernoulli polynomials. 
\begin{definition}[Bernoulli Polynomial]\label{def: Bernoulli polynomial}
    Let $n\geq0$. The $n$th Bernoulli polynomial $B_{n}(t)$ is the $n$th coefficient in the power series expansion 
    $$-\frac{xe^{tx}}{1-e^{x}}=\sum_{n\geq0}\frac{B_{n}(t)}{n!}x^{n}\in\QQ[t][[x]].$$
\end{definition}
We state some elementary properties of Bernoulli polynomials.  
\begin{lemma}\label{lem: properties of Bernoulli polynomials}
    The Bernoulli polynomials satisfy the following identities:
    \begin{enumerate}[label=(\roman*)]
        \item $B_{n}(0)=B_{n}$, 
        \item $B_{n}(t+1)-B_{n}(t)=nt^{n-1}$, and 
        \item $B_{n}(k)=B_{k}+n\cdot\sum_{i=0}^{k-1}i^{n-1}$ for $k\in\NN$. 
    \end{enumerate}
\end{lemma}
\begin{proof}[Proof of (i)]
    This is immediate from the definition. We have $-\frac{xe^{0\cdot x}}{1-e^{x}}=-\frac{x}{1-e^{x}}$ recovering \Cref{def: Bernoulli number}. 
\end{proof}
\begin{proof}[Proof of (ii)]
    The finite difference formula follows from 
    \begin{align*}
        \sum_{n\geq0}\left(B_{n}(t+1)-B_{n}(t)\right)\frac{x^{n}}{t!}&= \frac{xe^{(t+1)x}-xe^{tx}}{e^{x}-1} \\
        &= \frac{xe^{tx}(e^{x}-1)}{(e^{x}-1)} \\
        &= xe^{tx} \\
        &= \sum_{n\geq0}\frac{t^{n}}{n!}x^{n+1} && e^{tx}=\sum_{n\geq0}\frac{t^{n}}{n!}x^{n} \\
        &= \sum_{n\geq0}(nt^{n-1})\cdot\frac{x^{n}}{n!} 
    \end{align*}
    where the equality is given termwise. 
\end{proof}
\begin{proof}[Proof of (iii)]
    Rearranging (ii) we get the recursion $B_{n}(t+1)=nt^{n-1}+B_{n}(t)$ so by induction for any natural number $k$ we have
    \begin{align*}
        B_{n}(k)=B_{n}(0)+n\sum_{i=0}^{k-1}i^{n-1}
    \end{align*}
    as desired. 
\end{proof}

The first few Bernoulli polynomials are given as follows \cite{BernoulliPolynomial}. 
\begin{table}[h]\label{table: Bernoulli polynomials}
    \begin{align*}
        B_{0}(t)&=1 && B_{1}(t)=t-\frac{1}{2} \\
        B_{2}(t)&=t^{2}-t+\frac{1}{6} && B_{3}(t)=t^{3}-\frac{3}{2}t^{2}+\frac{1}{2}t\\
        B_{4}(t)&=t^{4}-2t^{3}+t^{2}-\frac{1}{30} && B_{5}(t)=t^{5}-\frac{5}{2}t^{4}+\frac{5}{3}t^{3}+\frac{1}{6}t
    \end{align*}
    \caption{Bernoulli polynomials $B_{n}(t)$ for $0\leq n\leq 5$.}
\end{table}

We now treat the asymptotics at other roots of unity, taking $q=\zeta_{m}\exp(h)$ where $\zeta_{m}$ is a primitive $m$th root of unity. 

\begin{lemma}\label{lem: summand expansion at roots of unity}
    Let $\zeta_{m}$ be a primitive $m$th root of unity and $q=\zeta_{m}\exp(h)$. Then 
    \begin{equation}\label{eqn: summand expansion at roots of unity}
        \frac{1}{\ell(1-q^{\ell})}\cdot t^{\ell} = -\sum_{n\geq0}\frac{t^{\ell}}{\ell^{2-n}}\left(\sum_{i=0}^{m-1}\zeta_{m}^{i\ell}\cdot B_{n}\left(\frac{i}{m}\right)\right)\frac{m^{n-1}}{n!}h^{n-1}.
    \end{equation}
\end{lemma}
\begin{proof}
    We compute 
    \begin{align*}
        \frac{1}{1-q^{\ell}} &=\frac{1}{1-\zeta_{m}^{\ell}e^{h\ell}} \\
        &= \frac{1}{1-(\zeta_{m}^{\ell}e^{h\ell})^{m}}\sum_{i=1}^{m-1}(\zeta_{m}^{\ell}e^{h\ell})^{i} && \frac{1}{1-x}=\frac{1+x+\dots+x^{m-1}}{1-x^{m}}
    \end{align*}
    so for each summand of (\ref{eqn: expresssion of logarithm of q exponential}) in \Cref{prop: logarithm at worst simple poles at roots of unity}, we have 
    \begin{align*}
        \frac{t^{\ell}}{\ell(1-q^{\ell})}&=\frac{t^{\ell}}{\ell}\cdot\frac{1}{1-(\zeta_{m}^{\ell}e^{h\ell})^{m}}\sum_{i=0}^{m-1}(\zeta_{m}^{\ell}e^{h\ell})^{i} \\
        &= \sum_{i=0}^{m-1}\frac{\zeta_{m}^{i\ell}e^{ih\ell}}{\ell(1-\zeta_{m}^{m\ell}e^{mh\ell})}\cdot t^{\ell} \\
        &= \sum_{i=0}^{m-1}\frac{\zeta_{m}^{i\ell}e^{ih\ell}}{\ell(1-e^{mh\ell})}\cdot t^{\ell} && \zeta_{m}^{m\ell}=1 \\
        &= \sum_{i=0}^{m-1}\frac{\zeta_{m}^{i\ell}e^{\frac{i}{m}x}}{1-e^{x}}\cdot \frac{t^{\ell}}{\ell} && x=mh\ell \\
        &= \frac{t^{\ell}}{\ell}\sum_{i=0}^{m-1}\zeta_{m}^{i\ell}\left(\frac{e^{\frac{i}{m}x}}{1-e^{x}}\right) \\
        &= \frac{t^{\ell}}{\ell}\sum_{i=0}^{m-1}\zeta_{m}^{i\ell}\cdot\frac{1}{x}\left(\frac{xe^{\frac{i}{m}x}}{1-e^{x}}\right) \\
        &= \frac{t^{\ell}}{\ell}\sum_{i=0}^{m-1}\zeta_{m}^{i\ell}\frac{1}{x}\left(-\sum_{n\geq0}\frac{B_{n}(\frac{i}{m})}{n!}x^{n}\right) && \text{\Cref{def: Bernoulli polynomial}} \\
        &= -\frac{t^{\ell}}{\ell}\sum_{i=0}^{m-1}\zeta_{m}^{i\ell}\left(\sum_{n\geq0}\frac{B_{n}(\frac{i}{m})}{n!}x^{n-1}\right) \\
        &= -\frac{t^{\ell}}{\ell}\sum_{n\geq0}\left(\sum_{i=0}^{m-1}\zeta_{m}^{i\ell}\cdot B_{n}\left(\frac{i}{m}\right)\right)\frac{x^{n-1}}{n!} \\
        &= -\frac{t^{\ell}}{\ell}\sum_{n\geq0}\left(\sum_{i=0}^{m-1}\zeta_{m}^{i\ell}\cdot B_{n}\left(\frac{i}{m}\right)\right)\frac{m^{n-1}h^{n-1}\ell^{n-1}}{n!} && x=mh\ell\\
        &= -\sum_{n\geq0}\frac{t^{\ell}}{\ell^{2-n}}\left(\sum_{i=0}^{m-1}\zeta_{m}^{i\ell}\cdot B_{n}\left(\frac{i}{m}\right)\right)\frac{m^{n-1}}{n!}h^{n-1}
    \end{align*}
    giving an expression of the power series in terms of $h$. 
\end{proof}
We will require the following statements in what follows. 
\begin{lemma}\label{lem: dilogarithm roots of unity sum}
    The dilogarithm satisfies the identity 
    \begin{equation}\label{eqn: dilogarithm roots of unity sum}
        \frac{1}{m^{n-1}}\cdot\Li_{n}(t^{m}) = \sum_{i=0}^{m-1}\Li_{n}(\zeta_{m}^{i}t)
    \end{equation}
    for $m,n\in\NN$. 
\end{lemma}
\begin{proof}
    We have 
    \begin{align*}
        \sum_{i=0}^{m-1}\Li_{n}(\zeta_{m}^{i}t) &= \sum_{i=0}^{m-1}\left(\sum_{\ell\geq1}\frac{(\zeta_{m}^{i}t)^{\ell}}{\ell^{n}}\right) \\
        &= \sum_{\ell\geq 1}\left(\sum_{i=0}^{m-1}\zeta_{m}^{i\ell}\right)\frac{t^{\ell}}{\ell^{n}}
    \end{align*}  
    and now noting 
    $$\sum_{i=0}^{m-1}\zeta_{m}^{i\ell}=\begin{cases}
        m & m|\ell \\
        0 & m\nmid\ell
    \end{cases}$$
    the summands above vanish if $\ell$ is not a multiple of $m$ so the sum is in fact given by the sum over $m$-multiples
    $$\sum_{\ell\geq1}\frac{mt^{m\ell}}{(m\ell)^{n}}=\frac{1}{m^{n-1}}\sum_{\ell\geq1}\frac{t^{m\ell}}{\ell^{n}}=\frac{1}{m^{n-1}}\cdot\Li_{n}(t^{m}).$$
\end{proof}
We recover the behavior $q\to\zeta_{m}$ as $h\to0$ so applying the expansion of \Cref{lem: summand expansion at roots of unity} to \Cref{prop: logarithm at worst simple poles at roots of unity}, we get the following asymptotic result. 
\begin{proposition}\label{prop: asymptotics of q t Pochhammer at root of unity}
    The $q$-Pochhammer symbol $(t;q)_{\infty}$ satisfies the asymptotic formula
    \begin{equation}\label{eqn: asymptotics of q t Pochhammer at root of unity}
        (t;q)_{\infty}\sim\exp\left(\frac{\Li_{2}(t^{m})}{m^{2}h}\right)\cdot\frac{\sqrt{1-t^{m}}}{\prod_{i=0}^{m-1}\cdot\left(1-\zeta_{m}t\right)^{i/m}}\cdot O(h)
    \end{equation}
    as $q\to \zeta_{m}$ with $\zeta_{m}$ a primitive $m$th root of unity and $O(h)\in\QQ(\zeta_{m})[t,\frac{1}{1-t^{m}}][[h]]$.
\end{proposition}
\begin{proof}
    We compute  
    \begin{align*}
        -\log(t;q)_{\infty}^{-1} &= -\sum_{\ell\geq1}\frac{1}{\ell(1-q^{\ell})}\cdot t^{\ell} \\
        &= \sum_{\ell\geq1}\left(\sum_{n\geq0}\frac{t^{\ell}}{\ell^{2-n}}\left(\sum_{i=0}^{m-1}\zeta_{m}^{i\ell}\cdot B_{n}\left(\frac{i}{m}\right)\right)\frac{m^{n-1}}{n!}h^{n-1}\right)&& \text{by }(\ref{eqn: summand expansion at roots of unity}) \\
        &=\sum_{n\geq0}\left(\sum_{\ell\geq 1}\frac{t^{\ell}}{\ell^{2-n}}\left(\sum_{i=0}^{m-1}\zeta_{m}^{i\ell}B_{n}\left(\frac{i}{m}\right)\right)\right)\frac{m^{n-1}}{n!}h^{n-1}
    \end{align*}
    by observation, the terms for $n\geq 2$ are power series in $h$, and so too is its exponent, so it remains to consider the first two terms of the series given by 
    \begin{align*}
        \left(\sum_{\ell\geq 1}\frac{t^{\ell}}{\ell^{2-n}}\left(\sum_{i=0}^{m-1}\zeta_{m}^{i\ell}B_{0}\left(\frac{i}{m}\right)\right)\right)\frac{1}{mh}&=\left(\sum_{\ell\geq 1}\frac{t^{\ell}}{\ell^{2-n}}\left(\sum_{i=0}^{m-1}\zeta_{m}^{i\ell}\right)\right)\frac{1}{mh}  && B_{0}(t)=1 \\
        &= \left(\sum_{\ell'\geq1}\frac{mt^{m\ell'}}{(m\ell')^{2}}\right)\frac{1}{mh} \\
        &=\frac{1}{m^{2}}\left(\sum_{\ell'\geq1}\frac{t^{m\ell'}}{\ell'^{2}}\right)\frac{1}{h} \\
        &= \frac{1}{m^{2}h}\Li_{2}(t^{m})
    \end{align*}
    and 
    \begin{align*}
        \sum_{\ell\geq 1}\frac{t^{\ell}}{\ell^{2-n}}\left(\sum_{i=0}^{m-1}\zeta_{m}^{i\ell}B_{1}\left(\frac{i}{m}\right)\right)&= \sum_{\ell\geq 1}\frac{t^{\ell}}{\ell^{2-n}}\left(\sum_{i=0}^{m-1}\zeta_{m}^{i\ell}\left(\frac{i}{m}-\frac{1}{2}\right)\right) && \text{Table \ref{table: Bernoulli polynomials}}\\
        &= \sum_{i=0}^{m-1}\frac{i}{m}\left(\sum_{\ell\geq1}\frac{(\zeta_{m}^{i}t)^{\ell}}{\ell}\right)-\frac{1}{2}\sum_{\ell\geq1}\left(\sum_{i=0}^{\infty}\zeta_{m}^{i\ell}\right) \\
        &= \sum_{i=0}^{m-1}\frac{i}{m}\Li_{1}(\zeta_{m}^{i}t) + \frac{1}{2}\log(1-t^{m}) \\
        &= \frac{1}{2}\log(1-t^{m})+\sum_{i=0}^{m-1}\frac{i}{m}\log(1-\zeta_{m}^{i}t)
    \end{align*}
    respectively. Exponentiating, we get, up to constants, 
    $$\exp\left(\frac{\Li_{2}(t^{m})}{m^{2}h}\right)\cdot\frac{\sqrt{1-t^{m}}}{\prod_{i=0}^{m-1}\cdot\left(1-\zeta_{m}t\right)^{i/m}}\cdot O(h).$$
\end{proof}
Qualitatively, this is quite similar to the asymptotic expansion gleaned in (\ref{eqn: asymptotics of q t Pochhammer at 1}) albeit with a more complicated factor of $O(h)$. In more general settings, the factor of $O(h)$ is the \'{e}tale regulator maps $K$-theory and becomes increasingly difficult to understand. 
% The expansion property of n\geq 2 terms should be able to be gleaned using the different approach outlined by grouping the Pochhammer symbol by residue classes modulo m. If q= \zeta_{m}\exp(h) so q^m is close to 1 and know the asymptotics of (q^i t; q^m)_{\infty}. 
\newpage
\printbibliography
\end{document}