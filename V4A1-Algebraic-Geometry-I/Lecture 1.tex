\section{Lecture 1 -- 7th October 2024}\label{sec: lecture 1}
All rings are to be taken as commutative.

This will be a lecture course on algebraic geometry based on the abstract perspective of sheaves and schemes. The basic construction to be introduced is sheaves, which are defined as presheaves that satisfy additional properties. Sheaves keep track of both local and global information on topological spaces.\marginpar{Sheaves can in fact be defined more abstractly using sites and Grothendieck topologies.} We first consider the case of presheaves. 
\begin{definition}[Presheaf]\label{def: presheaf}
    Let $\Csf$ be a category and $X$ a topological space. A $\Csf$-valued presheaf $\Fcal$ on $X$ consists of the data of 
    \begin{enumerate}[label=(\roman*)]
        \item An object $\Fcal(U)$ of $\Csf$ for each $U\subseteq X$ open. 
        \item A morphism $\res_{U,V}:\Fcal(U)\to\Fcal(V)$ in $\Csf$ for each $V\subseteq U\subseteq X$ open. 
    \end{enumerate}
    such that $\res_{U,U}:\Fcal(U)\to\Fcal(U)$ is the identity on $\Fcal(U)$ and the triangle 
    \begin{equation}\label{diag: presheaf triangle}
        % https://q.uiver.app/#q=WzAsMyxbMCwwLCJcXEZjYWwoVSkiXSxbMiwwLCJcXEZjYWwoVikiXSxbMSwxLCJcXEZjYWwoVykiXSxbMCwxLCJcXHJlc197VSxWfSJdLFsxLDIsIlxccmVzX3tWLFd9Il0sWzAsMiwiXFxyZXNfe1UsV30iLDJdXQ==
        \begin{tikzcd}
            {\Fcal(U)} && {\Fcal(V)} \\
            & {\Fcal(W)}
            \arrow["{\res_{U,V}}", from=1-1, to=1-3]
            \arrow["{\res_{U,W}}"', from=1-1, to=2-2]
            \arrow["{\res_{V,W}}", from=1-3, to=2-2]
        \end{tikzcd}
    \end{equation}
    is commutative for all $W\subseteq V\subseteq U\subseteq X$ open. 
\end{definition}
\begin{remark}
    Commutativity of the triangle is equivalent to $\res_{U,W}=\res_{V,W}\circ\res_{U,V}$ as morphisms in $\Csf$. 
\end{remark}
We will typically consider examples where $\Csf$ is the category of Abelian groups $\AbGrp$, of rings $\Ring$, of modules over a fixed ring $A$ $\Mod_{A}$ and their appropriate generalizations in the setting of schemes. While it is typically bad categorical practice to look ``within'' objects of a category, we refer to elements of $\Fcal(U)$\marginpar{We will also use the notation $\Gamma(U,\Fcal)$ and $H^{0}(U,\Fcal)$ for $\Fcal(U)$.} for $U\subseteq X$ open as (local) sections, with global sections being elements of $\Fcal(X)$. 
\begin{example}\label{ex: presheaves of continuous functions}
    Let $X$ be a topological space. $C_{X}$ which associates to $U\subseteq X$ open the set $C_{X}(U)$ of continuous functions from $U$ to $\RR$ and restriction maps given by restriction of continuous functions defines a presheaf on $X$. 
\end{example}
\begin{example}\label{ex: presheaves of continuous functions between topological spaces}
    Let $X,Y$ be topological spaces. $\Fcal$ which associates $U\subseteq X$ to the set $\Fcal(U)$ of continuous functions from $U$ to $Y$ and restriction maps given by restriction of continuous functions defines a presheaf on $X$.  
\end{example}
\begin{example}\label{ex: constant presheaves}
    Let $X$ be a topological space and $G$ an Abelian group. We can consider the association taking each open set $U\subseteq X$ to $G$ and restriction maps the identities. This construction defines a presheaf on $X$. 
\end{example}
We can alternatively define presheaves as a type of functor from the category of open sets of a topological space to $\Csf$. Recall here the definition of the category of open sets of a topological space $\Opens_{X}$.  
\begin{definition}[Category of Open Sets]\label{def: category of open sets}
    Let $X$ be a topological space. The category $\Opens_{X}$ of open sets of $X$ has objects open sets of $X$ and morphisms inclusions. 
\end{definition}
\begin{remark}
    Spelling things out, for $U,V$ open in $X$ there is one morphism from $V$ to $U$ if $V\subseteq U$ and no morphism otherwise. 
\end{remark}
We can show that $\Csf$-valued presheaves are merely contravariant functors from $\Opens_{X}$ to $\Csf$, which is essentially a formal result. 
\begin{proposition}\label{prop: presheaves are functors from open sets opposite}
    Let $X$ be a topological space. The data of a presheaf on $X$ corresponds uniquely to a functor $\Opens_{X}^{\Opp}\to\Csf$. 
\end{proposition}
\begin{proof}
    This statement follows by unwinding the definitions of functoriality. A functor associates $\Opens_{X}^{\Opp}\to\Csf$ associates to each open set of $X$ an object of $\Csf$ and by construction of the morphisms in $\Opens_{X}$ there is a morphism $V\to U$ if and only if $V\subseteq U$ corresponding to a morphism $U\to V$ in $\Opens_{X}^{\Opp}$ inducing the corresponding restriction maps in $\Csf$ by functoriality. Furthermore, commutativity of the triangle is preserved since the action of functors on morphisms distributes over composition and composition is unqiue.  
\end{proof}
One example of special interest in the study of algebraic geometry is the construction of a canonical presheaf on the spectrum of a commutative ring $A$. 
\begin{definition}[Spectrum of a Ring]\label{def: spectrum of a ring}
    Let $A$ be a ring. The spectrum $\spec(A)$ of $A$ is the topological space with points given by prime ideals of $A$ and closed sets of the form $V(a)$ for $a\in A$ consisting of the prime ideals containing $a$. 
\end{definition}
Note that no prime ideals contain the multiplicative unit of $A$ so for $a_{1},\dots,a_{n}$ generating $A$ as an $A$-module, the sets $D(a_{i})=\spec(A)\setminus V(a_{i})$ consisting of the prime ideals not containing $A$ form a basis for the toplogy of $\spec(A)$. 

We can now consider the following construction. 
\begin{proposition}\label{prop: structure presheaf on spec A}
    Let $A$ be a ring. Consider the association
    \begin{equation}\label{eqn: structure presheaf on spec A}
        U\mapsto \left\{s:U\to\coprod_{\pfrak\in U}A_{\pfrak}:\substack{\forall\pfrak\in U, s(\pfrak)\in A_{\pfrak} \text{ and } \\ \exists U'\subseteq X, p\in U'\subseteq U, \exists a,b\in A \text{ s.t. }b\notin\qfrak \forall\qfrak\in U', s(\qfrak)=\frac{a}{b}\in A_{\qfrak}}\right\}
    \end{equation}
    and for $V\subseteq U\subseteq X$ the forgetful maps. This association defines a presheaf of rings on $\spec(A)$.\marginpar{This defines a presheaf in terms of compatible stalks following \cite[\S 2.2]{Hartshorne}, which to me seems more artificial than in terms of localizations as in \cite[\S 4.1]{Vakil}}
\end{proposition}
\begin{proof}
    The category of rings is bicomplete, in particular admitting colimits and thus coproducts giving the sets of (\ref{eqn: structure presheaf on spec A}) the structure of a ring by pointwise operations. We get forgetful maps 
    \begin{align*}
        &\left\{s:U\to\coprod_{\pfrak\in U}A_{\pfrak}:\substack{\forall\pfrak\in U, s(\pfrak)\in A_{\pfrak} \text{ and } \\ \exists V\subseteq X, p\in U'\subseteq U, \exists a,b\in A \text{ s.t. }b\notin\qfrak \forall\qfrak\in U', s(\qfrak)=\frac{a}{b}\in A_{\qfrak}}\right\} \\
        &\hspace{0.5cm}\to\left\{s|_{V}:V\to\coprod_{\pfrak\in V}A_{\pfrak}:\substack{\forall\pfrak\in U', s(\pfrak)\in A_{\pfrak} \text{ and } \\ \exists U'\subseteq X, p\in U'\subseteq U, \exists a,b\in A \text{ s.t. }b\notin\qfrak \forall\qfrak\in U', s(\qfrak)=\frac{a}{b}\in A_{\qfrak}}\right\}
    \end{align*}
    for $U'\subseteq U$ that satisfy the commutativity of the diagram (\ref{diag: presheaf triangle}) and hence defines a presheaf. 
\end{proof}
We will set this as the structure presheaf $\Ocal_{\spec(A)}$ of  $\spec(A)$ since it captures important geometric phenomena that we seek to capture as motivated by the following example. 
\begin{example}\label{ex: regular functions on affine scheme}
    Let $k$ be an algebraically closed field and $A$ a finite type $k$-algebra. Consider an open $U\subseteq\spec(A)$ and $s:U\to\coprod_{\pfrak\in U}A_{\pfrak}$ as in (\ref{eqn: structure presheaf on spec A}). We can construct $\overline{s}:U\cap\mathrm{mSpec}(A)\to k$ whose value on each maximal ideal $\mfrak$ is the image of $s(\mfrak)\in A_{\mfrak}$ in the quotient $A_{\mfrak}/\mfrak A_{\mfrak}$ which is isomorphic to $k$ since $k$ is algebraically closed. $s$ is regular on $U$ and thus the association of (\ref{eqn: structure presheaf on spec A}) can be thought of as a presheaf of regular functions. 
\end{example}
Having discussed presheaves, we can define morphisms of presheaves as follows. 
\begin{definition}[Morphism of Presheaves]\label{def: morphism of presheaves}
    Let $X$ be a topological space and $\Fcal,\Fcal'$ $\Csf$-valued presheaves on $X$. A morphism of presheaves $\phi:\Fcal\to\Fcal'$ consists of the data of morphisms $\phi_{U}:\Fcal(U)\to\Fcal'(U)$ in $\Csf$ such that the diagram 
    $$% https://q.uiver.app/#q=WzAsNCxbMCwwLCJcXEZjYWwoVSkiXSxbMiwwLCJcXEZjYWwnKFUpIl0sWzAsMSwiXFxGY2FsKFYpIl0sWzIsMSwiXFxGY2FsJyhWKSJdLFswLDIsIlxccmVzX3tVLFZ9IiwyXSxbMSwzLCJcXHJlc197VSxWfSciXSxbMCwxLCJcXHBoaV97VX0iXSxbMiwzLCJcXHBoaV97Vn0iLDJdXQ==
    \begin{tikzcd}
        {\Fcal(U)} && {\Fcal'(U)} \\
        {\Fcal(V)} && {\Fcal'(V)}
        \arrow["{\phi_{U}}", from=1-1, to=1-3]
        \arrow["{\res_{U,V}}"', from=1-1, to=2-1]
        \arrow["{\res_{U,V}'}", from=1-3, to=2-3]
        \arrow["{\phi_{V}}"', from=2-1, to=2-3]
    \end{tikzcd}$$
    is commutative for $V\subseteq U\subseteq X$ open.
\end{definition}
\begin{remark}\label{rmk: morphism of presheaves is a natural transformation}
    A morphism of presheaves is equivalent to the data of a natural transformation between the corresponding functors $\Opens_{X}^{\Opp}\to\Csf$. 
\end{remark}
This data allows us to define the category of $\Csf$-valued presheaves $\PSh(X,\Csf)$ for a topological space $X$, where we omit the ``value-category'' $\Csf$ when it is clear from context. 
\begin{example}\label{ex: inclusion of differentiable to continuous functions}
    In analogy to \Cref{ex: presheaves of continuous functions}, let $X$ be a topological space, $C_{X}$ the presheaf of continuous functions on $X$, and $C^{\mathrm{diff}}_{X}$ the presheaf of differentiable functions on $X$. There is a natural morphism of presheaves $C_{X}^{\mathrm{diff}}\to C_{X}$ induced by the inclusions of differentiable into continuous functions on each open set. 
\end{example}
\begin{example}\label{ex: circle valued presheaves}
    Let $X$ be a topological space and $S^{1}$ the circle. We can define a presheaf of continuous functions valued in $S^{1}$ by exponentiating a locally continuous function $e^{2\pi i f}$ for $f$ locally continuous. This defines a morphism of presheaves $C_{X}$ to $S^{1}$-valued presheaves in analogue to \Cref{ex: presheaves of continuous functions between topological spaces} with $Y=S^{1}$. 
\end{example}
Having discussed presheaves in some depth, we can now specialize to the case of sheaves, which are presheaves satisfying additional structure. 
\begin{definition}[Sheaves]\label{def: sheaves}
    Let $X$ be a topological space and $\Csf$ be a category admitting arbitrary products.\marginpar{See also the discussion in \cite[\href{https://stacks.math.columbia.edu/tag/00VL}{Tag 00VL}]{stacks-project}.} A presheaf $\Fcal$ on $X$ is a sheaf if for all $U\subseteq X$ and all open covers $\{U_{i}\}_{i\in I}$ the sequence 
    $$% https://q.uiver.app/#q=WzAsMyxbMCwwLCJcXEZjYWwoVSkiXSxbMiwwLCJcXHByb2Rfe2lcXGluIEl9XFxGY2FsKFVfe2l9KSJdLFs0LDAsIlxccHJvZF97aSxqXFxpbiBJfVxcRmNhbChVX3tpfVxcY2FwIFVfe2p9KSJdLFswLDFdLFsxLDIsIiIsMCx7Im9mZnNldCI6LTF9XSxbMSwyLCIiLDIseyJvZmZzZXQiOjF9XV0=
    \begin{tikzcd}
        {\Fcal(U)} && {\prod_{i\in I}\Fcal(U_{i})} && {\prod_{i,j\in I}\Fcal(U_{i}\cap U_{j})}
        \arrow[from=1-1, to=1-3]
        \arrow[shift left, from=1-3, to=1-5]
        \arrow[shift right, from=1-3, to=1-5]
    \end{tikzcd}$$
    is an equalizer where the parallel maps are given by 
    $$(s_{i})_{i\in I}\mapsto (\res_{U_{i}, U_{i}\cap U_{j}}(s_{i}))_{i,j\in I}\text{ and }(s_{i})_{i\in I}\mapsto (\res_{U_{j}, U_{i}\cap U_{j}}(s_{j}))_{i,j\in I},$$
    respectively.
\end{definition}
\begin{remark}\label{rmk: sheaves valued in abelian categories}
    If $\Csf$ is an Abelian category, the equalizer condition can be replaced by $\Fcal(U)$ being the kernel of $\prod_{i\in I}\Fcal(U_{i})\to\prod_{i,j\in I}\Fcal(U_{i}\cap U_{j})$ by $(s_{i})_{i\in I}\mapsto (s_{i}|_{U_{i}\cap U_{j}}-s_{j}|_{U_{i}\cap U_{j}})_{i,j\in I}$. 
\end{remark}
Explicitly, the local behavior states that if there is an open set $U\subseteq X$ with an open cover $\{U_{i}\}_{i\in I}$, a collection of sections $(s_{i})_{i\in I}$ glues to a section on $U$ if and only if $s_{i}|_{U_{i}\cap U_{j}}=s_{j}|_{U_{i}\cap U_{j}}$ for all $i,j\in I$. This proves to be the quality of sheaves that makes them more geometric than presheaves: while functions restrict naturally, they also glue naturally which is behavior captured by sheaves but not by presheaves. 
\begin{example}
    Let $U$ be a topological space and $\{U_{1},U_{2}\}$ an open cover of $U$. Consider continous functions from $U$ to $\RR$. The sheaf axiom allows for the construction of a continuous function $f:U\to \RR$ from continuous functions $f_{1}:U_{1}\to R$ and $f_{2}:U_{2}\to\RR$ that agree on the intersection $U_{1}\cap U_{2}$. 
\end{example}
In particular, we can see that \Cref{ex: presheaves of continuous functions} and \Cref{prop: structure presheaf on spec A} in fact define sheaves on $X$ and $\spec(A)$, respectively. 