\section{Lecture 1 -- 11th October 2024}\label{sec: lecture 1}
The goal of this course is to develop a theory of analytic functions in several variables over non-Archimedean fields -- analogous to the theory of functions in several complex variables -- compatible with the algebraic geometry over such fields in the style of Serre's GAGA principle. 

Several problems arise from this desideratum:
\begin{itemize}
    \item The topology on $\QQ_{p}$ and $\CC_{p}$, the $p$-completion of its algebraic closure, are highly disconnected. 
    \item It is not reasonable for the germs of an analytic function to determine the global behavior of the function. 
    \item The unit disc $\{z\in\CC_{p}:|z|<1\}\subseteq\CC_{p}$ is non-compact. 
\end{itemize}
John Tate proposed a solution to this problem was to force certain sets to become quasicompact,\marginpar{Analogues of this theory are also visible in real algebraic geometry as developed by the school of Delfs-Knebusch \cite{DK}.} a perspective elaborated on by Bosch, Grauert, Gr\"{u}nzer, and Remmert (vis. \cite{BGR}). This was complemented by an alternative approach employed by Fresnel and van der Put which added points to the relevant spaces, later further developed by Berkovich and Huber (vis. \cite{Berkovich,FvdP,Huber}). 

We will largely follow the approach of Tate and his followers, highlighting contemporary applications. 

We now make the following recollections from point set topology. We will largely ignore set-theoretic issues throughout. 
\begin{definition}[Sieve]\label{def: sieve}
    Let $X$ be an object of a category $\Csf$. A sieve on $X$ is a collection of morphisms with target $X$ that is closed under composition on the left: if $f:Y\to X$ is in the sieve, then $g\circ f:Y'\to X$ is in the sieve for all $g:Y'\to Y$. 
\end{definition}
Here are some examples.
\begin{definition}[All Sieve]\label{def: all sieve}
    Let $X$ be an object of a category $\Csf$. The all sieve on $X$ consists of all morphisms in $\Csf$ with target $X$. 
\end{definition}
\begin{definition}[Empty Sieve]\label{def: empty sieve}
    Let $X$ be an object of a category $\Csf$. The empty sieve on $X$ consists of no morphisms. 
\end{definition}
Given a collection of morphisms with fixed target, we can construct a sieve as follows. 
\begin{definition}[Sieve Generated by a Collection]\label{def: sieve generated by a collection}
    Let $X$ be an object of a category $\Csf$ and $\{f_{i}:X_{i}\to X\}_{i\in I}$ a collection of morphisms in $\Csf$ with target $X$. The sieve generated by this collection consists of those morphisms $g:Y\to X$ such that $g$ factors through some $X_{i}$.\marginpar{Explicitly, there exists some $i\in I$ and a map $\gamma:Y\to X_{i}$ such that $g=f_{i}\circ\gamma$.}
\end{definition}
We should think of sieves as taking away objects that are ``small enough'' with respect to $X$ in the sense that they admit a map to $X$.  