\section{Lecture 6 -- 29th October 2024}\label{sec: lecture 6}
Recall the construction of the regular affine cubic using the Weierstrass $\wp$-functions \Cref{prop: holomorphic embedding by Weierstrass function}. A natural question of the solution to the inverse problem arises. In particular: given an affine regular cubic $E\subseteq\CC^{2}$, is there a lattice $\Omega$ such that the Weierstrass functions of $\Omega$ parametrize $E$? 

This can be shown using elliptic modular functions as hinted at in \Cref{sec: lecture 4} preceding \Cref{lem: Weierstrass zeta function} as further developed in \cite{Ahlfors}. We address the easier question of conditions of lattices that result in $K(\Omega),K(\Omega')$ being isomorphic. \Cref{thm: description of field of elliptic functions} connects this question to the former. We have the following preparatory lemmata. 
\begin{lemma}\label{lem: elliptic functions satisfy algebraic differential equation}
    Let $\Omega$ be a lattice. If $f(z)\in K(\Omega)$ an elliptic function then $f(z)$ satisifes an algebraic differential equation. 
\end{lemma}
\begin{proof}
    For a lattice $\Omega$, $K(\Omega)$ is a function field of dimension 1, that is, of transcendence degree over $\CC$ is 1. This implies that for any $f,g\in K(\Omega)$ there is an algebraic equation $F(x,y)$ such that $F(f,g)=0$ so applying this to an elliptic function $f(z)$ and its elliptic derivative $f'(z)$, we have the claim. 
\end{proof}
This extends to an algebraic relation between the elliptic functions $f(z),f(z+\omega)$ and the constant $f(\omega)$ to yield for $\omega\in\Omega$ an algebraic relation betewen $f(z)$ and $f(z+\omega)$. 


Let us now consider an alternative way to define elliptic functions. 
\begin{proposition}\label{prop: set of annulus functions form a field}
    Let $q\in\CC^{\times}$ with $0<|q|<1$. The set of meromorphic functions $f(z)$ on $\CC^{\times}$ satisfying $f(z)=f(qz)$ for all $z\in\CC^{\times}$ forms a field $K_{q}$. 
\end{proposition}
\begin{proof}
    By inspection, the set of such functions is closed under addition and multiplication, and multiplicative inversion. 
\end{proof}
Let $A_{q,r}$ be the annulus $\{z\in\CC^{\times}:r<|z|<\frac{r}{|q|}\}$. 
\begin{lemma}\label{lem: assumes all values in annulus}
    Let $q\in\CC^{\times}$ with $0<|q|<1$ and $f(z)$ meromorphic on $\CC^{\times}$ such that $f(z)=f(qz)$ for all $z\in\CC^{\times}$. Then $f$ assumes all values in $\widehat{\CC}$ already in the annulus $A_{q,r}$. Furthermore, if $f$ is holomorphic on $A_{q,r}$ then $f$ is constant. 
\end{lemma}
\begin{proof}
    If it does not assume all values of $\widehat{\CC}$ then it is bounded on the closure of the annulus and hence constant by Liouville's theorem \Cref{thm: Liouville}. 
\end{proof}
We can thus form the torus $T=\CC^{\times}/\langle q\rangle$ where $\langle q\rangle$ is the subgroup of $\CC^{\times}$ generated by multiples of $q$. This is a torus with fundamental domain represented by the annulus, identifying the inner boundary circle with the outer. 
\begin{lemma}\label{lem: q-invariant implies q-power-multiple}
    Let $f$ be holomorphic on $\CC^{\times}$ such that there is a constant $c\neq0$ satisfying $f(qz)=c\cdot f(z)$. Then $c=q^{k}$ for some $k\in\ZZ$ and $f(z)=f(1)\cdot z^{k}$. 
\end{lemma}
\begin{proof}
    Consider the Laurent series expansion of $f(z)$ given by $\sum_{n=-N}^{\infty}a_{n}z^{n}$ for $N\in\NN$. We can compare coefficients 
    $$\sum_{n=-N}^{\infty}ca_{n}z^{n}=\sum_{n=-N}^{\infty}a_{n}q^{n}z^{n}$$
    which are necessarily equal for all $n$, and which holds if and only if $a_{n}=0$ for all but one $n=k$ at which we have $ca_{k}z^{k}=a_{k}q^{k}z^{k}$ with the coefficent $a_{k}$ recovered as $f(1)$, giving the claim. 
\end{proof}
The subsequent lemma allows us to produce a class of meromorphic functions on $\CC^{\times}$ that are invariant under $q$-multiplication. 
\begin{lemma}\label{lem: producing q-scaling invariant functions}
    Let $q\in\CC^{\times}$ with $0<|q|<1$. Then the function 
    $$p(z)=\left(\prod_{n=1}^{\infty}(1-q^{n}z)\right)\left(\prod_{n=0}^{\infty}\left(1-\frac{q^{n}}{z}\right)\right)$$
    is holomorphic on $\CC^{\times}$, and satisfies $p(qz)=-\frac{1}{qz}p(z)$. 
\end{lemma}
\begin{proof}
    By inspection, the only pole of the function is at 0, and the second statement follows from a direct computation. 
\end{proof}
We can now show the main result. 
\begin{theorem}\label{thm: function on annulus with prescribed zeroes and poles}
    Let $a_{1},\dots,a_{n}$ and $a_{n+1},\dots,a_{\ell}$ be lists of distinct elements in the annulus
    $$A_{q,r}=\{z\in\CC^{\times}:r<|z|<\frac{r}{|q|}\}$$
    and $m_{1},\dots,m_{n},m_{n+1},\dots,m_{\ell}$ a list of positive integers. The following are equivalent:\marginpar{Theorem 6.2}
    \begin{enumerate}[label=(\alph*)]
        \item There exists a function $f\in K_{q}$ with zeroes at $a_{1},\dots,a_{n}$ with multiplicities $m_{1},\dots,m_{n}$ and poles at $a_{n+1},\dots,a_{\ell}$ with multiplicities $m_{n+1},\dots,m_{\ell}$, respectively. 
        \item $\ell=2n$ and $a_{1}\dots a_{n}q^{k}=a_{n+1}\dots a_{\ell}$ for some integer $k$. 
    \end{enumerate}
\end{theorem}
\begin{proof}
    We make some preliminary observations. Note $p$ has simple zeroes at each positive $q$-power by inspection. And by passage to the inverse of a function, can assume that counted with multiplicity there are more zeroes than poles. Setting $h(z)=\frac{p(z/a_{1})\dots p(a/a_{m})}{p(z/a_{n+1})\dots p(z/a_{\ell})}$, we have that $h(z)$ is meromorphic on $\CC^{\times}$ as it is obtained as a quotient of holomorphic functions, and we can set $\lambda=\frac{a_{1}\dots a_{n}}{a_{n+1}\dots a_{\ell}}$. 

    (a)$\Rightarrow$(b) Let $f\in K_{q}$ be given with prescribed zeroes and poles and set $F(z)=f(z)/h(z)$. The zeroes of $h$ are some $q$-power multiplied by $a$ so they cancel in the annulus gainst zeroes of $f$, that is, $F$ is holomorphic in the annulus. Now $F(qz)=\frac{1}{\lambda}F(z)$ implying $F$ is holomorphic on $\CC^{\times}$. Applying \Cref{lem: q-invariant implies q-power-multiple}, we get $\frac{1}{\lambda}q^{k}$ which gives the claim. 
    
    (b)$\Rightarrow$(a) The function $f(z)=c\cdot z^{k}\cdot\left(\frac{p(z/a_{1})\dots p(a/a_{m})}{p(z/a_{n+1})\dots p(z/a_{\ell})}\right)$ has the desired properties. 
\end{proof}
Returning to the discussion of elliptic functions, the construction above gives a map $K_{q}\to K(\Omega)$ by $f(z)\mapsto f(\exp(2\pi i z))$ where the image of $p(z)$ is the Weierstrass $\sigma$ function up to a correction factor. We additionally remark that elliptic functions are used to parametrize parabolas and ellipses, and are at times used in physical mathematics. 