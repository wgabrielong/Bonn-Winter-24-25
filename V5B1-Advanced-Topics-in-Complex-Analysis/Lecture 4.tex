\section{Lecture 4 -- 22nd October 2024}\label{sec: lecture 4}
Following our previous dicussion, we seek to express $\wp'(z)^{2}$ the elliptic function of order 6 in terms of a degree 3 polynomial in $\wp(z)$ with coefficients determined by the lattice. In particular, we have the following theorem. 
\begin{theorem}\label{thm: Weierstrass differential equation}
    Let $\Omega$ be a lattice. \marginpar{Theorem 3.4}Then the Weierstrass $\wp$-function of $\Omega$ satisfies the nonlinear differential equation
    \begin{equation}\label{eqn: Weierstrass differential equation}
        \wp'(z)^{2}=4\wp(z)^{3}-g_{2}(\Omega)\wp(z)-g_{3}(\Omega)
    \end{equation}
    where $g_{2}(\Omega)=60\sum_{\omega\in\Omega\setminus\{0\}}\frac{1}{\omega^{4}}$ and $g_{3}(\Omega)=140\sum_{\omega\in\Omega\setminus\{0\}}\frac{1}{\omega^{6}}$.
\end{theorem}
\begin{proof}
    We take Laurent expansions of $\wp(z),\wp'(z)^{2},\wp(z)^{3}$ to observe 
    \begin{align*}
        \wp(z) &= \frac{1}{z^{2}}+c_{2}z^{2}+c_{4}z^{4}+\dots \\
        \wp'(z)^{2} &= \frac{4}{z^{6}}-8c_{2}z+4c_{4}z^{3}+\dots \\
        \wp(z)^{3} &= \frac{1}{z^{6}} +3c_{2}\frac{1}{z^{2}} + 3c_{4}+\dots
    \end{align*}
    where we compute 
    $$\wp'(z)-4\wp(z)^{3}=-20c_{2}\frac{1}{z^{2}}-28c_{4}+\dots$$
    and iterating the process once more to use $\wp(z)$ write the pole of order 2 that 
    $$\wp'(z)^{2}-4\wp(z)^{3}+20c_{2}\wp(z)=28c_{4}+\dots$$
    where we now note that the function on the right is a holomorphic elliptic function and hence $28c_{4}$ itself. Rearranging, we yield $\wp'(z)^{2}=4\wp(z)^{3}-20c_{2}\wp(z)-28c_{4}$. To find $c_{2},c_{4}$ in terms of the lattice, we write $h(z)=\wp(z)-\frac{1}{z^{2}}=\sum_{\omega\in\Omega\setminus\{0\}}\left(\frac{1}{(z-\omega^{2})}-\frac{1}{\omega^{2}}\right)$ so taking the derivative we get $h^{(m)}(z)=(-1)^{m}(m-1)!\sum_{\omega\in\Omega\setminus\{0\}}\frac{1}{(z-\omega)^{m+2}}$. Now noting that $(2m)!c_{2m}=h^{(2m)}(0)$ we conclude that $c_{2m}=(2m+1)\sum_{\omega\in\Omega\setminus\{0\}}\frac{1}{\omega^{2m+2}}$ giving the claim. 
\end{proof}
We deduce the following as an immediate corollary. 
\begin{corollary}
    Let $\Omega$ be a lattice with Weierstrass function $\wp(z)$. Then 
    \begin{enumerate}[label=(\roman*)]
        \item $2\wp'(z)\wp''(z)=12\wp(z)^{2}\wp'(z)- g_{2}(\Omega)$. 
        \item $\wp''(z)=12\wp(z)\wp'(z)$. 
    \end{enumerate}
\end{corollary}
\begin{proof}
    This follows from a direct computation of the derivative of the function (\ref{eqn: Weierstrass differential equation}) in \Cref{thm: Weierstrass differential equation}.
\end{proof}
The field of elliptic functions with respect to a lattice $\Omega$ can be explicitly described as follows. 
\begin{theorem}\label{thm: description of field of elliptic functions}
    Let $\Omega$ be a lattice. Then the field of elliptic functions $K(\Omega)$ with respect to $\Omega$ is isomorphic to 
    $$\CC(X)[Y]/(Y^{2}-4X^{3}+g_{2}(\Omega)X+g_{3}(\Omega)).$$
\end{theorem}
\begin{proof}
    By \Cref{prop: elliptic functions in terms of Weierstrass}, the map $\CC(X)[Y]$ by $X\mapsto\wp(z), Y\mapsto\wp'(z)$ is surjective with kernel given by the relation imposed by the differential equation of \Cref{thm: Weierstrass differential equation}. 
\end{proof}
In particular, this algebraically realizes the field of elliptic functions as a degree 2 extension of the field of rational functions on $\CC$. 

By \Cref{prop: residue sum is zero}, we know that elliptic functions satisfy strong constraints on their zeroes and poles. However, given a set of zeroes and poles alongside their multiplicities, it is unclear if the inverse problem can be solved -- constructing an elliptic function with those zeroes and poles. This, however, turns out to be possible, and will necessitate the development of the theory of elliptic modular functions. 

We begin with the following lemma, which produces an example of such a function. 
\begin{lemma}\label{lem: Weierstrass zeta function}
    Let $\Omega$ be a lattice. The function 
    $$f(z)=\frac{1}{z}+\sum_{\omega\in\Omega\setminus\{0\}}\left(\frac{1}{z-\omega}+\frac{1}{\omega}+\frac{z}{\omega^{2}}\right)$$
    is convergent with derivative $-\wp(z)$. 
\end{lemma}
\begin{proof}
    This function arises as the integral $\int_{0}^{z}\wp(w)dw$ and can be seen to be convergent using the termwise estimate 
    $$\left|\frac{1}{z-\omega}+\frac{1}{\omega}+\frac{z}{\omega^{2}}\right|\leq\frac{|z|^{2}}{|\omega^{3}|\left(1-\frac{|z|}{|\omega|}\right)}\leq\frac{2|z|^{2}}{|\omega|^{3}}$$
    with the rightmost term convergent by \Cref{prop: absolute convergence of lattice sum}. 
\end{proof}
This is the Weierstrass $\zeta$ function. 
\begin{definition}[Weierstrass $\zeta$-Function]\label{def: Weierstrass zeta function}
    Let $\Omega$ be a lattice.\marginpar{Definition 4.1} The Weierstrass $\zeta$-function of $\Omega$ is given by 
    $$\zeta(z)=\frac{1}{z}+\sum_{\omega\in\Omega\setminus\{0\}}\left(\frac{1}{z-\omega}+\frac{1}{\omega}+\frac{z}{\omega^{2}}\right).$$
\end{definition}
Continuing along this line, we consider the the primitive of $\log\zeta(z)$. 
\begin{lemma}\label{lem: Weierstrass sigma function}
    Let $\Omega$ be a lattice. The function 
    $$f(z)=z\prod_{\omega\in\Omega\setminus\{0\}}\left(1-\frac{z}{\omega}\right)\exp\left(\frac{z}{\omega}+\frac{z^{2}}{2\omega^{2}}\right)$$
    has derivative $\log\zeta(z)$. 
\end{lemma}
\begin{proof}
    The logarithm of the product is given by $\sum_{\omega\in\Omega\setminus\{0\}}\left(\log\left(1-\frac{z}{\omega}\right)+\frac{z}{\omega}+\frac{z^{2}}{\omega^{2}}\right).$ Note that this is equal to $\int_{0}^{z}(\zeta(w)-\frac{1}{w})dw=\log\left(\frac{f(z)}{z}\right)$ and thus $\zeta(z)-\frac{1}{z}=\frac{\sigma'(z)}{\sigma(z)}-\frac{1}{z}$ as desired. 
\end{proof}
This is the Weierstrass $\sigma$-function. 
\begin{definition}[Weierstrass $\sigma$-Function]\label{def: Weierstrass sigma function}
    Let $\Omega$ be a lattice.\marginpar{Definition 4.1} The Weierstrass $\sigma$-function of $\Omega$ is given by 
    $$\sigma(z)=z\prod_{\omega\in\Omega\setminus\{0\}}\left(1-\frac{z}{\omega}\right)\exp\left(\frac{z}{\omega}+\frac{z^{2}}{2\omega^{2}}\right).$$
\end{definition}
The following property of the $\zeta$ and $\sigma$ functions will be used to construct an elliptic function given a compatible set of zeroes and poles with multiplicities. 
\begin{proposition}\label{prop: properties of zeta and sigma functions}
    Let $\Omega$ be a lattice and $\zeta(z),\sigma(z)$ the Weierstrass $\zeta$ and $\sigma$ functions with respect to $\Omega$. Then:
    \begin{enumerate}[label=(\roman*)]
        \item $\zeta(z+\omega_{i})=\zeta(z)+\eta_{i}$ for $\omega_{i}$ a generating element of the lattice and $\eta_{i}$ some constant depending on $\omega_{i}$. 
        \item $\sigma(z+\omega_{i})=\sigma(z)\exp(\eta_{i}z+c_{i})$ for $\omega_{i}$ a generating element of the lattice, $\eta_{i}$ the constant depending on $\omega_{i}$ identified above, and $c_{i}$ some other constant depending on $\omega_{i}$.
    \end{enumerate}
\end{proposition}
\begin{proof}[Proof of (i)]
    We know the derivative of $\zeta(z+\omega_{i})-\zeta(z)$ is given by $-\wp(z+\omega_{i})+\wp(z)=0$ so integrating, we have $\zeta(z+\omega_{i})=\zeta(z)+\eta_{i}$ for some constant $\omega_{i}$. 
\end{proof}
\begin{proof}[Proof of (ii)]
    From $\frac{\sigma'(z)}{\sigma(z)}=\zeta(z)$ we have 
    $$\frac{\sigma'(z+\omega_{i})}{\sigma(z+\omega)}=\zeta(z+\omega_{i})=\zeta(z)+\eta_{i}=\frac{\sigma'(z)}{\sigma(z)}+\eta_{i}$$
    so integrating, we get $\sigma(z+\omega_{i})=\sigma(z)\exp(\eta_{i}z+c_{i})$, as desired. 
\end{proof}
We can now prove the desired result. 
\begin{theorem}\label{thm: construction of elliptic function with given zeroes and poles}
    Let $\Omega$ be a lattice, $a_{1},\dots,a_{n},a_{n+1},\dots,a_{\ell}$ points in the interior of the fundamental period parallelogram $P_{\Omega}$ labeled by natural numbers $m_{1},\dots,m_{n},m_{n+1},\dots,m_{\ell}$ such that 
    $$\left(\sum_{i=1}^{n}m_{i}a_{i}\right)-\left(\sum_{i=n+1}^{\ell}m_{i}a_{i}\right)=0.$$
    Then there exists an elliptic function $f(z)$ with respect to $\Omega$ with zeroes at $a_{1},\dots,a_{n}$ and poles at $a_{n+1},\dots,a_{\ell}$ of multiplicities $m_{1},\dots,m_{n}$ and $m_{n+1},\dots,m_{\ell}$, respectively. 
\end{theorem}
\begin{proof}
    The function 
    $$f(z)=\frac{\prod_{i=1}^{n}\sigma(z-a_{i})^{m_{i}}}{\prod_{i=n+1}^{\ell}\sigma(z-a_{i})^{m_{i}}}$$
    suffifces. By construction of $\sigma(z)$, this function is meromorphic with the desired zeroes and poles. To see that it is periodic, we note 
    $$f(z+\omega_{j})=f(z)\exp(A_{j})$$
    where
    $$A_{i}=\sum_{i=1}^{n}m_{i}(\eta_{j}(z-a_{i})+c_{j})-\sum_{i=n+1}^{\ell}m_{i}(\eta_{j}(z-a_{i})+c_{j})$$
    which is zero modulo $\Omega$ by \Cref{prop: properties of zeta and sigma functions} (ii). 
\end{proof}
We conclude with the following remark.
\begin{remark}
    A lattice can be normalized to have $\Omega=\langle1,\tau\rangle$ with $\tau$ having positive imaginary part. Can define another function $\Theta(z,\tau)=\sum_{n\in\ZZ}\exp(\pi i\tau n^{2}+2\pi i n z + \pi i n)$ which is known as the $\Theta$-function of the lattice $\Omega$. Further exposition of this function can be found in a standard graduate text on complex analysis, but we will not discuss these functions in the course. 
\end{remark}