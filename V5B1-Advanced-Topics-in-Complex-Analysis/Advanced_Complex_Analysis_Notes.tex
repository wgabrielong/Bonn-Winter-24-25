\documentclass{amsart}
\usepackage[margin=1.5in]{geometry} 
\usepackage{amsmath}
\usepackage{tcolorbox}
\usepackage{amssymb}
\usepackage{amsthm}
\usepackage{lastpage}
\usepackage{fancyhdr}
\usepackage{accents}
\usepackage{hyperref}
\usepackage{xcolor}
\usepackage{color}
% Fields
\newcommand{\CC}{\mathbb{C}}
\newcommand{\RR}{\mathbb{R}}
\newcommand{\QQ}{\mathbb{Q}}
\newcommand{\ZZ}{\mathbb{Z}}
\newcommand{\HH}{\mathbb{H}}
\newcommand{\KK}{\mathbb{K}}
\newcommand{\NN}{\mathbb{N}}
\newcommand{\FF}{\mathbb{F}}
\newcommand{\PP}{\mathbb{P}}

% mathcal letters
\newcommand{\Acal}{\mathcal{A}}
\newcommand{\Bcal}{\mathcal{B}}
\newcommand{\Ccal}{\mathcal{C}}
\newcommand{\Dcal}{\mathcal{D}}
\newcommand{\Ecal}{\mathcal{E}}
\newcommand{\Fcal}{\mathcal{F}}
\newcommand{\Gcal}{\mathcal{G}}
\newcommand{\Hcal}{\mathcal{H}}
\newcommand{\Ical}{\mathcal{I}}
\newcommand{\Jcal}{\mathcal{J}}
\newcommand{\Kcal}{\mathcal{K}}
\newcommand{\Lcal}{\mathcal{L}}
\newcommand{\Mcal}{\mathcal{M}}
\newcommand{\Ncal}{\mathcal{N}}
\newcommand{\Ocal}{\mathcal{O}}
\newcommand{\Pcal}{\mathcal{P}}
\newcommand{\Qcal}{\mathcal{Q}}
\newcommand{\Rcal}{\mathcal{R}}
\newcommand{\Scal}{\mathcal{S}}
\newcommand{\Tcal}{\mathcal{T}}
\newcommand{\Ucal}{\mathcal{U}}
\newcommand{\Vcal}{\mathcal{V}}
\newcommand{\Wcal}{\mathcal{W}}
\newcommand{\Xcal}{\mathcal{X}}
\newcommand{\Ycal}{\mathcal{Y}}
\newcommand{\Zcal}{\mathcal{Z}}

% abstract categories
\newcommand{\Asf}{\mathsf{A}}
\newcommand{\Bsf}{\mathsf{B}}
\newcommand{\Csf}{\mathsf{C}}
\newcommand{\Dsf}{\mathsf{D}}
\newcommand{\Esf}{\mathsf{E}}
\newcommand{\Ssf}{\mathsf{S}}

% algebraic geometry
\newcommand{\spec}{\operatorname{Spec}}
\newcommand{\proj}{\operatorname{Proj}}

% categories 
\newcommand{\id}{\mathrm{id}}
\newcommand{\Obj}{\mathrm{Obj}}
\newcommand{\Mor}{\mathrm{Mor}}
\newcommand{\Hom}{\mathrm{Hom}}
\newcommand{\Aut}{\mathrm{Aut}}
\newcommand{\Sets}{\mathsf{Sets}}
\newcommand{\SSets}{\mathsf{SSets}}
\newcommand{\kVect}{\mathsf{Vect}_{k}}
\newcommand{\Vect}{\mathsf{Vect}}
\newcommand{\Alg}{\mathsf{Alg}}
\newcommand{\Ring}{\mathsf{Ring}}
\newcommand{\Mod}{\mathsf{Mod}}
\newcommand{\Grp}{\mathsf{Grp}}
\newcommand{\AbGrp}{\mathsf{AbGrp}}
\newcommand{\PSh}{\mathsf{PSh}}
\newcommand{\Sh}{\mathsf{Sh}}
\newcommand{\PSch}{\mathsf{PSch}}
\newcommand{\Sch}{\mathsf{Sch}}
\newcommand{\Top}{\mathsf{Top}}
\newcommand{\Com}{\mathsf{Com}}
\newcommand{\Coh}{\mathsf{Coh}}
\newcommand{\QCoh}{\mathsf{QCoh}}
\newcommand{\Opens}{\mathsf{Opens}}
\newcommand{\Opp}{\mathsf{Opp}}
\newcommand{\Cat}{\mathsf{Cat}}
\newcommand{\NatTrans}{\mathrm{NatTrans}}
\newcommand{\pr}{\mathrm{pr}}
\newcommand{\Fun}{\mathrm{Fun}}
\newcommand{\colim}{\mathrm{colim}}
\newcommand{\lifts}{\boxslash}
\DeclareMathOperator\squarediv{\lifts}
\newcommand{\Kan}{\mathsf{Kan}}
\newcommand{\Path}{\mathsf{Path}}
\newcommand{\SPSh}{\mathsf{SPSh}}
\newcommand{\SSh}{\mathsf{SSh}}
\newcommand{\Bord}{\mathsf{Bord}}

% simplicial sets
\newcommand{\DDelta}{\Updelta}
\newcommand{\Sing}{\operatorname{Sing}}

% ideal theory
\newcommand{\mfrak}{\mathfrak{m}}
\newcommand{\afrak}{\mathfrak{a}}
\newcommand{\bfrak}{\mathfrak{b}}
\newcommand{\pfrak}{\mathfrak{p}}
\newcommand{\qfrak}{\mathfrak{q}}

% number theory
\newcommand{\Tr}{\mathrm{Tr}}
\newcommand{\Nm}{\mathrm{Nm}}
\newcommand{\Gal}{\mathrm{Gal}}
\newcommand{\Frob}{\mathrm{Frob}}

\newcommand{\SL}{\mathrm{SL}}
\newcommand{\Li}{\mathrm{Li}}
\setlength{\headheight}{40pt}


\newenvironment{solution}
  {\renewcommand\qedsymbol{$\blacksquare$}
  \begin{proof}[Solution]}
  {\end{proof}}
\renewcommand\qedsymbol{$\blacksquare$}

\usepackage{amsmath, amssymb, tikz, amsthm, csquotes, multicol, footnote, tablefootnote, biblatex, wrapfig, float, quiver, mathrsfs, cleveref, enumitem, upgreek, stmaryrd, marginnote, todonotes}
\addbibresource{refs.bib}
\theoremstyle{definition}
\newtheorem{theorem}{Theorem}[section]
\newtheorem{lemma}[theorem]{Lemma}
\newtheorem{corollary}[theorem]{Corollary}
\newtheorem{exercise}[theorem]{Exercise}
\newtheorem{question}[theorem]{Question}
\newtheorem{example}[theorem]{Example}
\newtheorem{proposition}[theorem]{Proposition}
\newtheorem{conjecture}[theorem]{Conjecture}
\newtheorem{remark}[theorem]{Remark}
\newtheorem{definition}[theorem]{Definition}
\numberwithin{equation}{section}
\setuptodonotes{color=blue!20, size=tiny}
\begin{document}
\large
\title[Advanced Topics in Complex Analysis -- Bonn, Winter 2024/25]{V5B1 -- Advanced Topics in Complex Analysis \\ Winter Semester 2024/25}
\author{Wern Juin Gabriel Ong}
\address{Universit\"{a}t Bonn, Bonn, D-53113}
\email{wgabrielong@uni-bonn.de}
\urladdr{https://wgabrielong.github.io/}
\maketitle
\section*{Preliminaries}
These notes roughly correspond to the course \textbf{V5B1 - Advanced Topics in Complex Analysis} taught by Prof. Ingo Lieb at the Universit\"{a}t Bonn in the Winter 2024/25 semester. These notes are \LaTeX-ed after the fact with significant alteration and are subject to misinterpretation and mistranscription. Use with caution. Any errors are undoubtedly my own and any virtues that could be ascribed to these notes ought be attributed to the instructor and not the typist. 
\tableofcontents
\section{Lecture 1 -- 8th October 2024}\label{sec: lecture 1}
We first set the following conventions to be used throughout these notes:
\begin{itemize}
    \item Let $X$ be a topological space. A neighborhood of a point $p\in X$ is an open set $U\subseteq X$ containing $p$. 
    \item For $p=(p_{1},\dots,p_{n})\in\RR^{n}$ and $r\in\RR_{\geq0}$, $B_{r}(p)=\{x\in\RR^{n}:|x-p|^{2}<r\}$ is the open ball of radius $r$ centered at $p$. 
\end{itemize}
We begin with a review of point set topology. 

Recall the definition of locally Euclidean spaces. 
\begin{definition}[Locally Euclidean Space]\label{def: locally Euclidean space}
    Let $X$ be a topological space. $X$ is locally euclidean if each point $x\in X$ has a neighborhood homeomorphic to $\RR^{n}$ for some fixed $n$. 
\end{definition}
\begin{remark}
    Note that the definition above does not permit topological spaces with points $x,y\in X$ such that $x$ admits a neighborhood homeomorphic to $\RR^{n}$ and $y$ admits a neighborhood homeomorphic to $\RR^{m}$ for $m\neq n$. 
\end{remark}
On a locally Euclidean topological space, we can take the neighborhoods homeomorphic to $\RR^{n}$ and consider open subsets of such neighborhoods which also possess a map to $\RR^{n}$. 
\begin{definition}[Chart]\label{def: chart}
    Let $X$ be a locally Euclidean topological space. A chart $(U,\phi)$ consists of an open set $U\subseteq X$ and a continuous map $\phi:U\to\RR^{n}$ that is a homeomorphism onto its image. 
\end{definition}
Given a point $x\in X$ and a neighborhood, we can consider charts with a prescribed image $\phi(x)\in\RR^{n}$. An especially nice case is when $\phi(x)=0\in\RR^{n}$.
\begin{definition}[Centered Chart]\label{def: centered chart}
    Let $X$ be a locally Euclidean topological space. A chart $(U,\phi)$ is centered at $x\in U$ if $\phi(x)=0\in\RR^{n}$. 
\end{definition}
In fact, one can show that locally Euclidean topological spaces have charts centered at $x$ for all points $x\in X$. 
\begin{proposition}\label{prop: locally euclidean and centered charts}
    Let $X$ be a topological space. The following are equivalent:
    \begin{enumerate}[label=(\alph*)]
        \item $X$ is locally Euclidean. 
        \item For any point $x\in X$, there is a chart centered at $x$ with image the unit ball of $\RR^{n}$. 
        \item For any point $x\in X$, there is a chart centered at $x$ with image $\RR^{n}$. 
    \end{enumerate}
\end{proposition}
\begin{proof}
    (b)$\Longleftrightarrow$(c) by composing appropriately with the homeomorphism $B_{1}(0)\to\RR^{n}$ by fixing the origin and the map on the complement defined by $x\mapsto\frac{1}{1-\Vert x\Vert}$. Furthermore (c)$\Rightarrow$(a) since (c) is a homeomorphisms of a neighborhood to $\RR^{n}$ are in particular continuous maps to $\RR^{n}$ homeomorphic onto its image. 
    
    It remains to show (a)$\Rightarrow$(b). Consider a chart $(U,\phi)$. For $x\in U$, we can consider the map $U\to\RR^{n}$ by $y\mapsto y-\phi(x)$ yielding a chart centered at $x$. By scaling this map by some $\lambda\in\RR_{>0}$ we can consider a map $\widetilde{\phi}$ by $y\mapsto \lambda y-\lambda\phi(x)$ with image containing $B_{1}(0)$. Restriction to the preimage of $B_{1}(0)$ under $\widetilde{\phi}$ yields a chart centered at $x$ with image the unit ball $(U|_{\widetilde{\phi}^{-1}(B_{1}(0))}, \widetilde{\phi})$. 
\end{proof}
We now introduce the notion of Hausdorff spaces, which include the spaces of concern in this course, as well as a large proportion of spaces one will encounter over the course of one's mathematical life. 
\begin{definition}[Hausdorff]\label{def: Hausdorff}
    Let $X$ be a topological space. $X$ is Hausdorff if for any two distinct points $x,x'\in X$ there exist open neighborhoods $U,U'$ of $x,x'$, respectively, such that $U\cap U'=\emptyset$. 
\end{definition}
\begin{example}
    Euclidean space $\RR^{n}$ is Hausdorff. 
\end{example}
\begin{example}
    CW complexes are Hausdorff. 
\end{example}
\begin{example}\label{ex: R by units is not Hausdorff}
    Let $X$ be the topological space given by the set $\{0,1\}$ and open sets $\emptyset, \{0\}, \{0,1\}$. This space is not Hausdorff since the points 0 and 1 cannot be separated by open sets. This space is in fact the quotient space $\RR/\RR^{\times}$ with $\RR^{\times}$ acting on $\RR$ by multiplication. 
\end{example}
\begin{remark}
    As suggested by \Cref{ex: R by units is not Hausdorff}, quotient spaces are the prototypical example of non-Hausdorff spaces. 
\end{remark}
We can show the following properties of Hausdorff spaces. 
\begin{proposition}\label{prop: properties of Hausdorff spaces}
    Let $X$ be a Hausdorff topological space. Then:
    \begin{enumerate}[label=(\roman*)]
        \item Compact sequences have unique limits. 
        \item Compact subsets are closed. 
        \item One-point subsets are closed. 
    \end{enumerate}
\end{proposition}
\begin{proof}[Proof of (a)]
    Suppose to the contrary that there is a sequence $\{x_{i}\}_{i=1}^{\infty}$ with limit points $x,x'$ distinct. Since $X$ is Hausdorff, we can take open neighborhoods $U,U'$ of $x,x'$, respectively, such that $U\cap U'=\emptyset$. However we can take $N$ large we have both $x_{i}\in U$ and $x_{i}\in V$, a contradiction as $U,U'$ are disjoint. 
\end{proof}
\begin{proof}[Proof of (b)]
    Let $K\subseteq X$ be compact. We want to show that its complement $X\setminus K$ is open. Let $x\in X\setminus K$. Since $X$ is Hausdorff, we can consider a neighborhood $V_{y}$ for each $y\in K$ disjoint from (possibly varying) neighborhoods $U_{y}$ of $x$. Since $K$ is compact, $K$ is covered by finitely many $V_{y}$'s say $V_{y_{1}},\dots,V_{y_{n}}$ and set $U=\bigcap_{i=1}^{n}U_{y_{i}}$. Note that each $U_{y_{i}}$ is an open set of $X$ containing $x$ in the complement of $V_{y_{i}}$ in $X$ and as such their intersection contains $x$ and is in the complement of $K$. As such, any $x\in X\setminus K$ admits an open neighborhood disjoint from $K$ showing $K$ is closed.
\end{proof}
\begin{proof}[Proof of (c)]
    This is immediate from (b), for one-point sets are compact. 
\end{proof}
We now discuss bases and covers of topological spaces. 
\begin{definition}[Basis for a Topological Space]\label{def: basis of topological space}
    Let $X$ be a topological space. A collection $\Bcal$ of arbitrary subsets of $X$ is a basis of $X$ if for any $p\in X$ and any neighborhood $U$ of $p$ there exists an element of $B$ containing $p$ and contained in $U$. 
\end{definition}
It can be shown that any open set of a topological space can be written as a union of basis sets. 
\begin{proposition}\label{lem: basis iff every open is a union of elements}
    Let $X$ be a topological space and $\Bcal$ an arbitrary collection of subsets of $X$. $\Bcal$ is a basis of $X$ if and only if every open set of $X$ can be written as a union of sets of $\Bcal$. 
\end{proposition}
\begin{proof}
    $(\Rightarrow)$ Suppose $\Bcal$ is a basis of $X$ and let $U\subseteq X$ be open. For $x\in U$ consider $V_{x}\in\Bcal$ containing $x$ but contained in $U$ where we have $U=\bigcup_{x\in U}V_{x}$, writing $U$ as a union of basis sets. 

    $(\Leftarrow)$ Suppose for each open $U\subseteq X$ we can write $U=\bigcup_{i\in I}V_{i}$. As such, for each point $x\in U$ there is some $V_{i}$ contained in $U$ containing $X$ thus forming a basis. 
\end{proof}
We want to focus our attention on topological spaces that are appropriately ``small'' by imposing size conditions on the basis. 
\begin{definition}[Second Countable Space]\label{def: second countable space}
    Let $X$ be a topological space. $X$ is a second countable space if $X$ admits a countable basis $\Bcal$.
\end{definition}
The countability property is preserved under the following conditions. 
\begin{proposition}\label{prop: second countability preserved}
    Let $X$ be a topological space. Then:
    \begin{enumerate}[label=(\roman*)]
        \item If $X$ is second countable, then any subspace of $X$ with the subspace topology is second countable. 
        \item If $\{U_{i}\}_{i\in I}$ is a countable open cover of $X$ with each each $U_{i}$ second countable then $X$ is countable. 
        \item If $X$ is locally Euclidean and $\{K_{i}\}_{i=1}^{\infty}$ is a sequence of compact subsets such that $X=\bigcup_{i=1}^{\infty}K_{i}$ then $X$ is second countable. 
    \end{enumerate}
\end{proposition}
\begin{remark}
    The property of being second countable is not preserved under arbitrary quotients, though this holds when the quotient map is open. 
\end{remark}
We can describe the second countability property in terms of covers. 
\begin{proposition}\label{prop: second countability via covers}
    Let $X$ be a topological space. If $X$ is second countable then any open cover of $X$ admits a countable subcover. 
\end{proposition}
\begin{proof}
    Let $\Bcal$ be a countable basis for $X$ and $\{U_{i}\}_{i\in I}$ an open cover of $X$. Consider $\widetilde{\Bcal}$ consisting of those basis elements of $X$ contained in some $U_{i}$. Note that $\widetilde{\Bcal}$ is a cover of $X$ since for any point $x\in U_{i}$ there is an element of $\Bcal$ containing $x$ contained in $U_{i}$. For each $V\in\widetilde{\Bcal}$ of which there are countably many, consider $U_{V}\in\{U_{i}\}_{i\in I}$ such that $V\subseteq U_{V}$. These form a cover of $X$ indexed by a countable set $\widetilde{\Bcal}$ giving the claim. 
\end{proof}
We also introduce the following notion of compact exhaustability. 
\begin{definition}[Compact Exhaustability]\label{def: compact exhaustability}
    Let $X$ be a topological space. $X$ is compactly exhaustible if there exists a sequence of compact subsets $\{K_{i}\}_{i=1}^{\infty}$ of $X$ such that $K_{i}\subseteq K_{i+1}^{\circ}$ and $X=\bigcup_{i=1}^{\infty}K_{i}$.
\end{definition}
The condition of compact exhaustability is satisfied under relatively mild hypotheses. 
\begin{proposition}\label{prop: locally euclidean, Hausdorff, second countable implies compactly exhaustible}
    Let $X$ be a topological space. If $X$ is locally Euclidean, Hausdorff, and second countable, $X$ admits an exhaustion by compact subsets. 
\end{proposition}
\begin{proof}
    We first note that since $X$ is locally Euclidean, it admits a basis $\Bcal$ of open subsets having compact closure: for each chart $(U,\phi)$ we can take some $x\in U$ and set the image of the chart to be centered at $x$ homeomorphic onto the open unit ball by \Cref{prop: locally euclidean and centered charts} and produce a countable basis of the ball by smaller balls wich have compact closure. By taking preimages, we can consider the countable union of countable balls with compact closures inducing the respective property for each open of $X$. 
 
    Furthermore, since $X$ is second countable, it is covered -- up to a choice of bijection of the countable indexing set with the natural numbers -- by countably many sets $\{U_{i}\}_{i=1}^{\infty}$ with compact closure. Suppose $K_{1}=\overline{U_{1}}$. We proceed by induction and suppose that there are compact sets $K_{1},\dots,K_{m}$ such that $U_{i}\subseteq K_{i}$ for each $i$ and $K_{i}\subseteq K_{i+1}^{\circ}$ for $2\leq i\leq m-1$. Since $K_{m}$ is compact, there is some $N_{m}\geq m+1$ large such that $K_{m}\subseteq U_{1}\cup\dots\cup U_{N_{m}}$. If $K_{m+1}=\overline{U_{1}}\cup\dots\cup\overline{U_{N_{m}}}$ then $K_{m+1}$ is closed and thus compact with interior containing $K_{m}$ giving the claim. 
\end{proof}
\section{Lecture 2 -- 18th October 2024}\label{sec: lecture 2}
Let us revisit the $q$-Pochhammer function as an example of Nahm sums, and in particular to consider the asymptotic behavior of these rings as $q$ approaches roots of unity. Indeed, the study of such phenomena is precisely the study of the Habiro ring. 

We consider the theory of $q$-calculus and some of its more modern incarnations. The following table summarizes the analogy.

\begin{table}[h]\label{table: q-calculus comparison}
    \begin{tabular}{c c c}
        \textbf{Classical} & \textbf{$q$-deformed} & \textbf{$q$-deformed (adapted)}\\
        $\nabla:\ZZ[t]\to\ZZ[t]$ & $\nabla_{q}:\ZZ[q][t]\to\ZZ[q][t]$ & $\nabla_{q}':\ZZ[q][t]\to\ZZ[q][t]$ \\
        $f(t)\mapsto\lim_{h\to 0}\frac{f(t)-f(t+h)}{h}$ & $f(t)\mapsto\frac{f(t)-f(qt)}{t-qt}$ & $f(t)\mapsto\frac{f(t)-f(qt)}{t}$\\
        $t^{n}\mapsto nt^{n-1}$ & $t^{n}\mapsto \frac{1-q^{n}}{1-q}\cdot t^{n-1}=[n]_{q}t^{n-1}$ & $t^{n}\mapsto (1-q^{n})t^{n-1}$\\
    \end{tabular}
    \caption{Comparison between classical calculus and two variants of $q$-calculus.}
\end{table}
The $q$-deformed construction recovers the classical case as $q\to1$, but the adapted $q$-deformed variant often works better since the lack of $(1-q)$-factors ``treats all roots of unity the same,'' unlike in the classical $q$-deformed variant which ``singles out the first root of unity.''

Note that $\nabla$ is coordinate independent as a local operator, but $\nabla_{q}$ is not, and multiplication by $q$ is information that needs to be remembered. However, despite these issues, $q$-de Rahm cohomology groups turn out to be coordinate independent after $(q-1)$-adic completion, as implied by the theory of prismatic cohomology as developed in \cite{PrismsPrismatic}. In the $\nabla_{q}'$-variant, however, this coordinate independence does not hold as shown by Wagner in \cite{WagnerMSThesis} (vis. \cite{WagnerQWittQHodge}), but recent work of Meyer-Wagner shows the theory does still hold at some level of generality \cite{MeyerWagner}.

Let us now consider $q$-integration, to the end of considering solutions to $q$-difference equations. Classically, $\nabla f(t)=f(t)$ with initial value $f(0)=1$ yields the power series $f(t)=\sum_{n\geq0}\frac{t^{n}}{n!}=\exp(t)$. In the $q$-deformed setting, we have the following. 
\begin{proposition}\label{prop: q-deformed exponential}
    The $q$-difference equation with $\nabla_{q}f(t)=f(t)$ with initial value $f(0)=1$ has solution $f(t)=\sum_{n\geq0}\frac{t^{n}}{[n]_{q}!}$ where $[n]_{q}!=\frac{(q;q)_{n}}{(1-q)^{n}}$.
\end{proposition}
\begin{proof}
    This can be verified by a direct computation:
    \begin{align*}
        \nabla_{q}f(t) &= \sum_{n\geq0}\frac{[n]_{q}t^{n-1}}{[n]_{q}!} \\
        &=\sum_{n\geq0}\frac{t^{n-1}}{[n-1]_{q}!} = f(t)
    \end{align*}
    which satisfies the initial value condition by inspection. 
\end{proof}
Similarly in the case of adapted $q$-deformations, we have the following. 
\begin{proposition}\label{prop: adapted q-deformed exponential}
    The $q$-difference equation with $\nabla_{q}'f(t)=f(t)$ with initial value $f(0)=1$ has solution $f(t)=\sum_{n\geq0}\frac{t^{n}}{(q;q)_{n}}$. 
\end{proposition}
\begin{proof}
    Computing as above:
    \begin{align*}
        \nabla_{q}'f(t) &= \sum_{n\geq0}\frac{(1-q^{n})t^{n-1}}{(q;q)_{n}} \\
        &=\sum_{n\geq0}\frac{t^{n-1}}{(q;q)_{n-1}}
    \end{align*}
    which once again, by observation, satisfies the initial value condition. 
\end{proof}
\begin{remark}
    We will primarily focus on the adapted variant $\nabla_{q}'$. 
\end{remark}
\begin{remark}
    The above are examples of Nahm sums of \Cref{def: Nahm sum} for the case $N=1$ and $A=0$. 
\end{remark}
In the adapted variant, we can alternatively describe $f(t)$ as follows. 
\begin{proposition}\label{prop: Pochhammer as exponential}
    The $q$-difference equation with $\nabla_{q}'f(t)=f(t)$ with initial value $f(0)=1$ has solution $f(t)=(t;q)_{\infty}^{-1}$.
\end{proposition}
\begin{proof}
    Given $f(t)=\frac{f(t)-f(qt)}{t}$ we have that $(1-t)f(t)=f(qt)$ and thus 
    \begin{equation}\label{eqn: q Pochhammer expansion of exponential}
        f(t) = (1-t)f(qt)
    \end{equation} Applying the same manipulation to $f(qt)=\frac{f(qt)-f(q^{2}t)}{qt}$ we have $f(qt)=(1-qt)f(q^{2}t)$ which by induction and substituting into (\ref{eqn: q Pochhammer expansion of exponential}) we get $f(t)=(t;q)_{\infty}^{-1}$, yielding the claim. 
\end{proof}
As an immediate corollary, we deduce the following. 
\begin{corollary}
    There is an equality
    $$(t;q)_{\infty}^{-1}=\sum_{n\geq0}\frac{t^{n}}{(q;q)_{n}}$$
    in $\ZZ[q^{\pm},\frac{1}{1-q},\frac{1}{1-q^{2}}, \dots][[t]]$.
\end{corollary}
\begin{proof}
    This is immediate from \Cref{prop: adapted q-deformed exponential,prop: Pochhammer as exponential}.
\end{proof}
We are interested in two phenomena:
\begin{itemize}
    \item the asymptotics as $q\to 1$ recovering the classical theory, and 
    \item the asymptotics at roots of unity. 
\end{itemize}
One immediately observes that these functions have poles at roots of unity, but their logarithms converge as power series in $t$ with coefficients in $\QQ(q)$. We show the logarithm of $(t;q)_{\infty}^{-1}$ has at worst simple poles at roots of unity. 
\begin{proposition}\label{prop: logarithm at worst simple poles at roots of unity}
    There is an equality 
    \begin{equation}\label{eqn: expresssion of logarithm of q exponential}
        \log(t;q)_{\infty}^{-1}=\sum_{\ell\geq1}\frac{1}{\ell(1-q^{\ell})}\cdot t^{\ell}
    \end{equation}
    in $\QQ(q)[[t]]$. As such, $\log(t;q)_{\infty}^{-1}$ has at worst simple poles at all roots of unity. 
\end{proposition}
\begin{proof}
    We compute 
    \begin{align*}
        \log(t;q)_{\infty}^{-1} &= \sum_{n\geq0}\log(1-q^{n}t)^{-1} \\
        &= \sum_{n\geq0}\sum_{\ell\geq1}\frac{q^{n\ell}t^{\ell}}{\ell}&& \log(1-x)^{-1}=\sum_{\ell\geq1}\frac{x^{\ell}}{\ell} \\
        &= \sum_{\ell\geq1}\left(\sum_{n\geq0}q^{n\ell}\right)\frac{t^{\ell}}{\ell} \\
        &= \sum_{\ell\geq1}\left(\frac{1}{1-q^{\ell}}\right)\frac{t^{\ell}}{\ell} && \text{sum of geom. series} \\
        &= \sum_{\ell\geq1}\frac{1}{\ell(1-q^{\ell})}t^{\ell}
    \end{align*}
    giving the first claim. 

    For the second claim, observe that the denominator of (\ref{eqn: expresssion of logarithm of q exponential}) vanishes at order at most 1 at roots of unity, yielding the proposition. 
\end{proof}
We now consider the behavior at $q=1$, and to that end we consider $\log(t;q)_{\infty}^{-1}$ as an element of $\frac{1}{q-1}\QQ[[q-1,t]]$. To simplify computations, we make the variable change $q=\exp(h)$ and writing our power series in $\frac{1}{h}\QQ[[h,t]]$ since $\log(q)=\log(1-(q-1))=h$ with $\log(q)$ in $\QQ[[q^{-1}]]$ and understand the asymptotic behavior by writing equations as power series in the variable $h$. To that end, we recall the following definitions. 
\begin{definition}[Bernoulli Number]\label{def: Bernoulli number}
    Let $n\geq0$. The $n$th Bernoulli number $B_{n}$ is the $n$th coefficient in the power series expansion 
    $$-\frac{x}{1-e^{x}}=\sum_{n\geq0}\frac{B_{n}}{n!}x^{n}\in\QQ[[x]].$$
\end{definition}
\begin{definition}[Polylogarithm]\label{def: polylogarithm}
    Let $n\in\ZZ$. The $n$th polylogarithm is the function
    $$\Li_{n}(x)=\sum_{\ell\geq1}\frac{x^{\ell}}{\ell^{n}}\in\QQ[[x]].$$
\end{definition}
Let us consider some elementary properties of the polylogarithm. 
\begin{lemma}\label{lem: polylogarithm differential equation}
    The $n$th polylogarithm satisfies the differential equation $\nabla\Li_{n}(t)=\frac{1}{t}\Li_{n-1}(t)$ with initial condition $\Li_{n}(0)=0$. 
\end{lemma}
\begin{proof}
    We compute
    $$ \nabla\Li_{n}(t)=\sum_{\ell\geq1}\frac{\ell t^{\ell-1}}{\ell^{n}}=\sum_{\ell\geq1}\frac{t^{\ell-1}}{\ell^{n-1}}$$
    so multiplying by $t$ we have 
    $$t\cdot\nabla\Li_{n}(t)=\sum_{\ell\geq 1}\frac{t^{\ell}}{\ell^{n-1}}= \Li_{n-1}(t)$$
    so $\frac{1}{t}\Li_{n-1}(t)=\Li_{n}(t)$ with the intial condition holding since $\sum_{\ell\geq1}\frac{0^{\ell}}{\ell^{n}}=0$. 
\end{proof}
The some small values of the polylogarithm are given below \cite{Polylogarithm}. 

\begin{table}[h]\label{table: polylogarithm values}
    \begin{align*}
        \Li_{-2}(t)&=\frac{t(t+1)}{(1-t)^{3}} && \Li_{-1}(t)=\frac{t}{1-t} \\
        \Li_{0}(t)&=\frac{t}{1-t} && \Li_{1}(t)=-\log(1-t)
    \end{align*}
    \caption{Values of $\Li_{n}(t)$ for $-2\leq n\leq 1$.}
\end{table}

\begin{lemma}\label{lem: form of negative polylogarithms}
    $\Li_{n}(t)\in t\cdot\ZZ[t,\frac{1}{1-t}]$ for $n\leq 0$. 
\end{lemma}
\begin{proof}
    $\Li_{0}(t)$ satisfies this by Table \ref{table: polylogarithm values}. We proceed by induction, supposing that $\Li_{-k}(t)\in t\cdot\ZZ[t,\frac{1}{1-t}]$, we have by \Cref{lem: polylogarithm differential equation} that $\Li_{-k-1}(t)=t\cdot\nabla\Li_{-k}(t)$. The induction hypothesis implies $\Li_{-k}(t)$ is a $\ZZ$-linear combination of elements of the form $\frac{t^{a}}{(1-t)^{b}}$ so by the quotient rule, the derivative lies in $\ZZ[t,\frac{1}{1-t}]$ which suffices by the discussion above. 
\end{proof}
Elements of in the ring $t\cdot\ZZ[t,\frac{1}{1-t}]$ behave especially nicely with respect to exponentiation. 
\begin{lemma}\label{lem: behavior of nonpositive dilogarithms under exponentials}
    Let $A$ be a ring of characteristic 0. If $f(x)\in t\cdot A[t,\frac{1}{1-t}][[x]]$ then $\exp(f)$ admits a power series expansion in $\mathrm{Frac}(A)[t,\frac{1}{1-t}][[x]]$ as $x\to 0$. 
\end{lemma}
\begin{proof}
    Let us write $f(x)=\sum_{n\geq0}c_{n}(t)x^{n}$ with $c_{n}(t)\in t\cdot A[t,\frac{1}{1-t}]$ depending on $n$. We compute
    \begin{align*}
        \exp(f) &= \exp\left(\sum_{n\geq0}c_{n}(t)x^{n}\right) \\
        &= \prod_{n\geq0}\exp(c_{n}(t)x^{n}) \\
        &= \prod_{n\geq0}\sum_{k\geq0}\frac{c_{n}(t)^{k}}{k!}x^{nk}.
    \end{align*}
    However, for any fixed $N$, the coefficient of $x^{N}$ is a polynomial combination of terms $\frac{c_{n}(t)^{k}}{k!}$ where $n\leq N, k\leq N$ of which there are only finitely many, in particular given by some restriction of $\prod_{0\leq n\leq N}\sum_{0\leq k\leq N}\frac{c_{n}(t)^{k}}{k!}$ which lies in $\mathrm{Frac}(A)[t,\frac{1}{1-t}]$ since each term does. 
\end{proof}

With this language in hand, we deduce the following asymptotic result about the Pochhammer symbol $(t;q)_{\infty}$. 
\begin{proposition}\label{prop: asymptotics q t Pochhammer at 1}
    The $q$-Pochhammer symbol $(t;q)_{\infty}$ satisfies the asymptotic formula
    \begin{equation}\label{eqn: asymptotics of q t Pochhammer at 1}
        (t;q)_{\infty}\sim\exp\left(\frac{\Li_{2}(t)}{h}\right)\cdot\sqrt{1-t}\cdot O(h)
    \end{equation}
    as $q\to 1$ with $O(h)\in\QQ[t,\frac{1}{1-t}][[h]]$. 
\end{proposition}
\begin{proof}
    We compute 
    \begin{align*}
        \frac{t^{\ell}}{\ell(1-q^{\ell})} &= \frac{t^{\ell}}{\ell(1-e^{h\ell})} && q^{\ell}=(e^{h})^{\ell}=e^{h\ell}\\
        &= \frac{h\ell}{1-e^{h\ell}}\cdot\frac{t^{\ell}}{h\ell^{2}} \\
        &= -\sum_{k\geq0}\frac{B_{k}}{k!}(h\ell)^{k}\cdot\frac{t^{\ell}}{h\ell^{2}} && \frac{h\ell}{1-e^{h\ell}}=-\sum_{k\geq0}\frac{B_{k}}{k!}(h\ell)^{k} \\
        &= -\sum_{k\geq0}\left(\frac{t^{\ell}}{h\ell^{2}}\cdot\frac{B_{k}}{k!}\cdot(h\ell)^{k}\right)
    \end{align*}
    so applying this to $\log(t;q)_{\infty}^{-1}$, we have by \Cref{prop: logarithm at worst simple poles at roots of unity} that
    \begin{align*}
        -\log(t;q)_{\infty}^{-1} &= -\sum_{\ell\geq1}\frac{t^{\ell}}{\ell(1-q^{\ell})} \\ 
        &= \sum_{\ell\geq1}\left(\sum_{k\geq0}\frac{t^{\ell}}{h\ell^{2}}\cdot\frac{B_{k}}{k!}\cdot(h\ell)^{k}\right)&& \text{as above} \\
        &= \sum_{k\geq0}\left(\sum_{\ell\geq1}\frac{t^{\ell}}{\ell^{2-k}}\right)\frac{B_{k}}{k!}\cdot h^{k-1} && \\
        &= \sum_{k\geq0}\Li_{2-k}(t)\cdot\frac{B_{k}}{k!}\cdot h^{k-1} && \Li_{2-k}(t)=\sum_{\ell\geq1}\frac{t^{\ell}}{\ell^{2-k}}.
    \end{align*}
    We write this as 
    $$\Li_{2}(t)\cdot B_{0}\cdot\frac{1}{h} + \Li_{1}(t)\cdot B_{1}+\sum_{k\geq2}\Li_{2-k}(t)\cdot\frac{B_{k}}{k!}\cdot h^{k-1}.$$
    Note here that the third summand is a power series in $h$ with coefficients in $\QQ[t,\frac{1}{1-t}]$. Exponentiating, we get, up to constants, 
    $$\exp\left(\frac{\Li_{2}(t)}{h}\right)\cdot\sqrt{1-t}\cdot O(h)$$
    where the second factor follows from $B_{0}=\frac{1}{2}$ and $\exp(-\frac{1}{2}\log(1-t))=\sqrt{1-t}$, and the third factor from applying \Cref{lem: behavior of nonpositive dilogarithms under exponentials} to the observation above. 
\end{proof}
\begin{remark}
    Something similar to \Cref{prop: asymptotics q t Pochhammer at 1} is true for all Nahm sums. 
\end{remark}
\begin{remark}
    It is crucial here to do the expansion in terms of $h$ in order to get a simple result. Doing a power series expansion in other variables will necessitate the use of much more complicated functions. 
\end{remark}



The proofs we have encountered thus far have largely centered around explicit computation, yielding qualitative descriptions of the expansions. The qualitative features of the higher order terms $a_{i}(t)$ of (\ref{eqn: asymptotics of q t Pochhammer at 1}) can in fact be defined recursively by integrating lower order terms. The fact that the integrals of these rational functions remain rational without introducing exotic functions hints at the existence of additional underlying structure to these Nahm sums that may allow qualitative behavior to be deduced without explicit computation.

Returning to the broader discussion at hand, \Cref{prop: adapted q-deformed exponential} suggests that $(t;q)_{\infty}^{-1}$ is the $q$-analogue of the exponential function, and recovering the classical exponential as $q\to 1$, but the behavior we have deduced above is indeed much more complicated. This arises as a consequence of working with $\nabla_{q}'$ in place of $\nabla_{q}$.

To show the asymptotics at other roots of unity, we will require Bernoulli polynomials. 
\begin{definition}[Bernoulli Polynomial]\label{def: Bernoulli polynomial}
    Let $n\geq0$. The $n$th Bernoulli polynomial $B_{n}(t)$ is the $n$th coefficient in the power series expansion 
    $$-\frac{xe^{tx}}{1-e^{x}}=\sum_{n\geq0}\frac{B_{n}(t)}{n!}x^{n}\in\QQ[t][[x]].$$
\end{definition}
We state some elementary properties of Bernoulli polynomials.  
\begin{lemma}\label{lem: properties of Bernoulli polynomials}
    The Bernoulli polynomials satisfy the following identities:
    \begin{enumerate}[label=(\roman*)]
        \item $B_{n}(0)=B_{n}$, 
        \item $B_{n}(t+1)-B_{n}(t)=nt^{n-1}$, and 
        \item $B_{n}(k)=B_{k}+n\cdot\sum_{i=0}^{k-1}i^{n-1}$ for $k\in\NN$. 
    \end{enumerate}
\end{lemma}
\begin{proof}[Proof of (i)]
    This is immediate from the definition. We have $-\frac{xe^{0\cdot x}}{1-e^{x}}=-\frac{x}{1-e^{x}}$ recovering \Cref{def: Bernoulli number}. 
\end{proof}
\begin{proof}[Proof of (ii)]
    The finite difference formula follows from 
    \begin{align*}
        \sum_{n\geq0}\left(B_{n}(t+1)-B_{n}(t)\right)\frac{x^{n}}{t!}&= \frac{xe^{(t+1)x}-xe^{tx}}{e^{x}-1} \\
        &= \frac{xe^{tx}(e^{x}-1)}{(e^{x}-1)} \\
        &= xe^{tx} \\
        &= \sum_{n\geq0}\frac{t^{n}}{n!}x^{n+1} && e^{tx}=\sum_{n\geq0}\frac{t^{n}}{n!}x^{n} \\
        &= \sum_{n\geq0}(nt^{n-1})\cdot\frac{x^{n}}{n!} 
    \end{align*}
    where the equality is given termwise. 
\end{proof}
\begin{proof}[Proof of (iii)]
    Rearranging (ii) we get the recursion $B_{n}(t+1)=nt^{n-1}+B_{n}(t)$ so by induction for any natural number $k$ we have
    \begin{align*}
        B_{n}(k)=B_{n}(0)+n\sum_{i=0}^{k-1}i^{n-1}
    \end{align*}
    as desired. 
\end{proof}

The first few Bernoulli polynomials are given as follows \cite{BernoulliPolynomial}. 
\begin{table}[h]\label{table: Bernoulli polynomials}
    \begin{align*}
        B_{0}(t)&=1 && B_{1}(t)=t-\frac{1}{2} \\
        B_{2}(t)&=t^{2}-t+\frac{1}{6} && B_{3}(t)=t^{3}-\frac{3}{2}t^{2}+\frac{1}{2}t\\
        B_{4}(t)&=t^{4}-2t^{3}+t^{2}-\frac{1}{30} && B_{5}(t)=t^{5}-\frac{5}{2}t^{4}+\frac{5}{3}t^{3}+\frac{1}{6}t
    \end{align*}
    \caption{Bernoulli polynomials $B_{n}(t)$ for $0\leq n\leq 5$.}
\end{table}

We now treat the asymptotics at other roots of unity, taking $q=\zeta_{m}\exp(h)$ where $\zeta_{m}$ is a primitive $m$th root of unity. 

\begin{lemma}\label{lem: summand expansion at roots of unity}
    Let $\zeta_{m}$ be a primitive $m$th root of unity and $q=\zeta_{m}\exp(h)$. Then 
    \begin{equation}\label{eqn: summand expansion at roots of unity}
        \frac{1}{\ell(1-q^{\ell})}\cdot t^{\ell} = -\sum_{n\geq0}\frac{t^{\ell}}{\ell^{2-n}}\left(\sum_{i=0}^{m-1}\zeta_{m}^{i\ell}\cdot B_{n}\left(\frac{i}{m}\right)\right)\frac{m^{n-1}}{n!}h^{n-1}.
    \end{equation}
\end{lemma}
\begin{proof}
    We compute 
    \begin{align*}
        \frac{1}{1-q^{\ell}} &=\frac{1}{1-\zeta_{m}^{\ell}e^{h\ell}} \\
        &= \frac{1}{1-(\zeta_{m}^{\ell}e^{h\ell})^{m}}\sum_{i=1}^{m-1}(\zeta_{m}^{\ell}e^{h\ell})^{i} && \frac{1}{1-x}=\frac{1+x+\dots+x^{m-1}}{1-x^{m}}
    \end{align*}
    so for each summand of (\ref{eqn: expresssion of logarithm of q exponential}) in \Cref{prop: logarithm at worst simple poles at roots of unity}, we have 
    \begin{align*}
        \frac{t^{\ell}}{\ell(1-q^{\ell})}&=\frac{t^{\ell}}{\ell}\cdot\frac{1}{1-(\zeta_{m}^{\ell}e^{h\ell})^{m}}\sum_{i=0}^{m-1}(\zeta_{m}^{\ell}e^{h\ell})^{i} \\
        &= \sum_{i=0}^{m-1}\frac{\zeta_{m}^{i\ell}e^{ih\ell}}{\ell(1-\zeta_{m}^{m\ell}e^{mh\ell})}\cdot t^{\ell} \\
        &= \sum_{i=0}^{m-1}\frac{\zeta_{m}^{i\ell}e^{ih\ell}}{\ell(1-e^{mh\ell})}\cdot t^{\ell} && \zeta_{m}^{m\ell}=1 \\
        &= \sum_{i=0}^{m-1}\frac{\zeta_{m}^{i\ell}e^{\frac{i}{m}x}}{1-e^{x}}\cdot \frac{t^{\ell}}{\ell} && x=mh\ell \\
        &= \frac{t^{\ell}}{\ell}\sum_{i=0}^{m-1}\zeta_{m}^{i\ell}\left(\frac{e^{\frac{i}{m}x}}{1-e^{x}}\right) \\
        &= \frac{t^{\ell}}{\ell}\sum_{i=0}^{m-1}\zeta_{m}^{i\ell}\cdot\frac{1}{x}\left(\frac{xe^{\frac{i}{m}x}}{1-e^{x}}\right) \\
        &= \frac{t^{\ell}}{\ell}\sum_{i=0}^{m-1}\zeta_{m}^{i\ell}\frac{1}{x}\left(-\sum_{n\geq0}\frac{B_{n}(\frac{i}{m})}{n!}x^{n}\right) && \text{\Cref{def: Bernoulli polynomial}} \\
        &= -\frac{t^{\ell}}{\ell}\sum_{i=0}^{m-1}\zeta_{m}^{i\ell}\left(\sum_{n\geq0}\frac{B_{n}(\frac{i}{m})}{n!}x^{n-1}\right) \\
        &= -\frac{t^{\ell}}{\ell}\sum_{n\geq0}\left(\sum_{i=0}^{m-1}\zeta_{m}^{i\ell}\cdot B_{n}\left(\frac{i}{m}\right)\right)\frac{x^{n-1}}{n!} \\
        &= -\frac{t^{\ell}}{\ell}\sum_{n\geq0}\left(\sum_{i=0}^{m-1}\zeta_{m}^{i\ell}\cdot B_{n}\left(\frac{i}{m}\right)\right)\frac{m^{n-1}h^{n-1}\ell^{n-1}}{n!} && x=mh\ell\\
        &= -\sum_{n\geq0}\frac{t^{\ell}}{\ell^{2-n}}\left(\sum_{i=0}^{m-1}\zeta_{m}^{i\ell}\cdot B_{n}\left(\frac{i}{m}\right)\right)\frac{m^{n-1}}{n!}h^{n-1}
    \end{align*}
    giving an expression of the power series in terms of $h$. 
\end{proof}
We will require the following statements in what follows. 
\begin{lemma}\label{lem: dilogarithm roots of unity sum}
    The dilogarithm satisfies the identity 
    \begin{equation}\label{eqn: dilogarithm roots of unity sum}
        \frac{1}{m^{n-1}}\cdot\Li_{n}(t^{m}) = \sum_{i=0}^{m-1}\Li_{n}(\zeta_{m}^{i}t)
    \end{equation}
    for $m,n\in\NN$. 
\end{lemma}
\begin{proof}
    We have 
    \begin{align*}
        \sum_{i=0}^{m-1}\Li_{n}(\zeta_{m}^{i}t) &= \sum_{i=0}^{m-1}\left(\sum_{\ell\geq1}\frac{(\zeta_{m}^{i}t)^{\ell}}{\ell^{n}}\right) \\
        &= \sum_{\ell\geq 1}\left(\sum_{i=0}^{m-1}\zeta_{m}^{i\ell}\right)\frac{t^{\ell}}{\ell^{n}}
    \end{align*}  
    and now noting 
    $$\sum_{i=0}^{m-1}\zeta_{m}^{i\ell}=\begin{cases}
        m & m|\ell \\
        0 & m\nmid\ell
    \end{cases}$$
    the summands above vanish if $\ell$ is not a multiple of $m$ so the sum is in fact given by the sum over $m$-multiples
    $$\sum_{\ell\geq1}\frac{mt^{m\ell}}{(m\ell)^{n}}=\frac{1}{m^{n-1}}\sum_{\ell\geq1}\frac{t^{m\ell}}{\ell^{n}}=\frac{1}{m^{n-1}}\cdot\Li_{n}(t^{m}).$$
\end{proof}
We recover the behavior $q\to\zeta_{m}$ as $h\to0$ so applying the expansion of \Cref{lem: summand expansion at roots of unity} to \Cref{prop: logarithm at worst simple poles at roots of unity}, we get the following asymptotic result. 
\begin{proposition}\label{prop: asymptotics of q t Pochhammer at root of unity}
    The $q$-Pochhammer symbol $(t;q)_{\infty}$ satisfies the asymptotic formula
    \begin{equation}\label{eqn: asymptotics of q t Pochhammer at root of unity}
        (t;q)_{\infty}\sim\exp\left(\frac{\Li_{2}(t^{m})}{m^{2}h}\right)\cdot\frac{\sqrt{1-t^{m}}}{\prod_{i=0}^{m-1}\cdot\left(1-\zeta_{m}t\right)^{i/m}}\cdot O(h)
    \end{equation}
    as $q\to \zeta_{m}$ with $\zeta_{m}$ a primitive $m$th root of unity and $O(h)\in\QQ(\zeta_{m})[t,\frac{1}{1-t^{m}}][[h]]$.
\end{proposition}
\begin{proof}
    We compute  
    \begin{align*}
        -\log(t;q)_{\infty}^{-1} &= -\sum_{\ell\geq1}\frac{1}{\ell(1-q^{\ell})}\cdot t^{\ell} \\
        &= \sum_{\ell\geq1}\left(\sum_{n\geq0}\frac{t^{\ell}}{\ell^{2-n}}\left(\sum_{i=0}^{m-1}\zeta_{m}^{i\ell}\cdot B_{n}\left(\frac{i}{m}\right)\right)\frac{m^{n-1}}{n!}h^{n-1}\right)&& \text{by }(\ref{eqn: summand expansion at roots of unity}) \\
        &=\sum_{n\geq0}\left(\sum_{\ell\geq 1}\frac{t^{\ell}}{\ell^{2-n}}\left(\sum_{i=0}^{m-1}\zeta_{m}^{i\ell}B_{n}\left(\frac{i}{m}\right)\right)\right)\frac{m^{n-1}}{n!}h^{n-1}
    \end{align*}
    by observation, the terms for $n\geq 2$ are power series in $h$, and so too is its exponent, so it remains to consider the first two terms of the series given by 
    \begin{align*}
        \left(\sum_{\ell\geq 1}\frac{t^{\ell}}{\ell^{2-n}}\left(\sum_{i=0}^{m-1}\zeta_{m}^{i\ell}B_{0}\left(\frac{i}{m}\right)\right)\right)\frac{1}{mh}&=\left(\sum_{\ell\geq 1}\frac{t^{\ell}}{\ell^{2-n}}\left(\sum_{i=0}^{m-1}\zeta_{m}^{i\ell}\right)\right)\frac{1}{mh}  && B_{0}(t)=1 \\
        &= \left(\sum_{\ell'\geq1}\frac{mt^{m\ell'}}{(m\ell')^{2}}\right)\frac{1}{mh} \\
        &=\frac{1}{m^{2}}\left(\sum_{\ell'\geq1}\frac{t^{m\ell'}}{\ell'^{2}}\right)\frac{1}{h} \\
        &= \frac{1}{m^{2}h}\Li_{2}(t^{m})
    \end{align*}
    and 
    \begin{align*}
        \sum_{\ell\geq 1}\frac{t^{\ell}}{\ell^{2-n}}\left(\sum_{i=0}^{m-1}\zeta_{m}^{i\ell}B_{1}\left(\frac{i}{m}\right)\right)&= \sum_{\ell\geq 1}\frac{t^{\ell}}{\ell^{2-n}}\left(\sum_{i=0}^{m-1}\zeta_{m}^{i\ell}\left(\frac{i}{m}-\frac{1}{2}\right)\right) && \text{Table \ref{table: Bernoulli polynomials}}\\
        &= \sum_{i=0}^{m-1}\frac{i}{m}\left(\sum_{\ell\geq1}\frac{(\zeta_{m}^{i}t)^{\ell}}{\ell}\right)-\frac{1}{2}\sum_{\ell\geq1}\left(\sum_{i=0}^{\infty}\zeta_{m}^{i\ell}\right) \\
        &= \sum_{i=0}^{m-1}\frac{i}{m}\Li_{1}(\zeta_{m}^{i}t) + \frac{1}{2}\log(1-t^{m}) \\
        &= \frac{1}{2}\log(1-t^{m})+\sum_{i=0}^{m-1}\frac{i}{m}\log(1-\zeta_{m}^{i}t)
    \end{align*}
    respectively. Exponentiating, we get, up to constants, 
    $$\exp\left(\frac{\Li_{2}(t^{m})}{m^{2}h}\right)\cdot\frac{\sqrt{1-t^{m}}}{\prod_{i=0}^{m-1}\cdot\left(1-\zeta_{m}t\right)^{i/m}}\cdot O(h).$$
\end{proof}
Qualitatively, this is quite similar to the asymptotic expansion gleaned in (\ref{eqn: asymptotics of q t Pochhammer at 1}) albeit with a more complicated factor of $O(h)$. In more general settings, the factor of $O(h)$ is the \'{e}tale regulator maps $K$-theory and becomes increasingly difficult to understand. 
% The expansion property of n\geq 2 terms should be able to be gleaned using the different approach outlined by grouping the Pochhammer symbol by residue classes modulo m. If q= \zeta_{m}\exp(h) so q^m is close to 1 and know the asymptotics of (q^i t; q^m)_{\infty}. 
\section{Lecture 3 -- 17th October 2024}\label{sec: lecture 3}
We consider some additional properties of elliptic functions. 
\begin{proposition}\label{prop: sum of residues of elliptic functions}
    Let $\Omega$ be a lattice with period parallelogram $P_{\Omega}$ and $f$ an elliptic function with respect to $\Omega$.\marginpar{Proposition 2.5} Let $a_{1},\dots,a_{n}$ and $a_{n+1},\dots,a_{\ell}$ be the zeroes and poles of $f$, respectively, of orders $m_{1},\dots,m_{n}$ and $m_{n+1},\dots,m_{\ell}$, respectively. Then 
    \begin{equation}\label{eqn: residue sum of zeroes and poles of elliptic function}
        \left(\sum_{i=1}^{n}m_{i}a_{i}\right)-\left(\sum_{i=n+1}^{\ell}m_{i}a_{i}\right)\in\Omega.
    \end{equation}
\end{proposition}
\begin{proof}
    Without loss of generality, we can take these zeroes and poles to lie in the interior of the period parallelogram. Consider the function $g(z)=z\cdot\frac{f'(z)}{f(z)}$ which has simple poles at both the poles and zeroes of $f$ with residues $m_{i}a_{i}$ and $-m_{i}a_{i}$, respectively. In particular, the sum of (\ref{eqn: residue sum of zeroes and poles of elliptic function}) is given by $\frac{1}{2\pi i}\int_{\partial P_{\Omega}}g(z)dz$. Decomposing this as an integral over segments as in \Cref{prop: residue sum is zero}, we have 
    $$\int_{\partial P_{\Omega}}g(z)dz = \int_{[0,\omega_{1}]}g(z)dz + \int_{[\omega_{1},\omega_{1}+\omega_{2}]}g(z)dz + \int_{[\omega_{1}+\omega_{2},\omega_{2}]}g(z)dz+\int_{[\omega_{2},0]}g(z)dz.$$
    Considering the integral over the oriented segments $[0,\omega_{1}]$ and $[\omega_{1}+\omega_{2},\omega_{2}]$ we can write this integral 
    \begin{align*}
        \int_{[0,\omega_{1}]}g(z)dz + \int_{[\omega_{1}+\omega_{2},\omega_{2}]}g(z)dz &= \int_{[0,\omega_{1}]}g(z)dz - \int_{[\omega_{2},\omega_{1}+\omega_{2}]}g(z)dz \\ 
        &= \int_{[0,\omega_{1}]}\frac{zf'(z)}{f(z)}dz - \int_{[0,\omega_{1}]}\frac{(z+\omega_{2})f'(z+\omega_{2})}{f(z+\omega_{2})}dz \\
        &= \omega_{2}\int_{[0,\omega_{1}]}\frac{f'(z)}{f(z)}dz
    \end{align*}
    where we know that $f(z)=f(z+\omega_{2}), f'(z)=f'(z+\omega_{2})$ with $f'(z)$ elliptic by \Cref{prop: elliptic functions form a field}. In particular, this an integer multiple of $\omega_{2}$. Arguing similarly, we can see that 
    $$\int_{[\omega_{1},\omega_{1}+\omega_{2}]}g(z)dz + \int_{[\omega_{2},0]}g(z)dz = \omega_{1}\int_{[0,\omega_{2}]}\frac{f'(z)}{f(z)}dz$$
    which is also an integer multiple of $\omega_{1}$, giving the claim.  
\end{proof}
We return to a consideration of the field of elliptic functions more generally. 
\begin{proposition}\label{prop: fk is elliptic}
    Let $\Omega$ be a lattice.\marginpar{Proposition 3.1} The function 
    $$f_{k}(z)=\sum_{\omega\in\Omega}\frac{1}{(z-\omega)^{k}}$$
    is a nonconstant elliptic function for $k\geq 3$. 
\end{proposition}
\begin{proof}
    The function is elliptic by inspection so it remains to show that the function is locally uniformly convergent. For this, fix $r>0$ and consider the disc $B_{2r}=\{z\in\CC:|z|<2r\}$ and note that $|\Omega\cap B_{2r}|<\infty$ by discreteness of $\Omega$. For any $z$ with $|z|<r$ and $\omega$ with $|\omega|>2r$ we have $\frac{|\omega|}{2}\leq|\omega|-|z|\leq|\omega-z|$ so we have 
    $$\sum_{|\omega|\geq 2r}\frac{1}{|z-\omega|^{k}}\leq 2^{k}\sum_{\omega\in\Omega\setminus\{0\}}\frac{1}{|\omega|^{k}}$$
    where the latter is convergent by \Cref{prop: absolute convergence of lattice sum}, giving the claim. 
\end{proof}
This allows us to produce elliptic functions of orders at least 3. A natural question arises if there are elliptic functions of lower orders. By \Cref{prop: residue sum is zero} it is clear that elliptic functions of order 1 are not possible. Though, as it turns out, elliptic functions of order 2 will play an important role in the theory. 

We can rearrange the equation $f_{3}(z)$ as 
\begin{equation}\label{eqn: f3z rearranged}
    f_{3}(z)-\frac{1}{z^{3}}=\sum_{\omega\in\Omega\setminus\{0\}}\frac{1}{(z-\omega)^{3}}
\end{equation}
and we note that this function has poles at all $\Omega\setminus\{0\}$ with residue zero. As such, we can form the integral $\int_{0}^{z}\sum_{\omega\in\Omega\setminus\{0\}}\frac{1}{(w-\omega)^{3}}dw$ which by convergence of the sum is given by 
$$\sum_{\omega\in\Omega\setminus\{0\}}\int_{0}^{z}\frac{1}{(w-\omega)^{3}}dw=-\frac{1}{2}\sum_{\omega\in\Omega\setminus\{0\}}\left(\frac{1}{(z-\omega)^{2}}-\frac{1}{\omega^{2}}\right)$$
and similarly $\int_{0}^{z}\frac{1}{w^{3}}dw=-\frac{1}{2z^{2}}$ so we have by (\ref{eqn: f3z rearranged}) that 
$$\int f_{3}(z)dz=-\frac{1}{2z^{2}}-\frac{1}{2}\sum_{\omega\in\Omega\setminus\{0\}}\left(\frac{1}{(z-\omega)^{2}}-\frac{1}{\omega^{2}}\right).$$
This yields the Weierstrass $\wp$-function. 
\begin{definition}[Weierstrass $\wp$-Function]\label{def: Weierstrass P-function}
    Let $\Omega$ be a lattice. The Weierstrass $\wp$-function of $\Omega$ is given by 
    $$\wp(z)=\frac{1}{z^{2}}+\sum_{\omega\in\Omega\setminus\{0\}}\left(\frac{1}{(z-\omega)^{2}}-\frac{1}{\omega^{2}}\right)$$
\end{definition}
Moreover, this function has the expected properties. 
\begin{proposition}\label{prop: orders of weierstrass P-function}
    Let $\Omega$ be a lattice. Then:
    \begin{enumerate}[label=(\roman*)]
        \item $\wp(z)$ is elliptic of order 2 and 
        \item $\wp'(z)$ is elliptic of order 3. 
    \end{enumerate}
\end{proposition}
\begin{proof}[Proof of (i)]
    $\wp(z)$ is of order 2 by construction, with double poles at the lattice points. We now show it is elliptic. Noting that $\wp'(z+\omega)-\wp(z)=0$ we have that $\wp(z+\omega)-\wp(z)=C_{\omega}$ where $C_{\omega}$ is a constant depending on $\omega\in\Omega$. For the basis $\omega_{1},\omega_{2}$ of $\Omega$ we can consider for $z=-\frac{\omega_{i}}{2}$ that $\wp(\frac{\omega_{i}}{2})-\wp(-\frac{\omega_{i}}{2})=C_{\omega_{i}}$ but $\wp$ is even so $C_{\omega_{i}}=0$ for $i\in\{1,2\}$ showing that it is elliptic. 
\end{proof}
\begin{proof}[Proof of (ii)]
    This follows from the discussion above, for the Weierstrass $\wp$-function arises as an integral of $f_{3}(z)$ which is an elliptic function of order 3. 
\end{proof}
The Weierstrass $\wp$-function is extremely important to the study of elliptic functions, since every elliptic function can be written as a rational function in $\wp,\wp'$. We first prove the following preparatory lemma. 
\begin{lemma}\label{lem: ellitpic function with poles only in lattice is polynomial in Weierstrass}
    Let $\Omega$ be a lattice and $f(z)$ elliptic with respect to $\Omega$ with poles, if any, in $\Omega$. Then $f(z)=\sum_{i=0}^{n}a_{i}\wp(z)^{i}$ for $a_{i}\in\CC$. 
\end{lemma}
\begin{proof}
    If $f(z)$ is a constant, we are done. Otherwise, we can take the Laurent series expansion of $f(z)$ around the origin which is of the form $\frac{a_{-2n}}{z^{2n}}+\dots$ since $f$ is even. Now note that $f(z)-\frac{a_{-2n}}{\wp(z)^{n}}$ is even and elliptic with a pole of order at most $2n-2$. Thus, repeating this process finitely many times, we eventually arrive at a constant function where we can arrive at the desired claim by multiplying the expressions by $\wp(z)^{2n}$ to clear denominators.  
\end{proof}
Using the above lemma, we can in fact show that it suffices to use a rational function in $\wp(z)$ multiplied by $\wp'(z)$. 
\begin{proposition}\label{prop: elliptic functions in terms of Weierstrass}
    Let $\Omega$ be a lattice and $f(z)$ elliptic with respect to $\Omega$. Then $f(z)$ can be written as the product of $\wp'(z)$ and a rational function in $\wp(z)$. \marginpar{Proposition 3.2} 
\end{proposition}
\begin{proof}
    Note that if $f(z)$ is an elliptic function of odd order, then $f(z)/\wp(z)$ is an elliptic function of even order. So to prove the claim, it suffices to treat the case of $f(z)$ an elliptic function of even order. 

    Let $f(z)$ be even with poles $a_{1},\dots,a_{n}$ in $P_{\Omega}\setminus\{0\}$. Noting that $\wp(z)-\wp(a_{i})$ vanishes for all $a_{i}$, the function $(\wp(z)-\wp(a_{i}))^{m_{i}}f(z)$ will have pole nowhere in $P_{\Omega}\setminus\{0\}$ for $n_{i}$ sufficiently large. Then observing that $f(z)\prod_{i=1}^{n}(\wp(z)-\wp(a_{i}))^{m_{i}}$ is an even elliptic function with poles only at lattice points. Thus by \Cref{lem: ellitpic function with poles only in lattice is polynomial in Weierstrass}, it can be written as a polynomial in $\wp(z)$ allowing us to rewrite $f(z)$ as the quotient of this polynomial by the product $\prod_{i=1}^{n}(\wp(z)-\wp(a_{i}))^{m_{i}}$, that is, as a rational function in $\wp(z)$. 
\end{proof}
\begin{remark}
    The expression of an ellitpic function in terms of $\wp(z),\wp'(z)$ is not unique. 
\end{remark}
Consider the elliptic function $\wp'(z)^{2}$. This is an even elliptic function of order 6 with poles on $\Omega$. By \Cref{lem: ellitpic function with poles only in lattice is polynomial in Weierstrass}, we can write this as a degree 3 polynomial in $\wp(z)$, say $a_{0}+a_{1}\wp(z)+a_{2}\wp(z)^{2}+a_{3}\wp(z)^{3}$ with $a_{i}\in\CC$. The coefficients $a_{i}$, however, are in fact highly structured and can be deduced from the lattice. 
\section{Lecture 4 -- 15th November 2024}\label{sec: lecture 4}
We consider some properties of the dilogarithm function following \cite{ZagierDilogarithm}. 

Recall polylogarithms from \Cref{def: polylogarithm}. We can in fact alternatively define polylogarithms using the relation of \Cref{lem: polylogarithm differential equation}. 
\begin{lemma}\label{lem: dilogarithm as integral on split plane}
    The dilogarithm $\Li_{2}(t)$ satisfies the integral equation 
    \begin{equation}\label{eqn: dilogarithm integral equation}
        \Li_{2}(t)=-\int_{0}^{t}\log(1-z)dz
    \end{equation}
    on the slit plane $\CC\setminus[1,\infty)$. 
\end{lemma}
\begin{proof}
    The claim follows from \Cref{lem: polylogarithm differential equation} and $\Li_{1}(t)=-\log(1-t)$ of Table \ref{table: polylogarithm values}, noting that the logarithm is defined on the slit plane. 
\end{proof}
Special values of the dilogarithm are given as follows \cite[\S 1]{ZagierDilogarithm}. 

\begin{table}[h]\label{table: dilogarithm special values}
    \begin{tabular}{c c c}
        $\Li_{2}(0)=0$ & $\Li_{2}(1)=\frac{\pi^{2}}{6}$ & $\Li_{2}(-1)=-\frac{\pi^{2}}{12}$ \\
        $\Li_{2}\left(\frac{1}{2}\right)=\frac{\pi^{2}}{12}-\frac{1}{2}\log(2)^{2}$ & $\Li_{2}\left(\frac{1-\sqrt{5}}{2}\right)=\frac{\pi^{2}}{15}+\frac{1}{2}\log\left(\frac{1+\sqrt{5}}{2}\right)^{2}$
    \end{tabular}
    \caption{Special values of the dilogarithm.}
\end{table}
The dilogarithm has especially interesting behavior at roots of unity where it can be expressed in terms of the Dedekind Zeta function and exhibit close connections to $L$-functions \cite[\S 5]{ZagierDilogarithm}. Dilogarithms also satisfy a number of functional equations:

\begin{table}[h]\label{table: dilogarithm functional equations}
    \begin{equation}\label{eqn: dilogarithm and negative functional equation}
        \Li_{2}(1/t) = -\Li_{2}(t) - \frac{\pi^{2}}{6}-\frac{1}{2}\log(-t)^{2}
    \end{equation}
    \begin{equation}\label{eqn: dilogarithm and inverse functional equation}
        \Li_{2}(1-z)=-\Li_{2}(t) + \frac{\pi^{2}}{6}-\log(t)\log(1-t)
    \end{equation}
    \caption{Functional equations for the dilogarithm.}
\end{table}

The phenomena above, where products of logarithms appear, are quite common in the setting of dilogarithm functional equations. The dilogarithm can be analytically continued in the following way. 
\begin{theorem}\label{thm: analytic continuation for dilogarithm}
    The function 
    \begin{equation}\label{eqn: analytic continuation for dilogarithm}
        \Li_{2}(e^{h})+h\cdot\Li_{1}(e^{h})
    \end{equation}
    is well deined $\CC\setminus(2\pi i\ZZ)\to\CC/(2\pi i)^{2}\ZZ$. 
\end{theorem}
\begin{proof}
    Computing, we get 
    \begin{align*}
        \frac{d}{dh}\left(\Li_{2}(e^{h})+h\cdot\Li_{1}(e^{h})\right) &= -e^{h}\cdot\frac{1}{e^{h}}\Li_{1}(e^{h}) + \Li_{1}(e^{h}) + h\Li_{0}(e^{h}) \\
        &= \frac{he^{h}}{1-e^{h}} && \Li_{0}(t)=\frac{t}{1-t}
    \end{align*}
    which is a well-defined meromorphic function on $\CC$ with simple poles at integer multiples of $2\pi i$ and residues multiples of $2\pi i$ so the map descends to the quotient. 
\end{proof}
One can also consider an analogue of the dilogarithm is the Bloch-Wigner dilogarithm. 
\begin{definition}[Bloch-Wigner Dilogarithm]\label{def: Bloch-Wigner dilogarithm}
    The Bloch-Wigner dilogarithm is the function 
    $$D(z)=\img(\Li_{2}(z))+\arg(1-z)\log(|z|).$$
\end{definition}
\begin{remark}
    The Bloch-Wigner dilogarithm is a well-defined continuous function on $\CC\cup\{\infty\}$.
\end{remark}
The advantage of working with the Bloch-Wigner dilogarithm is that one no longer needs to consider the products of logarithms in functional equations that arose above. We state some functional equations below. 

\begin{table}[h]\label{table: BW dilogarithm functional equations}
    \begin{equation}\label{eqn: BW dilogarithm sixfold}
        D(z)=D\left(1-\frac{1}{z}\right)=D\left(\frac{1}{1-z}\right)=-D\left(\frac{1}{z}\right)=-D(1-z)=-D\left(\frac{-z}{1-z}\right)
    \end{equation}
    \begin{equation}\label{eqn: BW dilogarithm five term relation}
        D(x)+D(y)+D\left(\frac{1-x}{1-xy}\right)+D(1-xy)+D\left(\frac{1-y}{1-xy}\right)=0
    \end{equation}
    \caption{Functional equations for the Bloch-Wigner dilogarithm.}
\end{table}
Let us return to our discussion of asymptotics of Nahm sums as in \Cref{sec: lecture 2}. We focus on the case $A=a$ for $a\in\ZZ$ producing Nahm sums of the form 
$$f_{a}(t,q)=\sum_{n\geq0}\frac{q^{\frac{1}{2}an^{2}}}{(q;q)_{n}}t^{n}.$$
The $q$-difference equation this satisfies can be easily deduced. 
\begin{proposition}\label{prop: q-difference equation of 1x1 Nahm sum}
    Let $a\in\ZZ$. The $t$-deformed Nahm sum $f_{a}(t,q)=\sum_{n\geq0}\frac{q^{\frac{1}{2}an^{2}}}{(q;q)_{n}}t^{n}$ satisfies the $t$-deformed $q$-difference equation 
    \begin{equation}\label{eqn: q-difference equation of 1x1 Nahm sum}
        f_{a}(t,q) - f_{a}(qt,q)=tq^{a/2}f_{a}(q^{a}t,q).
    \end{equation}
\end{proposition}
\begin{proof}
    We compute 
    \begin{align*}
        f_{a}(t,q) - f_{a}(qt,q) &= \sum_{n\geq0}\frac{q^{\frac{1}{2}an^{2}}}{(q;q)_{n}}t^{n} - \sum_{n\geq0}\frac{q^{\frac{1}{2}an^{2}}}{(q;q)_{n}}q^{n}t^{n} \\
        &= \sum_{n\geq0}\left(\frac{q^{\frac{1}{2}an^{2}}(1-q^{n})}{(q;q)_{n}}\right)t^{n} \\
        &= \sum_{n\geq0}\frac{q^{\frac{1}{2}a(n+1)^{2}}}{(q;q)_{n}}t^{n+1} && \text{reindexing}\\
        &= \sum_{n\geq0}\frac{q^{\frac{1}{2}an^{2}+an+\frac{a}{2}}}{(q;q)_{n}}t^{n+1}\\
        &= tq^{a/2}\left(\sum_{n\geq0}\frac{q^{\frac{1}{2}an^{2}}}{(q;q)_{n}}q^{an}t^{n}\right)\\
        &= tq^{a/2}f_{a}(q^{a}t,q)
    \end{align*}
    as desired. 
\end{proof}
We can modify the Nahm sum to rid ourselves of the $q^{a/2}$ factor in (\ref{eqn: q-difference equation of 1x1 Nahm sum}).\marginpar{The exposition that follows is drawn from the erratum discussed at the begining of lecture 5. The reader is encouraged to consult \cite{NahmAsymptotics} for full details.}
\begin{corollary}\label{corr: q-difference equation of modified 1x1 Nahm sum}
    Let 
    $$f_{a}^{\mathrm{mod}}(t,q)=f_{a}(q^{-a/2}t,q)=\sum_{n\geq0}(-1)^{an}\frac{q^{\frac{1}{2}an^{2}-\frac{1}{2}an}}{(q;q)_{n}}t^{n}$$
    Then the modified Nahm sum satisfies the $t$-deformed $q$-difference equation 
    \begin{equation}\label{eqn: q-difference equation of modified 1x1 Nahm sum}
        f_{a}^{\mathrm{mod}}(t,q) - f_{a}^{\mathrm{mod}}(qt,q) = (-1)^{a}t\cdot f_{a}^{\mathrm{mod}}(q^{a}t,q).
    \end{equation}
\end{corollary}
\begin{proof}
    This follows from a similar computation as in \Cref{prop: q-difference equation of 1x1 Nahm sum}. 
\end{proof}
\begin{remark}
    $f_{a}(t,q)\in\ZZ[[t,\sqrt{q}]]$ but $f_{a}^{\mathrm{mod}}(t,q)\in\ZZ[[t,q]]$. 
\end{remark}
We will henceforth work with these modified Nahm sums. As previously discussed, we would expect the asymptotic expansion of such a Nahm sum to be expressed as a product of an exponential of a dilogarithm, a square root, and a power series in $h$ as $q\to 1$ and $q=\exp(h)$, mirroring the discussion of \Cref{prop: asymptotics of q t Pochhammer at root of unity}. 

We use the ansatz 
\begin{equation}\label{eqn: 1x1 Nahm sum ansatz}
    f_{a}(t,q)=\exp\left(\frac{V(t)}{h}\right)g_{a}(t,q)
\end{equation}
\begin{equation}\label{eqn: qt 1x1 Nahm sum ansatz}
    f_{a}(qt,q)=\exp\left(\frac{V(qt)}{h}\right)g_{a}(qt,q)
\end{equation}
in what follows, with $V(t)\in\QQ[[t]]$ satisfying $V(0)=0$. \Cref{eqn: 1x1 Nahm sum ansatz,eqn: qt 1x1 Nahm sum ansatz} takes are related by the formula we now describe. 
\begin{proposition}\label{prop: V functional equation for 1x1 Nahm ansatz}
    The equation $V(t)$ of the Nahm equation ansatz satisfies the logarithmic differential equation 
    \begin{equation}\label{eqn: V functional equation for 1x1 Nahm ansatz}
        V(e^{h}t) = V(t) + h(\partial^{\log}V)(t)+\frac{h^{2}}{2}((\partial^{\log})^{2}V)(t)
    \end{equation}
    with $\partial^{\log}V(t)=t\cdot V'(t)$.
\end{proposition}
\begin{remark}
    (\ref{eqn: V functional equation for 1x1 Nahm ansatz}) of \Cref{prop: V functional equation for 1x1 Nahm ansatz} should be thought of as a multiplicative analogue of the Taylor expansion. 
\end{remark}
Using this, we can compute the ratio of the leading factors of \Cref{eqn: 1x1 Nahm sum ansatz,eqn: qt 1x1 Nahm sum ansatz}. 
\begin{proposition}\label{prop: asymptotic expansion of modifed 1x1 Nahm sum at 1}
    The Nahm sum $f_{a}(t,q)$ satisfies the asymptotic formula 
    \begin{equation}\label{eqn: asymptotic expansion of modified 1x1 Nahm sum at 1}
        f_{a}(t,q)\sim\exp\left(\frac{V(t)}{h}\right)\cdot O(h)
    \end{equation}
    as $q\to 1$ and $O(h)\in\QQ[[t,h]]$. 
\end{proposition}
\begin{corollary}\label{corr: exponent ratio for 1x1 Nahm sum antsatz}
    Let $V(t)$ be as in the Nahm equation ansatz. The ratios of the exponential factors satisfy 
    $$\exp\left(\frac{V(qt)}{h}\right)/\exp\left(\frac{V(t)}{h}\right)\sim\exp((\partial^{\log}V)(t))(1+O(h))$$
    for $q=e^{h}$.  
\end{corollary}
\begin{proof}
    We have by \Cref{prop: V functional equation for 1x1 Nahm ansatz}
    \begin{align*}
        \exp\left(\frac{V(qt)}{h}\right) &= \exp\left(\frac{V(t)}{h}+(\partial^{\log}V)(t)+\frac{h}{2}((\partial^{\log})^{2}V)(t)\right)\\
        &=\exp\left(\frac{V(t)}{h}\right)\exp\left((\partial^{\log}V)(t)\right)\exp\left(\frac{h}{2}((\partial^{\log})^{2}V)(t)\right)
    \end{align*}
    where the first term of the product above cancels in the ratio and the final term of the product expands to a power series in $h$ with constant coefficient 1. 
\end{proof}
Denoting $Z(t)=\exp((\partial^{\log}V)(t))$ and $\widetilde{Z}(t,q)=\exp\left((\partial^{\log}V)(t)\right)\exp\left(\frac{h}{2}((\partial^{\log})^{2}V)(t)\right)$, we divide (\ref{eqn: q-difference equation of modified 1x1 Nahm sum}) by the exponential prefactor $\exp\left(\frac{V(t)}{h}\right)$ and the ansatz \Cref{eqn: 1x1 Nahm sum ansatz,eqn: qt 1x1 Nahm sum ansatz} to observe 
$$g_{a}(t,q)-\widetilde{Z}(t,q)g_{a}(qt,q)=(-1)^{a}\cdot t\cdot\prod_{i=0}^{a-1}\widetilde{Z}(q^{i}t,q)\cdot g_{a}(q^{a},t)$$
which we seek to show lies in $\QQ[[t,q-1]]$.

Specializing at $q=1$ (ie. $h=0$) produces 
\begin{equation}\label{eqn: specialized Z functional equation}
    1-Z(t)=(-1)^{a}t\cdot Z(t)^{a}
\end{equation}

We now explicitly describe $V(t)$. As suggested by the preceding discussion, it suffices to solve the differential equation $V'(t)=\frac{1}{t}\log(Z(t))$. 
\begin{proposition}\label{prop: V t dilogarithm equation}
    Let $V(t)$ be as in the Nahm equation ansatz and $Z(t)=\exp((\partial^{\log}V)(t))$. $V(t)$ satisfies the equation 
    \begin{equation}\label{eqn: V t dilogarithm equation}
        V(t) = -\Li_{2}(1-Z(t))-\frac{a}{2}\log(Z(t))^{2}
    \end{equation}
    with $V(0)=0$. 
\end{proposition}
\begin{proof}
    We compute 
    \begin{align*}
        V'(t) &= -\frac{Z'(t)}{1-Z(t)}\log(Z(t)) - a\cdot\log(Z(t))\cdot\frac{Z'(t)}{Z(t)} \\
        &= \log(Z(t))\cdot\frac{\left(Z(t)+a(1-Z(t))\right)Z'(t)}{(1-Z(t))Z(t)}
    \end{align*}
    and differentiating (\ref{eqn: specialized Z functional equation}) we get 
    $-Z'(t)=(-1)^{a}Z(t)^{a}+(-1)^{a}at\cdot Z(t)^{a-1}Z'(t)$. Now note $(-1)^{a}at\cdot Z(t)^{a-1}=\frac{1-Z(t)}{Z'(t)}$ so $-Z'(t)=\frac{Z'(t)}{Z(t)}\left(Z(t)+a(1-Z(t))\right)=-\frac{1-Z(t)}{t}$. Substituting this into the equation for $V'(t)$ we get the desired result. 
\end{proof}
We can also discuss asymptotics of Nahm sums at roots of unity $\zeta_{m}$. In parallel to \Cref{prop: asymptotics of q t Pochhammer at root of unity}, we have the following result for Nahm sums. 
\begin{theorem}\label{thm: Nahm sum asymptotics at roots of unity}
    The Nahm sum $f_{a}(t,q)$ satisfies the asymptotic formula 
    \begin{equation}\label{eqn: Nahm sum asymptotics at roots of unity}
        f_{a}(t,q)\sim\exp\left(\frac{V(t^{m})}{m^{2}h}\right)\cdot O(h)
    \end{equation}
    as $q\to\zeta_{m}$ with $\zeta_{m}$ a primitive $m$th root of unity and $O(h)\in\QQ(\zeta_{m})[[t,h]]$. 
\end{theorem}
\begin{proof}[Proof Outline]
    We use the ansatz 
    \begin{equation}\label{eqn: Nahm sum at roots of unity ansatz}
        f_{a}(t,q) = \exp\left(\frac{V(t^{m})}{m^{2}h}\right)g_{a}(t,h)
    \end{equation}
    where we want to show that $g_{a}(t,h)\in\QQ(\zeta_{m})[[t,h]]$ where \emph{a priori}, $g_{a}(t,q)\in\QQ(\zeta_{m})((h))[[t]]$. 

    By \Cref{prop: V t dilogarithm equation}, we have that $t\cdot V'(t)=\log(Z(t))$ so we can write
    $$\widetilde{Z}(t^{m},h)=\frac{\exp\left(\frac{V(q^{m}t^{m})}{m^{2}h}\right)}{\exp\left(\frac{V(t^{m})}{m^{2}h}\right)}=Z(t^{m})^{1/m}\cdot(1+O(h)).$$
    and thus 
    \begin{equation}\label{eqn: expansion of g function}
        g_{a}(t,h)-\widetilde{Z}(t^{m},h)g_{a}(\zeta_{m}e^{h}t,h)=(-1)^{a}t\cdot\prod_{j=0}^{a-1}\widetilde{Z}(e^{\pi i h}t^{m},h)\cdot g_{a}(\zeta_{m}^{a}e^{ah}t,h).
    \end{equation}
    shwoing specialization at $h=0$ is well-defined. As such, we can write $g_{a}(t,0)$ as 
    $$\sum_{j=0}^{a-1}t^{j}h_{j}(t^{m})$$
    and $h_{j}$ a function in $t$. As such, by (\ref{eqn: expansion of g function}) we have
    \begin{align*}
        \sum_{j=0}^{a-1}t^{j}h_{j}(t^{m}) - Z(t^{m})^{1/m}\sum_{j=0}^{a-1}\zeta_{m}^{j}t^{j}h_{j}(t^{m}) &= (-1)^{a}t\cdot Z(t^{m})^{a/m}\cdot\sum_{j=0}^{a-1}\zeta_{m}^{aj}t^{j}h_{j}(t^{m})
    \end{align*}
    and we separate the sum for each $j$ by taking residue classes of exponents of $t$ where we have 
    \begin{align*}
        h_{j}(t^{m})(1-\zeta_{m}^{j}Z(t^{m})^{1/m})&=(-1)^{a}\zeta_{m}^{a(j-1)}Z(t^{m})^{a/m}\cdot h_{j-1}(t^{m}) && j>0 \\
        h_{0}(t^{m})(1-Z(t^{m})^{1/m})&=(-1)^{a}\zeta_{m}^{a(m-1)}t^{m}Z(t^{m})^{a/m}h_{m-1}(t^{m}) && j=0
    \end{align*}
    where 
    $$\prod_{j=0}^{m-1}(1-\zeta_{m}^{j}Z(t^{m})^{1/m})=(-1)^{am}\prod_{j=0}^{m-1}\zeta_{m}^{aj}t^{m}Z(t^{m})^{a}$$
    holds as it simplifies to 
    $$1-Z(t^{m})=(-1)^{a}t^{m}\cdot Z(t^{m})^{a}$$
    by factorization results for cyclotomic polynomials. 
\end{proof}
\section{Lecture 5 -- 22nd November 2024}\label{sec: lecture 5}
We continue our discussion of $q$-series and in particular a property of the modified Nahm sum considered in \Cref{corr: q-difference equation of modified 1x1 Nahm sum}.

The following definition is due to Konsevich-Soibelman \cite{DTInvariants}.
\begin{definition}[Admissable Series]\label{def: admissable series}
    A series $f\in\ZZ((q))[[t]]$ such that $f\equiv 1\pmod{(t)}$ is admissable if it can be written as 
    $$f=\prod_{n\geq1}\prod_{i\in\ZZ}(q^{i}t^{n};q)_{\infty}^{a_{n,i}}$$
    such that for each $n$ only finitely many $a_{n,i}$ are nonzero. 
\end{definition}
\begin{remark}
    These $a_{n,i}$'s are precisely Donaldson-Thomas invariants that arise in Gromov-Witten theory and enumerative geometry. 
\end{remark}
Admissable series force an algebraicity condition on the $q$, allowing $f$ to be written as an element of $\ZZ[q][[t]]$. Up to a condition on the residue of the series $f$ mod $(t)$, series in $\ZZ((q))[[t]]$ admit such an expansion. 
\begin{proposition}
    Let $f\in\ZZ((q))[[t]]$. If $f\equiv1\pmod{(t)}$ then $f$ admits a unique expansion as a series of the form 
    $$f=\prod_{n\geq1}\prod_{i\in\ZZ}(q^{i}t^{n};q)_{\infty}^{a_{n,i}}.$$
\end{proposition}
\begin{proof}[Proof Outline]
    This can be solved for truncated polynomials so for $f\equiv 1\pmod{(t)}$, it suffices to consider $f\equiv 1\pmod{(t^{m})}$ and solve inductively to give an expression algebraic in $t$.
\end{proof}
This result is in fact much more general and it can be shown that the modified Nahm sum 
$$f_{a}(t,q)=\sum_{n\geq0}(-1)^{an}\frac{q^{\frac{1}{2}an^{2}-\frac{1}{2}an}}{(q;q)_{n}}t^{n}$$
as previously defined is admissable. The original proof is highly involved, and we will instead offer a simpler exposition of the same result. Recall from \Cref{prop: logarithm at worst simple poles at roots of unity}, we have 
\begin{equation}\label{eqn: modified Nahm sum as exponent of sum}
    (q^{i}t;q)_{\infty}=\exp\left(-\sum_{\ell\geq 1}\frac{1}{\ell}\cdot\frac{q^{i\ell}t^{n\ell}}{1-q^{\ell}}\right). 
\end{equation}
Furthermore, $\ZZ((q))[[t]]$ is a $\lambda$-ring as we now define. 
\begin{definition}[$\lambda$-Ring]\label{def: lambda-ring}
    A $\lambda$-ring $A$ is a ring equipped with set maps $\lambda^{k}:A\to A$ for $0\leq k\leq\infty$ such that 
    \begin{enumerate}[label=(\roman*)]
        \item $\lambda^{0}(a)=1$ for all $a\in A$.
        \item $\lambda^{1}(a)=a$ for all $a\in A$. 
        \item $\lambda^{n}(a+b)=\sum_{i+j=n}\lambda^{i}(a)\lambda^{i}(b)$ for all $a,b\in A$. 
    \end{enumerate}
\end{definition}
Such a ring admits Adams operations $\psi_{n}:\ZZ((q))[[t]]\to\ZZ((q))[[t]]$ by $t\mapsto t^{n},q\mapsto q^{n}$. The Adams operations allow us to rewrite (\ref{eqn: modified Nahm sum as exponent of sum}) as 
\begin{equation}\label{eqn: modified Nahm sum as exponent of Adams sum}
    \exp\left(-\sum_{\ell\geq1}\psi_{\ell}\left(\frac{q^{i}t^{n}}{1-q}\right)\right).
\end{equation}
We introduce the notion of the plethystic exponential.
\begin{definition}[Plethystic Exponential]\label{def: Plethystic exponential}
    Let $A$ be a $\lambda$-ring and $\sum_{\ell\geq 1}\frac{a_{\ell}}{\ell}$ a convergent series in $A$. The plethystic exponential is of the series is given by 
    $$\exp\left(\sum_{\ell\geq 1}\frac{1}{\ell}\psi_{\ell}(a_{\ell})\right).$$
\end{definition}
Now taking 
$$\phi(t,q)=-\sum_{n\geq1}\sum_{i\in\ZZ}a_{n,i}q^{i}t^{n}\in\ZZ((q))[[t]]$$
we have 
\begin{equation*}\label{eqn: adams admissable Nahm sum}
    \prod_{n\geq1}\prod_{i\in\ZZ}(q^{i}t;q)_{\infty}=\exp\left(\sum_{\ell\geq1}\frac{1}{\ell}\psi_{\ell}\left(\frac{\phi(t,q)}{1-q}\right)\right).
\end{equation*}
We can define the plethystic logarithm as the inverse of the plethystic exponential and observe $\frac{\phi(t,q)}{1-q}$ is the plethystic logarithm of the modified Nahm sum $f_{a}(t,q)$. We then seek to show that this plethystic logarithm $\frac{\phi(t,q)}{1-q}\in\frac{1}{1-q}\ZZ[q^{\pm}][[t]]$ which has a single simple pole at $q=1$. Indeed, this suffices as the behavior at other roots of unity are determined by the Adams operations. 


Observe the plethystic exponential gives an isomorphism $t\QQ[q^{\pm}][[t]]\to 1+t\QQ[q^{\pm}][[t]]$. It thus suffices to show that the plethystic logarithm of $f_{a}$ is a function that is a sum $\frac{\phi_{0}(t)}{1-q}$ with an element of $\QQ[q^{\pm}][[t]]$. 

Now using the ansatz
\begin{equation}\label{eqn: plethystic exponential ansatz}
    f_{a}(t,q)=\exp\left(\sum_{\ell\geq 1}\frac{1}{\ell}\frac{\phi_{0}(t^{\ell})}{1-q^{\ell}}\right)g_{a}(t,q)
\end{equation}
we show that there exists a chioce of function $\phi_{0}(t)$ satisfying the ansatz above would imply the factor $g_{a}(t,q)\in 1+t\QQ[q^{\pm}][[t]]$ and from which the result  of showing $g_{a}(t,q)$ lying in $1+t\QQ[t^{\pm}][[t]]$ would follow by application of the Plethystic exponential. 

But a choice of $\phi_{0}\in t\cdot\QQ[[t]]$ can be made such that $\sum_{\ell\geq 1}\frac{\phi_{0}(t^{\ell})}{\ell^{2}}=-V(t)$ whose plethystic exponential has leading term asymptotics agreeing with $f_{a}(t,q)$ at all roots of unity.

This is gives the desired result as stated below. 
\begin{theorem}[Kontsevich-Soibelman, Efimov]\label{thm: modified 1x1 Nahm sum is admissable}
    The $q$-series 
    $$f_{a}(t,q)=\sum_{n\geq0}(-1)^{an}\frac{q^{\frac{1}{2}an^{2}-\frac{1}{2}an}}{(q;q)_{n}}t^{n}$$
    is admissable. 
\end{theorem}
Let us unfurl some of the consequences of \Cref{thm: modified 1x1 Nahm sum is admissable}. 

To understand the $\phi_{0}(t)$ function better, we use compute its logarithmic derivative so in conjunction with \Cref{prop: V functional equation for 1x1 Nahm ansatz} we have 
$$\sum_{\ell\geq1}\frac{(\partial^{\log}\phi_{0})(t^{\ell})}{\ell}=-\log Z(t)$$
with $Z(t)$ is the logarithmic derivative of $V(t)$. Note that $Z(t)\in 1+t\ZZ[[t]]$ so $(\partial^{\log}\phi_{0})(t)=-\sum_{n\geq1}c_{n}t^{n}$ for $Z(t)=\prod_{n\geq1}(1-t^{n})^{c_{n}}$ for $c_{n}\in\ZZ$. This shows $n|c_{n}$. 
\begin{example}
    For $a=0$, $f_{1}(t,q)=(t;q)^{-1}_{\infty}$ so $Z(t)=1-t$ and $\phi_{0}(t)=\pm t$ which agrees with $V(t)$ being the dilogarithm (up to a sign). 
\end{example}
\begin{example}
    In the case $a=2$ which was discussed in \Cref{sec: lecture 1}, we recover $Z(t)$ as an alternating sum of the Catalan numbers. 
\end{example}
\section{Lecture 6 -- 25th October 2024}\label{lec: lecture 6}
Recall that for a topological space $X$ and a continuous map $f:X\to Y$, the induced functors $\Gamma(X,-):\Sh(X)\to\AbGrp$ and $f_{*}:\Sh(X)\to\Sh(Y)$ are not in general exact, but only left exact \Cref{def: sheaf cohomology,def: derived pushforward}. We can measure the failure of exactness by using $\delta$-functors which is computed using injective resolutions. However, injective resolutions are quite difficult to work with. We will consider some resolutions to the difficulty of computation today. 
\begin{definition}[$F$-Acyclic Objects]\label{def: F-acyclic}
    Let $F:\Asf\to\Bsf$ be a left exact additive functor between Abelian categories and $R^{i}F$ exists. An object $A$ of $\Asf$ is $F$-acyclic if $R^{i}F(A)=0$ for all $i\geq1$. 
\end{definition}
\begin{remark}
    Injective objects are $F$-acyclic per \Cref{lem: vanishing of higher derived image of injective object}. 
\end{remark}
Cohomology can in most cases be computed on acyclic resolutions in place of injective resolutions. 
\begin{proposition}\label{prop: acyclic resolutions replace }
    Let $F:\Asf\to\Bsf$ be a left exact additive functor between Abelian categories and $\Asf$ having enough injectives. Then the cohomology of any $F$-acyclic resolution is equal to the cohomology of any injective resolution. 
\end{proposition}
\begin{proof}
    These are both effacable $\delta$-functors and hence universal by \Cref{thm: Grothendieck tohoku effacable implies universal}.
\end{proof}
Let us now turn to the case of derived pushforward. 
\begin{proposition}\label{prop: derived pushforward is sheafification of inverse image cohomology sheaf}
    Let $f:X\to Y$ be a continuous map and $\Fcal$ a sheaf on $X$. Then $R^{i}f_{*}\Fcal$ is the sheafification of the presheaf $V\mapsto H^{i}(f^{-1}(V),\Fcal)$ for all sheaves $\Fcal$ on $X$. 
\end{proposition}
\begin{proof}
    Taking an injective resolution of $\Fcal$, we know its direct image is injective, extending the left exact functor $f_{*}$. So computing the cohomology of the complex $0\to f_{*}\Fcal\to f_{*}\Ical_{0}\to\dots$, we have 
    $$R^{i}f_{*}\Fcal=\frac{\ker\left(f_{*}\Ical_{i}\to f_{*}\Ical_{i+1}\right)}{\img\left(f_{*}\Ical_{i-1}\to f_{*}\Ical_{i}\right)}$$
    with sections over $V\subseteq Y$ open are given by
    \begin{align*}
        R^{i}f_{*}\Fcal(V) &= \frac{\ker\left(f_{*}\Ical_{i}(V)\to f_{*}\Ical_{i+1}(V)\right)}{\img\left(f_{*}\Ical_{i-1}(V)\to f_{*}\Ical_{i}(V)\right)}\\
        &=\frac{\ker\left(\Ical_{i}(f^{-1}(V))\to \Ical_{i+1}(f^{-1}(V))\right)}{\img\left(\Ical_{i-1}(f^{-1}(V))\to \Ical_{i}(f^{-1}(V))\right)} \\
        &= \frac{\ker(\Ical_{i}\to\Ical_{i+1})(f^{-1}(V))}{\img(\Ical_{i-1}\to\Ical_{i})(f^{-1}(V))}
    \end{align*} 
    where the claim follows, noting that the image sheaf in the quotient is the sheafification of the presheaf image. 
\end{proof}
In the case of computing sheaf cohomology, a large example of $\Gamma$-acyclic sheaves is provided by flasque sheaves. 
\begin{definition}[Flasque Sheaf]\label{def: flasque sheaf}
    Let $X$ be a topological space. A sheaf $\Fcal$ on $X$ is flasque if for all $U\subseteq X$ open and $V\subseteq U$ open $\res_{U,V}:\Fcal(U)\to\Fcal(V)$ is surjective. 
\end{definition}
We consider some elementary properties of flasque sheaves. 
\begin{proposition}\label{prop: properties of flasque sheaves}
    Let $X$ be a topological space. Then:
    \begin{enumerate}[label=(\roman*)]
        \item If $\Fcal$ is flasque, then $\Fcal|_{U}$ is flasque for all $U\subseteq X$ open. 
        \item If $\Ical$ is injective, then $\Ical$ is flasque. 
        \item If $f:X\to Y$ is a continuous map and $\Fcal$ a flasque sheaf on $X$ then $f_{*}\Fcal$ is a flasque sheaf on $Y$. 
    \end{enumerate}
\end{proposition}
\begin{proof}[Proof of (i)]
    This is immediate from the definition as for $W\subseteq V\subseteq U$ we have $\Fcal(V)=\Fcal|_{U}(V)\to\Fcal|_{U}(W)=\Fcal(W)$ surjective. 
\end{proof}
\begin{proof}[Proof of (ii)]
    See \cite[\href{https://stacks.math.columbia.edu/tag/01EA}{Tag 01EA}]{stacks-project}. 
\end{proof}
\begin{proof}[Proof of (iii)]
    For $W\subseteq V\subseteq Y$ we have $f^{-1}(W)\subseteq f^{-1}(V)$ giving so $\Fcal(f^{-1}(V))=f_{*}\Fcal(V)\to f_{*}\Fcal(W)=\Fcal(f^{-1}(W))$ is surjective. 
\end{proof}
As expected, flasque sheaves are $\Gamma$-acyclic. We can say more:
\begin{proposition}\label{prop: flasque sheaves are gamma acyclic}
    Let $X$ be a topological space. If $0\to\Fcal\to\Gcal\to\Hcal\to0$ is a short exact sequence of sheaves and $\Fcal$ is flasque then $0\to\Fcal(U)\to\Gcal(U)\to\Hcal(U)\to0$ is exact for all $U\subseteq X$ open. 
\end{proposition}
\begin{proof}
    Taking sections is generally left exact by \Cref{prop: sections does not preserve exactness} giving an exact sequence 
    $$0\longrightarrow\Fcal(U)\longrightarrow\Gcal(U)\longrightarrow\Hcal(U).$$ 
    To show exactness on the right, then, it suffices to show that the morphism $\Gcal(U)\to\Hcal(U)$ is surjective. Consider the induced sequence on stalks for some $p\in U$ which is exact. So, passing to germs, for any section $s\in\Fcal(U)$ there exists an open covering $\{U_{i}\}_{i\in I}$ for an ordered indexing set $I$ on which $s_{i}$ is the image of $t_{i}\in\Fcal(U_{i})$, though the $t_{i}$ need not glue as sections of $\Gcal(U)$. 
    \\\\
    We proceed by induction. For some fixed $j\in I$ suppose that $t_{i}$ are such that they glue to $t$ in $U_{(j)}=\bigcup_{i<j}U_{i}$. Consider $t|_{U_{(j)}\cap U_{j}}-t_{j}|_{U_{(j)}\cap U_{j}}$ which lies in the subsheaf $\Fcal'(U_{(j)}\cap U_{j})$ as its image vanishes in $\Hcal(U_{(j)}\cap U_{j})$. But $\Fcal$ is flasque so there is $r_{j}\in\Fcal(U_{j})$ with image $t|_{U_{(j)}\cap U_{j}}-t_{j}|_{U_{(j)}\cap U_{j}}$ in $\Fcal(U_{(j)}\cap U_{j})$. Now note that $r_{j}+t_{j}$ is compatible with $t$: on $U_{i}\cap U_{j}$ for $i<j$ is given by 
    $$t|_{U_{i}\cap U_{j}}= (t|_{U_{i}\cap U_{j}} - t_{j}|_{U_{i}\cap U_{j}}) + t_{j}|_{U_{i}\cap U_{j}}.$$
    The section $r_{j}+t_{j}\in\Fcal(U)$ hence extends to a section on $\bigcup_{i\leq j}U_{i}$ by the gluability axiom of the sheaf $\Gcal$. Repeating this process inductively yields a section of $\Gcal(U)$ with image $s$, showing surjectivity, and thus exactness on the right. 
\end{proof}
We conclude with a final property of flasque sheaves. 
\begin{proposition}\label{prop: flasque sheaves are 2 of 3}
    Let $X$ be a topological space. If 
    $$0\to\Fcal\to\Gcal\to\Hcal\to0$$
    is a short exact sequence of sheaves with $\Fcal,\Gcal$ flasque then $\Hcal$ is flasque. 
\end{proposition}
\begin{proof}
    For suppose $V\subseteq U$ and the exactness result from \Cref{prop: flasque sheaves are gamma acyclic} gives 
    $$
    \begin{tikzcd}
        0 & {\Fcal(U)} & {\Gcal(U)} & {\Hcal(U)} & 0 \\
        0 & {\Fcal(V)} & {\Gcal(V)} & {\Hcal(V)} & 0
        \arrow[from=1-1, to=1-2]
        \arrow[from=1-2, to=1-3]
        \arrow[from=1-3, to=1-4]
        \arrow[from=1-4, to=1-5]
        \arrow[from=2-1, to=2-2]
        \arrow[from=2-2, to=2-3]
        \arrow[from=2-3, to=2-4]
        \arrow[from=2-4, to=2-5]
        \arrow[from=1-4, to=2-4]
        \arrow[from=1-3, to=2-3]
        \arrow[from=1-2, to=2-2]
    \end{tikzcd}$$
    a diagram with vertical maps given by restrictions. Furthermore, the diagram is commutative by the definition of morphisms of schemes which commute with restrictions. $\Gcal(U)\to\Gcal(V)$ is surjective by flasqueness of $\Gcal$, $\Gcal(U)\to\Hcal(U)$ and $\Gcal(V)\to\Hcal(V)$ surjective by exactness of the sequence. In particular, the composite $\Fcal(U)\to\Gcal(U)\to\Hcal(V)$ is surjective, and hence $\Hcal(U)\to\Hcal(V)$ is surjective -- recalling here that if $f:A\to B$, $g:B\to C$ are such that $g\circ f:A\to B\to C$ is surjective then $g$ is surjective.
\end{proof}
We introduce \v{C}ech cohomology which will be a key tool for computing sheaf cohomology, and will agree with sheaf cohomology in many cases. This will be done by computing the cohomology of the \v{C}ech complex associated to a cover. 
\begin{proposition}\label{prop: cech complex is a complex}
    Let $X$ be a topological space, $\{U_{i}\}_{i\in I}$ a cover of $X$ with $I$ a totally ordered set, and $\Fcal$ a sheaf on $X$. Consider the data of
    \begin{itemize}
        \item An Abelian group for each $p\geq0$
        $$C^{p}(\{U_{i}\}_{i\in I},\Fcal)=\prod_{i_{0}<i_{1}<\dots<i_{p}}\Fcal(U_{i_{0}}\cap U_{i_{1}}\cap \dots\cap U_{i_{p}})$$
        \item Morphisms $C^{p}(\{U_{i}\}_{i\in I},\Fcal)\to C^{p+1}(\{U_{i}\}_{i\in I},\Fcal)$ by 
        $$(s_{i_{0},\dots,i_{p}})_{i_{0}<i_{1}<\dots<i_{p}}\mapsto\left(\sum_{j=0}^{p+1}(-1)^{j}s_{i_{0},\dots,\widehat{i_{j}},\dots,i_{p+1}}|_{U_{i_{0}\cap\dots \cap U_{i_{p+1}}}}\right)_{i_{0}<i_{1}<\dots<i_{p}<i_{p+1}}$$
    \end{itemize}
    giving a diagram of Abelian groups 
    \begin{equation}\label{eqn: Cech complex}
        0\to C^{0}(\{U_{i}\}_{i\in I},\Fcal)\to C^{1}(\{U_{i}\}_{i\in I},\Fcal)\to C^{2}(\{U_{i}\}_{i\in I},\Fcal)\to\dots.
    \end{equation}
    The diagram (\ref{eqn: Cech complex}) is a chain complex of Abelian groups. 
\end{proposition}
\begin{proof}
    We verify the map $C^{p-1}(\{U_{i}\}_{i\in I},\Fcal)\to C^{p+1}(\{U_{i}\}_{i\in I},\Fcal)$ is the zero map via direct computation. For a section $(s_{i_{0},\dots,i_{p-1}})_{i_{0}<\dots<i_{p-1}}$, its image in $C^{p+1}(\{U_{i}\}_{i\in I},\Fcal)$ is given by 
    \begin{align*}
        &\sum_{j=0}^{p+1}(-1)^{j}\left(\sum_{k=0}^{p}(-1)^{k}s_{i_{0},\dots,\widehat{i_{k}},\dots,i_{p}}|_{U_{i_{0},\dots,i_{p}}}\right)_{i_{0},\dots,\widehat{i_{k}},\dots,i_{p_1}} \\
        &= \sum_{j=0}^{p+1}(-1)^{j}\left(\sum_{k=0}^{j-1}(-1)^{k}s_{i_{0},\dots,\widehat{i_{k}},\dots,\widehat{i_{j}},\dots,i_{p+1}}+\sum_{k=j+1}^{p+1}(-1)^{k-1}s_{i_{0},\dots,\widehat{i_{j}},\dots,\widehat{i_{k}},\dots,i_{p+1}}\right)|_{U_{i_{0},\dots,i_{p+1}}}
    \end{align*}
    but the sum telescopes, giving the claim. 
\end{proof}
\begin{remark}
    The construction of \Cref{prop: cech complex is a complex}, in words, states that given a family of sections $s_{i_{0},\dots,i_{p}}$ defined on $U_{i_{0}}\cap\dots\cap U_{i_{p}}$ for each ordered subset of $I$ of size $p+1$ to a section on $U_{i_{0}}\cap\dots\cap U_{i_{p}}\cap U_{i_{p+1}}$ by taking the alternating sum of the restriction of sections $s_{i_{0},\dots,i_{p}}|_{U_{i_{0}}\cap\dots\cap U_{i_{p+1}}}$ over all $p+1$ element subsets of $i_{0},\dots,i_{p+1}$. 
\end{remark}
\begin{remark}
    To the end of getting better intuition for the construction, let's consider some cases of the \v{C}ech complex for $|I|$ small. Let $X$ admit a cover by $U_{0},U_{1},U_{2}$ and $\Fcal$ a sheaf on $X$. For a section $s\in\Fcal(X)$ the construction of the \v{C}ech complex takes $s$ to the tuple of sections $(s|_{U_{0}},s|_{U_{1}},s|_{U_{2}})$, the tuple of sections $(s|_{U_{0}},s|_{U_{1}},s|_{U_{2}})$ to $((s|_{U_{1}}-s|_{U_{0}})|_{U_{0}\cap U_{1}}, (s|_{U_{2}}-s|_{U_{0}})|_{U_{0}\cap U_{2}}, (s|_{U_{2}}-s|_{U_{1}})|_{U_{1}\cap U_{2}})$.
\end{remark}
As such we are justified in making the following definition. 
\begin{definition}[\v{C}ech Complex]\label{def: Cech complex}
    Let $X$ be a topological space, $\{U_{i}\}_{i\in I}$ a cover of $X$ with $I$ a totally ordered set, and $\Fcal$ a sheaf on $X$. The \v{C}ech complex of $\Fcal$ with respect to the cover $U$ is the chain complex 
    $$0\to C^{0}(\{U_{i}\}_{i\in I},\Fcal)\to C^{1}(\{U_{i}\}_{i\in I},\Fcal)\to C^{2}(\{U_{i}\}_{i\in I},\Fcal)\to\dots$$
    where $C^{p}(\{U_{i}\}_{i\in I},\Fcal)$ and the differentials given by 
    $$(s_{i_{0},\dots,i_{p}})_{i_{0}<i_{1}<\dots<i_{p}}\mapsto\left(\sum_{j=0}^{p+1}(-1)^{j}s_{i_{0},\dots,\widehat{i_{j}},\dots,i_{p+1}}|_{U_{i_{0}\cap\dots \cap U_{i_{p+1}}}}\right)_{i_{0}<i_{1}<\dots<i_{p}<i_{p+1}}.$$
\end{definition}
The \v{C}ech cohomology of a sheaf is merely the cohomology of the corresponding \v{C}ech complex. 
\begin{definition}\label{def: Cech cohomology}
    Let $X$ be a topological space, $\{U_{i}\}_{i\in I}$ a cover of $X$ with $I$ a totally ordered set, and $\Fcal$ a sheaf on $X$. The \v{C}ech cohomology of the sheaf 
    $$\check{H}^{p}(X,\Fcal)=\frac{\ker\left(C^{p}(\{U_{i}\}_{i\in I},\Fcal)\to C^{p+1}(\{U_{i}\}_{i\in I},\Fcal)\right)}{\img\left(C^{p-1}(\{U_{i}\}_{i\in I},\Fcal)\to C^{p}(\{U_{i}\}_{i\in I},\Fcal)\right)}$$
    is the cohomology of the corresponding \v{C}ech complex. 
\end{definition}
We can use this to compute the sheaf cohomology of $\underline{\ZZ}$ on the circle $S^{1}$. 
\begin{example}
    Let $S^{1}$ be the unit circle and $U_{0},U_{1}$ a cover of the upper and lower semicircles intersecting around $(-1,0)$ and $(1,0)$. Then the \v{C}ech complex is given by $H^{0}(S^{1},\underline{\ZZ})\to C^{0}(\{U_{0},U_{1}\},\Fcal)\to C^{1}(\{U_{0},U_{1}\},\Fcal)$ by $\ZZ\to\ZZ\oplus\ZZ\to\ZZ\oplus\ZZ$ where the first map is by $s\mapsto (s|_{U_{0}},s|_{U_{1}})$ and the second by $(s|_{U_{0}},s|_{U_{1}})\mapsto (s|_{U_{1}}-s|_{U_{0}},s|_{U_{1}}-s|_{U_{0}})$ yielding $\check{H}^{0}(S^{1},\underline{\ZZ})=\check{H}^{1}(S^{1},\underline{\ZZ})=\ZZ$. 
\end{example}
\section{Lecture 7 -- 31st October 2024}\label{sec: lecture 7}
We turn to the study of analytic number theory, and in particular the prime number theorem. Recall Riemann's zeta function. 
\begin{definition}[Riemann Zeta Function]\label{def: Riemann zeta function}
    The Riemann zeta function is defined as the infinite sum 
    $$\zeta(s)=\sum_{n=1}^{\infty}\frac{1}{n^{s}}.$$
\end{definition}
\begin{remark}
    $\zeta(s)$ converges for $\RE(s)>1$: $n^{-s}=e^{-s\log(n)}$ here taking the real logarithm, so $|n^{-s}|=n^{-\RE(s)}$ and the series converges -- in fact, uniformly -- on $\RE(s)>1$. 
\end{remark}
We now define the Mellin transform which will be a crucial construction, which will be justified by the following lemma. 
\begin{lemma}\label{lem: Mellin transform existence conditions}
    Let $f:[1,\infty)\to\RR$ be a locally integrable function such that $|f(x)|\leq C\cdot x^{k}$ for some constant $C$ and $k\in\NN$. Then the integral $\int_{1}^{\infty}f(x)x^{-s-1}dx$ exists for $\RE(s)>k$. 
\end{lemma}
\begin{proof}
    For $s$ such that $\RE(s)>k$ we have that $|f(x)|x^{-s-1}\leq C\cdot x^{k-s-1}$ where $\RE(k-s-1)<0$ so the function is bounded above by $C$, giving the claim. 
\end{proof}
\begin{remark}
    Recall that a locally integrable function is a function where the integral over any finite interval exists. 
\end{remark}
\begin{definition}[Mellin Transform]\label{def: Mellin transform}
    Let $f:[1,\infty)\to\RR$ be a locally integrable function such that $|f(x)|\leq C\cdot x^{k}$ for some constant $C$ and $k\in\NN$. The Mellin transform of $f$ is given by 
    $$\Mcal_{f}(s)=s\int_{1}^{\infty}f(x)x^{-s-1}dx.$$
\end{definition}
The Mellin transform for the function $x$ and the Gauss bracket will play a key role in estimating the distribution of primes. 
\begin{lemma}\label{lem: Mellin transform of x}
    The Mellin transform of $f(x)=x$ is given by $1+\frac{1}{s-1}$. 
\end{lemma}
\begin{proof}
    We compute 
    \begin{align*}
        s\int_{1}^{\infty}x\cdot x^{-s-1}dx &= s\int_{1}^{\infty}x^{-s}dx \\
        &= \frac{s}{s-1}.
    \end{align*}
\end{proof}
We now define the Gauss bracket and compute its Mellin transform. 
\begin{definition}[Gauss Bracket]\label{def: Gauss bracket}
    The Gauss bracket $[x]$ is given by the largest integer at most $x$. 
\end{definition}
\begin{lemma}\label{lem: Mellin transform of Gauss bracket}
    The Mellin transform of the Gauss bracket $f(x)=[x]$ is given by the Riemann zeta function $\zeta(s)$. 
\end{lemma}
\begin{proof}
    We compute 
    \begin{align*}
        s\int_{1}^{\infty}[x]x^{-s-1}dx &= \sum_{n=1}^{\infty}s\int_{n}^{n+1}[x]x^{-s-1}dx \\
        &= \sum_{n=1}^{\infty}ns\int_{n}^{n+1}x^{-s-1}dx \\
        &= \sum_{n=1}^{\infty}ns\left(\frac{x^{-s}}{-s}\mid^{n+1}_{x=n}\right) \\
        &= \sum_{n=1}^{\infty}n(n^{-s}-(n+1)^{-s})
    \end{align*}
    note the series covnerges for $\RE(s)>2$ where we decompose the sum 
    \begin{align*}
        \sum_{n=1}^{\infty}n(n^{-s}-(n+1)^{-s}) &= \sum_{n=1}^{\infty}n^{1-s}-\sum_{n=1}^{\infty}(n+1-1)(n+1)^{-s} \\
        &= \sum_{n=1}^{\infty}n^{1-s} - \sum_{n=2}^{\infty}n^{1-s} + \sum_{n=2}^{\infty}n^{-s} \\
        &= 1+(\zeta(s)-1) \\
        &= \zeta(s)
    \end{align*}
    as desired. 
\end{proof}
From this, we can deduce the Mellin transform $\Mcal_{[x]-x}(s)$. 
\begin{proposition}\label{prop: Mellin transform of bracket minus x}
    The Mellin transform of the difference $[x]-x$ is given by 
    $$\Mcal_{[x]-x}(s)=\zeta(s)-\frac{1}{s-1}-1$$
    on $\RE(s)>1$ and is meromorphic in that halfspace with pole at $s=1$ with residue 1.\marginpar{Theorem 1}
\end{proposition}
\begin{proof}
    For $\RE(s)>2$ the statement follows from \Cref{lem: Mellin transform of x,lem: Mellin transform of Gauss bracket} and noting that $\frac{s}{s-1}=\frac{1}{s-1}+1$, and extends to $\RE(s)>1$ by the identity theorem \Cref{thm: identity theorem} as the functions agree in the right halfspace $\RE(s)>2$ but are holomorphic for $\RE(s)>1$ away from the pole at 1. The function $\frac{1}{s-1}$ is the sole contributor to the residue at 1 and has residue 1 by inspection. 
\end{proof}
The Riemann zeta function admits an alternative description via infinite products. 
\begin{theorem}\label{thm: zeta function by infinite products}
    The Riemann zeta function $\zeta(s)$ is equal to the infinite product $\prod_{p\text{ prime}}\frac{1}{1-p^{-s}}$ for $\RE(s)>1$.\marginpar{Theorem 2} 
\end{theorem}
\begin{proof}
    The convergence of the sum $\sum_{n=1}^{\infty}\frac{1}{n^{s}}$ implies the convergence of the product since it contains $\sum_{p\text{ prime}}\frac{1}{p^{s}}$ is a subsequence. Now let $\Pcal_{k}\subseteq\Pcal$ be the finite set of the first $k$ primes and consider the restricted product $\prod_{p\in\Pcal_{k}}\frac{1}{1-p^{-s}}$ which is equal to $\prod_{\Pcal_{k}}\sum_{m=0}^{\infty}\frac{1}{(p^{m})^{s}}$. Moreover since the product and sum are absoltuely convergent, we can rewrite this as $\sum_{n\in S_{k}}\frac{1}{n^{s}}$ where $S$ is the set of natural numbers with all factors in $\Pcal_{k}$. Passing to the limit as $k\to\infty$, we get the claim. 
\end{proof}
\begin{remark}
    This already implies that there are infinitely many primes, since if not we would have a finite product equalling a divergent harmonic series, a contradiction. 
\end{remark}
We also show the following structure result for zeroes of the Riemann zeta function of real part 1. 
\begin{theorem}\label{thm: no zeroes at real part 1}
    The Riemann zeta function $\zeta(s)$ has no zeroes with $\RE(s)=1$.\marginpar{Theorem 3}
\end{theorem}
\begin{proof}
    Suppose to the contrary there is a zero $s$ with $\RE(s)=1$, say of the form $1+ti$. Consider the function 
    $$F(s)=\zeta(s)^{3}\zeta(s+it)^{4}\zeta(s+2it).$$
    Observe that if $\zeta(1+it)=0$ then at the pole of $\zeta(s)$ at $s=1$ is canceled by the zero of order four in the second factor. The function $F(s)$ thus vanishes at $s=1$ and is holomorphic in a neighborhood around that point. Passing to logarithms, we have that $\lim_{s\to 1}\log|F(s)|=-\infty$. 

    Now consider the product expression of \Cref{thm: zeta function by infinite products} where we compute 
    \begin{align*}
        \log|\zeta(s)| &= \RE\left(\sum_{p\text{ prime}}\log(1-p^{-s})^{-1}\right) \\
        &= \RE\left(\sum_{p\text{ prime}}\sum_{n=1}^{\infty}\frac{1}{n\cdot p^{ns}}\right) \\
        &= \RE\left(\sum_{n=1}^{\infty}\frac{a_{n}}{n^{s}}\right)
    \end{align*}
    where the coefficient $a_{n}$ is given by 
    $$a_{n}=\begin{cases}
        \frac{1}{r} & n = p^{r} \\
        0 & n\neq p^{r}
    \end{cases}$$
    where on substituting to our sum we have for $s>1$ real
    \begin{align*}
        \log|F(s)| &= 3\log|\zeta(s)| + 4\log|\zeta(s+ti)| + \log|\zeta(s+2ti)| \\
        &=  \sum_{n=1}^{\infty}a_{n}\RE\left(3n^{-s}+ 4n^{-s-ti}+n^{-s-2ti}\right) \\
        &= \sum_{n=1}^{\infty}\frac{a_{n}}{n^{s}}\left(3+4cos(t\cdot\log(n))+cos(2t\cdot log(n))\right)
    \end{align*}
    where we note that $\frac{a_{n}}{n^{s}}\geq0$ and 
    \begin{align*}
        3+4\cos(t\cdot\log(n))+\cos(2t\cdot \log(n))&=3+4\cos(t\cdot\log(n))+2\cos^{2}(t\cdot \log(n))-1\\
        &=2(1+\cos(t\cdot\log(n)))^{2}
    \end{align*}
    which is also positive. In particular $\lim_{s\to 1}\log|F(s)|\neq-\infty$, a contradiction.
\end{proof}
We now introduce some language relating to the prime number theorem. 
\begin{definition}[Prime Counting Function]\label{def: prime counting function}
    The prime counting function $\pi(x)$ is the number of primes at most $x$. 
\end{definition}
\begin{definition}[Theta Function]\label{def: theta function}
    The theta function is given by 
    $$\vartheta(x)=\sum_{p\text{ prime}, p\leq x}\log(p).$$
\end{definition}
The prime number theorem concerns the asymptotics of $\pi(x)$, though it is often easier to consider $\vartheta(x)$. These are linked by the following result. 
\begin{proposition}\label{prop: equivalence of PNT limits}
    The existence of the following limits are equivalent 
    \begin{enumerate}[label=(\alph*)]
        \item $\lim_{x\to\infty}\frac{\pi(x)\cdot\log(x)}{x}$
        \item $\lim_{x\to\infty}\frac{\vartheta(x)}{x}$
    \end{enumerate}
    and they are equal if they exist. 
\end{proposition}
\begin{proof}
    $\left(\frac{\vartheta(x)}{x}\leq\frac{\pi(x)\log(x)}{x}\right)$ Observe 
    \begin{align*}
        \vartheta(x)&=\sum_{p\text{ prime, } p\leq x}\log(p)\\
        &\leq \sum_{p\text{ prime, } p\leq x}\frac{\log(x)}{\log(p)}\cdot\log(p)\\
        &=\log(x)\sum_{p\text{ prime, } p\leq x}1=\pi(x)\log(x).
    \end{align*}

    $\left(\frac{\pi(x)\log(x)}{x}\leq\frac{\vartheta(x)}{x}\right)$ Conversely, we can estimate for $1<y<x$
    \begin{align*}
        \pi(x) &= \pi(y)+\sum_{p\text{ prime, } y<p\leq x}1 \\
        &\leq \pi(y)+\frac{1}{\log(y)}\sum_{p\text{ prime, } y<p\leq x}\log(p) \\
        &y+\frac{1}{\log(y)}\vartheta(x).
    \end{align*}
    We now have 
    $$\frac{\pi(x)\log(x)}{x}\leq\frac{y\log(x)}{x}+\frac{\log(x)}{\log(y)}\frac{\vartheta(x)}{x}.$$
    Taking $y=\frac{x}{(\log(x))^{2}}$ we yield 
    $$\frac{\pi(x)\log(x)}{x}\leq\frac{1}{\log(x)}+\frac{\log(x)}{\log(x)-2\log(\log(x))}\frac{\vartheta(x)}{x}$$
    where we have 
    \begin{align*}
        \lim_{x\to\infty}\frac{1}{\log(x)} &= 0 \\
        \lim_{x\to\infty}\frac{\log(x)}{\log(x)-2\log(\log(x))}&= 1
    \end{align*}
    giving the first claim. 

    The second claim follows immediately from the inequalities.
\end{proof}
\section{Lecture 8 -- 5th November 2024}\label{sec: lecture 8}
We continue with the discussion of tangent spaces. 
\begin{definition}[Tangent Bundle]\label{def: tangent bundle}
    Let $M$ be a smooth manifold. The tangent bundle $TM$ of $M$ is given by $\coprod_{p\in M}T_{p}M$. 
\end{definition}
\begin{remark}
    Elements of the tangent bundle are denoted by pairs $(p,v)\in M\times T_{p}M$ and there is a natural forgetful map $TM\to M$ by $(p,v)\mapsto p$. 
\end{remark}
\begin{proposition}\label{prop: tangent bundle is smooth manifold}
    Let $M$ be a smooth manifold. The tangent bundle can be endowed with a structure of a smooth manifold such that the map $TM\to M$ is smooth. 
\end{proposition}
\begin{remark}
    We will soon see that the data of a map of smooth manifolds with fiberes fector spaces forms a vector bundle. In fact, this map is sufficiently functorial and is a functor from the category of smooth manifolds to the category of vector bundles. 
\end{remark}
We now discuss submersions, immersions, and embeddings, which are special classes of smooth maps. 
\begin{definition}[Rank of Smooth Map]\label{def: rank of smooth map}
    Let $F:M\to N$ be a morphism of smooth manifolds. The rank of $F$ at $p$ is the rank of the linear map $dF_{p}:T_{p}M\to T_{F(p)}N$. 
\end{definition}
Smooth maps of full rank are particularly important. 
\begin{definition}[Submersion]\label{def: submersion}
    Let $M,N$ be smooth $m,n$ manifolds, respectively, and $F:M\to N$ a smooth map. $F$ is a submersion if $dF_{p}$ is surjective for all $p\in M$. 
\end{definition}
\begin{definition}[Immersion]\label{def: immersion}
    Let $M,N$ be smooth $m,n$ manifolds, respectively, and $F:M\to N$ a smooth map. $F$ is an immersion if $dF_{p}$ is injective for all $p\in M$.
\end{definition}
\begin{remark}
    A necessary but insufficient condition for a submersion is that $m\geq n$, and a necessary but insufficient condition for an immersion is that $m\leq n$.   
\end{remark}
To define immersions, we require the following lemma. 
\begin{lemma}\label{lem: subset of matrices of full rank is open}
    Let $m,n\in\NN$. The set of matrices of rank $\min\{m,n\}$ is open in $\mathrm{Mat}_{m\times n}(\RR)$. 
\end{lemma}
\begin{proof}
    Fix some full rank matrix $A\in\mathrm{Mat}_{m\times n}(\RR)$. Up to transposition, it suffices to consider the case $m\leq n$. Now note that if $A$ is of full rank, there is an invertible submatrix $A'$ of $A$ obtained by deleting $m-n$ columns from $A$. Consider the map $\mathrm{Mat}_{m\times n}(\RR)\to\mathrm{Mat}_{m\times m}(\RR)\to\RR$ by deleting the columns and taking the determinant, respectively. Observe that this map is continuous and that the image of $A$ is nonzero. As such, the preimage of any open neighborhood of the image of $A$ not intersecting zero gives an open neighborhood of $A$ in $\mathrm{Mat}_{m\times n}(\RR)$ of matrices of full rank, as desired. 
\end{proof}
We can show that immersions and submersions are local conditions. 
\begin{proposition}\label{prop: immersions and submersions are local}
    Let $M,N$ be smooth $m,n$ manifolds, respectively, and $F:M\to N$ a smooth map. Then:
    \begin{enumerate}[label=(\roman*)]
        \item If $dF_{p}$ is injective, there exists an open neighborhood of $p$ on which $dF_{(-)}$ is injective. 
        \item If $dF_{p}$ is surjective, there exists an open neighborhood of $p$ on which $dF_{(-)}$ is surjective.
    \end{enumerate}
\end{proposition}
\begin{proof}
    On passage to charts, we can reduce to the case of $M\subseteq\RR^{m},N\subseteq\RR^{n}$ where we note that $dF_{(-)}:M\to\mathrm{Mat}_{m\times n}(\RR)$ has image in the full rank matrices -- of full column rank in the case of injectivity and of full row rank in the case of surjectivity -- both of which are open conditions by \Cref{lem: subset of matrices of full rank is open} yielding the claim. 
\end{proof}
We can now define local diffeomorphisms. 
\begin{definition}[Local Diffeomorphism]\label{def: local diffeomorphism}
    Let $M,N$ be smooth $m,n$ manifolds, respectively, and $F:M\to N$ a smooth map. $F$ is a local diffeomorphism if it is both an immersion and a submersion. 
\end{definition}
The rank theorem will give a necessary and sufficient conditions for a smooth map to be a local diffeomorphism. 
\begin{example}
    Let $S^{1}\subseteq\CC$. The map $S^{1}\to S^{1}$ by $z\mapsto z^{2}$ is a local diffeomorphism but not a (global) diffeomorphism. 
\end{example}
We define embeddings as follows. 
\begin{definition}\label{def: embedding}
    Let $M,N$ be smooth $m,n$ manifolds, respectively, and $F:M\to N$ a smooth map. $F$ is an embedding if it is an immersion and a homeomorphism onto its image endowed with the subspace topology. 
\end{definition}
\begin{example}
    The inclusion $S^{1}\to\CC$ is an embedding. 
\end{example}
\begin{example}
    More generally, the inclusion $S^{n}\to\RR^{n+1}$ is an embedding. 
\end{example}
\begin{example}
    The map $\RR\to\TT^{2}$ the 2-torus by $t\mapsto (t,\alpha t)$ for $\alpha\in\RR\setminus\QQ$ is an immersion since the map on tangent spaces is injective, but not an embedding as it is not a homeomorphism onto its image with the subspace topology.
\end{example}
\section{Lecture 9 -- 20th December 2024}\label{sec: lecture 9}
We consider the notion of relative $K$-theory, a variant of $K$-theory in which $K$-theoretic classes are more easily described, and is as computable as $K$-theory. In this way, relative $K$-theory can be seen as more flexible than $K$-theory. 
\begin{definition}[Relative $K$-Theory]\label{def: relative K-theory}
    Let $R$ be a ring and $M$ an Abelian group with a map $M\to R^{\times}$. The relative $K$-theory 
    $$K(R/\ZZ[M])= K(R)\otimes_{\SSS[*/M]}\SSS$$
    where $K(R)$ is considered as a spectrum over $\SSS[*/M]$ by the map $*/M\to */R^{\times}\to R$. 
\end{definition}
\begin{remark}
    By the universal property of group rings, the existence of a map $M\to R^{\times}$ is equivalent to the existence of a map from the group ring $\ZZ[M]$ to $R$. 
\end{remark}
\begin{remark}
    One can generalize \Cref{def: relative K-theory} to the setting of $M$ a commutative monoid $M$ to $R$ which corresponds to a map $\ZZ[M]\to R$. One can then define 
    $$K(R/\ZZ[M])=K^{\log}(R,M)\otimes_{\SSS[*/M^{\mathsf{gp}}]}\SSS$$
    where $K^{\log}$ is logarithmic $K$-theory. 
\end{remark}
\begin{remark}
    The construction of \Cref{def: relative K-theory} above is equivalent to taking the homotopy orbits $K(R)_{h(*/M)}$ which in the case of $F$ a field produces $K(F)_{h(*/F^{\times})}=K(F/\ZZ[F^{\times}])$. 
\end{remark}
The upshot of this construction is that all constructions in $K$-theory generalize to the setting of relative $K$-theory. Of importance to us are regulators and polylogarithms. In particular, the dilogarithm is most naturally expressed in the setting of relative $K$-theory. 

Moreover, relative $K$-theory overcomes the rigidity of $K$-theory in high weights. For example, it is conjectured that the Bloch group of $\CC$ and of $\overline{\QQ}$ coincide. On the other hand, classes in relative $K$-theory is plentiful. As such, the regulators are now highly interesting special functions and not merely numbers and provides a coherent organizing principle for functions like dilogarithms. 

These constructions also arise in $p$-adic geometry. Let $R$ be a smooth algebra over $\CC_{p}$, the $p$-completion of the algebraic closure of $\QQ_{p}$. We often pick a system of coordinates $T_{1},\dots,T_{d}\in R$ invertible and consider an \'{e}tale map $\CC_{p}[T_{1}^{\pm},\dots,T_{d}^{\pm}]\to R$ and pass to
$$R_{\infty}=R\otimes_{\CC_{p}[T_{1}^{\pm},\dots,T_{d}^{\pm}]}\CC_{p}\left[T_{1}^{\pm\frac{1}{p^{\infty}}},\dots,T_{d}^{\pm\frac{1}{p^{\infty}}}\right]$$
which on further passage to an appropriate completion is a perfectoid algebra. This is similar to the data required for relative $K$-theory since the group generated by $T_{1}^{\pm},\dots,T_{d}^{\pm}$ naturally maps to the units of $R$. In this case, the \'{e}tale $p$-complete relative $K$-theory 
$$K_{\mathsf{\acute{e}t}}(R/\ZZ[T_{1}^{\pm},\dots,T_{d}^{\pm}])^{\wedge}_{p}$$
turns out to be the $p$-completed \'{e}tale $K$-theory $K_{\mathsf{\acute{e}t}}(R_{\infty})$. Morally, what happens here is the $p$-power roots vanish in the $K$-theory of the $p$-completion. See \cite{PrismaticCohDelta} for a further discussion on this topic, viewing syntomic cohomology as a form of $p$-adic $K$-theory. 

Going forward, we will take the perspective of polylogarithms being relative $K$-theory classes and functional equations of polylogarithms are already identities of relative $K$-theory. 

\begin{example}
    Let $R=\ZZ[t^{\pm},\frac{1}{1-t}]$ and $M=t^{\ZZ}$. There is a natural map $M\to R^{\times}$. We have 
    $$K(R/\ZZ[t^{\pm}])=K(R)\otimes_{\SSS[*/t^{\ZZ}]}\SSS$$
    inducing an exact triangle 
    $$K(R)\to K(R/\ZZ[t^{\pm}])\to K(R/\ZZ[t^{\pm}])[2]$$
    where the map $K(R/\ZZ[t^{\pm}])\to K(R/\ZZ[t^{\pm}])[2]$ can be thought of as a logarithmic $t$-derivative $\nabla^{\log}_{t}$ by the exact sequence of $\SSS[*/t^{\ZZ}]$-modules
    $$\SSS[*/t^{\ZZ}]\to\SSS\to\SSS[2].$$
    In particular, any $K$-theory class gives rise to a relative $K$-theory classes and $K$-theory classes can be recovered from those relative $K$-theory classes that vanish under the differential. 
\end{example}

Now recall the existence of a motivic filtration on $K$-theory \cite{MotivicFiltrationKTheory} and is still a topic of contemporary interest with T. Bouis' recent results in the mixed characteristic case \cite{BouisThesis}. Running this machinery on relative $K$-theory, this produces a filtration by relative motivic complexes 
$$\ZZ(n)\left(R/\ZZ[M]\right)$$
and relative motivic cohomology of a ring relative to a group algebra taking $\ZZ[M]\to R$ to $\ZZ(n)(R/\ZZ[M])$. Note that rationalized $K$-theory can be recovered from relative motivic cohomology $K(R/\ZZ[M])\cong\bigoplus_{n\geq0}\QQ(n)(R/\ZZ[M])[2n]$.

We have an exact triangle
$$\ZZ(n)(R)\to\ZZ(n)\left(R/\ZZ[t^{\pm}]\right)\to\ZZ(n-1)\left(R/\ZZ[t^{\pm}]\right)$$
which in weight $\leq 2$ computes $K$-theory in small degrees. Explicitly, we have $\ZZ(0)(R)=\ZZ$ and $\ZZ(1)(R)=R^{\times}[-1]$. By $\mathbb{A}^{1}$-invariance for regular rings, we have $K(\ZZ[t])\cong K(\ZZ)$ so we have $K(\ZZ[t^{\pm},\frac{1}{t-1}])\cong K(\ZZ)\oplus K(\ZZ)[1]\oplus K(\ZZ)[1]= K(\ZZ)\oplus K(\ZZ)[1]^{\oplus 2}$. In the relative setting, the exact triangle allow us to compute relative $K$-theory 
$$% https://q.uiver.app/#q=WzAsNCxbMCwwLCJcXFpaKDEpKFIpIl0sWzIsMCwiXFxaWigxKShSL1xcWlpbdF57XFxwbX1dKSJdLFs0LDAsIlxcWlooMCkoUi9cXFpaW3Ree1xccG19XSkiXSxbNiwwLCJSXntcXHRpbWVzfSJdLFswLDFdLFsxLDJdLFsyLDMsIjFcXG1hcHN0byB0Il1d
\begin{tikzcd}
	{\ZZ(1)(R)} && {\ZZ(1)(R/\ZZ[t^{\pm}])} && {\ZZ(0)(R/\ZZ[t^{\pm}])} && {R^{\times}}
	\arrow[from=1-1, to=1-3]
	\arrow[from=1-3, to=1-5]
	\arrow["{1\mapsto t}", from=1-5, to=1-7]
\end{tikzcd}$$
and where making the appropriate substitutions on rationalization gives 
$$0\longrightarrow \QQ(2)(R/\ZZ[t^{\pm}])\longrightarrow \QQ[-1]$$
so in fact there is an isomorphism $\QQ(2)(R/\ZZ[t^{\pm}])\to \QQ[-1]$ induced by the so-called universal dilogarithm that takes $\Li_{2}^{\mathrm{univ}}(t)$ to $(1-t)$ where we note that $\QQ[-1]$ is generated by $1-t$. 

We now want to observe that this so-called universal dilogarithm satisfies the expected functional equations. 
\begin{proposition}\label{prop: equality of the universal dilogarithm}
    There is an equality $\Li_{2}^{\mathrm{univ}}(t)=-\Li_{2}^{\mathrm{univ}}(1-t)$ in relative rational motivic cohomology 
    $$H^{1}\left(\QQ(2)\left(\ZZ\left[t^{\pm},\frac{1}{1-t}\right]/\ZZ[t^{\pm}, (1-t)^{\pm}]\right)\right).$$ 
\end{proposition}
\begin{proof}
    There is a Koszul-like complex computing $\ZZ(n)(R)$ as the limit of 
    $$% https://q.uiver.app/#q=WzAsMyxbMCwwLCJcXFpaKG4pXFxsZWZ0KFIvXFxaWlt0X3sxfV57XFxwbX0sdF97Mn1ee1xccG19XVxccmlnaHQpIl0sWzIsMCwiXFxaWihuLTEpXFxsZWZ0KFIvXFxaWlt0X3sxfV57XFxwbX0sdF97Mn1ee1xccG19XVxccmlnaHQpXntcXG9wbHVzIDJ9Il0sWzMsMCwiXFxaWihuLTIpXFxsZWZ0KFIvXFxaWlt0X3sxfV57XFxwbX0sdF97Mn1ee1xccG19XVxccmlnaHQpIl0sWzAsMSwiKFxcbmFibGFfe3RfezF9fV57XFxsb2d9LFxcbmFibGFfe3RfezJ9fV57XFxsb2d9KSJdLFsxLDJdXQ==
    \begin{tikzcd}
        {\ZZ(n)\left(R/\ZZ[t_{1}^{\pm},t_{2}^{\pm}]\right)} && {\ZZ(n-1)\left(R/\ZZ[t_{1}^{\pm},t_{2}^{\pm}]\right)^{\oplus 2}} & {\ZZ(n-2)\left(R/\ZZ[t_{1}^{\pm},t_{2}^{\pm}]\right)}
        \arrow["{(\nabla_{t_{1}}^{\log},\nabla_{t_{2}}^{\log})}", from=1-1, to=1-3]
        \arrow[from=1-3, to=1-4]
    \end{tikzcd}$$
    so noting that $\QQ(2)(R)\cong 0$ and 
    $$\QQ(i)\left(\ZZ\left[t^{\pm},\frac{1}{1-t}\right]/\ZZ[t^{\pm}, (1-t)^{\pm}]\right)$$
    vanishes for $i\in\{0,1\}$ we have that 
    $$\QQ(2)\left(\ZZ\left[t^{\pm},\frac{1}{1-t}\right]/\ZZ[t^{\pm}, (1-t)^{\pm}]\right)\cong\QQ[-1].$$
    So equality follows by considering the composite. 
\end{proof}
\begin{remark}
    In general if the dilogarithm on a ring $R$ the condition of $\sum_{i}\Li_{2}(f_{i}(t))$ being constant implies that $\sum_{i}f_{i}(t)\wedge (1-f_{i}(t))=0$ in $\bigwedge^{2}R^{\times}$. This holds for the five-term relation \Cref{eqn: BW dilogarithm five term relation}. 
\end{remark}
We now consider the Borel (complex) regulator on algebraic $K$-theory. There is a map 
$$K_{3}\left(\ZZ\left[t^{\pm},\frac{1}{1-t}\right]/\ZZ[t^{\pm}]\right)\to K_{3}(\CC/\ZZ[\CC^{\times}])\to K_{3}^{\cont}(\CC/\ZZ[\CC^{\times}])$$
where we define $K_{3}^{\cont}(\CC/\ZZ[\CC^{\times}])=\pi_{3}\left(K^{\cont}(\CC)\otimes_{\SSS[*/M]}\SSS\right)$ and which statisfies a functional equation. So for $t\in\CC\setminus\{0,1\}$, its image in $K_{3}^{\cont}(\CC/\ZZ[\CC^{\times}])$ satisfies a functional equation as well. 

Rationally, we can consider rationalized relative motivic complexes $\QQ(n)^{\cont}(\CC/\ZZ[\CC^{\times}])$ where in the case of $\QQ(2)^{\cont}(\CC/\ZZ[\CC^{\times}])$ we can compute using $\QQ(0)^{\cont}(\CC/\ZZ[\CC^{\times}])\cong\QQ, \QQ(1)^{\cont}(\CC/\ZZ[\CC^{\times}])\cong(\CC^{\times}/\CC^{\times})[-1]\cong0$ so the fiber sequence computes $\QQ(2)^{\cont}(\CC)$ as $(\CC/(2\pi i)^{2}\QQ)[-1]$ and is given by an extension $E$ fitting into the short exact sequence 
$$% https://q.uiver.app/#q=WzAsNSxbMCwwLCIwIl0sWzEsMCwiXFxDQy8oMlxccGkgaSleezJ9XFxRUSJdLFsyLDAsIkUiXSxbMywwLCJcXFFRXFxvdGltZXNfe1xcWlp9XFxiaWd3ZWRnZV57Mn1cXENDXntcXHRpbWVzfSJdLFs0LDAsIjAiXSxbMCwxXSxbMSwyXSxbMiwzXSxbMyw0XV0=
\begin{tikzcd}
	0 & {\CC/(2\pi i)^{2}\QQ} & E & {\QQ\otimes_{\ZZ}\bigwedge^{2}\CC^{\times}} & 0
	\arrow[from=1-1, to=1-2]
	\arrow[from=1-2, to=1-3]
	\arrow[from=1-3, to=1-4]
	\arrow[from=1-4, to=1-5]
\end{tikzcd}$$
where the map $\CC\setminus\{0,1\}\to E$ by $t\mapsto (t)\wedge(1-t)$ satisfies the five-term relation. This extends to a map $\QQ[\CC\setminus\{0,1\}]$ to $E$ which necessarily factors over the pre-Bloch group $\wp_{2}(\CC)$ of \Cref{def: pre-Bloch group} since the map satisfies the five-term relation. This in turn induces a map $B_{2}(\CC)\to \CC/(2\pi i)^{2}\QQ$ as expected. More explicitly, $E$ is obtained as the cokernel of $\CC\otimes_{\ZZ}\CC\to \CC/(2\pi i)^{2}\QQ\oplus\CC^{\times}\otimes_{\ZZ}\CC$ by $x\otimes y\mapsto (xy, \exp(x)\otimes y+\exp(y)\otimes x)$. 
\section{Lecture 10 -- 17th January 2025}\label{sec: lecture 10}
We continue our discussion of relative motivic cohomology. Just as classical motivic cohomology has realization maps that recover de Rham, Betti, \'{e}tale, and prismatic cohomology, so too does relative motivic cohomology. 

Observe that we can consider the ring of integers Nahm number field 
$$R=\ZZ[t_{1},\dots,t_{N},z_{1},\dots,z_{N}]/\left(1-z_{i}=(-1)^{A_{ii}}t_{i}z_{1}^{A_{i1}}\dots z_{N}^{A_{iN}}\right)$$
of \Cref{eqn: number field of Nahm equation} as a ring over $\ZZ[t_{1},\dots,t_{N}]$. Recall that for $R=\ZZ[t,\frac{1}{1-t}]$ as discussed in \Cref{sec: lecture 9}, we can define relative motivic cohomology $H^{i}(\ZZ(n)(R/\ZZ[t^{\pm}]))$ where there is a map 
$$H^{1}(\ZZ(2)(R/\ZZ[t^{\pm}]))\longrightarrow H^{i}(\ZZ(1)(R/\ZZ[t^{\pm}]))=R[1/t]^{\times}/t^{\ZZ}$$
induced by the logarithmic derivative taking $-\Li^{\mathrm{univ}}_{2}(t)$ to $[\frac{1}{1-t}]$. Under ``de Rham realization,'' the diloagarithm class goes to $-\log(1-t)$. This theory of realizations of motivic cohomology also explains the factor $\prod_{i=0}^{m-1}(1-\zeta_{m}t)^{i/m}$ at the expansion of Nahm sums at $m$th roots of unity \Cref{thm: Nahm sum asymptotics at roots of unity}.

For $R$ a ring, motivic cohomology admits Betti and \'{e}tale realizations
$$H^{i}(R,\ZZ(n))\to H^{i}_{\mathsf{sing}}(\spec(R)(\CC),\ZZ), H^{i}(R,\ZZ(n))\to H^{i}_{\mathsf{\acute{e}t}}(\spec(R[1/m]),\ZZ/m\ZZ(n)).$$
We can describe this explicitly in motivic weight $n=1$. This produces maps $R^{\times}$ to $\ZZ$-torsors to $\spec(R)(\CC)$ taking $f$ to a $\ZZ$-torsor of choices of $\log(f)$ in the Betti setting and $R^{\times}$ to $\mu_{m}$-torsors over $\spec(R)$. Analogous constructions can be made for relative motivic cohomology. In relative motivic cohomology, the same constructions yield maps
$$H^{i}(\ZZ(n)(R/\ZZ[t^{\pm}]))\to H^{i}_{\mathsf{sing}}\left(\spec(R)(\CC)\times_{(\CC^{\times})^{N}}\CC^{N}\right)$$
$$ H^{i}(\ZZ(n)(R/\ZZ[t^{\pm}]))\to H^{i}_{\mathsf{\acute{e}t}}\left(\spec(R[1/m, t_{1}^{1/m},\dots,t_{N}^{1/m}]),\ZZ/m\ZZ(n)\right)$$
where in the first case we the fibered product is taken over the map $\exp:\CC^{N}\to(\CC^{\times})^{N}$. Here, we observe that relative motivic cohomology still give \'{e}tale cohomology classes, but these only start to appear after extracting some roots of the coordinates. 

In this framework, it can be seen that the Betti realization of the diloagarithm is a well-defined function on $\CC\setminus(2\pi i)\ZZ$ with at most simple poles at $(2\pi i)\ZZ$, and residues $\pm(2\pi i)n$ at $2\pi i n$. In particular, the diloagarithm realizes to a $\ZZ$-local system on $\CC\setminus(2\pi i)\ZZ$ whose mononodromy around $2\pi i n$ is $n$. Similarly, the \'{e}tale realization produces precisely the product $\prod_{i=0}^{m-1}(1-\zeta_{m}t)^{i/m}$ alluded to above. This recovers a construction of Calegari-Garoufalidis-Zagier as well as the cyclic quantum dilogarithm. 
\section{Lecture 11 -- 15th November 2024}\label{sec: lecture 11}
We begin our discussion of transversality. 
\begin{definition}[Transverse Submanifolds at a Point]\label{def: transverse submanifolds at a point}
    Let $M$ be a smooth manifold and $S,S'$ submanifolds of $M$. $M$ and $M'$ are transverse at $p\in S\cap S'$ if the subspaces $T_{p}S$ and $T_{p}S'$ span $T_{p}M$. 
\end{definition}
\begin{definition}[Transverse Submanifolds]\label{def: transverse submanifolds}
    Let $M$ be a smooth manifold and $S,S'$ submanifolds of $M$. $S$ and $S'$ are transverse submanifolds $S$ -- $S\pitchfork S'$ -- if they are transverse at each point $p\in S\cap S'$. 
\end{definition}
\begin{example}
    The union of coordinate axes in $\RR^{2}$ is transverse. 
\end{example}
\begin{example}
    The generic intersection of a circle and a line is transverse. 
\end{example}
\begin{example}
    A line tangent to a circle is not transverse, as the tangent spaces at the intersection point is just the line, while the tangent space of $\RR^{2}$ at that point is $\RR^{2}$. 
\end{example}
Transverse intersections behave especially nicely, insofar as their intersections are smooth submanifolds. On the other hand, non-transverse intersections might not even be a topological manifold. 
\begin{example}\label{ex: graph of crossing lines}
    Let $f:\RR^{2}\to\RR$ by $f(x,y)=x^{2}-y^{2}$ and $g(x,y)=0$. Let $S$ be the graph of $f$ in $\RR^{3}$ given by $\{(x,y,z)\in\RR^{3}:z=f(x,y)\}$ and $S'$ the graph of $g$ in $\RR^{3}$ given by $\{(x,y,z)\in\RR^{3}:z=g(x,y)=0\}$. The intersection $S\cap S'$ is given by $\{(x,y,z)\in\RR^{3}:x^{2}-y^{2}=z=0\}$ which is two lines intersecting transversely in $\RR^{3}$ and not a topological manifold. 
\end{example}
We now show the lemma. 
\begin{lemma}\label{lem: transversality is a manifold}
    Let $M$ be a smooth manifold and $S,S'$ submanifolds of $M$. If $S,S'$ are transverse, then $S\cap S'$ is a smooth submanifold. 
\end{lemma}
\begin{proof}
    By composing with a chart centered around zero in $\RR^{m}$ and the slice \Cref{thm: slice}, it suffices to show that the intersection is a smooth submanifold ina  neighborhood of 0. Using the rank theorem \ref{thm: rank theorem}, after possibly shrinking $U$, that $S=f^{-1}(0)$ for $f:U\to\RR^{m-\dim(S)}$ and $S'=g^{-1}(0)$ for $g:U\to\RR^{m-\dim(S')}$ with $f,g$ of full rank. 

    Consider $H:U\to\RR^{m-\dim(S)}\oplus\RR^{m-\dim(S')}$ by $p\mapsto(f(p),g(p))$. It suffices to show that $H$ is surjective, where injectivity follows from $S,S'$ being submanifolds. We first observe that $H^{-1}(0)=f^{-1}(0)\cap g^{-1}(0)=S\cap S'$. To see the surjectivity of $H$ at the origin, note that $T_{0}S+ T_{0}S'\to T_{0}U$ is a linear isomorphism since $S,S'$ are transverse and that there is a map $dH_{0}:T_{0}U\to\RR^{m-\dim(S)}\oplus\RR^{m-\dim(S')}$ fitting into 
    $$% https://q.uiver.app/#q=WzAsNCxbMCwwLCJUX3swfVNcXG9wbHVzIFRfezB9UyciXSxbMCwxLCJUX3swfVUiXSxbMiwwLCJUX3swfVMvKFRfezB9U1xcY2FwIFRfezB9UycpXFxvcGx1cyBUX3swfVMnLyhUX3swfVNcXGNhcCBUX3swfVMnKSJdLFsyLDEsIlxcUlJee20tXFxkaW0oUyl9XFxvcGx1c1xcUlJee20tXFxkaW0oUycpfSJdLFsxLDNdLFsyLDMsIihkZl97MH0sZGdfezB9KSJdLFswLDJdLFswLDEsIlxcd3IiLDJdXQ==
    \begin{tikzcd}
        {T_{0}S+ T_{0}S'} && {T_{0}S/(T_{0}S\cap T_{0}S')\oplus T_{0}S'/(T_{0}S\cap T_{0}S')} \\
        {T_{0}U} && {\RR^{m-\dim(S)}\oplus\RR^{m-\dim(S')}}
        \arrow[from=1-1, to=1-3]
        \arrow["\wr"', from=1-1, to=2-1]
        \arrow["{(df_{0},dg_{0})}", from=1-3, to=2-3]
        \arrow[from=2-1, to=2-3]
    \end{tikzcd}$$
    The top horizontal map is well-defined as if $v+w=v'+w'$ then $v-v'=w-w'$ is in $T_{0}S\cap T_{0}S'$. It remains to show the right vertical map is surjective, we show in particular it is an isomorphism. Observe the map is injective as as the kernels of $df_{0},dg_{0}$ in $T_{0}S,T_{0}S'$ is exactly the intersection. We then consider the short exact sequence 
    $$% https://q.uiver.app/#q=WzAsNSxbMCwwLCIwIl0sWzEsMCwiVF97MH1TXFxjYXAgVF97MH1TJyJdLFsyLDAsIlRfezB9U1xcb3BsdXMgVF97MH1TJyJdLFszLDAsIlRfezB9VSJdLFs0LDAsIjAiXSxbMCwxXSxbMSwyXSxbMiwzXSxbMyw0XV0=
    \begin{tikzcd}
        0 & {T_{0}S\cap T_{0}S'} & {T_{0}S\oplus T_{0}S'} & {T_{0}U} & 0
        \arrow[from=1-1, to=1-2]
        \arrow[from=1-2, to=1-3]
        \arrow[from=1-3, to=1-4]
        \arrow[from=1-4, to=1-5]
    \end{tikzcd}$$
    where the maps are $v\mapsto (v,v)$ and $(a,b)\mapsto a-b$. Computing the dimensions, $\dim(T_{0}S\cap T_{0}S')+m=\dim(S)+\dim(S')$ and thus $\dim(T_{0}S/(T_{0}S\cap T_{0}S'))=\dim(S')-(\dim(S)+\dim(S')-m)=m-\dim(S)$ and similarly $\dim(T_{0}S'/(T_{0}S\cap T_{0}S'))=\dim(S)-(\dim(S)+\dim(S')-m)=m-\dim(S')$ showing that the final map is surjective. The claim follows. 
\end{proof}
We can generalize transversality of manifolds to transversality of maps. This is motivated by the goal of producing fibered products in the category of smooth manifolds. 
\begin{definition}[Fibered Product]\label{def: fibered product}
    Let $f:X\to Z, g:Y\to Z$ be continuous maps of topological spaces. The fibered product $X\times_{Z}Y$ is given by 
    $$\{(x,y)\in X\times Y:f(x)=g(y)\in\ZZ\}\subseteq X\times Y$$
    with the subspace topology on the product. 
\end{definition}
\begin{remark}
    \Cref{def: fibered product} implies the universal property of fibered products. For all topological spaces $W$ making the solid diagram commute 
    $$% https://q.uiver.app/#q=WzAsNSxbMSwxLCJYXFx0aW1lc197Wn1ZIl0sWzEsMiwiWSJdLFszLDEsIlgiXSxbMywyLCJaIl0sWzAsMCwiVyJdLFsyLDMsImYiXSxbMSwzLCJnIiwyXSxbMCwxXSxbMCwyXSxbNCwxLCIiLDAseyJvZmZzZXQiOjEsImN1cnZlIjoyfV0sWzQsMiwiIiwwLHsiY3VydmUiOi0xfV0sWzQsMCwiXFxleGlzdHMhIiwxLHsic3R5bGUiOnsiYm9keSI6eyJuYW1lIjoiZGFzaGVkIn19fV1d
    \begin{tikzcd}
        W \\
        & {X\times_{Z}Y} && X \\
        & Y && Z
        \arrow["{\exists!}"{description}, dashed, from=1-1, to=2-2]
        \arrow[curve={height=-6pt}, from=1-1, to=2-4]
        \arrow[shift right, curve={height=12pt}, from=1-1, to=3-2]
        \arrow[from=2-2, to=2-4]
        \arrow[from=2-2, to=3-2]
        \arrow["f", from=2-4, to=3-4]
        \arrow["g"', from=3-2, to=3-4]
    \end{tikzcd}$$
    there is a unique map $W\to X\times_{Z}Y$ making the entire diagram commute. 
\end{remark}
Note that neither the categories $\Mfld$ of \Cref{def: category of topological manifolds} nor $\SmMfld$ of \Cref{def: category of smooth manifolds} admit fibered products. 
\begin{example}
    In the setup of \Cref{ex: graph of crossing lines}, the fibered product $S\times_{\RR^{3}}S'$ is the union of transversely intersecting lines in $\RR^{3}$ which is not a topological manifold even though $S,S'$ are topological and even smooth manifolds (cf. \Cref{ex: graphs are mflds}).
\end{example}
Extending \Cref{def: transverse submanifolds at a point,def: transverse submanifolds}, we can define transverse maps as follows. 
\begin{definition}[Transverse Maps at a Point]\label{def: transverse maps at a point}
    Let $M_{1},M_{2},N$ be smooth manifolds and $F:M_{1}\to N,G:M_{2}\to N$ be smooth maps. $F$ and $G$ are transverse at $F(p_{1})=F(p_{2})\in N$ if $\mathrm{im}(dF_{p_{1}})+\mathrm{im}(dG_{p_{2}})$ spans $T_{q}N$. 
\end{definition}
\begin{definition}[Transverse Maps]\label{def: transverse maps}
    Let $M_{1},M_{2},N$ be smooth manifolds and $F:M_{1}\to N,G:M_{2}\to N$ be smooth maps. $F$ and $G$ are transverse -- $F\pitchfork G$ -- if it is transverse at all $q\in Z$. 
\end{definition}
\begin{remark}
    Transversality of maps generalizes transversality of manifolds by taking $F:S\to M, G:S'\to M$ for smooth submanifolds $S,S'\subseteq M$. 
\end{remark}
\section{Lecture 12 -- 18th November 2024}\label{sec: lecture 12}
We continue with some formal properties of fibered products. 
\begin{proposition}\label{prop: triple fibered products are associative}
    Let $X\to S, Y\to S, W\to T, Y\to T$ be morphisms in a category admitting fibered products. Then there is a unique isomorphism $(X\times_{S}Y)\times_{T}W\cong X\times_{S}(Y\times_{T}W)$. 
\end{proposition}
\begin{proof}
    The ? of the diagram 
    $$% https://q.uiver.app/#q=WzAsOCxbMCwwLCI/Il0sWzAsMSwiWFxcdGltZXNfe1N9WSJdLFsyLDAsIllcXHRpbWVzX3tUfVciXSxbMiwxLCJZIl0sWzQsMCwiVyJdLFs0LDEsIlQiXSxbMiwyLCJTIl0sWzAsMiwiWCJdLFszLDVdLFsxLDNdLFszLDZdLFsxLDddLFs3LDZdLFs0LDVdLFsyLDRdLFsyLDNdLFswLDJdLFswLDFdXQ==
    \begin{tikzcd}
        {?} && {Y\times_{T}W} && W \\
        {X\times_{S}Y} && Y && T \\
        X && S
        \arrow[from=1-1, to=1-3]
        \arrow[from=1-1, to=2-1]
        \arrow[from=1-3, to=1-5]
        \arrow[from=1-3, to=2-3]
        \arrow[from=1-5, to=2-5]
        \arrow[from=2-1, to=2-3]
        \arrow[from=2-1, to=3-1]
        \arrow[from=2-3, to=2-5]
        \arrow[from=2-3, to=3-3]
        \arrow[from=3-1, to=3-3]
    \end{tikzcd}$$
    with all squares cartesian can be filled by $X\times_{S}(Y\times_{T}W)$ and $(X\times_{S}Y)\times_{T}W$ by considering the vertical and horizontal rectangles, respectively. But in any such diagram, the rectangles are Cartesian as well so both of the objects above satisfy the same universal property and are thus isomorphic. 
\end{proof}
We now show fibered products exist in general. We do this in a sequence of lemmas.\marginpar{We follow the presentation of \cite[\href{https://stacks.math.columbia.edu/tag/01JO}{Tag 01JO}]{stacks-project} in place of \cite[Thm. II.3.3]{Hartshorne} presented in class.}
\begin{lemma}
    Let $f:X\to S,g:Y\to S$ be morphisms of schemes and suppose $X\times_{S}Y$ exists. If $U\subseteq S, V\subseteq X, W\subseteq Y$ are open such that $f(V)\subseteq U$ and $g(W)\subseteq U$ then there is a unique morphism $V\times_{U}W\to X\times_{S}Y$ and $V\times_{U}W\subseteq X\times_{S}Y$ is open. 
\end{lemma}
\begin{proof}
    For any other scheme $Z$ admitting maps to $V,W$ whose compatible with $f|_{V},g|_{W}$, there is a unique morphism $Z\to X\times_{S}Y$ which is contained in the open subscheme $\pr_{X}^{-1}(V)\cap\pr_{Y}^{-1}(W)$ of $X\times_{S}Y$, giving an identification $V\times_{U}W\cong\pr_{X}^{-1}(V)\cap\pr_{Y}^{-1}(W)$ by the uniqueness of fibered products. 

    Uniqueness of the map follows from the diagram 
    $$% https://q.uiver.app/#q=WzAsOCxbMywxLCJYIl0sWzEsMiwiWSJdLFszLDIsIlMiXSxbMSwxLCJYXFx0aW1lc197U31ZIl0sWzQsMywiVSJdLFs0LDAsIlYiXSxbMCwzLCJXIl0sWzAsMCwiVlxcdGltZXNfe1V9VyJdLFszLDBdLFswLDJdLFszLDFdLFsxLDJdLFs1LDRdLFs0LDJdLFs1LDBdLFs2LDFdLFs2LDRdLFs3LDMsIlxcZXhpc3RzISIsMSx7InN0eWxlIjp7ImJvZHkiOnsibmFtZSI6ImRhc2hlZCJ9fX1dLFs3LDZdLFs3LDVdXQ==
    \begin{tikzcd}
        {V\times_{U}W} &&&& V \\
        & {X\times_{S}Y} && X \\
        & Y && S \\
        W &&&& U.
        \arrow[from=1-1, to=1-5]
        \arrow["{\exists!}"{description}, dashed, from=1-1, to=2-2]
        \arrow[from=1-1, to=4-1]
        \arrow[from=1-5, to=2-4]
        \arrow[from=1-5, to=4-5]
        \arrow[from=2-2, to=2-4]
        \arrow[from=2-2, to=3-2]
        \arrow[from=2-4, to=3-4]
        \arrow[from=3-2, to=3-4]
        \arrow[from=4-1, to=3-2]
        \arrow[from=4-1, to=4-5]
        \arrow[from=4-5, to=3-4]
    \end{tikzcd}$$
\end{proof}
\begin{proposition}\label{prop: fibered products exist}
    Let $X\to S, Y\to S$ be morphisms of schemes. Then the fibered product $X\times_{S}Y$ exists in the category of schemes. 
\end{proposition}
\begin{proof}
    Let $\{U_{i}\}_{i\in I}$ be an affine open cover of $S$, $\{V_{j}\}_{j\in J_{i}}$ an affine open cover of $f^{-1}(U_{i})$ for each $i$, and $\{W_{k}\}_{k\in K_{i}}$ an affine open cover of $g^{-1}(U_{i})$ for each $i$. By \Cref{prop: fibered products of affine schemes}, each $V_{j}\times_{U_{i}}W_{k}$ is affine and satisfies the universal property of the fibered product for morphisms factoring through $V_{j},W_{k}$ that agree on $U_{i}$ and any scheme $Z$ admitting maps to $X,Y$ that agree on $S$ is given by the data of maps to each such $V_{j}\times_{U_{i}}W_{k}$. Moreover these schemes satisfy the hypothesis of \Cref{prop: gluing schemes} so these schemes glue to give the fibered product $X\times_{S}Y$ which satisfies the expected universal property. 
\end{proof}
Note that fibered products of schemes can behave unexpectedly. 
\begin{example}
    Let $X=\spec(k[x_{1}]),Y=\spec(k[x_{2}]),S=\spec(k)$ for $k$ algebraically closed. $X\times_{S}Y=\spec(k[x_{1}]\otimes_{k}k[x_{2}])=\spec(k[x_{1},x_{2}])=\A^{2}_{k}$ but the underlying topological space $|\A^{2}_{k}|$ is not $|\A^{1}_{k}|\times|\A^{1}_{k}|$ -- the latter has points given by pairs of prime ideals in $k[x_{1}]\oplus k[x_{2}]$ but $(x_{1}-x_{2})$ is a point of $\A^{2}_{k}$ not of this form. 
\end{example}
\begin{example}
    Let $X=Y=\spec(\CC)$ and $S=\spec(\RR)$. $X\times_{S}Y=\spec(\RR[x]/(x^{2}-1)\otimes_{\RR}\CC)=\spec(\CC[x]/(x^{2}-1))$ which consists of two points given by the maximal ideals $(x-i),(x+i)$. But the product $|X|\times|Y|$ is just one point, so there is not even a natural map $|X\times Y|\to|X\times_{S}Y|$. 
\end{example}
One place fibered products are ubiquitous is in the computation of fibers of a morphism. 
\begin{definition}[Fiber]\label{def: fiber of morphism}
    Let $f:X\to Y$ be a morphism of schemes and $y\in Y$ with residue field $\kappa(y)$. The fiber $X_{y}$ of $f$ over $Y$ is the fibered product $X\times_{Y}\spec(\kappa(y))$. 
\end{definition}
More explicitly, the fiber is induced by the following diagram.
$$% https://q.uiver.app/#q=WzAsNCxbMCwwLCJYX3t5fT1YXFx0aW1lc197WX1cXHNwZWMoXFxrYXBwYSh5KSkiXSxbMCwxLCJcXHNwZWMoXFxrYXBwYSh5KSkiXSxbMiwwLCJYIl0sWzIsMSwiWSJdLFswLDJdLFsyLDNdLFsxLDNdLFswLDFdXQ==
\begin{tikzcd}
	{X_{y}=X\times_{Y}\spec(\kappa(y))} && X \\
	{\spec(\kappa(y))} && Y
	\arrow[from=1-1, to=1-3]
	\arrow[from=1-1, to=2-1]
	\arrow[from=1-3, to=2-3]
	\arrow[from=2-1, to=2-3]
\end{tikzcd}$$
In the case of $Y$ having a generic point, we can construct generic and closed fibers. 
\begin{definition}[Generic Fiber]\label{def: generic fiber}
    Let $f:X\to Y$ be a morphism of schemes and $\eta\in Y$ the unique generic point of $Y$ with residue field $\kappa(\eta)$. The generic fiber $X_{\eta}$ of $f$ over $Y$ is the fibered product $X\times_{Y}\spec(\kappa(\eta))$.
\end{definition}
\begin{definition}[Closed Fiber]\label{def: closed fiber}
    Let $f:X\to Y$ be a morphism of schemes and $y\in Y$ a closed point with residue field $\kappa(y)$. The closed fiber $X_{y}$ of $f$ over $Y$ is the fibered product $X\times_{Y}\spec(\kappa(y))$.
\end{definition}
\begin{example}
    Let $A$ be a discrete valuation ring. Then $\spec(A)=\{\eta,\pi\}$ where $\eta$ is the generic point and $\pi$ the prime ideal corresponding to the uniformizer. A scheme $X$ over $\spec(A)$ has two fibers: the generic fiber $X_{\eta}$ and the closed fiber $X_{\pi}$. 
\end{example}
Fibered products are also a key tool in working in Grothendieck's ``relative point of view.''
\begin{definition}[Category of $S$-Schemes]\label{def: category of S-schemes}
    The category of $S$-schemes $\Sch_{S}$ has objects morphisms of schemes $X\to S$ and morphisms commutative diagrams 
    $$% https://q.uiver.app/#q=WzAsMyxbMCwwLCJYIl0sWzIsMCwiWSJdLFsxLDEsIlMiXSxbMCwyXSxbMSwyXSxbMCwxXV0=
    \begin{tikzcd}
        X && Y \\
        & S.
        \arrow[from=1-1, to=1-3]
        \arrow[from=1-1, to=2-2]
        \arrow[from=1-3, to=2-2]
    \end{tikzcd}$$
\end{definition}
\begin{definition}[Category of $k$-Schemes]\label{def: category of k-schemes}
    Let $k$ be field. The category of $k$-schemes $\Sch_{k}$ has objects morphisms of schemes $X\to \spec(k)$ and morphisms commutative diagrams 
    $$% https://q.uiver.app/#q=WzAsMyxbMCwwLCJYIl0sWzIsMCwiWSJdLFsxLDEsIlxcc3BlYyhrKSJdLFswLDJdLFsxLDJdLFswLDFdXQ==
    \begin{tikzcd}
        X && Y \\
        & {\spec(k).}
        \arrow[from=1-1, to=1-3]
        \arrow[from=1-1, to=2-2]
        \arrow[from=1-3, to=2-2]
    \end{tikzcd}$$
\end{definition}
\begin{remark}
    When working in the setting of $k$-schemes and considering the fibered product of $X\to\spec(k),Y\to\spec(k)$ we will write $X\times_{k}Y$ in place of $X\times_{\spec(k)}Y$. 
\end{remark}
The fibered product gives us a way to ``extend scalars'' on schemes defined over a field. 
\begin{definition}[Base Change]\label{def: base change}
    Let $k$ be a field, $\Sch_{k}$ the category of $k$-schemes, and $L/k$ a field extension. The base change functor $(-)_{L}:\Sch_{k}\to\Sch_{L}$ is given by $X\mapsto X_{L}=X\times_{k}\spec(L)$ and morphisms those induced morphisms of $L$-schemes. 
\end{definition}
More precisely, for a morphism $X\to Y$ of $k$-schemes, we have a diagram 
$$% https://q.uiver.app/#q=WzAsNixbNSwyLCJZIl0sWzMsMiwiWCJdLFs0LDMsIlxcc3BlYyhrKSJdLFswLDAsIlhfe0x9Il0sWzIsMCwiWV97TH0iXSxbMSwxLCJcXHNwZWMoTCkiXSxbMywxXSxbNCwwXSxbNSwyXSxbMSwwXSxbMSwyXSxbMCwyXSxbMyw1XSxbNCw1XV0=
\begin{tikzcd}
	{X_{L}} && {Y_{L}} \\
	& {\spec(L)} \\
	&&& X && Y \\
	&&&& {\spec(k)}
	\arrow[from=1-1, to=2-2]
	\arrow[from=1-1, to=3-4]
	\arrow[from=1-3, to=2-2]
	\arrow[from=1-3, to=3-6]
	\arrow[from=2-2, to=4-5]
	\arrow[from=3-4, to=3-6]
	\arrow[from=3-4, to=4-5]
	\arrow[from=3-6, to=4-5]
\end{tikzcd}$$
with both rectangles Cartesian so there is a unique map $X_{L}\to Y_{L}$ making the diagram 
$$% https://q.uiver.app/#q=WzAsNixbMywyLCJcXHNwZWMoaykiXSxbMSwyLCJcXHNwZWMoTCkiXSxbMywxLCJZIl0sWzEsMSwiWV97TH0iXSxbMSwwLCJYIl0sWzAsMCwiWF97TH0iXSxbMSwwXSxbNSwxXSxbNSw0XSxbNCwyXSxbMiwwXSxbMywyXSxbMywxXSxbNSwzLCJcXGV4aXN0cyEiLDEseyJzdHlsZSI6eyJib2R5Ijp7Im5hbWUiOiJkYXNoZWQifX19XV0=
\begin{tikzcd}
	{X_{L}} & X \\
	& {Y_{L}} && Y \\
	& {\spec(L)} && {\spec(k)}
	\arrow[from=1-1, to=1-2]
	\arrow["{\exists!}"{description}, dashed, from=1-1, to=2-2]
	\arrow[from=1-1, to=3-2]
	\arrow[from=1-2, to=2-4]
	\arrow[from=2-2, to=2-4]
	\arrow[from=2-2, to=3-2]
	\arrow[from=2-4, to=3-4]
	\arrow[from=3-2, to=3-4]
\end{tikzcd}$$
commute. 

These constructions are especially important in arithmetic applications. 
\begin{definition}[Rational Points]\label{def: rational points}
    Let $X$ be a $k$-scheme and $L/k$ a field extension. The set of $L$-rational points of $X$ is the set $X(L)=\Mor_{\Sch_{k}}(\spec(L),X)$. 
\end{definition}
One can easily show that this is invariant under base change of the scheme to $L$. 
\begin{lemma}\label{lem: bijection of rational points}
    Let $X$ be a $k$-scheme and $L/k$ a field extension. Then $X(L)=X_{L}(L)$ as sets. 
\end{lemma}
\begin{proof}
    Any morphism $\spec(L)\to X$ factors over a morphism to $X_{L}$
    $$% https://q.uiver.app/#q=WzAsNSxbMSwyLCJcXHNwZWMoTCkiXSxbMywyLCJcXHNwZWMoaykiXSxbMywxLCJYIl0sWzEsMSwiWF97TH0iXSxbMCwwLCJcXHNwZWMoTCkiXSxbNCwyLCIiLDEseyJjdXJ2ZSI6LTF9XSxbNCwwLCJcXGlkX3tcXHNwZWMoTCl9IiwyLHsiY3VydmUiOjF9XSxbMywwXSxbMywyXSxbMiwxXSxbMCwxXSxbNCwzLCJcXGV4aXN0cyEiLDEseyJzdHlsZSI6eyJib2R5Ijp7Im5hbWUiOiJkYXNoZWQifX19XV0=
    \begin{tikzcd}
        {\spec(L)} \\
        & {X_{L}} && X \\
        & {\spec(L)} && {\spec(k)}
        \arrow["{\exists!}"{description}, dashed, from=1-1, to=2-2]
        \arrow[curve={height=-6pt}, from=1-1, to=2-4]
        \arrow["{\id_{\spec(L)}}"', curve={height=6pt}, from=1-1, to=3-2]
        \arrow[from=2-2, to=2-4]
        \arrow[from=2-2, to=3-2]
        \arrow[from=2-4, to=3-4]
        \arrow[from=3-2, to=3-4]
    \end{tikzcd}$$
    giving the claim. 
\end{proof}
The most ``absolute'' form of base change is the geometric fiber. 
\begin{definition}[Geometric Fiber]\label{def: geometric fiber}
    Let $f:X\to Y$ be a morphism of schemes,$y\in Y$ with residue field $\kappa(y)$, and $\overline{\kappa(y)}$ a choice of algebraic closure of $\kappa(y)$. The geometric fiber $X_{\overline{y}}$ is defined to be $X\times_{Y}\spec(\overline{\kappa(y)})$. 
\end{definition}
\begin{remark}
    The topology may change under passage to the geometric fiber. This often has better topological behavior as the underling topological spaces of schemes over algebraically closed fields $k$ are often identical to the set of $k$-rational points. 
\end{remark}
\begin{example}
    Let $X$ be a scheme over $\spec(\ZZ_{(p)})$, the localization of $\ZZ$ at the prime ideal $(p)$. $\ZZ_{(p)}$ consists of two points $\{\eta,\mfrak\}$ corresponding to the generic and maximal ideal. The generic fiber $X_{\eta}$ is a scheme over $\spec(\QQ)$ and the closed fiber $X_{\mfrak}$ is a scheme over $\spec(\FF_{p})$. On the other hand, the generic and closed fibers $X_{\overline{\eta}},X_{\overline{\mfrak}}$ are schemes over $\spec(\overline{\QQ}),\spec(\overline{\FF_{p}})$, respectively. 
\end{example}
Returning to a discussion of arithmetic, we consider conjugate $k$-schemes. 
\begin{definition}[Conjugate $k$-Schemes]\label{def: conjugate k-schemes}
    Let $k$ be a field, $X$ a $k$-scheme, and $\sigma\in\Aut(k)$. The conjugate $k$-scheme is defined as the fibered product 
    $$% https://q.uiver.app/#q=WzAsNCxbMCwwLCJYX3tcXHNpZ21hfSJdLFsyLDAsIlgiXSxbMiwxLCJcXHNwZWMoaykiXSxbMCwxLCJcXHNwZWMoaykiXSxbMywyLCJcXHNpZ21hIl0sWzEsMl0sWzAsMV0sWzAsM11d
    \begin{tikzcd}
        {X_{\sigma}} && X \\
        {\spec(k)} && {\spec(k).}
        \arrow[from=1-1, to=1-3]
        \arrow[from=1-1, to=2-1]
        \arrow[from=1-3, to=2-3]
        \arrow["\sigma", from=2-1, to=2-3]
    \end{tikzcd}$$
\end{definition}
Note that $X_{\sigma}\to X$ is an isomorphism of abstract schemes, but not necessarily as $k$-schemes, since $X_{\sigma}$ has a different structure map that commutes with the structure map of $X$ up to $\sigma$. 
\begin{example}\label{ex: non-isomorphic as schemes over the base}
    Note that $\Aut(\RR)$ is the trivial group. Consider the $\spec(\QQ(\sqrt{2}))$ scheme $X=\spec(\RR)$ induced by the inclusion $\QQ(\sqrt{2})\hookrightarrow\RR$. Let $\sigma\in\Gal(\QQ(\sqrt{2})/\QQ)$ be the non-trivial element of the Galois group of the quadratic extension. We can consider $X_{\sigma}$ obtained as the fibered product 
    $$% https://q.uiver.app/#q=WzAsNCxbMCwwLCJYX3tcXHNpZ21hfSJdLFsyLDAsIlgiXSxbMiwxLCJcXHNwZWMoXFxRUShcXHNxcnR7Mn0pKSJdLFswLDEsIlxcc3BlYyhcXFFRKFxcc3FydHsyfSkpIl0sWzMsMiwiXFxzaWdtYSIsMl0sWzEsMl0sWzAsM10sWzAsMV1d
    \begin{tikzcd}
        {X_{\sigma}} && X \\
        {\spec(\QQ(\sqrt{2}))} && {\spec(\QQ(\sqrt{2})).}
        \arrow[from=1-1, to=1-3]
        \arrow[from=1-1, to=2-1]
        \arrow[from=1-3, to=2-3]
        \arrow["\sigma"', from=2-1, to=2-3]
    \end{tikzcd}$$
    If $X$ and $X_{\sigma}$ are isomorphic as $\QQ(\sqrt{2})$-schemes, there would be an automorphism of $\RR$ by $\sqrt{2}\mapsto-\sqrt{2}$, a contradiction. 
\end{example}
\section{Lecture 13 -- 22nd November 2024}\label{sec: lecture 13}
Following \Cref{ex: non-isomorphic as schemes over the base}, we consider the following. 
\begin{example}
    Note that $\pi,e^{\pi}$ are algebraically independent -- there does not exist a polynomial $F\in\QQ[x_{1},x_{2}]$ where $F(\pi,e^{\pi})=0$. Let $X=\PP^{1}_{\CC}\setminus\{0,1,\infty,\pi\}$ considered as a $\CC$-scheme. Let $\sigma\in\Aut(\QQ(\pi,e^{\pi}))$ be the element of the automorphism group $\pi\mapsto e^{\pi},e^{\pi}\mapsto\pi$ considered as an automorphism of $\CC$. We can compute $X_{\sigma}=\PP^{1}_{\CC}\setminus\{0,1,\infty,e^{\pi}\}$. Any isomorphism of $X_{\sigma}$ and $X$ as a $\CC$-scheme would be induced by a linear fractional transformation defining an automorphism of $\PP^{1}_{\CC}$, but this would produce an algebraic dependence between $\pi,e^{\pi}$, a contradiction. 
\end{example}
For those more arithmetically minded, we can use the language of rational points of \Cref{def: rational points} to understand Galois actions on schemes. 
\begin{proposition}\label{prop: map from Galois group to automorphisms of scheme}
    Let $X$ be a $k$-scheme and $K/k$ a Galois extension with Galois group $G$. There exists a group homomorphism $G\to\Aut_{k}(X_{K})$. 
\end{proposition}
\begin{proof}
    Consider the diagram 
    $$% https://q.uiver.app/#q=WzAsNixbMywxLCJYIl0sWzMsMiwiXFxzcGVjKGspIl0sWzEsMiwiXFxzcGVjKEspIl0sWzEsMSwiWF97S30iXSxbMCwwLCJYX3tLfSJdLFswLDIsIlxcc3BlYyhLKSJdLFs1LDIsIlxcc2lnbWEiLDJdLFsyLDFdLFswLDFdLFszLDJdLFszLDBdLFs0LDBdLFs0LDVdLFs0LDMsIlxcZXhpc3RzISIsMSx7InN0eWxlIjp7ImJvZHkiOnsibmFtZSI6ImRhc2hlZCJ9fX1dXQ==
    \begin{tikzcd}
        {X_{K}} \\
        & {X_{K}} && X \\
        {\spec(K)} & {\spec(K)} && {\spec(k)}
        \arrow["{\exists!}"{description}, dashed, from=1-1, to=2-2]
        \arrow[from=1-1, to=2-4]
        \arrow[from=1-1, to=3-1]
        \arrow[from=2-2, to=2-4]
        \arrow[from=2-2, to=3-2]
        \arrow[from=2-4, to=3-4]
        \arrow["\sigma"', from=3-1, to=3-2]
        \arrow[from=3-2, to=3-4]
    \end{tikzcd}$$
    where the $X_{K}$ of the upper-left corner is obtained by the fibered product over the $\spec(K)$ of the lower-left corner. The diagram commutes since $\sigma|_{k}$ is the identity and the unique morphism induced by the universal property of the fibered product is an isomorphism of $X_{K}$ to itself as $k$-schemes as its structure as a $k$-scheme is unchanged under $\sigma$. 
\end{proof}
Such methods were used by Weil to understand $\FF_{q}$-points of algebraic varieties by considering fixed points of the action of $\Gal(\overline{\FF_{q}}/\FF_{q})$ on $X(\overline{\FF_{q}})$. 

The language of fibered products also allow us to describe graphs and diagonals, the latter of which will play a key role in definitions of properties of schemes. 
\begin{definition}[Graph Morphism]\label{def: graph morphism}
    Let $f:X\to Y$ be a morphism of $S$-schemes. The graph of $f$ is the unique morphism $X\to X\times_{S}Y$ induced by the diagram 
    $$% https://q.uiver.app/#q=WzAsNSxbMSwxLCJYXFx0aW1lc197U31ZIl0sWzMsMSwiWCJdLFsxLDIsIlkiXSxbMywyLCJTIl0sWzAsMCwiWCJdLFsyLDNdLFsxLDNdLFswLDFdLFswLDJdLFs0LDEsIlxcaWRfe1h9IiwwLHsiY3VydmUiOi0xfV0sWzQsMiwiZiIsMix7ImN1cnZlIjoxfV0sWzQsMCwiXFxleGlzdHMhIiwxLHsic3R5bGUiOnsiYm9keSI6eyJuYW1lIjoiZGFzaGVkIn19fV1d
    \begin{tikzcd}
        X \\
        & {X\times_{S}Y} && X \\
        & Y && S.
        \arrow["{\exists!}"{description}, dashed, from=1-1, to=2-2]
        \arrow["{\id_{X}}", curve={height=-6pt}, from=1-1, to=2-4]
        \arrow["f"', curve={height=6pt}, from=1-1, to=3-2]
        \arrow[from=2-2, to=2-4]
        \arrow[from=2-2, to=3-2]
        \arrow[from=2-4, to=3-4]
        \arrow[from=3-2, to=3-4]
    \end{tikzcd}$$
\end{definition}
\begin{definition}[Diagonal Morphism]\label{def: diagonal morphism}
    Let $X$ be an $S$-scheme. The diagonal morphism $\Delta_{X/S}$ is the unique morphism $X\to X\times_{S}X$ induced by the diagram 
    $$% https://q.uiver.app/#q=WzAsNSxbMSwxLCJYXFx0aW1lc197U31YIl0sWzMsMSwiWCJdLFsxLDIsIlgiXSxbMywyLCJTIl0sWzAsMCwiWCJdLFsyLDNdLFsxLDNdLFswLDFdLFswLDJdLFs0LDEsIlxcaWRfe1h9IiwwLHsiY3VydmUiOi0xfV0sWzQsMiwiXFxpZF97WH0iLDIseyJjdXJ2ZSI6MX1dLFs0LDAsIlxcZXhpc3RzISIsMSx7InN0eWxlIjp7ImJvZHkiOnsibmFtZSI6ImRhc2hlZCJ9fX1dXQ==
    \begin{tikzcd}
        X \\
        & {X\times_{S}X} && X \\
        & X && S.
        \arrow["{\exists!}"{description}, dashed, from=1-1, to=2-2]
        \arrow["{\id_{X}}", curve={height=-6pt}, from=1-1, to=2-4]
        \arrow["{\id_{X}}"', curve={height=6pt}, from=1-1, to=3-2]
        \arrow[from=2-2, to=2-4]
        \arrow[from=2-2, to=3-2]
        \arrow[from=2-4, to=3-4]
        \arrow[from=3-2, to=3-4]
    \end{tikzcd}$$
\end{definition}
\begin{remark}
    In particular, the diagonal morphism \Cref{def: diagonal morphism} is the graph \Cref{def: graph morphism} of $\id_{X}$. 
\end{remark}
\begin{remark}
    The absolute variants of \Cref{def: graph morphism,def: diagonal morphism} can be recovered by taking $S=\spec(\ZZ)$. 
\end{remark}
The name ``diagonal'' is justified by the followng example. 
\begin{example}\label{ex: A1 is separated}
    Let $k$ be a field $X=\A^{1}_{k},S=\spec(k)$. The diagonal map $\A^{1}_{k}\to \A^{1}_{k}\times_{k}\A^{1}_{k}\cong\A^{2}_{k}$ is induced by the ring map $k[x_{1},x_{2}]=k[x_{1}]\otimes_{k}k[x_{2}]\to k[x]$ with kernel $x_{1}-x_{2}$ whose vanishing locus is the diagonal line in $\A^{2}_{k}$. 
\end{example}
While in the above example the diagonal had a closed image, this is not always the case. The collection of schemes where this is satisfied are precisely the separated schemes. 
\begin{definition}[Separated Scheme]\label{def: separated scheme}
    Let $X$ be an $S$-scheme. $X$ is separated if the image of $\Delta_{X/S}$ is closed. 
\end{definition}
\begin{remark}
    This will be the analogue of Hausdorffness in the setting of schemes. 
\end{remark}
\begin{example}
    \Cref{ex: A1 is separated} shows that $\A^{1}_{k}$ is a separated $k$-scheme. On the other hand, the affine line with doubled origin \Cref{ex: doubled origin} is not separated. 
\end{example}
Before considering properties of morphisms of schemes, we consider stability properties of morphisms under certain operations. Thus far, we have only seen fibered products, and the appropriate notion is defined as follows. 
\begin{definition}[Stable Under Base Change]\label{def: stable under base change}
    Let P be a property of morphisms of schemes. The property P is stable under base change if for all $f:X\to Y$ with the property P, and all diagrams Cartesian diagrams 
    $$% https://q.uiver.app/#q=WzAsNCxbMiwwLCJYIl0sWzIsMSwiWSJdLFswLDAsIlgnIl0sWzAsMSwiWSciXSxbMiwzLCJmJyIsMl0sWzMsMV0sWzAsMSwiZiJdLFsyLDBdXQ==
    \begin{tikzcd}
        {X'} && X \\
        {Y'} && Y
        \arrow[from=1-1, to=1-3]
        \arrow["{f'}"', from=1-1, to=2-1]
        \arrow["f", from=1-3, to=2-3]
        \arrow[from=2-1, to=2-3]
    \end{tikzcd}$$
    $f'$ has the property $P$. 
\end{definition}

We now turn to properties of schemes. We first consider topological properties -- properties determined by the map of underlying topological spaces. 
\begin{definition}[Open Morphism]\label{def: open morphism}
    Let $f:X\to Y$ be a morphism of schemes. $f$ is an open morphism if for all $U\subseteq X$ open, $f(U)\subseteq Y$ is open. 
\end{definition}
\begin{definition}[Closed Morphism]\label{def: closed morphism}
    Let $f:X\to Y$ be a morphism of schemes. $f$ is an closed morphism if for all $Z\subseteq X$ closed, $f(Z)\subseteq Y$ is closed.
\end{definition}
\begin{definition}[Dominant]\label{def: dominant morphism}
    Let $f:X\to Y$ be a morphism of schemes. $f$ is a dominant morphism if $f(X)\subseteq Y$ is dense. 
\end{definition}
\begin{remark}
    If $Y$ is integral, and thus with a unique geometric point $\eta$, a morphism $f:X\to Y$ is dense if $\eta\in f(X)$. 
\end{remark}
\begin{definition}[Quasicompact Morphism]\label{def: quasicompact morphism}
    Let $f:X\to Y$ be a morphism of schemes. $f$ is a quasicompact morphism if for all $V\subseteq Y$ quasicompact, $f^{-1}(V)\subseteq X$ is quasicompact. 
\end{definition}
\begin{remark}
    It can be shown that quasicompactness of \Cref{def: quasicompact morphism} can be verified on affine schemes. 
\end{remark}
\begin{definition}[Quasifinite Morphism]\label{def: quasifinite}
    Let $f:X\to Y$ be a morphism of schemes. $f$ is a quasifinite morphism if for all $y\in Y$, $f^{-1}(y)$ is a finite set. 
\end{definition}
We can now consider some properties of locally ringed spaces.
\begin{definition}[Open Immersion]\label{def: open immerrsion}
    Let $f:X\to Y$ be a morphism of schemes. $f$ is an open immersion if there exists some $V\subseteq Y$ open such that $f$ factors over the isomorphism of schemes $X\cong V$.  
\end{definition}
\begin{definition}[Closed Immersion]\label{def: closed immersion}
    Let $f:X\to Y$ be a morphism of schemes. $f$ is a closed immersion if there exists some $W\subseteq Y$ closed such that $f$ factors over the isomorphism of schemes $X\cong W$. 
\end{definition}
\begin{definition}[Immersion]\label{def: immersion}
    Let $f:X\to Y$ be a morphism of schemes. $f$ is an immersion if there exists a closed subscheme $W\subseteq Y$ and an open subscheme $V\subseteq W$ such that $f$ factors over the isomorphism $X\cong V$. 
\end{definition}
\begin{example}
    Let $X=\A^{1}_{k}\setminus\{0\}=D(x)=\spec(k[x^{\pm}])$. THe inclusion to $\A^{2}_{k}$ by $k[x_{1},x_{2}]\to k[x_{1}^{\pm}]$ taking $x_{1}\mapsto x_{1},x_{2}\mapsto0$ is an immersion taking $W=V(x_{2})$ and $V=W\setminus\{0\}$. 
\end{example}
\begin{remark}
    If $f$ is an immersion, it factors as a closed followed by an open immersion. 
\end{remark}
Finally, we turn to scheme-theoretic properties. 
\begin{definition}[Locally Finite Type Morphism]\label{def: locally finite type}
    Let $f:X\to Y$ be a morphism of schemes. $f$ is locally of finite type if there exists an affine open cover $\{V_{j}\}_{j\in J}$ and each affine open covering $\{U_{ij}\}_{i\in I_{j}}$ of $f^{-1}(V_{j})$, $\Gamma(U_{ij},\Ocal_{X})$ is a finite type $\Gamma(V_{j},\Ocal_{Y})$-algebra. 
\end{definition}
\begin{definition}[Finite Type Morphism]\label{def: finite type}
    Let $f:X\to Y$ be a morphism of schemes. $f$ is of finite type if there exists an affine open cover $\{V_{j}\}_{j\in J}$ such that each $f^{-1}(V_{j})$ has a finite cover $\{U_{ij}\}_{i=1}^{n_{j}}$ and $\Gamma(U_{ij},\Ocal_{X})$ is a finite type $\Gamma(V_{j},\Ocal_{Y})$-algebra.
\end{definition}
\begin{definition}[Affine Morphism]\label{def: affine morphism}
    Let $f:X\to Y$ be a morphism of schemes. $f$ is an affine morphism if there exists an open cover $\{V_{j}\}_{j\in J}$ of $Y$ such that $f^{-1}(V_{j})$ is affine for all $j$. 
\end{definition}
\begin{definition}[Finite Morphism]\label{def: finite morphism}
    Let $f:X\to Y$ be a morphism of schemes. $f$ is a finite morphism if it is an affine morphism and there exists an open cover $\{V_{j}\}_{j\in J}$ of $Y$ such that $\Gamma(f^{-1}(V_{j}),\Ocal_{X})$ is a finite module over $\Gamma(V_{j},\Ocal_{Y})$. 
\end{definition}
\begin{remark}
    A number of these conditions on morphisms imply others. The condition for ``existence of an affine open cover'' can often be replaced by ``for all open covers'' using \Cref{lem: affine communication}.
\end{remark}
\section{Lecture 14 -- 26th November 2024}\label{sec: lecture 14}
We now show the Weierstrass preparation theorem. 
\begin{theorem}[Weierstrass Preparation]\label{thm: Weierstrass preparation}
    Let $U\subseteq\CC^{n}$ be open and $D_{R}(0)=D\subseteq\CC$ a disc. If $f:U\times D\to\CC$ is holomorphic of and of order $k$ with respect to the coordinate $w$ on $D$ then there are functions $e(z,w)$ holomorphic nonvanishing on $U\times D$ and $\omega(z,w)$ holomorphic on $U\times D$ of the form $w^{k}+a_{k-1}(z)w^{k-1}+\dots+a_{1}(z)w+a_{0}(z)$ such that $f(z,w)=e(z,w)\cdot\omega(z,w)$.
\end{theorem}
\begin{proof}
    Take $r$ such that $f|_{U\times K}$ on the annulus $K=K(r,R)$ satisfies the hypotheses for \Cref{prop: pre-Weierstrass preparation}. As such, taking $h(z,w),c(z)$ such that $w^{k}e^{h(z,w)}=c(z)\cdot f(z,w)$. Now applying the Laurent decomposition \Cref{prop: Laurent decomposition} we get $h(z,w)=h_{0}(z,w)+h_{\infty}(z,w)$ with $h_{0},h_{\infty}$ holomorphic on $|w|<R,|w|>r$, respectively, and $\lim_{w\to\infty}h_{\infty}(z,w)=0$. 

    We have 
    $$f(z,w)=\frac{w^{k}e^{h_{0}(z,w)+h_{\infty}(z,w)}}{c(z)}=\frac{w^{k}e^{h_{0}(z,w)}}{c(z)}e^{h_{\infty}(z,w)}.$$
    Taking $e(z,w)=\frac{e^{h_{0}}(z,w)}{c(z)}$, we have 
    $$f(z,w)=e(z,w)e^{h_{\infty}(z,w)}.$$
    Moreover, since $\lim_{w\to\infty}h_{\infty}(z,w)=0$, $\lim_{w\to\infty}e^{h_{\infty}(z,w)}=1$ for fixed $z$. We can thus write 
    $$e^{h_{\infty}(z,w)}=1+\sum_{m=1}^{\infty}a_{m}(z)w^{-m}$$
    and decomposing the sum we have 
    \begin{align*}
        w^{k}e^{h_{\infty}(z,w)}&=w^{k}\left(\sum_{m=1}^{k}a_{m}(z)w^{-m}\right)+w^{k}\cdot\sum_{m=k+1}^{\infty}a_{m}(z)w^{-m}\\
        &=\omega(z,w)+\mathcal{R}_{\infty}(z,w)
    \end{align*}
    where the summand $\omega(z,w)=w^{k}\left(\sum_{m=1}^{k}a_{m}(z)w^{-m}\right)$ of the desired form. To prove the result, it suffices to show that $\mathcal{R}_{\infty}(z,w)$ is of the desired form. 

    Now we have 
    $$0=\frac{f(z,w)}{e(z,w)}-\omega(z,w)-\mathcal{R}_{\infty}(z,w)=\mathcal{R}_{0}(z,w)-\mathcal{R}_{\infty}(z,w)$$
    where by uniqueness of Laurent decompositions on the annulus as $\omega(z,w)$ has the same zeroes as $f$. 
\end{proof}
We now consider some special types of functions in the convergent power series ring $\CC\{z_{1},\dots,z_{n}\}$. 
\begin{definition}[Regular of Fixed Order]\label{def: regular of fixed order}
    Let $f\in\CC\{z_{1},\dots,z_{n}\}$. $f$ is $z_{n}$-regular of order $k$ if $f(0,z_{n})$ is not identically zero and has a zero of order $k$ at $z_{n}=0$.\marginpar{Definition 4.1} 
\end{definition}
The Weierstrass theorem shows that zeroes of holomorphic functions are highly structured. 
\begin{lemma}\label{lem: regular transformations}
    If $f$ a holomorphic function on an open neighborhood $U$ around the origin and vanishing at the origin in $\CC^{n+1}$, then there exists a linear change of coordinates $T$ such that $f\circ T$ is $z_{n}$ regular of some order.\marginpar{Lemma 4.5}
\end{lemma}
\begin{proof}
    Consider a complex line $L$ on which $f$ is not identically zero. Setting this line $L$ as the $w$-coordinate, we get that $f$ vanishes at $w=0$. 
\end{proof}
\begin{example}
    $f(z_{1},z_{2})$ is neither $z_{1}$ nor $z_{2}$-regular. But under the transformation $z_{1}=u_{1},z_{2}=u_{1}+u_{2}$, we have that $f(u_{1},u_{2})=u_{1}^{2}+u_{1}u_{2}$ is $u_{1}$-regular of order 2. 
\end{example}
\begin{lemma}\label{lem: order k function}
    Let $f$ be a $z_{n}$-regular function of order $k$. There are constants $0<r', 0<r<R$ such that $f$ converges on $D_{r'}(0)\times D_{R}(0)\subseteq\CC^{n-1}\times\CC$, has no zeroes outside the disc $D_{r'}(0)\times D_{r}(0)\subseteq\CC^{n-1}\times\CC$, and $f(z',z_{n})$ is of order $k$ with respect to $z_{n}$.\marginpar{Lemma 4.6}
\end{lemma}
\begin{proof}
    This is a direct application of \Cref{thm: Weierstrass preparation} as the function $f(0,z_{n})$ has an order $k$ zero at the origin. 
\end{proof}
More generally, we can define Weierstrass polynomials as follows. 
\begin{definition}[Weierstrass Polynomial]\label{def: Weierstrass polynomials}
    A Weierstrass polynomial of order $k$ is a function in $(z_{1},\dots,z_{n-1},z_{n})=(z',z_{n})$ given by 
    $$\omega(z',z_{n})=z_{n}^{k}+a_{k-1}(z')z_{n}^{k-1}+\dots+a_{1}(z')z_{n}+a_{0}(z')$$
    where $a_{k}(0)=0$. 
\end{definition}
We will require the following result about $z_{n}$-regular functions of order $k$ in the following exposition. 
\begin{theorem}[Weierstrass Formal]\label{thm: euclidean division for formal power series}
    Let $f$ be a $z_{n}$-regular function of order $k$. For each $g\in\CC\{z_{1},\dots,z_{n}\}$, there exists a unique decomposition $g=qf+r$ where $q\in\CC\{z_{1},\dots,z_{n}\}$ and $r\in\CC\{z_{1},\dots,z_{n-1}\}[z_{n}]$ of degree smaller than $k$ in $z_{n}$.\marginpar{Theorem 4.8}
\end{theorem}
\begin{proof}
    By the Weierstrass preparation theorem, we can write $f=e\omega$ for $\omega$ a Weierstrass polynomial of order $k$. Up to setting $g=g/e$, we can take $f$ to be a Weierstrass polynomial. Now consider $g/\omega$ and let $r,R,r'$ be such that $g/\omega$ is holomorphic on $D_{r'}(0)\times K(r,R)$, the product of a disc and annulus. By the Laurent decomposition \cref{prop: Laurent decomposition}, we can write $\frac{g}{\omega}=q+h$ where $q,h$ are holomorphic in $D_{r'}(0)\times D_{R}(0)$ and $D_{r'}(0)\times K(r,R)$, respectively. Now recalling $\omega$ is given as $z_{n}^{k}+a_{k-1}(z')z_{n}^{k-1}+\dots+a_{1}(z')z_{n}+a_{0}(z')$,\marginpar{Once again denoting $z'$ for the coordinates $z_{1},\dots,z_{n-1}$.} we have that for $h(z',z_{n})=\sum_{m=1}^{\infty}c_{m}(z')z_{n}^{-m}$, the product $\omega h$ is given by
    $$\omega h=b_{1}(z')z_{n}^{k-1}+\dots+b_{0}(z')+\sum_{m=1}^{\infty}b_{m}(z')z_{n}^{-m}.$$
    Writing $r(z',z_{n})=b_{1}(z')z_{n}^{k-1}+\dots+b_{0}(z'),\mathcal{R}(z',z_{n})=\sum_{m=1}^{\infty}b_{m}(z')z_{n}^{-m}$, we have $g=q\omega+r(z',z_{n})+\mathcal{R}(z',z_{n})$ and it remains to show the latter summand vanishes. 

    Rewriting this $0=-g+q\omega+r(z',z_{n})+\mathcal{R}(z',z_{n})$, we have that $-g+q\omega+r$ is holomorphic on $D_{r'}(0)\times D_{R}(0)$ and $\mathcal{R}(z',z_{n})$ is holomorphic outside $D_{r}(0)$ and hence the above is the Laurent decomposition of the zero function showing $\mathcal{R}(z',z_{n})$ vanishes. Uniqueness of $q,r$ follows from uniqueness of the Laurent decomposition. 
\end{proof}
\appendix
\section{Basic Results in Complex Analysis}\label{app: basic results}
In this appendix, we collect some basic results of complex analysis, largely following the text of Stein and Shakarchi \cite{SteinShakarchi}.
\begin{theorem}[Identity]\label{thm: identity theorem}
    Let $D\subseteq\CC$ be a domain and $f,g$ holomorphic functions on $D$. If the set $\{z\in D:f(z)=g(z)\}$ contains an accumulation point, then $f(z)=g(z)$ for all $z\in D$. 
\end{theorem}
\begin{theorem}[Liouville]\label{thm: Liouville}
    If $f(z)$ is a holomorphic function such that $|f(z)|\leq M$ for some $M\in\RR_{\geq0}$ then $f$ is constant. 
\end{theorem}
\newpage
\printbibliography
\end{document}