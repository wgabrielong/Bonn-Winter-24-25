\section{Lecture 4 -- 15th November 2024}\label{sec: lecture 4}
We consider some properties of the dilogarithm function following \cite{ZagierDilogarithm}. 

Recall polylogarithms from \Cref{def: polylogarithm}. We can in fact alternatively define polylogarithms using the relation of \Cref{lem: polylogarithm differential equation}. 
\begin{lemma}\label{lem: dilogarithm as integral on split plane}
    The dilogarithm $\Li_{2}(t)$ satisfies the integral equation 
    \begin{equation}\label{eqn: dilogarithm integral equation}
        \Li_{2}(t)=-\int_{0}^{t}\log(1-z)dz
    \end{equation}
    on the slit plane $\CC\setminus[1,\infty)$. 
\end{lemma}
\begin{proof}
    The claim follows from \Cref{lem: polylogarithm differential equation} and $\Li_{1}(t)=-\log(1-t)$ of Table \ref{table: polylogarithm values}, noting that the logarithm is defined on the slit plane. 
\end{proof}
Special values of the dilogarithm are given as follows \cite[\S 1]{ZagierDilogarithm}. 

\begin{table}[h]\label{table: dilogarithm special values}
    \begin{tabular}{c c c}
        $\Li_{2}(0)=0$ & $\Li_{2}(1)=\frac{\pi^{2}}{6}$ & $\Li_{2}(-1)=-\frac{\pi^{2}}{12}$ \\
        $\Li_{2}\left(\frac{1}{2}\right)=\frac{\pi^{2}}{12}-\frac{1}{2}\log(2)^{2}$ & $\Li_{2}\left(\frac{1-\sqrt{5}}{2}\right)=\frac{\pi^{2}}{15}+\frac{1}{2}\log\left(\frac{1+\sqrt{5}}{2}\right)^{2}$
    \end{tabular}
    \caption{Special values of the dilogarithm.}
\end{table}
The dilogarithm has especially interesting behavior at roots of unity where it can be expressed in terms of the Dedekind Zeta function and exhibit close connections to $L$-functions \cite[\S 5]{ZagierDilogarithm}. Dilogarithms also satisfy a number of functional equations:

\begin{table}[h]\label{table: dilogarithm functional equations}
    \begin{equation}\label{eqn: dilogarithm and negative functional equation}
        \Li_{2}(1/t) = -\Li_{2}(t) - \frac{\pi^{2}}{6}-\frac{1}{2}\log(-t)^{2}
    \end{equation}
    \begin{equation}\label{eqn: dilogarithm and inverse functional equation}
        \Li_{2}(1-z)=-\Li_{2}(t) + \frac{\pi^{2}}{6}-\log(t)\log(1-t)
    \end{equation}
    \caption{Functional equations for the dilogarithm.}
\end{table}

The phenomena above, where products of logarithms appear, are quite common in the setting of dilogarithm functional equations. The dilogarithm can be analytically continued in the following way. 
\begin{theorem}\label{thm: analytic continuation for dilogarithm}
    The function 
    \begin{equation}\label{eqn: analytic continuation for dilogarithm}
        \Li_{2}(e^{h})+h\cdot\Li_{1}(e^{h})
    \end{equation}
    is well deined $\CC\setminus(2\pi i\ZZ)\to\CC/(2\pi i)^{2}\ZZ$. 
\end{theorem}
\begin{proof}
    Computing, we get 
    \begin{align*}
        \frac{d}{dh}\left(\Li_{2}(e^{h})+h\cdot\Li_{1}(e^{h})\right) &= -e^{h}\cdot\frac{1}{e^{h}}\Li_{1}(e^{h}) + \Li_{1}(e^{h}) + h\Li_{0}(e^{h}) \\
        &= \frac{he^{h}}{1-e^{h}} && \Li_{0}(t)=\frac{t}{1-t}
    \end{align*}
    which is a well-defined meromorphic function on $\CC$ with simple poles at integer multiples of $2\pi i$ and residues multiples of $2\pi i$ so the map descends to the quotient. 
\end{proof}
One can also consider an analogue of the dilogarithm is the Bloch-Wigner dilogarithm. 
\begin{definition}[Bloch-Wigner Dilogarithm]\label{def: Bloch-Wigner dilogarithm}
    The Bloch-Wigner dilogarithm is the function 
    $$D(z)=\img(\Li_{2}(z))+\arg(1-z)\log(|z|).$$
\end{definition}
\begin{remark}
    The Bloch-Wigner dilogarithm is a well-defined continuous function on $\CC\cup\{\infty\}$.
\end{remark}
The advantage of working with the Bloch-Wigner dilogarithm is that one no longer needs to consider the products of logarithms in functional equations that arose above. We state some functional equations below. 

\begin{table}[h]\label{table: BW dilogarithm functional equations}
    \begin{equation}\label{eqn: BW dilogarithm sixfold}
        D(z)=D\left(1-\frac{1}{z}\right)=D\left(\frac{1}{1-z}\right)=-D\left(\frac{1}{z}\right)=-D(1-z)=-D\left(\frac{-z}{1-z}\right)
    \end{equation}
    \begin{equation}\label{eqn: BW dilogarithm five term relation}
        D(x)+D(y)+D\left(\frac{1-x}{1-xy}\right)+D(1-xy)+D\left(\frac{1-y}{1-xy}\right)=0
    \end{equation}
    \caption{Functional equations for the Bloch-Wigner dilogarithm.}
\end{table}
Let us return to our discussion of asymptotics of Nahm sums as in \Cref{sec: lecture 2}. We focus on the case $A=a$ for $a\in\ZZ$ producing Nahm sums of the form 
$$f_{a}(t,q)=\sum_{n\geq0}\frac{q^{\frac{1}{2}an^{2}}}{(q;q)_{n}}t^{n}.$$
The $q$-difference equation this satisfies can be easily deduced. 
\begin{proposition}\label{prop: q-difference equation of 1x1 Nahm sum}
    Let $a\in\ZZ$. The $t$-deformed Nahm sum $f_{a}(t,q)=\sum_{n\geq0}\frac{q^{\frac{1}{2}an^{2}}}{(q;q)_{n}}t^{n}$ satisfies the $t$-deformed $q$-difference equation 
    \begin{equation}\label{eqn: q-difference equation of 1x1 Nahm sum}
        f_{a}(t,q) - f_{a}(qt,q)=tq^{a/2}f_{a}(q^{a}t,q).
    \end{equation}
\end{proposition}
\begin{proof}
    We compute 
    \begin{align*}
        f_{a}(t,q) - f_{a}(qt,q) &= \sum_{n\geq0}\frac{q^{\frac{1}{2}an^{2}}}{(q;q)_{n}}t^{n} - \sum_{n\geq0}\frac{q^{\frac{1}{2}an^{2}}}{(q;q)_{n}}q^{n}t^{n} \\
        &= \sum_{n\geq0}\left(\frac{q^{\frac{1}{2}an^{2}}(1-q^{n})}{(q;q)_{n}}\right)t^{n} \\
        &= \sum_{n\geq0}\frac{q^{\frac{1}{2}a(n+1)^{2}}}{(q;q)_{n}}t^{n+1} && \text{reindexing}\\
        &= \sum_{n\geq0}\frac{q^{\frac{1}{2}an^{2}+an+\frac{a}{2}}}{(q;q)_{n}}t^{n+1}\\
        &= tq^{a/2}\left(\sum_{n\geq0}\frac{q^{\frac{1}{2}an^{2}}}{(q;q)_{n}}q^{an}t^{n}\right)\\
        &= tq^{a/2}f_{a}(q^{a}t,q)
    \end{align*}
    as desired. 
\end{proof}
We can modify the Nahm sum to rid ourselves of the $q^{a/2}$ factor in (\ref{eqn: q-difference equation of 1x1 Nahm sum}). 
\begin{corollary}\label{corr: q-difference equation of modified 1x1 Nahm sum}
    Let $f_{a}^{\mathrm{mod}}(t,q)=f_{a}(q^{-a/2}t,q)$. Then the modified Nahm sum satisfies the $t$-deformed $q$-difference equation 
    \begin{equation}\label{eqn: q-difference equation}
        f_{a}^{\mathrm{mod}}(t,q) - f_{a}^{\mathrm{mod}}(qt,q) = t\cdot f_{a}^{\mathrm{mod}}(q^{a}t,q).
    \end{equation}
\end{corollary}
\begin{proof}
    This follows from a similar computation as in \Cref{prop: q-difference equation of 1x1 Nahm sum}. 
\end{proof}
\begin{remark}
    $f_{a}(t,q)\in\ZZ[[t,\sqrt{q}]]$ but $f_{a}^{\mathrm{mod}}(t,q)\in\ZZ[[t,q]]$. 
\end{remark}
We will henceforth work with these modified Nahm sums. As previously discussed, we would expect the asymptotic expansion of such a Nahm sum to be expressed as a product of an exponential of a dilogarithm, a square root, and a power series in $h$ as $q\to 1$ and $q=\exp(h)$, mirroring the discussion of \Cref{prop: asymptotics of q t Pochhammer at root of unity}. 

We use the ansatz 
\begin{equation}\label{eqn: 1x1 Nahm sum ansatz}
    f_{a}(t,q)=\exp\left(\frac{V(t)}{h}\right)g_{a}(t,q)
\end{equation}
\begin{equation}\label{eqn: qt 1x1 Nahm sum ansatz}
    f_{a}(t,q)=\exp\left(\frac{V(qt)}{h}\right)g_{a}(qt,q)
\end{equation}
in what follows, with $V(t)\in\QQ[[t]]$ satisfying $V(0)=0$. \Cref{eqn: 1x1 Nahm sum ansatz,eqn: qt 1x1 Nahm sum ansatz} takes are related by the formula we now describe. 
\begin{proposition}\label{prop: V functional equation for 1x1 Nahm ansatz}
    The equation $V(t)$ of the Nahm equation ansatz satisfies the logarithmic differential equation 
    \begin{equation}\label{eqn: V functional equation for 1x1 Nahm ansatz}
        V(e^{h}t) = V(t) + h(\partial^{\log}V)(t)+\frac{h^{2}}{2}((\partial^{\log})^{2}V)(t)
    \end{equation}
    with $\partial^{\log}V(t)=t\cdot V'(t)$.
\end{proposition}
\begin{remark}
    (\ref{eqn: V functional equation for 1x1 Nahm ansatz}) of \Cref{prop: V functional equation for 1x1 Nahm ansatz} should be thought of as a multiplicative analogue of the Taylor expansion. 
\end{remark}
Using this, we can compute the ratio of the leading factors of \Cref{eqn: 1x1 Nahm sum ansatz,eqn: qt 1x1 Nahm sum ansatz}. 
\begin{corollary}\label{corr: exponent ratio for 1x1 Nahm sum antsatz}
    Let $V(t)$ be as in the Nahm equation ansatz. The ratios of the exponential factors satisfy 
    $$\exp\left(\frac{V(qt)}{h}\right)/\exp\left(\frac{V(t)}{h}\right)\sim\exp((\partial^{\log}V)(t))(1+O(h)).$$ 
\end{corollary}
\begin{proof}
    We have 
    \begin{align*}
        \exp\left(\frac{V(qt)}{h}\right) &= \exp\left(\frac{V(t)}{h}+(\partial^{\log}V)(t)+\frac{h}{2}((\partial^{\log})^{2}V)(t)\right)\\
        &=\exp\left(\frac{V(t)}{h}\right)\exp\left((\partial^{\log}V)(t)\right)\exp\left(\frac{h}{2}((\partial^{\log})^{2}V)(t)\right)
    \end{align*}
    where the first term of the product above cancels in the ratio and the final term of the product expands to a power series in $h$. 
\end{proof}
Of special interest to us will be the final terms of the product above, where we denote $Z(t)=\exp((\partial^{\log}V)(t))$ and $\widetilde{Z}(t,q)=\exp\left((\partial^{\log}V)(t)\right)\exp\left(\frac{h}{2}((\partial^{\log})^{2}V)(t)\right)$.
This allows us to produce a functional equation of the $g_{a}$ appearing in \Cref{eqn: 1x1 Nahm sum ansatz,eqn: qt 1x1 Nahm sum ansatz} which we will soon revisit. Note specializing at $q-1$ produces 
\begin{equation}\label{eqn: specialized Z functional equation}
    1-Z(t)=t\cdot Z(t)^{a}
\end{equation}

We now explicitly describe $V(t)$. As suggested by the preceding discussion, it suffices to solve the differential equation $V'(t)=\frac{1}{t}\log(Z(t))$. 
\begin{proposition}\label{prop: V t dilogarithm equation}
    Let $V(t)$ be as in the Nahm equation ansatz and $Z(t)=\exp((\partial^{\log}V)(t))$. $V(t)$ satisfies the equation 
    \begin{equation}\label{eqn: V t dilogarithm equation}
        V(t) = -\Li_{2}(1-Z(t))-\frac{a}{2}\log(Z(t))^{2}
    \end{equation}
    with $V(0)=0$. 
\end{proposition}
\begin{proof}
    We compute 
    \begin{align*}
        V'(t) &= -\frac{Z'(t)}{1-Z(t)}\log(Z(t)) - a\cdot\log(Z(t))\frac{Z'(t)}{Z(t)} \\
        &= \frac{(Z(t)+a(1-Z(t)))Z'(t)}{(1-Z(t))Z(t)}\log(Z(t))
    \end{align*}
    so taking the derivative of (\ref{eqn: specialized Z functional equation}), we have 
    $$-Z'(t)=Z(t)^{a}+at\cdot Z(t)^{a-1}Z'(t)=Z'(t)\left(1+at\cdot Z(t)^{a-1}\right)$$
    but $t\cdot Z(t)^{a-1}=\frac{1-Z(t)}{Z'(t)}$ so $-Z'(t)=\frac{Z'(t)}{Z(t)}(Z(t)+a(1-Z(t)))=-\frac{1-Z(t)}{t}$ which on substitution into the final line of the first series of displayed equations above yields the claim. 
\end{proof}
We can also discuss asymptotics of Nahm sums at roots of unity $\zeta_{m}$. In parallel to \Cref{prop: asymptotics of q t Pochhammer at root of unity}, we have the following result for Nahm sums. 
\begin{theorem}\label{thm: Nahm sum asymptotics at roots of unity}
    The Nahm sum $f_{a}(t,q)$ satisfies the asymptotic formula 
    \begin{equation}\label{eqn: Nahm sum asymptotics at roots of unity}
        f_{a}(t,q)\sim\exp\left(\frac{V(t^{m})}{m^{2}h}\right)\cdot O(h)
    \end{equation}
    as $q\to\zeta_{m}$ with $\zeta_{m}$ a primitive $m$th root of unity and $O(h)\in\QQ(\zeta_{m})[[t,h]]$. 
\end{theorem}
\begin{proof}[Proof Outline]
    We use the ansatz 
\begin{equation}\label{eqn: Nahm sum at roots of unity ansatz}
    f_{a}(t,q) = \exp\left(\frac{V(t^{m})}{m^{2}h}\right)g_{a,m}(t,h)
\end{equation}
where we want to show that $g_{a,m}(t,h)\in\QQ(\zeta_{m})[[t,h]]$. As before we have 
$$\exp\left(\frac{V((qt)^{m})}{m^{2}h}\right)=\exp\left(\frac{V(e^{mh}t^{m})}{m^{2}h}\right)$$
and in the Taylor expansion we have 
$$\exp\left(\frac{V(t^{m})}{m^{2}h}+\frac{1}{m}\log(Z(t^{m}))+O(h)\right).$$
\end{proof}