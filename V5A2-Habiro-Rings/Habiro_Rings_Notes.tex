\documentclass{amsart}
\usepackage[margin=1.5in]{geometry} 
\usepackage{amsmath}
\usepackage{tcolorbox}
\usepackage{amssymb}
\usepackage{amsthm}
\usepackage{lastpage}
\usepackage{fancyhdr}
\usepackage{accents}
\usepackage{hyperref}
\usepackage{xcolor}
\usepackage{color}
\input{shortcuts.tex}
\setlength{\headheight}{40pt}


\newenvironment{solution}
  {\renewcommand\qedsymbol{$\blacksquare$}
  \begin{proof}[Solution]}
  {\end{proof}}
\renewcommand\qedsymbol{$\blacksquare$}

\usepackage{amsmath, amssymb, tikz, amsthm, csquotes, multicol, footnote, tablefootnote, biblatex, wrapfig, float, quiver, mathrsfs, cleveref, enumitem, upgreek, stmaryrd, marginnote, todonotes, euscript, lscape}
\addbibresource{refs.bib}
\theoremstyle{definition}
\newtheorem{theorem}{Theorem}[section]
\newtheorem{lemma}[theorem]{Lemma}
\newtheorem{corollary}[theorem]{Corollary}
\newtheorem{exercise}[theorem]{Exercise}
\newtheorem{question}[theorem]{Question}
\newtheorem{example}[theorem]{Example}
\newtheorem{proposition}[theorem]{Proposition}
\newtheorem{conjecture}[theorem]{Conjecture}
\newtheorem{remark}[theorem]{Remark}
\newtheorem{definition}[theorem]{Definition}
\numberwithin{equation}{section}
\setuptodonotes{color=blue!20, size=tiny}
\begin{document}
\large
\title[Habiro Rings -- Bonn, Winter 2024/25]{V5A2 -- The Habiro Ring of a Number Field \\ Winter Semester 2024/25}
\author{Wern Juin Gabriel Ong}
\address{Universit\"{a}t Bonn, Bonn, D-53113}
\email{wgabrielong@uni-bonn.de}
\urladdr{https://wgabrielong.github.io/}
\maketitle
\section*{Dedication}
The course is dedicated to the memory of Tobias Kreutz, a postdoctoral fellow at the Max Planck Institute for Mathematics, who passed away in August 2024. 
\section*{Preliminaries}
These notes roughly correspond to the course \textbf{V5A2 -- The Habiro Ring of a Number Field} taught by Prof. Peter Scholze at the Universit\"{a}t Bonn in the Winter 2024/25 semester. These notes are \LaTeX-ed after the fact with significant alteration and are subject to misinterpretation and mistranscription. Use with caution. Any errors are undoubtedly my own and any virtues that could be ascribed to these notes ought be attributed to the instructor and not the typist. Recordings of the lecture are availible at the following link:
\begin{center}
  \href{https://archive.mpim-bonn.mpg.de/id/eprint/5132/}{\texttt{archive.mpim-bonn.mpg.de/id/eprint/5132/}}
\end{center}

\tableofcontents
\section{Lecture 1 -- 7th October 2024}\label{sec: lecture 1}
All rings are to be taken as commutative.

This will be a lecture course on algebraic geometry based on the abstract perspective of sheaves and schemes. The basic construction to be introduced is sheaves, which are defined as presheaves that satisfy additional properties. Sheaves keep track of both local and global information on topological spaces.\marginpar{Sheaves can in fact be defined more abstractly using sites and Grothendieck topologies.} We first consider the case of presheaves. 
\begin{definition}[Presheaf]\label{def: presheaf}
    Let $\Csf$ be a category and $X$ a topological space. A $\Csf$-valued presheaf $\Fcal$ on $X$ consists of the data of 
    \begin{enumerate}[label=(\roman*)]
        \item An object $\Fcal(U)$ of $\Csf$ for each $U\subseteq X$ open. 
        \item A morphism $\res_{U,V}:\Fcal(U)\to\Fcal(V)$ in $\Csf$ for each $V\subseteq U\subseteq X$ open. 
    \end{enumerate}
    such that $\res_{U,U}:\Fcal(U)\to\Fcal(U)$ is the identity on $\Fcal(U)$ and the triangle 
    \begin{equation}\label{diag: presheaf triangle}
        % https://q.uiver.app/#q=WzAsMyxbMCwwLCJcXEZjYWwoVSkiXSxbMiwwLCJcXEZjYWwoVikiXSxbMSwxLCJcXEZjYWwoVykiXSxbMCwxLCJcXHJlc197VSxWfSJdLFsxLDIsIlxccmVzX3tWLFd9Il0sWzAsMiwiXFxyZXNfe1UsV30iLDJdXQ==
        \begin{tikzcd}
            {\Fcal(U)} && {\Fcal(V)} \\
            & {\Fcal(W)}
            \arrow["{\res_{U,V}}", from=1-1, to=1-3]
            \arrow["{\res_{U,W}}"', from=1-1, to=2-2]
            \arrow["{\res_{V,W}}", from=1-3, to=2-2]
        \end{tikzcd}
    \end{equation}
    is commutative for all $W\subseteq V\subseteq U\subseteq X$ open. 
\end{definition}
\begin{remark}
    Commutativity of the triangle is equivalent to $\res_{U,W}=\res_{V,W}\circ\res_{U,V}$ as morphisms in $\Csf$. 
\end{remark}
We will typically consider examples where $\Csf$ is the category of Abelian groups $\AbGrp$, of rings $\Ring$, of modules over a fixed ring $A$ $\Mod_{A}$ and their appropriate generalizations in the setting of schemes. While it is typically bad categorical practice to look ``within'' objects of a category, we refer to elements of $\Fcal(U)$\marginpar{We will also use the notation $\Gamma(U,\Fcal)$ and $H^{0}(U,\Fcal)$ for $\Fcal(U)$.} for $U\subseteq X$ open as (local) sections, with global sections being elements of $\Fcal(X)$. 
\begin{example}\label{ex: presheaves of continuous functions}
    Let $X$ be a topological space. $C_{X}$ which associates to $U\subseteq X$ open the set $C_{X}(U)$ of continuous functions from $U$ to $\RR$ and restriction maps given by restriction of continuous functions defines a presheaf on $X$. 
\end{example}
\begin{example}\label{ex: presheaves of continuous functions between topological spaces}
    Let $X,Y$ be topological spaces. $\Fcal$ which associates $U\subseteq X$ to the set $\Fcal(U)$ of continuous functions from $U$ to $Y$ and restriction maps given by restriction of continuous functions defines a presheaf on $X$.  
\end{example}
\begin{example}\label{ex: constant presheaves}
    Let $X$ be a topological space and $G$ an Abelian group. We can consider the association taking each open set $U\subseteq X$ to $G$ and restriction maps the identities. This construction defines a presheaf on $X$. 
\end{example}
We can alternatively define presheaves as a type of functor from the category of open sets of a topological space to $\Csf$. Recall here the definition of the category of open sets of a topological space $\Opens_{X}$.  
\begin{definition}[Category of Open Sets]\label{def: category of open sets}
    Let $X$ be a topological space. The category $\Opens_{X}$ of open sets of $X$ has objects open sets of $X$ and morphisms inclusions. 
\end{definition}
\begin{remark}
    Spelling things out, for $U,V$ open in $X$ there is one morphism from $V$ to $U$ if $V\subseteq U$ and no morphism otherwise. 
\end{remark}
We can show that $\Csf$-valued presheaves are merely contravariant functors from $\Opens_{X}$ to $\Csf$, which is essentially a formal result. 
\begin{proposition}\label{prop: presheaves are functors from open sets opposite}
    Let $X$ be a topological space. The data of a presheaf on $X$ corresponds uniquely to a functor $\Opens_{X}^{\Opp}\to\Csf$. 
\end{proposition}
\begin{proof}
    This statement follows by unwinding the definitions of functoriality. A functor associates $\Opens_{X}^{\Opp}\to\Csf$ associates to each open set of $X$ an object of $\Csf$ and by construction of the morphisms in $\Opens_{X}$ there is a morphism $V\to U$ if and only if $V\subseteq U$ corresponding to a morphism $U\to V$ in $\Opens_{X}^{\Opp}$ inducing the corresponding restriction maps in $\Csf$ by functoriality. Furthermore, commutativity of the triangle is preserved since the action of functors on morphisms distributes over composition and composition is unqiue.  
\end{proof}
One example of special interest in the study of algebraic geometry is the construction of a canonical presheaf on the spectrum of a commutative ring $A$. 
\begin{definition}[Spectrum of a Ring]\label{def: spectrum of a ring}
    Let $A$ be a ring. The spectrum $\spec(A)$ of $A$ is the topological space with points given by prime ideals of $A$ and closed sets of the form $V(a)$ for $a\in A$ consisting of the prime ideals containing $a$. 
\end{definition}
Note that no prime ideals contain the multiplicative unit of $A$ so for $a_{1},\dots,a_{n}$ generating $A$ as an $A$-module, the sets $D(a_{i})=\spec(A)\setminus V(a_{i})$ consisting of the prime ideals not containing $A$ form a basis for the toplogy of $\spec(A)$. 

We can now consider the following construction. 
\begin{proposition}\label{prop: structure presheaf on spec A}
    Let $A$ be a ring. Consider the association
    \begin{equation}\label{eqn: structure presheaf on spec A}
        U\mapsto \left\{s:U\to\coprod_{\pfrak\in U}A_{\pfrak}:\substack{\forall\pfrak\in U, s(\pfrak)\in A_{\pfrak} \text{ and } \\ \exists U'\subseteq X, p\in U'\subseteq U, \exists a,b\in A \text{ s.t. }b\notin\qfrak \forall\qfrak\in U', s(\qfrak)=\frac{a}{b}\in A_{\qfrak}}\right\}
    \end{equation}
    and for $V\subseteq U\subseteq X$ the forgetful maps. This association defines a presheaf of rings on $\spec(A)$.\marginpar{This defines a presheaf in terms of compatible stalks following \cite[\S 2.2]{Hartshorne}, which to me seems more artificial than in terms of localizations as in \cite[\S 4.1]{Vakil}}
\end{proposition}
\begin{proof}
    The category of rings is bicomplete, in particular admitting colimits and thus coproducts giving the sets of (\ref{eqn: structure presheaf on spec A}) the structure of a ring by pointwise operations. We get forgetful maps 
    \begin{align*}
        &\left\{s:U\to\coprod_{\pfrak\in U}A_{\pfrak}:\substack{\forall\pfrak\in U, s(\pfrak)\in A_{\pfrak} \text{ and } \\ \exists V\subseteq X, p\in U'\subseteq U, \exists a,b\in A \text{ s.t. }b\notin\qfrak \forall\qfrak\in U', s(\qfrak)=\frac{a}{b}\in A_{\qfrak}}\right\} \\
        &\hspace{0.5cm}\to\left\{s|_{V}:V\to\coprod_{\pfrak\in V}A_{\pfrak}:\substack{\forall\pfrak\in U', s(\pfrak)\in A_{\pfrak} \text{ and } \\ \exists U'\subseteq X, p\in U'\subseteq U, \exists a,b\in A \text{ s.t. }b\notin\qfrak \forall\qfrak\in U', s(\qfrak)=\frac{a}{b}\in A_{\qfrak}}\right\}
    \end{align*}
    for $U'\subseteq U$ that satisfy the commutativity of the diagram (\ref{diag: presheaf triangle}) and hence defines a presheaf. 
\end{proof}
We will set this as the structure presheaf $\Ocal_{\spec(A)}$ of  $\spec(A)$ since it captures important geometric phenomena that we seek to capture as motivated by the following example. 
\begin{example}\label{ex: regular functions on affine scheme}
    Let $k$ be an algebraically closed field and $A$ a finite type $k$-algebra. Consider an open $U\subseteq\spec(A)$ and $s:U\to\coprod_{\pfrak\in U}A_{\pfrak}$ as in (\ref{eqn: structure presheaf on spec A}). We can construct $\overline{s}:U\cap\mathrm{mSpec}(A)\to k$ whose value on each maximal ideal $\mfrak$ is the image of $s(\mfrak)\in A_{\mfrak}$ in the quotient $A_{\mfrak}/\mfrak A_{\mfrak}$ which is isomorphic to $k$ since $k$ is algebraically closed. $s$ is regular on $U$ and thus the association of (\ref{eqn: structure presheaf on spec A}) can be thought of as a presheaf of regular functions. 
\end{example}
Having discussed presheaves, we can define morphisms of presheaves as follows. 
\begin{definition}[Morphism of Presheaves]\label{def: morphism of presheaves}
    Let $X$ be a topological space and $\Fcal,\Fcal'$ $\Csf$-valued presheaves on $X$. A morphism of presheaves $\phi:\Fcal\to\Fcal'$ consists of the data of morphisms $\phi_{U}:\Fcal(U)\to\Fcal'(U)$ in $\Csf$ such that the diagram 
    $$% https://q.uiver.app/#q=WzAsNCxbMCwwLCJcXEZjYWwoVSkiXSxbMiwwLCJcXEZjYWwnKFUpIl0sWzAsMSwiXFxGY2FsKFYpIl0sWzIsMSwiXFxGY2FsJyhWKSJdLFswLDIsIlxccmVzX3tVLFZ9IiwyXSxbMSwzLCJcXHJlc197VSxWfSciXSxbMCwxLCJcXHBoaV97VX0iXSxbMiwzLCJcXHBoaV97Vn0iLDJdXQ==
    \begin{tikzcd}
        {\Fcal(U)} && {\Fcal'(U)} \\
        {\Fcal(V)} && {\Fcal'(V)}
        \arrow["{\phi_{U}}", from=1-1, to=1-3]
        \arrow["{\res_{U,V}}"', from=1-1, to=2-1]
        \arrow["{\res_{U,V}'}", from=1-3, to=2-3]
        \arrow["{\phi_{V}}"', from=2-1, to=2-3]
    \end{tikzcd}$$
    is commutative for $V\subseteq U\subseteq X$ open.
\end{definition}
\begin{remark}\label{rmk: morphism of presheaves is a natural transformation}
    A morphism of presheaves is equivalent to the data of a natural transformation between the corresponding functors $\Opens_{X}^{\Opp}\to\Csf$. 
\end{remark}
This data allows us to define the category of $\Csf$-valued presheaves $\PSh(X,\Csf)$ for a topological space $X$, where we omit the ``value-category'' $\Csf$ when it is clear from context. 
\begin{example}\label{ex: inclusion of differentiable to continuous functions}
    In analogy to \Cref{ex: presheaves of continuous functions}, let $X$ be a topological space, $C_{X}$ the presheaf of continuous functions on $X$, and $C^{\mathrm{diff}}_{X}$ the presheaf of differentiable functions on $X$. There is a natural morphism of presheaves $C_{X}^{\mathrm{diff}}\to C_{X}$ induced by the inclusions of differentiable into continuous functions on each open set. 
\end{example}
\begin{example}\label{ex: circle valued presheaves}
    Let $X$ be a topological space and $S^{1}$ the circle. We can define a presheaf of continuous functions valued in $S^{1}$ by exponentiating a locally continuous function $e^{2\pi i f}$ for $f$ locally continuous. This defines a morphism of presheaves $C_{X}$ to $S^{1}$-valued presheaves in analogue to \Cref{ex: presheaves of continuous functions between topological spaces} with $Y=S^{1}$. 
\end{example}
Having discussed presheaves in some depth, we can now specialize to the case of sheaves, which are presheaves satisfying additional structure. 
\begin{definition}[Sheaves]\label{def: sheaves}
    Let $X$ be a topological space and $\Csf$ be a category admitting arbitrary products.\marginpar{See also the discussion in \cite[\href{https://stacks.math.columbia.edu/tag/00VL}{Tag 00VL}]{stacks-project}.} A presheaf $\Fcal$ on $X$ is a sheaf if for all $U\subseteq X$ and all open covers $\{U_{i}\}_{i\in I}$ the sequence 
    $$% https://q.uiver.app/#q=WzAsMyxbMCwwLCJcXEZjYWwoVSkiXSxbMiwwLCJcXHByb2Rfe2lcXGluIEl9XFxGY2FsKFVfe2l9KSJdLFs0LDAsIlxccHJvZF97aSxqXFxpbiBJfVxcRmNhbChVX3tpfVxcY2FwIFVfe2p9KSJdLFswLDFdLFsxLDIsIiIsMCx7Im9mZnNldCI6LTF9XSxbMSwyLCIiLDIseyJvZmZzZXQiOjF9XV0=
    \begin{tikzcd}
        {\Fcal(U)} && {\prod_{i\in I}\Fcal(U_{i})} && {\prod_{i,j\in I}\Fcal(U_{i}\cap U_{j})}
        \arrow[from=1-1, to=1-3]
        \arrow[shift left, from=1-3, to=1-5]
        \arrow[shift right, from=1-3, to=1-5]
    \end{tikzcd}$$
    is an equalizer where the parallel maps are given by 
    $$(s_{i})_{i\in I}\mapsto (\res_{U_{i}, U_{i}\cap U_{j}}(s_{i}))_{i,j\in I}\text{ and }(s_{i})_{i\in I}\mapsto (\res_{U_{j}, U_{i}\cap U_{j}}(s_{j}))_{i,j\in I},$$
    respectively.
\end{definition}
\begin{remark}\label{rmk: sheaves valued in abelian categories}
    If $\Csf$ is an Abelian category, the equalizer condition can be replaced by $\Fcal(U)$ being the kernel of $\prod_{i\in I}\Fcal(U_{i})\to\prod_{i,j\in I}\Fcal(U_{i}\cap U_{j})$ by $(s_{i})_{i\in I}\mapsto (s_{i}|_{U_{i}\cap U_{j}}-s_{j}|_{U_{i}\cap U_{j}})_{i,j\in I}$. 
\end{remark}
Explicitly, the local behavior states that if there is an open set $U\subseteq X$ with an open cover $\{U_{i}\}_{i\in I}$, a collection of sections $(s_{i})_{i\in I}$ glues to a section on $U$ if and only if $s_{i}|_{U_{i}\cap U_{j}}=s_{j}|_{U_{i}\cap U_{j}}$ for all $i,j\in I$. This proves to be the quality of sheaves that makes them more geometric than presheaves: while functions restrict naturally, they also glue naturally which is behavior captured by sheaves but not by presheaves. 
\begin{example}
    Let $U$ be a topological space and $\{U_{1},U_{2}\}$ an open cover of $U$. Consider continous functions from $U$ to $\RR$. The sheaf axiom allows for the construction of a continuous function $f:U\to \RR$ from continuous functions $f_{1}:U_{1}\to R$ and $f_{2}:U_{2}\to\RR$ that agree on the intersection $U_{1}\cap U_{2}$. 
\end{example}
In particular, we can see that \Cref{ex: presheaves of continuous functions} and \Cref{prop: structure presheaf on spec A} in fact define sheaves on $X$ and $\spec(A)$, respectively. 
\section{Lecture 2 -- 11th October 2024}
We continue our discussion of topological manifolds in general and bases and covers in particular. 
\begin{definition}[Locally Finite]\label{def: locally finite}
    Let $X$ be a topological space and $\Ccal$ a collection of subsets of $X$. $\Ccal$ is locally finite if for every $x\in X$ there exists a neighborhood $U$ of $X$ such that $U$ intersects only finitely many elements of $\Ccal$. 
\end{definition}
\begin{example}
    Let $X=\RR$ in the usual topology and $\Ccal=\{(a-1, a+1): a\in\ZZ\}$. This is locally finite since every sufficiently small ball will intersect at most two elements in $\Ccal$. 
\end{example}
\begin{example}
    Let $X=\RR$ in the usual topology and $\Ccal=\{(a-1,a+1):a\in\QQ\}$. This is not locally finite since $\QQ$ is dense in $\RR$. 
\end{example}
We can now define paracompactness in terms of a refinement condition. 
\begin{definition}[Refinement]\label{def: refinement}
    Let $X$ be a topological space and $\{U_{i}\}_{i\in I}$ an open cover of $X$. A cover $\{V_{j}\}_{j\in J}$ is a refinement of $\{U_{i}\}_{i\in I}$ if for all elements $U_{i}$ there is some $V_{j}\subseteq U_{i}$.
\end{definition}
\begin{definition}[Paracompact]\label{def: paracompact}
    Let $X$ be a topological space. $X$ is paracompact if each cover of $X$ has a refinement by a locally finite cover. 
\end{definition}
This is a weaker condition than compactness but still captures a number of desirable properties. 
\begin{lemma}\label{lem: hausdorff and compact exhaustion is paracompact}
    Let $X$ be a Hausdorff topological space admitting a compact exhaustion. Then for any basis $\Bcal$ of $X$, any open cover admits a locally finite subcover by basis elements. In particular, $X$ is paracompact. 
\end{lemma}
\begin{proof}
    By assumption, there is a sequence $\{K_{i}\}_{i=1}^{\infty}$ of compact sets with $K_{i}\subseteq K_{i+1}^{\circ}$ and $\bigcup_{i=1}^{\infty}K_{i}=X$. Let $\{U_{j}\}_{j\in J}$ be an open cover. For $m\in\ZZ$, set $V_{m}=K_{m+1}\setminus K_{m}^{\circ}$ for $m\geq0$ and $\emptyset$ otherwise. First note that the $V_{m}$ are compact as it is a closed set of a compact set and that $\bigcup_{m\in\ZZ}V_{m}=X$, and that $V_{m}\cap V_{m-1}=\partial K_{m}$ is compact it being a closed subset of a compact space. Further noting that $\{U_{j}\cap K_{m+1}^{\circ}\cap K_{m-1}^{c}\}_{j\in J}$ forms an open cover of $V_{j}$. Moreover, since $\Bcal$ is a basis, we can find a refinement of this cover by basis elements $W_{1},\dots,W_{n}$. This cover suffices as it is a refinement of $\{U_{j}\}_{j\in J}$ and is locally finite since for any $x\in X$ we have that $x\in V_{m}$ for some $m$ and thus $x\in K_{m+2}^{\circ}\cap K_{m-1}^{c}$ hence intersecting only finitely many of the $W$'s. 

    The latter claim follows immediately from the former. 
\end{proof}
From this, we conclude the following corollary. 
\begin{corollary}\label{corr: manifolds are paracompact}
    If $X$ is a locally Euclidean, Hausdorff, and second countable topological, then $X$ is paracompact. 
\end{corollary}
\begin{proof}
    This follows from previous results. Being locally Euclidean and second countable implies compact exhaustion by \Cref{prop: locally euclidean Hausdorff second countable implies compactly exhaustible}, which in turn implies paracompactness by \Cref{lem: hausdorff and compact exhaustion is paracompact}. 
\end{proof}
% Comparison for theories of manifolds. 
We can now begin a discussion of topological manifolds. 
\begin{definition}[Topological Manifold]\label{def: topological manifold}
    A topological space $M$ is a topological manifold if it is locally Euclidean, Hausdorff, and second countable. 
\end{definition}
\begin{remark}
    The Hausdorffness condition is required here to ensure the collection of objects we are considering is not too large. 
\end{remark}
These objects naturally assemble into a category, in fact a full subcategory of the category of topological spaces. 
\begin{definition}[Category of Topological Manifolds]\label{def: category of topological manifolds}
    The category of topological manifolds $\Mfld$ consists of objects topological manifolds and morphisms continuous maps. 
\end{definition}
\begin{remark}\label{rmk: mfld is a full subcategory}
    Fullness as a subcategory follows from the definition, and as such equivalences in $\Mfld$ are homeomorphisms. 
\end{remark}
We have already encountered a number of examples. 
\begin{example}\label{ex: Rn is a mfld}
    $\RR^{n}$ is a topological manifold. 
\end{example}
\begin{example}\label{ex: fd real vs is a mfld}
    A finite-dimensional $\RR$-vector space is a topological manifold under the metric topology. 
\end{example}
\begin{example}\label{ex: open subsets of Rn are mflds}
    Any open subset of $\RR^{n}$ is a topological manifold. 
\end{example}
\begin{example}\label{ex: graphs are mflds}
    Let $U\subseteq\RR^{n}$ open and $f:U\to\RR^{m}$ be a continuous function. Set $\Gamma(f)=\{(x,y)\in U\times\RR^{m}:f(x)=y\}$. Then $\Gamma(f)$ is a manifold. 
\end{example}
\begin{example}\label{ex: spheres are mflds}
    The $n$-sphere $S^{n}\subseteq\RR^{n+1}$ is a smooth manifold. 
\end{example}
\begin{example}\label{ex: boundary of cube is mfld}
    Let $C^{n}$ be the boundary of the $n$-cube. Then $C^{n}$ is homeomorphic to the sphere $S^{n}$ and hence a manifold. 
\end{example}
\begin{example}\label{ex: torus is mfld}
    Let $\TT^{n}=\RR^{n}/\ZZ^{n}$ with the quotient topology be the $n$-torus. Then $\TT^{n}$ is a manifold. 
\end{example}
\begin{example}\label{ex: projective space is a mfld}
    Real projective space $\RR\PP^{n}$ is a manifold. 
\end{example}
\begin{example}\label{ex: klein bottle is a mfld}
    The Klein bottle is a manifold. 
\end{example}
\begin{remark}
    The examples of \Cref{ex: projective space is a mfld,ex: klein bottle is a mfld} are examples of non-orientable manifolds. 
\end{remark}
We can also define manifolds with boundary, where charts are taken to be homeomorphic to the upper-half space. 
\begin{definition}[Upper-Half Space]\label{def: upper-half space}
    The upper-half space $\HH^{n}$ is given by 
    $$\HH^{n}=\{(x_{1},\dots,x_{n})\in\RR^{n}:x_{i}\geq0, \forall 1\leq i\leq n\}.$$
\end{definition}
Manifolds with boundary are then defined as follows. 
\begin{definition}[Manifold with Boundary]\label{def: manifold with boundary}
    A topological space $M$ is a manifold with boundary if it is Hausdorff, second countable, and each point has a neighborhood homeomorphic to an open subset of $\HH^{n}$. 
\end{definition}
\begin{remark}
    As such, every manifold is a manifold with boundary, translating the image of charts such that it does not intersect $x_{1}=\dots=x_{n}=0$ in $\HH^{n}$. 
\end{remark}
\begin{example}
    $\HH^{n}$ is a manifold with boundary. 
\end{example}
\begin{example}
    $S^{n}\cap\HH^{n+1}$ is a manifold with boundary, and is in fact homeomorphic to the closed unit disc. 
\end{example}
Interior and boundar points of manifolds with boundary are defined as follows. 
\begin{definition}[Interior Point]\label{def: interior point}
    Let $M$ be a manifold with boundary. A point $x\in M$ is an interior point if it has a neighborhood homeomorphic to $\RR^{n}$. 
\end{definition}
\begin{definition}[Boundary Point]\label{def: boundary point}
    Let $M$ be a manifold with boundary. A point $x\in M$ is an interior point if it does not have a neighborhood homeomorphic to $\RR^{n}$. 
\end{definition}
\section{Lecture 3 -- 14th October 2024}\label{sec: lecture 3}
Let us return to the example of the spectrum of a commutative ring as discussed in \Cref{prop: structure presheaf on spec A}. 
\begin{proposition}\label{prop: structure presheaf on spec A is a sheaf}
    Let $A$ be a ring. Consider the association
    $$U\mapsto \left\{s:U\to\coprod_{\pfrak\in U}A_{\pfrak}:\substack{\forall\pfrak\in U, s(\pfrak)\in A_{\pfrak} \text{ and } \\ \exists U'\subseteq X, p\in U'\subseteq U, \exists a,b\in A \text{ s.t. }b\notin\qfrak \forall\qfrak\in U', s(\qfrak)=\frac{a}{b}\in A_{\qfrak}}\right\}$$
    and for $V\subseteq U\subseteq X$ the forgetful maps. This association defines a sheaf of rings on $\spec(A)$.
\end{proposition}
\begin{proof}
    This was already shown to be a presheaf in \Cref{prop: structure presheaf on spec A}, which is a sheaf since it satisfies the local compatibility condition of \Cref{def: sheafification}.  
\end{proof}
We shall denote this ring $\Ocal_{\spec(A)}$, and its sections admit a more explicit description as follows. 
\begin{proposition}\label{prop: sections of OspecA}
    Let $A$ be a ring with spectrum $\spec(A)$ and structure sheaf $\Ocal_{\spec(A)}$. Then: 
    \begin{enumerate}[label=(\roman*)]
        \item $\Ocal_{\spec(A),\pfrak}\cong A_{\pfrak}$ for all prime ideals $\pfrak\subseteq A$. 
        \item $\Ocal_{\spec(A)}(D(f))\cong A_{f}$ for $D(f)=\{\pfrak\subseteq A:f\notin\pfrak\}\subseteq\spec(A)$ and all $f\in A$. 
    \end{enumerate}
\end{proposition}
\begin{proof}[Proof of (i)]
    Note that there is a natural homomorphism $\Ocal_{\spec(A),\pfrak}\to A_{\pfrak}$ by $s\mapsto s(\pfrak)$ which is in $A_{\pfrak}$ by hypothesis. 
    
    This map is surjective since each element of $A_{\pfrak}$ is of the form $\frac{a}{b}$ for $b\in A\setminus\pfrak$. As such $D(b)$ gives an open neighborhood of $\pfrak$ and $\frac{a}{b}$ is an element defining a section of $\Ocal_{\spec(A)}(D(b))$ whose value at $\pfrak$ is exactly $\frac{a}{b}$ giving surjectivity. 
    
    For injectivity, let $U\subseteq\spec(A)$ be an open set containing $\pfrak$ and $s,s'\in\Ocal_{\spec(A)}(U)$ such that $s(\pfrak)=s'(\pfrak)$. Taking $U$ to be sufficiently small, we have that $s=\frac{a}{b},s'=\frac{a'}{b'}$ for $a,a'\in A$ and $b,b'\in A\setminus\pfrak$. Since these elements are equivalent in the localization, there exists $c\in A\setminus\pfrak$ such that $c(ab'-a'b)=0$ in $A$. So $s=s'$ in all $A_{\qfrak}$ for $b,b',c\notin\qfrak$. But this is precisely $D(b)\cap D(b')\cap D(c)$ containing $\pfrak$ so $s=s'$ in a neighborhood of $\pfrak$ and thus give the same stalk showing injectivity, and that the map is an isomorphism. 
\end{proof}
\begin{proof}[Proof of (ii)]
    We now define a homomorphism $A_{f}\to\Ocal_{\spec(A)}(D(f))$ by $\frac{a}{f^{n}}\mapsto (\frac{a}{f^{n}}\mapsto (\frac{a}{f^{n}})_{\pfrak\in D(f)})$. 
    
    We first show the map is injective. Suppose there is some $\frac{a}{f^{n}},\frac{a'}{f^{n'}}$ mapping to the same element in $A_{\pfrak}$ for all $\pfrak\in D(f)$. So for each such $\pfrak$ there is $c_{\pfrak}\in A\setminus\pfrak$ such that $c_{\pfrak}(af^{n'}-a'f^{n})=0$ in $A$. Now note that $\Ann(af^{n'}-a'f^{n})\not\subseteq\pfrak$ for any $\pfrak\in D(f)$ since $\Ann(af^{n'}-a'f^{n})$ contains $c_{\pfrak}\in A\setminus\pfrak$. As such, $V(\Ann(af^{n'}-a'f^{n}))\subseteq V(f)=\spec(A)\setminus D(f)$ from which we conclude that $f\in\sqrt{\Ann(af^{n'}-a'f^{n})}$ and there is some $N$ large such that $f^{N}(af^{n'}-a'f^{n})=0$ showing injectivity. 

    For surjectivity, take $s\in\Ocal_{\spec(A)}(D(f))$ with $s:D(f)\to\coprod_{\pfrak\in D(f)}A_{\pfrak}$ such that for all $\pfrak\in D(f)$ we have that $s(\pfrak)\in A_{\pfrak}$ and there exists $U\subseteq D(f)$ containing $\pfrak$ and $a,b\in A$ such that $b\neq\qfrak$ for all $\qfrak\in U$ and $s(\qfrak)=\frac{a}{b}\in A_{\qfrak}$. Let $\{U_{i}\}_{i\in I}$ be an open cover of $U$ on which $s$ has image $\frac{a_{i}}{b_{i}}$ with $b_{i}\notin\pfrak$ for all $\pfrak\in U_{i}$. Since distinguished opens form a basis for the open sets of the Zariski topology on $\spec(A)$, we can take $U_{i}=D(r_{i})$ with $D(r_{i})\subseteq D(b_{i})$. We thus have $V((b_{i}))\subseteq V((r_{i}))$ and thus $\sqrt{(r_{i})}\subseteq\sqrt{(b_{i})}$. In particular, $r_{i}^{n}=cb_{i}$ for some $c$ so we can write $a_{i}b_{i}=ca_{i}r_{i}^{n}$ and since $D(r_{i})=D(r_{i}^{n})$ we can assume that $D(f)$ is covered by $D(r_{1}),\dots,D(r_{m})$ given quasicompactness of the spectrum of a ring on which $s$ is given by $\frac{a_{i}}{r_{i}}$. Now on $D(r_{i})\cap D(r_{j})=D(r_{i}r_{j})$ we have the image of $s$ given by both $\frac{a_{i}}{r_{i}}$ and $\frac{a_{j}}{r_{j}}$ giving $(h_{i}h_{j})^{N}(h_{j}a_{i}-h_{i}a_{j})=0$. Rewriting this equation and picking $N$ large, we have that $\frac{a}{f^{n}}=\frac{a_{i}}{r_{i}}$ on $D(r_{i})$ giving surjectivity and the claim. 
\end{proof}
We return to some generalities on sheaf theory, and discuss the kernel, cokernel, and image sheaves. This is easiest to do in the case of the kernel as justified by the following lemma. 
\begin{lemma}\label{lem: presheaf kernel is a sheaf}
    Let $X$ be a topological space and $\phi:\Fcal\to\Gcal$ a morphism of sheaves of Abelian groups on $X$. The association 
    $$U\mapsto\ker(\Fcal(U)\to\Gcal(U))$$
    is a sheaf on $X$. 
\end{lemma}
\begin{proof}
    For $\{U_{i}\}_{i\in I}$ an open cover of $U$, we have the following diagram 
    $$% https://q.uiver.app/#q=WzAsOSxbMCwwLCJcXGtlcihcXHBoaV97VX0pIl0sWzAsMSwiXFxwcm9kX3tpXFxpbiBJfVxca2VyKFxccGhpX3tVX3tpfX0pIl0sWzAsMiwiXFxwcm9kX3tpLGpcXGluIEl9XFxrZXIoXFxwaGlfe1Vfe2l9XFxjYXAgVV97an19KSJdLFsyLDAsIlxcRmNhbChVKSJdLFsyLDEsIlxccHJvZF97aVxcaW4gSX1cXEZjYWwoVV97aX0pIl0sWzIsMiwiXFxwcm9kX3tpLGpcXGluIEl9XFxGY2FsKFVfe2l9XFxjYXAgVV97an0pIl0sWzQsMCwiXFxHY2FsKFUpIl0sWzQsMSwiXFxwcm9kX3tpXFxpbiBJfVxcR2NhbChVX3tpfSkiXSxbNCwyLCJcXHByb2Rfe2ksalxcaW4gSX1cXEdjYWwoVV97aX1cXGNhcCBVX3tqfSkiXSxbMCwzXSxbMyw2XSxbNiw3XSxbNyw4XSxbNCw1XSxbMSwyXSxbMCwxXSxbMSw0XSxbNCw3XSxbMyw0XSxbMiw1XSxbNSw4XV0=
    \begin{tikzcd}
        {\ker(\phi_{U})} && {\Fcal(U)} && {\Gcal(U)} \\
        {\prod_{i\in I}\ker(\phi_{U_{i}})} && {\prod_{i\in I}\Fcal(U_{i})} && {\prod_{i\in I}\Gcal(U_{i})} \\
        {\prod_{i,j\in I}\ker(\phi_{U_{i}\cap U_{j}})} && {\prod_{i,j\in I}\Fcal(U_{i}\cap U_{j})} && {\prod_{i,j\in I}\Gcal(U_{i}\cap U_{j})}
        \arrow[from=1-1, to=1-3]
        \arrow[from=1-1, to=2-1]
        \arrow[from=1-3, to=1-5]
        \arrow[from=1-3, to=2-3]
        \arrow[from=1-5, to=2-5]
        \arrow[from=2-1, to=2-3]
        \arrow[from=2-1, to=3-1]
        \arrow[from=2-3, to=2-5]
        \arrow[from=2-3, to=3-3]
        \arrow[from=2-5, to=3-5]
        \arrow[from=3-1, to=3-3]
        \arrow[from=3-3, to=3-5]
    \end{tikzcd}$$
    realizing $\Fcal(U),\Gcal(U)$ as the kernels of the maps between the products in the lower-right corner. As such, we have the descent condition. 
\end{proof}
This defines the sheaf kernel. 
\begin{definition}[Sheaf Kernel]\label{def: sheaf kernel}
    Let $X$ be a topological space and $\phi:\Fcal\to\Gcal$ a morphism of sheaves on $X$. The sheaf kernel $\ker(\phi)$ is the sheaf
    $$U\mapsto\ker(\Fcal(U)\to\Gcal(U)).$$ 
\end{definition}
However, in the case of the cokernel and the image, the na\"{i}vely defined presheaf is often not a sheaf as illustrated by the following example. 
\begin{example}
    Let $X=\{0,1\}$ with the discrete topology, $G$ an Abelian group, and $\Fcal=\Gcal=\underline{G}$. Define a morphism of sheaves $\phi:\Fcal\to\Gcal$ which is the identity on $G$ over $X$ but the trivial map over any proper open subset of $X$. The cokernel is then 0 over $X$ but $G$ over any proper open subset of $X$ so the sheaf condition does not hold. 
\end{example}
As such, the definition of the cokernel and image necessitates sheafification. 
\begin{definition}[Sheaf Cokernel]\label{def: sheaf cokernel}
    Let $X$ be a topological space and $\phi:\Fcal\to\Gcal$ a morphism of sheaves of Abelian groups on $X$. The sheaf cokernel is the sheafification of the presheaf cokernel 
    $$U\mapsto\coker(\Fcal(U)\to\Gcal(U)).$$
\end{definition}
\begin{definition}[Sheaf Image]\label{def: sheaf image}
    Let $X$ be a topological space and $\phi:\Fcal\to\Gcal$ a morphism of sheaves of Abelian groups on $X$. The sheaf image is the sheafification of the presheaf image 
    $$U\mapsto\img(\Fcal(U)\to\Gcal(U)).$$
\end{definition}
Evidently these are sheaves. We show they satisfy the expected universal properties. 
\begin{proposition}\label{prop: coker and im satisfy universal properties}
    Let $X$ be a topological space and $\phi:\Fcal\to\Gcal$ a morphism of sheaves of Abelian groups on $X$. Then:
    \begin{enumerate}[label=(\roman*)]
        \item For any sheaf $\Gcal'$ admitting a morphism from $\Gcal$ such that the composite $\Fcal\to\Gcal\to\Gcal'$ is the zero morphism, there is a unique morphism making the diagram 
        $$% https://q.uiver.app/#q=WzAsNCxbMCwwLCJcXEZjYWwiXSxbMiwwLCJcXEdjYWwiXSxbNCwwLCJcXGNva2VyKFxccGhpKSJdLFs0LDEsIlxcR2NhbCciXSxbMCwxLCJcXHBoaSIsMix7ImxhYmVsX3Bvc2l0aW9uIjo3MH1dLFsxLDJdLFswLDIsIjAiLDAseyJjdXJ2ZSI6LTJ9XSxbMCwzLCIwIiwyLHsiY3VydmUiOjF9XSxbMSwzXSxbMiwzLCJcXGV4aXN0cyEiLDAseyJzdHlsZSI6eyJib2R5Ijp7Im5hbWUiOiJkYXNoZWQifX19XV0=
        \begin{tikzcd}
            \Fcal && \Gcal && {\coker(\phi)} \\
            &&&& {\Gcal'}
            \arrow["\phi"'{pos=0.7}, from=1-1, to=1-3]
            \arrow["0", curve={height=-12pt}, from=1-1, to=1-5]
            \arrow["0"', curve={height=6pt}, from=1-1, to=2-5]
            \arrow[from=1-3, to=1-5]
            \arrow[from=1-3, to=2-5]
            \arrow["{\exists!}", dashed, from=1-5, to=2-5]
        \end{tikzcd}$$
        commute. 
        \item For any sheaf $\Gcal'$ admitting a morphism from $\Fcal$ and a monomorphism to $\Gcal$, there exists a unique morphism making the diagram 
        $$% https://q.uiver.app/#q=WzAsNCxbMCwwLCJcXEZjYWwiXSxbMiwwLCJcXGltZyhcXHBoaSkiXSxbNCwwLCJcXEdjYWwiXSxbMiwxLCJcXEdjYWwnIl0sWzAsM10sWzMsMl0sWzEsMl0sWzAsMV0sWzEsMywiXFxleGlzdHMhIiwwLHsic3R5bGUiOnsiYm9keSI6eyJuYW1lIjoiZGFzaGVkIn19fV0sWzAsMiwiXFxwaGkiLDAseyJjdXJ2ZSI6LTJ9XV0=
        \begin{tikzcd}
            \Fcal && {\img(\phi)} && \Gcal \\
            && {\Gcal'}
            \arrow[from=1-1, to=1-3]
            \arrow["\phi", curve={height=-12pt}, from=1-1, to=1-5]
            \arrow[from=1-1, to=2-3]
            \arrow[from=1-3, to=1-5]
            \arrow["{\exists!}", dashed, from=1-3, to=2-3]
            \arrow[from=2-3, to=1-5]
        \end{tikzcd}$$
        commute. 
    \end{enumerate}
\end{proposition}
\begin{proof}[Proof of (i)]
    By definition the map from $\Fcal$ to the presheaf cokernel is the zero map inducing the solid diagram 
    $$% https://q.uiver.app/#q=WzAsNSxbMCwwLCJcXEZjYWwiXSxbMiwwLCJcXEdjYWwiXSxbNCwwLCJcXGNva2VyX3tcXFBTaH0oXFxwaGkpIl0sWzUsMCwiXFxjb2tlcihcXHBoaSk9XFxjb2tlcl97XFxQU2h9KFxccGhpKV57XFwjfSJdLFs1LDEsIlxcR2NhbCciXSxbMCwxLCJcXHBoaSJdLFsxLDJdLFswLDIsIjAiLDEseyJjdXJ2ZSI6LTJ9XSxbMCwzLCIwIiwxLHsiY3VydmUiOi01fV0sWzAsNF0sWzEsNF0sWzIsNCwiXFxleGlzdHMhIiwxLHsic3R5bGUiOnsiYm9keSI6eyJuYW1lIjoiZGFzaGVkIn19fV0sWzMsNCwiXFxleGlzdCEiLDAseyJzdHlsZSI6eyJib2R5Ijp7Im5hbWUiOiJkYXNoZWQifX19XSxbMiwzXV0=
    \begin{tikzcd}
        \Fcal && \Gcal && {\coker_{\PSh}(\phi)} & {\coker(\phi)=\coker_{\PSh}(\phi)^{\#}} \\
        &&&&& {\Gcal'}
        \arrow["\phi", from=1-1, to=1-3]
        \arrow["0"{description}, curve={height=-12pt}, from=1-1, to=1-5]
        \arrow["0"{description}, curve={height=-30pt}, from=1-1, to=1-6]
        \arrow[from=1-1, to=2-6]
        \arrow[from=1-3, to=1-5]
        \arrow[from=1-3, to=2-6]
        \arrow[from=1-5, to=1-6]
        \arrow["{\exists!}"{description}, dashed, from=1-5, to=2-6]
        \arrow["{\exists!}", dashed, from=1-6, to=2-6]
    \end{tikzcd}$$
    which extends to the one above by the universal property of sheafification since $\Gcal'$ is a sheaf. 
\end{proof}
\begin{proof}[Proof of (ii)]
    Arguing similarly, $\img_{\PSh}(\phi)$ fits into $\Fcal\to\img_{\PSh}(\phi)\to\Gcal$ where the morphism to $\Gcal'$ is induced by the universal property of sheafification. 
\end{proof}
With this language in mind, we want to be able to discuss isomorphisms of sheaves. We define these via monomorphisms and epimorphisms of sheaves. 
\begin{proposition}
    Let $X$ be a topological space and $\phi:\Fcal\to\Gcal$ a morphism of sheaves of Abelian groups on $X$. The following are equivalent:
    \begin{enumerate}[label=(\alph*)]
        \item $\phi$ is a monomorphism. 
        \item $\ker(\phi)$ is the zero sheaf. 
        \item For all $U\subseteq X$, $\phi_{U}$ is injective. 
        \item For all $x\in X$, $\phi_{x}$ is injective. 
    \end{enumerate}
\end{proposition}
\begin{proof}
    (a)$\Leftrightarrow$(b) Suppose $\phi$ is a monomorphism. The map $\ker(\phi)\to\Fcal$ necessarily factors through the zero object but for any $U\subseteq X$ we have $\ker(\phi_{U})\hookrightarrow\Fcal(U)\to\ker(\phi)(U)=0$ showing that $\ker(\phi)=0$. Conversely, for $\ker(\phi)=0$, the universal property for the zero object implies that $\phi$ is a monomorphism. 

    (b)$\Leftrightarrow$(c) Since the sheaf kernel agrees with the presheaf kernel as a presheaf by \Cref{lem: presheaf kernel is a sheaf} we have that $\phi_{U}$ injective implies $\ker(\phi)(U)=\ker(\phi_{U})=0$ which glues to $\ker(\phi)=0$. 

    (c)$\Rightarrow$(d) taking colimits is left exact so the kernel of $\phi_{x}$ is $\ker(\phi)_{x}$ which is zero. 
    
    (d)$\Rightarrow$(c) Supposing that $\phi_{x}$ is injective for all $x$, we can take germs and find sufficiently small neighborhoods gluing to $U$ to show $\phi_{U}$ is injective. 
\end{proof}
\section{Lecture 4 -- 18th October 2024}\label{sec: lecture 4}
Continuing the discussion of properties of morphisms of sheaves, we characterize epimorphisms and isomorphisms. 
\begin{proposition}\label{prop: equivalent conditions on epimorphisms}
    Let $X$ be a topological space and $\phi:\Fcal\to\Gcal$ a morphism of sheaves of Abelian groups on $X$. The following are equivalent:
    \begin{enumerate}[label=(\alph*)]
        \item $\phi$ is an epimorphism. 
        \item $\coker(\phi)$ is the zero sheaf. 
        \item For all $x\in X$, $\phi_{x}$ is surjective. 
    \end{enumerate}
\end{proposition}
\begin{proof}
    \todo{To do.}
\end{proof}
These conditions are implied by surjectivity of $\phi_{U}:\Fcal(U)\to\Gcal(U)$. 
\begin{proposition}\label{prop: surjective implies epimorphisms}
    Let $X$ be a topological space and $\phi:\Fcal\to\Gcal$ a morphism of sheaves of Abelian groups on $X$. If $\phi_{U}:\Fcal(U)\to\Gcal(U)$ is surjective for all $U\subseteq X$ open then:
    \begin{enumerate}[label=(\roman*)]
        \item $\phi$ is an epimorphism. 
        \item $\coker(\phi)$ is the zero sheaf. 
        \item For all $x\in X$, $\phi_{x}$ is surjective. 
    \end{enumerate}
\end{proposition}
\begin{proof}
    \todo{To do.}
\end{proof}
This allows us to characterize isomorphisms in the category of sheaves via their stalks. 
\begin{corollary}\label{corr: isomorphism of sheaves}
    Let $X$ be a topological space and $\phi:\Fcal\to\Gcal$ a morphism of sheaves of Abelian groups on $X$. The following are equivalent:
    \begin{enumerate}[label=(\alph*)]
        \item $\phi$ is an isomorphism. 
        \item For all $x\in X$, $\phi_{x}$ is an isomorphism. 
    \end{enumerate}
\end{corollary}
\begin{proof}
    By the definition of sheaves as compatible germs, we have (a)$\Rightarrow$(b). 

    Conversely, for all open neighborhoods $U\subseteq X$ we have a commutative diagram 
    $$% https://q.uiver.app/#q=WzAsNCxbMCwwLCJcXEZjYWwoVSkiXSxbMiwwLCJcXEdjYWwoVSkiXSxbMiwxLCJcXEdjYWxfe3h9Il0sWzAsMSwiXFxGY2FsX3t4fSJdLFszLDJdLFsxLDJdLFswLDNdLFswLDFdXQ==
    \begin{tikzcd}
        {\Fcal(U)} && {\Gcal(U)} \\
        {\Fcal_{x}} && {\Gcal_{x}.}
        \arrow[from=1-1, to=1-3]
        \arrow[from=1-1, to=2-1]
        \arrow[from=1-3, to=2-3]
        \arrow[from=2-1, to=2-3]
    \end{tikzcd}$$
    We first show injectivity. Suppose that there is $s\in\Fcal(U)$ such that $\phi_{U}(s)=0$ in $\Gcal(U)$. Thus for all $x\in U$ the germ $\phi_{U}(s)_{x}$ of $\phi_{U}(s)$ at $x$ is zero thus $\phi_{x}$ is injective. But since $\Fcal$ is a sheaf, $s=0$ in $\Fcal(U)$. For surjectivity, suppose there is some $t\in\Gcal(U)$ which on shrinking to sufficiently small neighborhoods has germ $t_{x}$ with preimage $s_{x}$ since $\phi_{x}$ is an isomorphism and in particular surjective. The open set on which $s_{x}$ is a representative section glues to a section in the preimage of $t$ defined over $U$ since injectivity implies that these preimages agree on overlaps. 
\end{proof}
These properties of morphisms of sheaves have a natural extension in the consideration of complexes. 
\begin{definition}[Exact]\label{def: exact}
    Let $X$ be a topological space and $\Fcal,\Gcal,\Hcal$ be sheaves of Abelian groups on $X$ fitting into a diagram 
    $$% https://q.uiver.app/#q=WzAsMyxbMCwwLCJcXEZjYWwiXSxbMSwwLCJcXEdjYWwiXSxbMiwwLCJcXEhjYWwiXSxbMCwxLCJcXHBoaSJdLFsxLDIsIlxccHNpIl1d
    \begin{tikzcd}
        \Fcal & \Gcal & \Hcal.
        \arrow["\phi", from=1-1, to=1-2]
        \arrow["\psi", from=1-2, to=1-3]
    \end{tikzcd}$$
    The diagram is exact at $\Gcal$ if $\img(\phi)=\ker(\psi)$. 
\end{definition}
\begin{definition}[Short Exact Sequence]\label{def: short exact sequence}
    Let $X$ be a topological space and $\Fcal,\Gcal,\Hcal$ be sheaves of Abelian groups on $X$ fitting into a diagram 
    $$% https://q.uiver.app/#q=WzAsNSxbMSwwLCJcXEZjYWwiXSxbMiwwLCJcXEdjYWwiXSxbMywwLCJcXEhjYWwiXSxbMCwwLCIwIl0sWzQsMCwiMCJdLFswLDEsIlxccGhpIl0sWzEsMiwiXFxwc2kiXSxbMywwXSxbMiw0XV0=
    \begin{tikzcd}
        0 & \Fcal & \Gcal & \Hcal & 0.
        \arrow[from=1-1, to=1-2]
        \arrow["\phi", from=1-2, to=1-3]
        \arrow["\psi", from=1-3, to=1-4]
        \arrow[from=1-4, to=1-5]
    \end{tikzcd}$$
    The diagram is a short exact sequence if $\img(\phi)=\ker(\psi)$.
\end{definition}
One fundamental operation on sheaves, however, does not preserve short exact sequences. 
\begin{proposition}\label{prop: sections does not preserve exactness}
    Let $X$ be a topological space and 
    $$% https://q.uiver.app/#q=WzAsNSxbMSwwLCJcXEZjYWwiXSxbMiwwLCJcXEdjYWwiXSxbMywwLCJcXEhjYWwiXSxbMCwwLCIwIl0sWzQsMCwiMCJdLFswLDEsIlxccGhpIl0sWzEsMiwiXFxwc2kiXSxbMywwXSxbMiw0XV0=
    \begin{tikzcd}
        0 & \Fcal & \Gcal & \Hcal & 0
        \arrow[from=1-1, to=1-2]
        \arrow["\phi", from=1-2, to=1-3]
        \arrow["\psi", from=1-3, to=1-4]
        \arrow[from=1-4, to=1-5]
    \end{tikzcd}$$
    a short exact sequence of sheaves on $X$. Then 
    $$% https://q.uiver.app/#q=WzAsNCxbMSwwLCJcXEdhbW1hKFgsXFxGY2FsKSJdLFsyLDAsIlxcR2FtbWEoWCxcXEdjYWwpIl0sWzMsMCwiXFxHYW1tYShYLFxcSGNhbCkiXSxbMCwwLCIwIl0sWzMsMF0sWzAsMV0sWzEsMl1d
    \begin{tikzcd}
        0 & {\Gamma(X,\Fcal)} & {\Gamma(X,\Gcal)} & {\Gamma(X,\Hcal)}
        \arrow[from=1-1, to=1-2]
        \arrow[from=1-2, to=1-3]
        \arrow[from=1-3, to=1-4]
    \end{tikzcd}$$
    is exact. 
\end{proposition}
\begin{proof}
    \todo{To do.}
\end{proof}
\begin{remark}
    In particular, the final morphism fails to be surjective. 
\end{remark}
However, as we have seen before, an exactness condition can be deduced on stalks. 
\begin{proposition}\label{prop: exactness on stalks}
    Let $X$ be a topological space and 
    $$% https://q.uiver.app/#q=WzAsNSxbMSwwLCJcXEZjYWwiXSxbMiwwLCJcXEdjYWwiXSxbMywwLCJcXEhjYWwiXSxbMCwwLCIwIl0sWzQsMCwiMCJdLFswLDEsIlxccGhpIl0sWzEsMiwiXFxwc2kiXSxbMywwXSxbMiw0XV0=
    \begin{tikzcd}
        0 & \Fcal & \Gcal & \Hcal & 0
        \arrow[from=1-1, to=1-2]
        \arrow[from=1-2, to=1-3]
        \arrow[from=1-3, to=1-4]
        \arrow[from=1-4, to=1-5]
    \end{tikzcd}$$
    a diagram of sheaves on $X$. The following are equivalent:
    \begin{enumerate}[label=(\alph*)]
        \item $$% https://q.uiver.app/#q=WzAsNSxbMSwwLCJcXEZjYWwiXSxbMiwwLCJcXEdjYWwiXSxbMywwLCJcXEhjYWwiXSxbMCwwLCIwIl0sWzQsMCwiMCJdLFswLDEsIlxccGhpIl0sWzEsMiwiXFxwc2kiXSxbMywwXSxbMiw0XV0=
    \begin{tikzcd}
        0 & \Fcal & \Gcal & \Hcal & 0
        \arrow[from=1-1, to=1-2]
        \arrow[from=1-2, to=1-3]
        \arrow[from=1-3, to=1-4]
        \arrow[from=1-4, to=1-5]
    \end{tikzcd}$$
    is a short exact sequence of sheaves on $X$. 
    \item For all $x\in X$ the diagram 
    $$% https://q.uiver.app/#q=WzAsNSxbMSwwLCJcXEZjYWwiXSxbMiwwLCJcXEdjYWwiXSxbMywwLCJcXEhjYWwiXSxbMCwwLCIwIl0sWzQsMCwiMCJdLFswLDEsIlxccGhpIl0sWzEsMiwiXFxwc2kiXSxbMywwXSxbMiw0XV0=
    \begin{tikzcd}
        0 & \Fcal_{x} & \Gcal_{x} & \Hcal_{x} & 0
        \arrow[from=1-1, to=1-2]
        \arrow[from=1-2, to=1-3]
        \arrow[from=1-3, to=1-4]
        \arrow[from=1-4, to=1-5]
    \end{tikzcd}$$
    is a short exact sequence of Abelian groups. 
    \end{enumerate}
\end{proposition}
\begin{proof}
    \todo{To do.}
\end{proof}
In summary, we can deduce the following theorem. 
\begin{theorem}\label{thm: sheaves of Abelian groups form an Abelian category}
    Let $X$ be a topolgoical space. The category $\Sh(X)$ of sheaves of Abelian groups on $X$ is an Abelian category. 
\end{theorem}
\begin{proof}
    \todo{To do.}
\end{proof}
Let us now consider how these sheaf categories behave with respect to continuous maps of topological spaces. 
\begin{definition}[Direct Image Sheaf]\label{def: direct image sheaf}
    Let $f:X\to Y$ be a continuous map of topological spaces and $\Fcal$ a sheaf of Abelian groups on $X$. The direct image sheaf $f_{*}\Fcal$ is the sheaf on $Y$ given by $V\mapsto f_{*}\Fcal(V)=\Fcal(f^{-1}(V))$. 
\end{definition}
\begin{definition}[Inverse Image Sheaf]\label{def: inverse image sheaf}
    Let $f:X\to Y$ be a continuous map of topological spaces and $\Gcal$ a sheaf of Abelian groups on $Y$. The inverse image sheaf $f^{-1}\Gcal$ is the sheaf on $X$ given by $U\mapsto(\colim_{f(U)\subseteq V}\Gcal(V))^{\#}$. 
\end{definition}
This data amalgamates into a functor in the evident way. Moreover, we can show the following exactness properties. 
\begin{proposition}\label{prop: exactness of inverse image}
    Let $f:X\to Y$ be a continuous map of topological spaces. Then $f^{-1}:\Sh(Y)\to\Sh(X)$ is an exact functor. 
\end{proposition}
\begin{proposition}\label{prop: left exactness of direct image}
    Let $f:X\to Y$ be a continuous map of topological spaces. Then $f^{*}:\Sh(X)\to\Sh(Y)$ is a left exact functor. 
\end{proposition}
\section{Lecture 5 -- 22nd November 2024}\label{sec: lecture 5}
We continue our discussion of $q$-series and in particular a property of the modified Nahm sum considered in \Cref{corr: q-difference equation of modified 1x1 Nahm sum}.\marginpar{As it stands, the proofs and structure for this lecture are more rough than usual, and will be updated in due course.}

The following definition is due to Konsevich-Soibelman \cite{DTInvariants}.
\begin{definition}[Admissable Series]\label{def: admissable series}
    A series $f\in\ZZ((q))[[t]]$ is admissable if it can be written as 
    $$f=\prod_{n\geq1}\prod_{i\in\ZZ}(q^{i}t^{n};q)_{\infty}^{a_{n,i}}$$
    such that for each $n$ only finitely many $a_{n,i}$ are nonzero. 
\end{definition}
\begin{remark}
    These $a_{n,i}$'s are precisely Donaldson-Thomas invariants. 
\end{remark}
Admissable series force an algebraicity condition on the $q$, allowing $f$ to be written as an element of $\ZZ[q][[t]]$. Up to a condition on the residue of the series $f$ mod $(t)$, series in $\ZZ((q))[[t]]$ admit such an expansion. 
\begin{proposition}
    Let $f\in\ZZ((q))[[t]]$. If $f\equiv1\pmod{(t)}$ then $f$ admits a unique expansion as an admissable series. 
\end{proposition}
This result is in fact much more general and it can be shown that the modified Nahm sum 
$$f_{a}(t,q)=\sum_{n\geq0}(-1)^{an}\frac{q^{\frac{1}{2}an^{2}-\frac{1}{2}an}}{(q;q)_{n}}t^{n}$$
as previously defined is admissable. The original proof is highly involved, and we will instead offer a simpler exposition of the same result. Recall from \Cref{prop: logarithm at worst simple poles at roots of unity}, we have 
\begin{equation}\label{eqn: modified Nahm sum as exponent of sum}
    (q^{i}t;q)_{\infty}=\exp\left(-\sum_{\ell\geq 1}\frac{1}{\ell}\cdot\frac{q^{i\ell}t^{n\ell}}{1-q^{\ell}}\right)
\end{equation}
and further recall that $\ZZ((q))[[t]]$ is a $\lambda$-ring -- admits an action with $\NN$ as a multiplicative monoid -- and admits Adams operations $\psi_{n}:\ZZ((q))[[t]]\to\ZZ((q))[[t]]$ by $t\mapsto t^{n},q\mapsto q^{n}$. The Adams operations allow us to rewrite (\ref{eqn: modified Nahm sum as exponent of sum}) as 
\begin{equation}\label{eqn: modified Nahm sum as exponent of Adams sum}
    \exp\left(-\sum_{\ell\geq1}\psi_{\ell}\left(\frac{q^{i}t^{n}}{1-q}\right)\right).
\end{equation}
We introduce the notion of the plethystic exponential.
\begin{definition}[Plethystic Exponential]\label{def: Plethystic exponential}
    Let $A$ be a $\lambda$-ring and $\sum_{\ell\geq 1}\frac{a_{\ell}}{\ell}$ a convergent series in $A$. The plethystic exponential is of the series is given by 
    $$\exp\left(\sum_{\ell\geq 1}\frac{1}{\ell}\psi_{\ell}(a_{\ell})\right).$$
\end{definition}
Now taking 
$$\phi(t,q)=-\sum_{n\geq1}\sum_{i\in\ZZ}a_{n,i}q^{i}t^{n}\in\ZZ((q))[[t]]$$
writing $\frac{\phi(t,q)}{1-q}$ as the plethystic logarithm of $f_{a}(t,q)$, inverse to the plethystic exponential, the coefficients of the expansion as an admissable series will be those coefficients of the plethystic logarithm since the plethystic exponential gives an isomorphism $t\QQ[q^{\pm}][[t]]\to 1+t\QQ[q^{\pm}][[t]]$. It thus suffices to show that the plethystic logarithm of $f_{a}$ is a function that is a sum $\frac{\phi_{0}(t)}{1-q}$ with an element of $\QQ[q^{\pm}][[t]]$. 

Now using the ansatz
\begin{equation}\label{eqn: plethystic exponential ansatz}
    f_{a}(t,q)=\exp\left(\sum_{\ell\geq 1}\frac{1}{\ell}\frac{\phi_{0}(t^{\ell})}{1-q^{\ell}}\right)g_{a}(t,q)
\end{equation}
we show that there exists a chioce of function $\psi_{0}(t)$ so that the remainder $g_{a}(t,q)\in 1+t\QQ[q^{\pm}][[t]]$ and from which the result would follow by application of the Plethystic exponential. But a choice of $\phi_{0}\in t\cdot\QQ[[t]]$ can be made  such that $\sum_{\ell\geq 1}\frac{\phi_{0}(t^{\ell})}{\ell^{2}}=-V(t)$. 

This is gives the desired result as stated below. 
\begin{theorem}[Kontsevich-Soibelman, Efimov]\label{thm: modified 1x1 Nahm sum is admissable}
    The $q$-series 
    $$f_{a}(t,q)=\sum_{n\geq0}(-1)^{an}\frac{q^{\frac{1}{2}an^{2}-\frac{1}{2}an}}{(q;q)_{n}}t^{n}$$
    is admissable. 
\end{theorem}


\section{Lecture 6 -- 25th October 2024}\label{sec: lecture 6}
We consider some applications of partitions of unity. 
\begin{definition}[Bump Function]\label{def: bump function}
    Let $X$ be a topological space and $A,U\subseteq X$ be a closed and open subspace, respectively. A bump function for $A$ supported in $U$ is a function $\psi:X\to\RR$ such that $\psi|_{A}\equiv1$ and $\supp(\psi)\subseteq U$. 
\end{definition}
Using partitions of unity, we can construct a bump function supported on any clsoed subset of a smooth manifold. 
\begin{proposition}\label{prop: existence of bump functions on smooth manifolds}
    Let $M$ be a smooth manifold, $A\subseteq M$ closed and $U\subseteq M$ an open neighborhood of $A$. There exists a smooth bump function for $A$ supported in $U$. 
\end{proposition}
\begin{proof}
    Let $U$ be as above and $V=M\setminus A$. These form an open cover $\{U,V\}$ of $M$. By \Cref{thm: existence of partitions of unity}, there exists a partition of unity $\psi_{U},\psi_{V}$ subordinate to this cover. $\psi_{U}$ suffices since $\psi_{V}\equiv0$ on $A$ and $\psi_{U}+\psi_{V}\equiv 1$ so it is idenitcally 1 on $A$.
\end{proof}
We also make the following definition, allowing us to consider smooth functions on a closed subset of a smooth manifold. 
\begin{definition}[Smooth Function on Closed Subset]\label{def: smooth on closed}
    Let $M,N$ be smooth manifolds and $A\subseteq M$ a closed subset. A continuous map $f:M\to N$ is smooth on $A$ if it admits a smooth extension in an open neighborhood of each $x\in A$. 
\end{definition}
\begin{remark}
    Recall from \Cref{def: smooth functions on upper half space} that we require that for each $x\in A$ there is an open neighborhood $U_{x}$ of $x$ such that $\widetilde{f}:U_{x}\to N$ is smooth and $\widetilde{f}|_{U_{x}\cap A}=f|_{U_{x}\cap A}$. 
\end{remark}
In particular, we can construct smmooth functions from a closed set of a smoth manifold to $\RR^{m}$ whose support is contained in an fixed open neighborhood of $A$. 
\begin{proposition}\label{prop: smooth function on closed with fixed support}
    Let $M$ be a smooth manifold, $A\subseteq M$ closed, and $f:A\to\RR^{m}$ a smooth function. For $U\subseteq M$ containing $A$, there exists a smooth extension $\widetilde{f}:M\to\RR^{m}$ such that $\widetilde{f}|_{A}=f$ and $\supp(\widetilde{f})\subseteq U$. 
\end{proposition}
\begin{proof}
    For each $x\in A$ consider a neighborhood $W_{x}\subseteq U$ containing $x$. By hypothesis, there exists $\widetilde{f}_{x}:W_{x}\to\RR^{m}$ such that $\widetilde{f}_{x}|_{W_{x}\cap A}=f|_{W_{x}\cap A}$. Now note that $\{W_{p}\}_{p\in A}\cup\{M\setminus A\}$ is an open cover of $M$. Let $\{\psi_{x}\}_{x\in A}\cup\{\psi_{M\setminus A}\}$ be a partition of unity subordinate to the cover. Set $\widetilde{f}(y)=\sum_{x\in A}\widetilde{f}_{x}(y)\psi_{x}(y)$. Observe that since $\{\supp(\psi_{x})\}_{x\in A}$ is locally finite, only fintely many terms of the sum are nonzero in a neighborhood of any point of $M$ showing $\widetilde{f}$ is smooth. Additionally, if $y\in A$ then $\psi_{M\setminus A}(y)=0$ and thus $\widetilde{f}(y)=\psi_{M\setminus A}(y)+\sum_{x\in A}\widetilde{f}_{x}(y)\psi_{x}(y)=f(y)$ showing $\widetilde{f}$ is an extension of $f$. Finally, the condition on the support holds as $\supp(\psi_{x})\subseteq U$ for each $p$
\end{proof}
\begin{remark}
    \Cref{prop: smooth function on closed with fixed support} fails in the category of topological manifolds. Take $A=S^{1}\subseteq\RR^{2}$ and $f:S^{1}\to S^{1}$ the identity map. $f$ does not admit a continuous extension to $\RR^{2}$ obstructed by the homotopy groups. 
\end{remark}
Partitions of unity can also be used to construct smooth functions with prescribed properties. One example of this is in exhaustion functions. 
\begin{definition}[Exhaustion Function]\label{def: exhaustion function}
    Let $X$ be a topological space. An exhaustion function $f:X\to\RR$ is a continuous function such that for all $c\in\RR$ $f^{-1}((-\infty,c])\subseteq X$ is open. 
\end{definition}
\begin{remark}
    This recovers the notion of an exhaustion by compact sets indexing over the natural numbers. 
\end{remark}
\begin{example}
    $f:\RR\to\RR$ by $x\mapsto x^{2}$ has closed and hence compact preimage. 
\end{example}
\begin{example}
    $f:\RR\to\RR$ by $x\mapsto x$ is a non-example since the preimage of $(-\infty,c]$ which is $(-\infty,c]$ is non-compact. 
\end{example}
It can be shown that every smooth manifold admits a smooth exhaustion function. 
\begin{proposition}\label{prop: existence of smooth exhaustion function}
    If $M$ is a smooth manifold, $M$ admits a smooth positive exhaustion function. 
\end{proposition}
\begin{proof}
    Let $\{U_{i}\}_{i=1}^{\infty}$ be a countable open cover having compact closure and $\{\psi_{i}\}_{i=1}^{\infty}$ a partition of unity subordinate to this cover. Set $f(x)=\sum_{i=1}^{\infty}i\cdot\psi_{i}(x)$. $f$ is smooth since only finitely many terms of the sum are nonzero in the neighborhood of any point and positive by construcion. 

    To show that $f$ is an exhaustion function, we have for any $c\in\RR$ and a natural number $N>c$ that $f^{-1}((-\infty,c])$ is a closed subset of the compact set $f^{-1}((-\infty,N])=\bigcup_{i=1}^{N}\overline{U_{i}}$ and hence compact. If $x\notin\bigcup_{i=1}^{N}\overline{U_{i}}$ then 
    $$f(x)=\sum_{i=N+1}^{\infty}i\psi_{i}(x)\geq\sum_{i=N+1}^{\infty}N\psi_{i}(x)=N\sum_{i=1}^{\infty}\psi_{i}(x)=N>c$$
    showing that $x\notin f^{-1}((-\infty,c])$ and conversely if $f(x)\leq c$ then $x\in\bigcup_{i=1}^{N}\overline{U_{i}}$ showing that $f^{-1}((-\infty,c])\subseteq \bigcup_{i=1}^{N}\overline{U_{i}}$ yielding the claim.  
\end{proof}
The construction of exhaustion functions is closely linked to the fact that any closed subset of a smooth manifold can be obtained as the preimage of 0 of a smooth function $f:M\to\RR$. We deduce the general case as a consequence of the following lemma. 
\begin{lemma}\label{lem: closed subset is level set of Rn}
    Let $A\subseteq\RR^{n}$ be a closed subset. There exists a nonnegative smooth function $f:\RR^{n}\to\RR$ such that $f^{-1}(0)=A$. 
\end{lemma}
\begin{proof}
    Let $\{B_{r_{i}}(x_{i})\}_{i=1}^{\infty}$ be a countable open cover of $M\setminus A$ by balls. By \Cref{lem: existence of cutoff functions}, there exists a cutoff function $H$ that is equal to 1 on $\overline{B_{1/2}(0)}$ and supported in $B_{1}(0)$.  Now for each $i$, let $C_{i}\geq 1$ be a constant such that $C_{i}>\sup_{x\in\RR^{n}}\{\partial^{\alpha}H:|\alpha|\leq i\}$. We show
    $$f(x)=\sum_{i=1}^{\infty}\frac{r_{i}^{i}}{2^{i}C_{i}}H\left(\frac{x-x_{i}}{r_{i}}\right)$$
    suffices. 

    First note that each term $\frac{r_{i}^{i}}{2^{i}C_{i}}H\left(\frac{x-x_{i}}{r_{i}}\right)$ is bounded by $\frac{1}{2^{i}}$ since $r_{i}<1$ and that the sequence $\sum_{i=1}^{\infty}\frac{1}{2^{i}}$ converges so the function is continuous. To see that $f$ is smooth, we proceed by induction on the hypothesis that partial derivatives of order up to $k$ exist and are continuous. Noting that an order $k+1$ partial derivative is of the form 
    $$\partial^{\alpha}\left(\frac{r^{i}}{2^{i}C_{i}}H\left(\frac{x-x_{i}}{r_{i}}\right)\right)=\frac{r^{i}}{2^{i}C_{i}}\partial^{\alpha}\left(H\left(\frac{x-x_{i}}{r_{i}}\right)\right)$$
    which is once again bounded above by $\frac{1}{2^{i}}$ by construction of $H$ so repeating the argument above, this is continuous showing $f$ is smooth. 
\end{proof}
We deduce the following general statement. 
\begin{theorem}\label{thm: closed subset is level set of smooth manifold}
    Let $M$ be a smooth manifold and $A\subseteq M$ closed. There exists a nonnegative smooth function $f:M\to\RR$ such that $f^{-1}(0)=A$. 
\end{theorem}
\begin{proof}
    Let $M$ be a smooth manifold as stated and $A\subseteq M$ closed. Let $(\phi_{\alpha},U_{\alpha})_{\alpha\in\Acal}$ be an atlas of $M$. Without loss of generality, we can take $\phi_{\alpha}(U_{\alpha})=\RR^{n}$ for some fixed $n$. Furthermore, $A\cap U_{\alpha}$ is closed in the subspace topology of $U_{\alpha}$ and thus has closed image $\phi_{\alpha}(A\cap U_{\alpha})\subseteq\phi_{\alpha}(U_{\alpha})=\RR^{n}$. By \Cref{lem: closed subset is level set of Rn} there are functions $f_{\alpha}:\RR^{n}\to[0,\infty)$ such that $f^{-1}_{\alpha}(0)=\phi_{\alpha}(A\cap U_{\alpha})$. Moreover, the functions $f_{\alpha}\circ\phi_{\alpha}:U_{\alpha}\to\RR$ are smooth since they are the composite of smooth functions. It remains to glue these $f_{\alpha}\circ\phi_{\alpha}$ into a smooth function $M\to[0,\infty)$. Let $(\psi_{\alpha})_{\alpha\in\Acal}$ be a partition of unity subordinate to the cover $\{U_{\alpha}\}_{\alpha\in\Acal}$. We show that the function $f(x)=\sum_{\alpha\in\Acal}\left(\psi_{\alpha}(x)\cdot(f_{\alpha}\circ\phi_{\alpha})(x)\right)$ which is smooth as the product and sum of smooth functions. We have $A=f^{-1}(0)$ since the summand $\psi_{\alpha}(x)\cdot (f_{\alpha}\circ\phi_{\alpha})(x)$ is zero if and only if $x\in A\cap U_{\alpha}$ and undefined on the complement $M\setminus U_{\alpha}$ so preimage of 0 is given by the union of the component functions.
\end{proof}
Having discussed partitions of unity, a central technical tool in the study of smooth manifolds, we begin a consideration of tangent vectors. 
\begin{definition}[Tangent Space]\label{def: tangent space}
    Let $M$ be a smooth manifold and $x\in M$. The tangent space $T_{x}M$ of $M$ at $x$ is the set of equivalence classes of smooth curves $\gamma:(-\varepsilon,\varepsilon)\to M$ such that $\gamma(0)=x$ where $\gamma\sim\gamma'$ if and only if for every smooth function $f$ defined near $p$ there is an equality $(f\circ\gamma)'(0)=(f\circ\gamma')'(0)$. 
\end{definition}
\begin{remark}
    This more abstract definition may seem quite foreign to those encountered in single and multivariable calculus. However, providing an embedding-independent description of tangent spaces is required for a study of smooth manifolds in the appropriate generality. 
\end{remark}
The formation of these tangent spaces behave well with respect to smooth maps. 
\begin{definition}[Differential of a Smooth Map]\label{def: differential of a smooth map}
    Let $f:M\to N$ be a morphism of smooth manifolds. The differential of $f$ at $x$ is the map $df_{x}:T_{x}M\to T_{f(x)}N$ by $[\gamma]\mapsto[f\circ\gamma]$. 
\end{definition}
\begin{remark}
    The map is well-defined and in fact functorial. 
\end{remark}
\section{Lecture 7 -- 28th October 2024}\label{sec: lecture 7}
We continue our discussion of \v{C}ech cohomology and its relation to derived functor cohomology. We first define a sheaf-variant of \v{C}ech cohomology. 
\begin{definition}[\v{C}ech Complex of Sheaves]\label{def: Cech complex of sheaves}
    Let $X$ be a topological space, $\{U_{i}\}_{i\in I}$ a cover of $X$ with $I$ a totally ordered set, and $\Fcal$ a sheaf on $X$. The \v{C}ech complex of sheaves 
    $$\Ccal^{0}(\{U_{i}\}_{i\in I},\Fcal)\to\Ccal^{1}(\{U_{i}\}_{i\in I},\Fcal)\to\Ccal^{2}(\{U_{i}\}_{i\in I},\Fcal)\to\dots$$
    is the chain complex of sheaves on $X$ with 
    $$\Ccal^{p}(\{U_{i}\}_{i\in I},\Fcal)=j_{*}\left(\prod_{i_{0}<i_{1}<\dots<i_{p}}\Fcal|_{U_{i_{0}}\cap\dots\cap U_{i_{p}}}\right)$$
    for $j:U_{i_{0}}\cap\dots\cap U_{i_{p}}\hookrightarrow X$ the inclusion map, and differentials those induced by taking sections on open sets. 
\end{definition}
\begin{remark}
    As such, $\Ccal^{\bullet}(\{U_{i}\}_{i\in I},\Fcal)$ is a complex of sheaves on $X$ with the property that $\Gamma(X,\Ccal^{p}(\{U_{i}\}_{i\in I},\Fcal))=C^{p}(\{U_{i}\}_{i\in I},\Fcal)$. 
\end{remark}
This complex is in fact a long exact sequence of sheaves. 
\begin{proposition}\label{prop: Cech complex of sheaves is a long exact sequence}
    Let $X$ be a topological space, $\{U_{i}\}_{i\in I}$ a cover of $X$ with $I$ a totally ordered set, and $\Fcal$ a sheaf on $X$. The \v{C}ech complex of sheaves 
    $$\Ccal^{0}(\{U_{i}\}_{i\in I},\Fcal)\to\Ccal^{1}(\{U_{i}\}_{i\in I},\Fcal)\to\Ccal^{2}(\{U_{i}\}_{i\in I},\Fcal)\to\dots$$
    is a long exact sequence of sheaves on $X$. 
\end{proposition}
\begin{proof}
    See \cite[Lem. 4.2]{Hartshorne}. 
\end{proof}
The formation of this complex behaves as expected on flasque sheaves. 
\begin{proposition}
    Let $X$ be a topological space, $\{U_{i}\}_{i\in I}$ a cover of $X$ with $I$ a totally ordered set, and $\Fcal$ a flasque sheaf on $X$. Then $\check{H}^{p}(\{U_{i}\}_{i\in I},\Fcal)=0$ for $p>0$. 
\end{proposition}
\begin{proof}
    For $\Fcal$ flasque, $\Ccal^{\bullet}(\{U_{i}\}_{i\in I},\Fcal)$ is a complex of flasque sheaves by construction so $\check{H}^{p}(\{U_{i}\}_{i\in I},\Fcal)=R^{i}\Gamma(X,\Ccal^{\bullet}(\{U_{i}\}_{i\in I},\Fcal))=0$.
\end{proof}
We anre now prepared to show the comparison theorem with derived functor cohomology. 
\begin{theorem}\label{thm: Cech to derived functor comparison}
    Let $X$ be a topological space, $\{U_{i}\}_{i\in I}$ a cover of $X$ with $I$ a totally ordered set, and $\Fcal$ a sheaf on $X$. There is a functorial comparison morphism $\check{H}^{i}(\{U_{i}\}_{i\in I},\Fcal)\to H^{i}(X,\Fcal)$ for all $i$. 
\end{theorem}
\begin{proof}
    Using the long exact sequence with the \v{C}ech complex of sheaves $0\to\Fcal\to\Ccal^{0}(\{U_{i}\}_{i\in I},\Fcal)\to\Ccal^{1}(\{U_{i}\}_{i\in I},\Fcal)\to\dots$ and an injective resolution $0\to\Fcal\to\Ical_{0}\to\Ical_{1}\to\dots$, the universal property of injective objects induces canonical maps $\Ccal^{i}(\{U_{i}\}_{i\in I},\Fcal)\to\Ical_{i}$ descending to a canonical map on cohomology. 
\end{proof}
In the case where a certain condition on higher cohomology is satisfied -- and as we will show holds in the case of schemes -- this functorial comparison morphism is an isomorphism. 
\begin{proposition}\label{prop: comparison between Cech and derived functor cohomology is an isomorphism}
    Let $X$ be a topological space, $\{U_{i}\}_{i\in I}$ a cover of $X$ with $I$ a totally ordered set. If $\Fcal$ a sheaf on $X$ such that $H^{p}(U_{i}\cap U_{j},\Fcal)=0$ for all $i,j\in I$ and $p\geq 1$ then there is an isomorphism $\check{H}^{i}(\{U_{i}\}_{i\in I},\Fcal)\cong H^{i}(X,\Fcal)$.
\end{proposition}
\begin{proof}
    Let $\Fcal$ be such a sheaf, in which case $R^{p}\Gamma(X,\Ccal^{k}(\{U_{i}\}_{i\in I},\Fcal))=0$ for $p>0, k\geq0$ and the result follows from \Cref{thm: Cech to derived functor comparison}. 
\end{proof}
We can also define a cover-independent variant of \v{C}ech cohomology as follows. 
\begin{definition}[Refinement of Cover]\label{def: refinement of cover}
    Let $X$ be a topological space and $\{U_{i}\}_{i\in I}$ and $\{V_{j}\}_{j\in J}$ be two covers of $X$. The cover $\{V_{j}\}_{j\in J}$ refines $\{U_{i}\}_{i\in I}$ if there is an order-preserving function $\rho:J\to I$ such that $V_{j}\subseteq U_{\rho(j)}$. 
\end{definition}
Note that for a refinement $\{V_{j}\}_{j\in J}$ of $\{U_{i}\}_{i\in I}$ the termwise map on \v{C}ech complexes 
$$\prod_{\rho^{-1}(j_{0})<\dots<\rho^{-1}(j_{p})}\Fcal(U_{\rho^{-1}(j_{0})}\cap\dots\cap U_{\rho^{-1}(j_{p})})\to\prod_{j_{0}<\dots<j_{p}}\Fcal(U_{j_{0}}\cap\dots\cap U_{j_{p}})$$
inducing a map on cohomology $\check{H}^{p}(\{U_{i}\}_{i\in I},\Fcal)\to\check{H}^{p}(\{V_{j}\}_{j\in J},\Fcal)$. Absolute \v{C}ech cohomology is defined by passage to colimits on refinements of covers. 
\begin{definition}[Absolute \v{C}ech Cohomology]\label{def: absolute Cech cohomology}
    Let $X$ be a topological space and $\Fcal$ a sheaf on $X$. The absolute \v{C}ech cohomology $\check{H}^{p}(X,\Fcal)$ is given by 
    $$\colim_{\substack{\{U_{i}\}_{i\in I}\to\{V_{j}\}_{j\in J} \\ \{V_{j}\}_{j\in J} \text{ refines } \{U_{i}\}_{i\in I}}}\check{H}^{p}(\{U_{i}\}_{i\in I},\Fcal).$$
\end{definition}
\begin{remark}
    It is often the case that \v{C}ech cohomology with respect to a cover is not equivalent to derived functor cohomology, but absolute \v{C}ech cohomology is.
\end{remark}
\begin{remark}
    It can be shown that derived functor cohomology agrees with absolute \v{C}ech cohomology in degree 1, that is, $\check{H}^{1}(X,\Fcal)\cong H^{1}(X,\Fcal)$, though not in general. 
\end{remark}

We are now prepared to define schemes, which we will exhibit as a special class of ringed spaces. 
\begin{definition}[Ringed Space]\label{def: ringed space}
    A ringed space $(X,\Ocal_{X})$ consists of a topological space $X$ and a sheaf of rings $\Ocal_{X}$ on $X$ known as the structure sheaf on $X$. 
\end{definition}
\begin{definition}[Morphism of Ringed Spaces]\label{def: morphism of ringed spaces}
    A morphism of ringed spaces $(f,f^{\sharp}):(X,\Ocal_{X})\to (Y,\Ocal_{Y})$ is the data of a continuous map $f:X\to Y$ and a morphism $f^{\sharp}:\Ocal_{Y}\to f_{*}\Ocal_{X}$ of sheaves of rings on $Y$. 
\end{definition}
\begin{remark}
    Note that $f_{*}\Ocal_{X}$ is automatically a sheaf of rings on $Y$ with no need for sheafification since $\Ocal_{X}$ is a sheaf on $X$ (cf. \Cref{def: direct image sheaf}). 
\end{remark}
\begin{remark}
    In many cases the induced map on sheaves $f^{\sharp}$ is obvious from the morphism $f$ and will be left implicit. 
\end{remark}
Let us consider some examples. 
\begin{example}
    Let $V$ be a $\CC$-vector space and $U\subseteq V$ open. The sheaf of holomorphic functions on $U$ endows $U$ with the structure of a ringed space with the natural restriction maps. 
\end{example}
\begin{example}
    Let $f:X\to Y$ be a continuous map of topological spaces. There is a natural map $C_{Y}\to f_{*}C_{X}$ of sheaves of continuous functions on $Y$ by composition. This is as such a morphism of ringed spaces. 
\end{example}
\section{Lecture 8 -- 13th December 2024}\label{sec: lecture 8}
We discuss the computation of algebraic $K$-theory groups of a field $F$ in degrees $\leq 3$. In fact, the first nontrivial case is $K_{3}$ as the 0th, 1st, and 2nd $K$-theory groups are given by $\ZZ$, $F^{\times}$, and $K^{\mathsf{M}}_{2}(F)$ the Milnor $K$-theory of $F$, respectively. 

In the case of interest, computing $K_{3}$ of a number field, it suffiecs to understand the homology of $\GL_{2}(F)$ in degrees at most 3. To do so, we consider the action of $\GL_{2}(F)$ on $\PP^{1}(F)$ by linear fractional transformations. The computation relies on the following lemma. 
\begin{proposition}\label{prop: resolutions by free Abelian groups}
    Let $(\PP^{1}(F))^{n}_{\neq}$ be the set of $n$ pairwise distinct points on $\PP^{1}(F)$. There is a functorial exact complex 
    $$\dots\to\ZZ[(\PP^{1}(F))^{2}_{\neq}]_{\Sigma_{2}}\to\ZZ[\PP^{1}(F)]\to\ZZ\to0$$
    where $(-)_{\Sigma_{n}}$ denotes the coinvariants of the natural action of the symmetric groups which is an exact complex of $\GL_{2}(F)$-modules. 
\end{proposition}
The resolution of \Cref{prop: resolutions by free Abelian groups} produces a spectral sequence converging to $H_{i+j}(*/\GL_{2}(F))$ with $E_{1}$-page given as in (\ref{diag: sseq E1 page}). We compute the homology in each case. We do so degree by degree in the complex of \Cref{prop: resolutions by free Abelian groups}
\begin{proposition}\label{prop: action on P1}
    The action of $\GL_{2}(F)$ on $\PP^{1}(F)$ by linear fractional transformations is transitive, and the stabilizer of $\infty$ is given by the Borel subgroup $\Bcal_{2}(F)$ of upper triangular matrices. Moreover, 
    $$H_{i}(*/\Bcal_{2}(F))\otimes\QQ\cong H_{i}(*/(F^{\times})^{2})\cong\bigwedge^{i}\left((F^{\times})^{2}\otimes\QQ\right).$$
\end{proposition}
In the case of $\ZZ[(\PP^{1}(F))^{2}_{\neq}]_{\Sigma_{2}}$, we have the following. 
\begin{proposition}\label{prop: action on P1 2}
    The homology of $H_{0}\left(\GL_{2}(F),\ZZ[(\PP^{1}(F))^{2}_{\neq}]_{\Sigma_{2}}\right)\otimes\QQ$ is given by 
    $$H_{0}\left(\GL_{2}(F),\ZZ[(\PP^{1}(F))^{2}_{\neq}]_{\Sigma_{2}}\right)\otimes\QQ=\begin{cases}
        0 & i = 0 \\ F^{\times}\otimes\QQ & i=1 \\ \bigwedge^{2}((F^{\times})^{2}\otimes\QQ)_{\Sigma_{2}} & i=2.
    \end{cases}$$
\end{proposition}
And in degrees $\ZZ[(\PP^{1}(F))^{3}_{\neq}]_{\Sigma_{3}}$ and $\ZZ[(\PP^{1}(F))^{4}_{\neq}]_{\Sigma_{4}}$, we have the following. 
\begin{proposition}\label{prop: action on P1 3}
    The action of $\GL_{2}(F)$ on $(\PP^{1}(F))^{3}_{\neq}$ factors over the simply transtive action of $\mathrm{PGL}_{2}(F)$ on $(\PP^{1}(F))^{3}_{\neq}$. In particular, the homology vanishes. 
\end{proposition}
\begin{proposition}\label{prop: action on P1 4}
    The 0th homology of $\ZZ[(\PP^{1}(F))^{4}_{\neq}]_{\Sigma_{4}}$ is given by 
    $$H_{0}\left(\GL_{2}(F),\ZZ[(\PP^{1}(F))^{4}_{\neq}]_{\Sigma_{4}}\right)\otimes\QQ=\QQ[F^{\times}\setminus\{1\}]_{\Sigma_{4}}.$$
\end{proposition}
This produces the necessary data for the $E_{1}$-page. For the $E_{2}$-page, we further need to understand $H_{0}\left(\GL_{2}(F),\ZZ[(\PP^{1}(F))^{5}_{\neq}]_{\Sigma_{5}}\right)\otimes\QQ$. 
\begin{proposition}\label{prop: action on P1 5}
    The 0th rational homology of $\ZZ[(\PP^{1}(F))^{5}_{\neq}]_{\Sigma_{5}}$ is given the $\QQ$-vector subspace of $\QQ[F^{\times}\setminus\{1\}]$ generated by the five term relations of the Bloch-Wigner dilogarithm. 
\end{proposition}
Substituting the results of \Cref{prop: action on P1,prop: action on P1 2,prop: action on P1 3,prop: action on P1 4} into (\ref{diag: sseq E1 page}) we get on the $E_{1}$-page (\ref{diag: sseq E1 page substituted}). In conjunction with \Cref{prop: action on P1 5}, we get on the $E_{2}$-page 
\begin{equation}\label{diag: sseq E2 page}
    % https://q.uiver.app/#q=WzAsMTAsWzMsMCwiXFxRUSJdLFszLDEsIigoRl57XFx0aW1lc30pXnsyfVxcb3RpbWVzXFxRUSkiXSxbMywyLCJcXGJpZ3dlZGdlXnsyfSgoRl57XFx0aW1lc30pXnsyfVxcb3RpbWVzXFxRUSkiXSxbMywzLCJcXGJpZ3dlZGdlXnszfSgoRl57XFx0aW1lc30pXnsyfVxcb3RpbWVzXFxRUSkiXSxbMiwwLCIwIl0sWzIsMSwiMCJdLFsyLDIsIjAiXSxbMSwwLCIwIl0sWzEsMSwiMCJdLFswLDAsIlxcd3AoRikiXV0=
\begin{tikzcd}
	{\wp(F)} & 0 & 0 & \QQ \\
	& 0 & 0 & {((F^{\times})^{2}\otimes\QQ)} \\
	&& 0 & {\bigwedge^{2}((F^{\times})^{2}\otimes\QQ)} \\
	&&& {\bigwedge^{3}((F^{\times})^{2}\otimes\QQ)}
\end{tikzcd}
\end{equation}
with no differentials. On the $E_{3}$-page, the map $\wp(F)\to\bigwedge^{2}((F^{\times})^{2}\otimes\QQ)$ recovers the second Bloch group \Cref{def: Bloch group} as the kernel and thus Bloch's result \Cref{thm: Bloch rational computation of K3} for $F=\KK$ a number field. 
\begin{remark}
    For this computation, one can also instead consider the action of the Picard groupoid $\sfPic(F)$ on the $K$-theory anima as in \Cref{def: K-theory of a ring} and consider the homotopy orbits $K(F)_{h(*/F^{\times})}\otimes\QQ$ which also recovers the pre-Bloch group as its third homotopy group. This exhibits $K(F)_{h(*/F^{\times})}$ as the group completion of $$\left(\coprod_{n\geq0}*/\GL_{n}(F)\right)_{h(*/F^{\times})}.$$
\end{remark}
Recalling \Cref{thm: Bloch rational computation of K3}, and the subsequent discussion we have regulator maps for each complex embedding $\tau:\KK\to\CC$ a Bloch regulator map $K_{3}(F)\cong B(F)\to\CC/(2\pi i)^{2}\QQ$ which can be extended to a map to $\RR$ by taking the imaginary part, with $\sum_{i}n_{i}[x_{i}]\mapsto\sum_{i}n_{i}D(\tau(x_{i}))$, here taking the Bloch-Wigner dilogarithm. 
\section{Lecture 9 -- 20th December 2024}\label{sec: lecture 9}
We consider the notion of relative $K$-theory, a variant of $K$-theory in which $K$-theoretic classes are more easily described, and is as computable as $K$-theory. In this way, relative $K$-theory can be seen as more flexible than $K$-theory. 
\begin{definition}[Relative $K$-Theory]\label{def: relative K-theory}
    Let $R$ be a ring and $M$ an Abelian group with a map $M\to R^{\times}$. The relative $K$-theory 
    $$K(R/\ZZ[M])= K(R)\otimes_{\SSS[*/M]}\SSS$$
    where $K(R)$ is considered as a spectrum over $\SSS[*/M]$ by the map $*/M\to */R^{\times}\to R$. 
\end{definition}
\begin{remark}
    By the universal property of group rings, the existence of a map $M\to R^{\times}$ is equivalent to the existence of a map from the group ring $\ZZ[M]$ to $R$. 
\end{remark}
\begin{remark}
    One can generalize \Cref{def: relative K-theory} to the setting of $M$ a commutative monoid $M$ to $R$ which corresponds to a map $\ZZ[M]\to R$. One can then define 
    $$K(R/\ZZ[M])=K^{\log}(R,M)\otimes_{\SSS[*/M^{\mathsf{gp}}]}\SSS$$
    where $K^{\log}$ is logarithmic $K$-theory. 
\end{remark}
\begin{remark}
    The construction of \Cref{def: relative K-theory} above is equivalent to taking the homotopy orbits $K(R)_{h(*/M)}$ which in the case of $F$ a field produces $K(F)_{h(*/F^{\times})}=K(F/\ZZ[F^{\times}])$. 
\end{remark}
The upshot of this construction is that all constructions in $K$-theory generalize to the setting of relative $K$-theory. Of importance to us are regulators and polylogarithms. In particular, the dilogarithm is most naturally expressed in the setting of relative $K$-theory. 

Moreover, relative $K$-theory overcomes the rigidity of $K$-theory in high weights. For example, it is conjectured that the Bloch group of $\CC$ and of $\overline{\QQ}$ coincide. On the other hand, classes in relative $K$-theory is plentiful. As such, the regulators are now highly interesting special functions and not merely numbers and provides a coherent organizing principle for functions like dilogarithms. 

These constructions also arise in $p$-adic geometry. Let $R$ be a smooth algebra over $\CC_{p}$, the $p$-completion of the algebraic closure of $\QQ_{p}$. We often pick a system of coordinates $T_{1},\dots,T_{d}\in R$ invertible and consider an \'{e}tale map $\CC_{p}[T_{1}^{\pm},\dots,T_{d}^{\pm}]\to R$ and pass to
$$R_{\infty}=R\otimes_{\CC_{p}[T_{1}^{\pm},\dots,T_{d}^{\pm}]}\CC_{p}\left[T_{1}^{\pm\frac{1}{p^{\infty}}},\dots,T_{d}^{\pm\frac{1}{p^{\infty}}}\right]$$
which on further passage to an appropriate completion is a perfectoid algebra. This is similar to the data required for relative $K$-theory since the group generated by $T_{1}^{\pm},\dots,T_{d}^{\pm}$ naturally maps to the units of $R$. In this case, the \'{e}tale $p$-complete relative $K$-theory 
$$K_{\mathsf{\acute{e}t}}(R/\ZZ[T_{1}^{\pm},\dots,T_{d}^{\pm}])^{\wedge}_{p}$$
turns out to be the $p$-completed \'{e}tale $K$-theory $K_{\mathsf{\acute{e}t}}(R_{\infty})$. Morally, what happens here is the $p$-power roots vanish in the $K$-theory of the $p$-completion. See \cite{PrismaticCohDelta} for a further discussion on this topic, viewing syntomic cohomology as a form of $p$-adic $K$-theory. 

Going forward, we will take the perspective of polylogarithms being relative $K$-theory classes and functional equations of polylogarithms are already identities of relative $K$-theory. 

\begin{example}
    Let $R=\ZZ[t^{\pm},\frac{1}{1-t}]$ and $M=t^{\ZZ}$. There is a natural map $M\to R^{\times}$. We have 
    $$K(R/\ZZ[t^{\pm}])=K(R)\otimes_{\SSS[*/t^{\ZZ}]}\SSS$$
    inducing an exact triangle 
    $$K(R)\to K(R/\ZZ[t^{\pm}])\to K(R/\ZZ[t^{\pm}])[2]$$
    where the map $K(R/\ZZ[t^{\pm}])\to K(R/\ZZ[t^{\pm}])[2]$ can be thought of as a logarithmic $t$-derivative $\nabla^{\log}_{t}$ by the exact sequence of $\SSS[*/t^{\ZZ}]$-modules
    $$\SSS[*/t^{\ZZ}]\to\SSS\to\SSS[2].$$
    In particular, any $K$-theory class gives rise to a relative $K$-theory classes and $K$-theory classes can be recovered from those relative $K$-theory classes that vanish under the differential. 
\end{example}

Now recall the existence of a motivic filtration on $K$-theory \cite{MotivicFiltrationKTheory} and is still a topic of contemporary interest with T. Bouis' recent results in the mixed characteristic case \cite{BouisThesis}. Running this machinery on relative $K$-theory, this produces a relative motivic filtration and relative motivic cohomology of a ring relative to a group algebra taking $\ZZ[M]\to R$ to $\ZZ(n)(R/\ZZ[M])$. 

We have an exact triangle
$$\ZZ(n)(R)\to\ZZ(n)\left(R/\ZZ[t^{\pm}]\right)\to\ZZ(n-1)\left(R/\ZZ[t^{\pm}]\right)$$
which in weight $\leq 2$ computes $K$-theory in small degrees. Explicitly, we have $\ZZ(0)(R)=\ZZ$ and $\ZZ(1)(R)=R^{\times}[-1]$. By $\mathbb{A}^{1}$-invariance for regular rings, we have $K(\ZZ[t])\cong K(\ZZ)$ so we have $K(\ZZ[t^{\pm},\frac{1}{t-1}])\cong K(\ZZ)\oplus K(\ZZ)[1]\oplus K(\ZZ)[1]= K(\ZZ)\oplus K(\ZZ)[1]^{\oplus 2}$. In the relative setting, the exact triangle allow us to compute relative $K$-theory 
$$% https://q.uiver.app/#q=WzAsNCxbMCwwLCJcXFpaKDEpKFIpIl0sWzIsMCwiXFxaWigxKShSL1xcWlpbdF57XFxwbX1dKSJdLFs0LDAsIlxcWlooMCkoUi9cXFpaW3Ree1xccG19XSkiXSxbNiwwLCJSXntcXHRpbWVzfSJdLFswLDFdLFsxLDJdLFsyLDMsIjFcXG1hcHN0byB0Il1d
\begin{tikzcd}
	{\ZZ(1)(R)} && {\ZZ(1)(R/\ZZ[t^{\pm}])} && {\ZZ(0)(R/\ZZ[t^{\pm}])} && {R^{\times}}
	\arrow[from=1-1, to=1-3]
	\arrow[from=1-3, to=1-5]
	\arrow["{1\mapsto t}", from=1-5, to=1-7]
\end{tikzcd}$$
and where making the appropriate substitutions on rationalization gives 
$$0\longrightarrow \QQ(2)(R/\ZZ[t^{\pm}])\longrightarrow \QQ[-1]$$
so in fact there is an isomorphism $\QQ(2)(R/\ZZ[t^{\pm}])\to \QQ[-1]$ induced by the so-called universal dilogarithm that takes $\Li_{2}^{\mathrm{univ}}(t)$ to $(1-t)$ where we note that $\QQ[-1]$ is generated by $1-t$. 

We now want to observe that this so-called universal dilogarithm satisfies the expected functional equations. 
\begin{proposition}\label{prop: equality of the universal dilogarithm}
    There is an equality $\Li_{2}^{\mathrm{univ}}(t)=-\Li_{2}^{\mathrm{univ}}(1-t)$ in relative rational motivic cohomology 
    $$H^{1}\left(\QQ(2)\left(\ZZ\left[t^{\pm},\frac{1}{1-t}\right]/\ZZ[t^{\pm}, (1-t)^{\pm}]\right)\right).$$ 
\end{proposition}
\begin{proof}
    There is a Koszul-like complex computing $\ZZ(n)(R)$ as the limit of 
    $$% https://q.uiver.app/#q=WzAsMyxbMCwwLCJcXFpaKG4pXFxsZWZ0KFIvXFxaWlt0X3sxfV57XFxwbX0sdF97Mn1ee1xccG19XVxccmlnaHQpIl0sWzIsMCwiXFxaWihuLTEpXFxsZWZ0KFIvXFxaWlt0X3sxfV57XFxwbX0sdF97Mn1ee1xccG19XVxccmlnaHQpXntcXG9wbHVzIDJ9Il0sWzMsMCwiXFxaWihuLTIpXFxsZWZ0KFIvXFxaWlt0X3sxfV57XFxwbX0sdF97Mn1ee1xccG19XVxccmlnaHQpIl0sWzAsMSwiKFxcbmFibGFfe3RfezF9fV57XFxsb2d9LFxcbmFibGFfe3RfezJ9fV57XFxsb2d9KSJdLFsxLDJdXQ==
    \begin{tikzcd}
        {\ZZ(n)\left(R/\ZZ[t_{1}^{\pm},t_{2}^{\pm}]\right)} && {\ZZ(n-1)\left(R/\ZZ[t_{1}^{\pm},t_{2}^{\pm}]\right)^{\oplus 2}} & {\ZZ(n-2)\left(R/\ZZ[t_{1}^{\pm},t_{2}^{\pm}]\right)}
        \arrow["{(\nabla_{t_{1}}^{\log},\nabla_{t_{2}}^{\log})}", from=1-1, to=1-3]
        \arrow[from=1-3, to=1-4]
    \end{tikzcd}$$
    so noting that $\QQ(2)(R)\cong 0$ and 
    $$\QQ(i)\left(\ZZ\left[t^{\pm},\frac{1}{1-t}\right]/\ZZ[t^{\pm}, (1-t)^{\pm}]\right)$$
    vanishes for $i\in\{0,1\}$ we have that 
    $$\QQ(2)\left(\ZZ\left[t^{\pm},\frac{1}{1-t}\right]/\ZZ[t^{\pm}, (1-t)^{\pm}]\right)\cong\QQ[-1].$$
    So equality follows by considering the composite. 
\end{proof}
\begin{remark}
    In general if the dilogarithm on a ring $R$ the condition of $\sum_{i}\Li_{2}(f_{i}(t))$ being constant implies that $\sum_{i}f_{i}(t)\wedge (1-f_{i}(t))=0$ in $\bigwedge^{2}R^{\times}$. This holds for the five-term relation \Cref{eqn: BW dilogarithm five term relation}. 
\end{remark}
We now consider the Borel (complex) regulator on algebraic $K$-theory. There is a map 
$$K_{3}\left(\ZZ\left[t^{\pm},\frac{1}{1-t}\right]/\ZZ[t^{\pm}]\right)\to K_{3}(\CC/\ZZ[\CC^{\times}])\to K_{3}^{\cont}(\CC/\ZZ[\CC^{\times}])$$
which statisfies a functional equation so for $t\in\CC\setminus\{0,1\}$, its image in $K_{3}^{\cont}(\CC/\ZZ[\CC^{\times}])$ satisfies a functional equation as well. 

Rationally, we can consider rationalized relative motivic complexes $\QQ(n)^{\cont}(\CC/\ZZ[\CC^{\times}])$ where in the case of $\QQ(2)^{\cont}(\CC/\ZZ[\CC^{\times}])$ we can compute using $\QQ(0)^{\cont}(\CC/\ZZ[\CC^{\times}])\cong\QQ, \QQ(1)^{\cont}(\CC/\ZZ[\CC^{\times}])\cong(\CC^{\times}/\CC^{\times})[-1]\cong0$ so the fiber sequence computes $\QQ(2)^{\cont}(\CC)$ as $(\CC/(2\pi i)^{2}\QQ)[-1]$ and is given by an extension $E$ fitting into the short exact sequence 
$$% https://q.uiver.app/#q=WzAsNSxbMCwwLCIwIl0sWzEsMCwiXFxDQy8oMlxccGkgaSleezJ9XFxRUSJdLFsyLDAsIkUiXSxbMywwLCJcXFFRXFxvdGltZXNfe1xcWlp9XFxiaWd3ZWRnZV57Mn1cXENDXntcXHRpbWVzfSJdLFs0LDAsIjAiXSxbMCwxXSxbMSwyXSxbMiwzXSxbMyw0XV0=
\begin{tikzcd}
	0 & {\CC/(2\pi i)^{2}\QQ} & E & {\QQ\otimes_{\ZZ}\bigwedge^{2}\CC^{\times}} & 0
	\arrow[from=1-1, to=1-2]
	\arrow[from=1-2, to=1-3]
	\arrow[from=1-3, to=1-4]
	\arrow[from=1-4, to=1-5]
\end{tikzcd}$$
where the map $\CC\setminus\{0,1\}\to E$ by $t\mapsto (t)\wedge(1-t)$ satisfies the five-term relation. This extends to a map $\QQ[\CC\setminus\{0,1\}]$ to $E$ which necessarily factors over the pre-Bloch group $\wp_{2}(\CC)$ of \Cref{def: pre-Bloch group} since the map satisfies the five-term relation. This in turn induces a map $B_{2}(\CC)\to \CC/(2\pi i)^{2}\QQ$ as expected. More explicitly, $E$ is obtained as the cokernel of $\CC\otimes_{\ZZ}\CC\to \CC/(2\pi i)^{2}\QQ\oplus\CC^{\times}\otimes_{\ZZ}\CC$ by $x\otimes y\mapsto (xy, \exp(x)\otimes y+\exp(y)\otimes x)$. 
\section{Lecture 10 -- 12th November 2024}\label{sec: lecture 10}
Continuing our discussion of real and complex differentiation, we note that Wirtinger derivatives are closely related to the multivariable Cauchy-Riemann equations. 
\begin{proposition}\label{prop: Wirtinger are partials}
    Let $U\subseteq\CC^{n}$ be an open set and $f:U\to\CC$ a real differentiable function of the form 
    $$f(z)-f(\tau)=\sum_{j=1}^{n}\Delta_{j}(z)(z_{j}-\tau_{j})+\sum_{j=1}^{n}E_{j}(z)(\overline{z}_{j}-\overline{\tau}_{j}).$$
    Then $$\partial_{z_{j}}f(z)=\Delta_{j}(z)=\frac{1}{2}\left(\partial_{x_{j}}f(z)-i\cdot\partial_{y_{j}}f(z)\right)$$
    and 
    $$\partial_{\overline{z_{j}}}f(z)=E_{j}(z)=\frac{1}{2}\left(\partial_{x_{j}}f(z)+i\cdot\partial_{y_{j}}f(z)\right).$$
\end{proposition}
\begin{proof}
    Suppsose $z_{1}=\tau_{1},\dots,z_{j-1}=\tau_{j-1},z_{j+1}=\tau_{j+1},\dots,z_{n}=\tau_{n}$ so $f'(z)=\lim_{z_{j}\to\tau_{j}}\frac{f(z)-f(\tau)}{z_{j}-\tau_{j}}=\Delta_{j}(\tau)$, the second equality in the first line follows from the first by applying the chain rule, and the second set of equalities follows by an analogous computation. 
\end{proof}
As such, we can show the following theorem. 
\begin{theorem}\label{thm: holomorphic iff differential equations}
    Let $U\subseteq\CC^{n}$ be an open set and $f:U\to\CC$ a function. $f$ is complex holomorphic if and only if it is real differentiable and satisfies the system of partial differential equations 
    $$\partial_{\overline{z_{1}}}f(z)=\dots=\partial_{\overline{z_{n}}}f(z)=0.$$
\end{theorem}
\begin{proof}
    \Cref{prop: Wirtinger are partials} expresses $\partial_{\overline{z_{j}}}f(z)$ in terms of $E_{j}(z)$ which vanish identically for a holomorphic function by \Cref{def: complex differentiable function}. 
\end{proof}
\begin{remark}
    Evidently a holomoprhic function is holomorphic in each variable. That is, for $f:U\to\CC$ holomorphic, the function $f(\tau_{1},\dots,\tau_{j-1},z_{j},\tau_{j+1},\dots,\tau_{n})$ is a univariate holomorphic function in $z_{j}$. Furthermore, while in real analysis there exist functions that are differentiable in each variable but are not even continuous, differentiability in each variable implies global differentiability in complex analysis by Hartogs' theorem. This is highly subtle and is beyond the scope of the course, and an account can be found in the text of H\"{o}rmander \cite{Hormander}. 
\end{remark}
We can show that holomorphic functions on a domain behave well algebraically and form a ring. 
\begin{proposition}\label{prop: holomorphic functions on a domain form a ring}
    Let $U\subseteq\CC^{n}$ be open. The set of holomorphic functions $\Ocal_{U}$ is a $\CC$-algebra that contains $\CC[z_{1},\dots,z_{n}]$. 
\end{proposition}
\begin{proof}
    Elements of $\CC[z_{1},\dots,z_{n}]$ are holomorphic on $\CC^{n}$ and hence on $U$, and the constants are holomorphic on $U$. Complex differentiable functions are preserved sums and products, and thus so too are holomorphic functions. 
\end{proof}
We now state and prove the Cauchy integral formula for functions of several complex variables. To do so, we will define integrals over the distinguished boundary of a polydisc. 
\begin{definition}[Distinguished Boundary of Polydisc]\label{def: distinguished boundary}
    Let $D_{r}(\tau)$ be a polydisc of polyradius $r$ around $\tau$. The distinguished boundary $T_{r}(\tau)$ of $D_{r}(\tau)$ is given by 
    $$\{z\in\CC^{n}:|z_{j}-\tau_{j}|=r_{j}\}.$$
\end{definition}
\begin{remark}
    The distinguished boundary is the product of $n$ copies of the topological circle $S^{1}$, that is, is of real dimension $n$. 
\end{remark}
We can now turn to a discussion of the multivariate Cauchy integral formula. 
\begin{theorem}[Multivariate Cauchy Integral]\label{thm: multivariate cauchy integral}
    Let $U\subseteq\CC^{n}$ be an open set and $f:U\to\CC$ a holomorphic function. If $D_{r}(\tau)\subseteq U$ is a polydisc with distinguished boundary $T_{r}(\tau)$ then for all $\alpha\in U$ 
    $$f(\alpha)=\frac{1}{(2\pi i)^{n}}\int_{T_{r}(\tau)}\frac{f(z)}{(z_{1}-\alpha_{1})\dots(z_{n}-\alpha_{n})}dz.$$
\end{theorem}
\begin{proof}
    The $n=1$ case is the univariate Cauchy integral formula. We proceed by induction on $n$, supposing it holds for the case $k$. Consider the case $k+1$ with a function $f(z_{1},\dots,z_{k+1})$. Let $g(z_{1})=f(z_{1},\alpha_{2},\dots,\alpha_{k+1})$ be a univariate function. By Cauchy's integral formula in one dimension, we have 
    $$f(\alpha)=f(\alpha_{1},\dots,\alpha_{k+1})=\int_{T_{r_{1}}(\tau_{1})}\frac{g(z_{1})}{z_{1}-\alpha_{1}}dw.$$
    But by the induction hypothesis, for fixed $w$, we have 
    $$f(z_{1},\alpha_{2},\dots,\alpha_{k+1})=\int_{T_{(r_{2},\dots,r_{k+1})}((\tau_{2},\dots,\tau_{k+1}))}\frac{f(z_{1},z_{2},\dots,z_{n})}{(z_{2}-\alpha_{2})\dots(z_{n}-\alpha_{n})}$$
    and combining the two integrals yields the claim. 
\end{proof}
We can alternatively phrase \Cref{thm: multivariate cauchy integral} in terms of Cauchy kernels. 
\begin{definition}[Cauchy Kernel]\label{def: cauchy kernel}
    Let $U\subseteq\CC^{n}$ be an open set, $f:U\to\CC$ a continuous function, and $D_{r}(\tau)\subseteq U$ is a polydisc with distinguished boundary $T_{r}(\tau)$. Then the Cauchy kernel is given by 
    $$C_{f}(z)=\frac{1}{(2\pi i)^{n}}\int_{T_{r}(\tau)}\frac{f(w)}{(w_{1}-z_{1})\dots(w_{n}-z_{n})}dw_{1}\dots dw_{n}.$$
\end{definition}
\begin{remark}
    By \Cref{thm: multivariate cauchy integral}, the Cauchy kernel agrees with $f$ if $f$ is holomorphic. 
\end{remark}
The Cauchy kernel allows us to deduce that holomorphic functions of several variables are infinitely differentiable. 
\begin{proposition}\label{prop: multivariate holomorphic are infinitey differentiable}
    Let $U\subseteq\CC^{n}$ be an open set containing a polydisc $D_{r}(\tau)$ with distinguished boundary $T_{r}(\tau)$ and $f:U\to\CC$ a holomorphic function. Then $f$ admits holomorphic partial derivatives of all orders.\marginpar{Theorem 1.2} 
\end{proposition}
\begin{proof}
    Differentiating the Cauchy kernel under the integral sign, we have 
    $$\partial^{\nu}_{z}C_{f}(z)=\frac{\nu_{1}!\dots\nu_{n}!}{(2\pi i)^{n}}\int_{T_{r}(\tau)}\frac{f(w)}{(w_{1}-z_{1})^{\nu_{1}+1}\dots(w_{n}-z_{n})^{\nu_{n}+1}}d_{w_{1}}\dots dw_{n}$$
    giving the claim. 
\end{proof}
We will now consider analogues of key results in complex analysis in the multivariate setting. This discussion will require the following lemma which often allows us to reduce to the single variable case. 
\begin{lemma}\label{lem: testing holomorphic by lines}
    Let $\alpha,\beta\in\CC^{n}$ and $\lambda:\CC\mapsto\CC^{n}$ by $t\mapsto\alpha+t\beta$. If $f$ is holomorphic on $U\subseteq\CC^{n}$ open then $f\circ\lambda$ is holomorphic on $\lambda^{-1}(U)$. 
\end{lemma}
\begin{proof}
    We compute $$\frac{d}{dt}(f\circ\lambda)(t)=\sum_{j=1}^{n}\partial_{z_{j}}(\lambda(t))\beta_{j}$$
    which is a $\beta_{j}$-weighted sum of holomorphic functions and hence holomorphic on $\lambda^{-1}(U)=f|_{L}$. 
\end{proof}
We can now prove the multivariate analogues of the identity theorem and maximum principle. 
\begin{theorem}[Multivariate Identity]\label{thm: multivariate identity}
    Let $f$ be holomorphic on a domain $G$ and identically zero on a nonempty open subset $U$ of $G$. Then $f\equiv0$ on $G$.\marginpar{Theorem 1.5} 
\end{theorem}
\begin{proof}
    Let $L$ be a line the image of $\lambda$ through $G$ and consider $f\circ\lambda$ the induced holomorphic function of one variable. By the identity theorem in one variable \Cref{thm: identity theorem}, $f$ is identically zero on all of $L$. Writing $G$ as the union of all lines passing through it gives the claim.  
\end{proof}
\begin{theorem}[Multivariate Maximum Modulus]\label{thm: multivariate maximum modulus}
    Let $f$ be holomophric on a domain $G$ and $|f|$ has local maximum at $\tau\in G$. Then $f(z)=f(\tau)$ is constant.\marginpar{Theorem 1.6}
\end{theorem}
\begin{proof}
    Arguing as before, let $L$ be a line the image of $\lambda$ through $G$ and consider $f\circ\lambda$ the induced holomorphic function of one variable. By the maximum maximum modulus principle in one variable, $f$ is constant on all of $L$. Writing $G$ as the union of all lines passing through it gives the claim. 
\end{proof}
We now turn to a consequence of \Cref{thm: multivariate cauchy integral} and some surprising consequences. 
\begin{proposition}\label{prop: holomorphic extensions over polyannuli}
    Let $G\subseteq\CC^{n}$ for $n\geq2$. If $f$ is holomorphic in a punctured neighborhood of $\tau\in G$, then $f$ is holomorphic on $G$. 
\end{proposition}
\begin{proof}
    By translation, it suffices to consider the case of a function $f$ holomorphic on $D_{1}(0)\setminus\overline{D_{1/2}(0)}$ in $\CC^{n}$, here considering polydiscs of fixed radius. By the univariate Cauchy integral formula we have 
    $$f(\alpha)=\frac{1}{(2\pi i)^{n}}\int_{T_{r}(\tau)}\frac{f(z)}{(z_{1}-\alpha_{1})\dots(z_{n}-\alpha_{n})}dz$$
    but fixing any $\alpha_{j}$ the Cauchy integral formula in one variable holds but $\alpha_{j}$ is arbitrary so the function extends in each variable to all of $D_{1}(0)$ and so it does overall. 
\end{proof}
This implies that holomorphic functions have zeroes and thus poles along a subset of $\CC^{n}$ of codimension at least 1. 
\begin{corollary}\label{corr: nonisolated zeroes and poles}
    Let $f$ be a holomorphic function on $U\subseteq\CC^{n}$ open. Then $\{z\in\CC^{n}:f(z)=0\}$ and $\{z\in\CC^{n}:f(z)=\infty\}$ are non-isolated. 
\end{corollary}
\begin{proof}
    The poles of $f$ are non-isolated by \Cref{prop: holomorphic extensions over polyannuli} and thus so too zeroes are non-isolated under passage to $1/f(z)$. 
\end{proof}
With the language of holomorphic functions in hand, we can discuss holomorphic maps. 
\begin{definition}[Holomorphic Map]\label{def: holomorphic map}
    Let $U\subseteq\CC^{n},V\subseteq\CC^{m}$ be open sets and $f=(f_{1},\dots,f_{m}):U\to V$ be a continuous function. $f$ is a holomorphic map if each component function $f_{k}:U\to\CC$ is holomorphic. 
\end{definition}
The chain rule generalizes to the multivariate setting. 
\begin{proposition}[Multivariate Chain Rule]\label{prop: multivariate chain rule}
    Let $U\subseteq\CC^{n},V\subseteq\CC^{m}$ be open sets and $f=(f_{1},\dots,f_{m}):U\to V$ and $g:V\to\CC$ holomorphic functions. Then 
    $$\partial z_{j}(g\circ f)=\sum_{k=1}^{m}\frac{\partial g}{\partial w_{k}}\frac{\partial f}{\partial z_{j}}.$$
\end{proposition}
\begin{proof}
    This is immediate from the chain rule and the Cauchy-Riemann equations. 
\end{proof}
\section{Lecture 11 -- 15th November 2024}\label{sec: lecture 11}
We begin our discussion of transversality. 
\begin{definition}[Transverse Submanifolds at a Point]\label{def: transverse submanifolds at a point}
    Let $M$ be a smooth manifold and $S,S'$ submanifolds of $M$. $M$ and $M'$ are transverse at $p\in S\cap S'$ if the subspaces $T_{p}S$ and $T_{p}S'$ span $T_{p}M$. 
\end{definition}
\begin{definition}[Transverse Submanifolds]\label{def: transverse submanifolds}
    Let $M$ be a smooth manifold and $S,S'$ submanifolds of $M$. $S$ and $S'$ are transverse submanifolds $S$ -- $S\pitchfork S'$ -- if they are transverse at each point $p\in S\cap S'$. 
\end{definition}
\begin{example}
    The union of coordinate axes in $\RR^{2}$ is transverse. 
\end{example}
\begin{example}
    The generic intersection of a circle and a line is transverse. 
\end{example}
\begin{example}
    A line tangent to a circle is not transverse, as the tangent spaces at the intersection point is just the line, while the tangent space of $\RR^{2}$ at that point is $\RR^{2}$. 
\end{example}
Transverse intersections behave especially nicely, insofar as their intersections are smooth submanifolds. On the other hand, non-transverse intersections might not even be a topological manifold. 
\begin{example}\label{ex: graph of crossing lines}
    Let $f:\RR^{2}\to\RR$ by $f(x,y)=x^{2}-y^{2}$ and $g(x,y)=0$. Let $S$ be the graph of $f$ in $\RR^{3}$ given by $\{(x,y,z)\in\RR^{3}:z=f(x,y)\}$ and $S'$ the graph of $g$ in $\RR^{3}$ given by $\{(x,y,z)\in\RR^{3}:z=g(x,y)=0\}$. The intersection $S\cap S'$ is given by $\{(x,y,z)\in\RR^{3}:x^{2}-y^{2}=z=0\}$ which is two lines intersecting transversely in $\RR^{3}$ and not a topological manifold. 
\end{example}
We now show the lemma. 
\begin{lemma}\label{lem: transversality is a manifold}
    Let $M$ be a smooth manifold and $S,S'$ submanifolds of $M$. If $S,S'$ are transverse, then $S\cap S'$ is a smooth submanifold. 
\end{lemma}
\begin{proof}
    By composing with a chart centered around zero in $\RR^{m}$ and the slice \Cref{thm: slice}, it suffices to show that the intersection is a smooth submanifold ina  neighborhood of 0. Using the rank theorem \ref{thm: rank theorem}, after possibly shrinking $U$, that $S=f^{-1}(0)$ for $f:U\to\RR^{m-\dim(S)}$ and $S'=g^{-1}(0)$ for $g:U\to\RR^{m-\dim(S')}$ with $f,g$ of full rank. 

    Consider $H:U\to\RR^{m-\dim(S)}\oplus\RR^{m-\dim(S')}$ by $p\mapsto(f(p),g(p))$. It suffices to show that $H$ is surjective, where injectivity follows from $S,S'$ being submanifolds. We first observe that $H^{-1}(0)=f^{-1}(0)\cap g^{-1}(0)=S\cap S'$. To see the surjectivity of $H$ at the origin, note that $T_{0}S+ T_{0}S'\to T_{0}U$ is a linear isomorphism since $S,S'$ are transverse and that there is a map $dH_{0}:T_{0}U\to\RR^{m-\dim(S)}\oplus\RR^{m-\dim(S')}$ fitting into 
    $$% https://q.uiver.app/#q=WzAsNCxbMCwwLCJUX3swfVNcXG9wbHVzIFRfezB9UyciXSxbMCwxLCJUX3swfVUiXSxbMiwwLCJUX3swfVMvKFRfezB9U1xcY2FwIFRfezB9UycpXFxvcGx1cyBUX3swfVMnLyhUX3swfVNcXGNhcCBUX3swfVMnKSJdLFsyLDEsIlxcUlJee20tXFxkaW0oUyl9XFxvcGx1c1xcUlJee20tXFxkaW0oUycpfSJdLFsxLDNdLFsyLDMsIihkZl97MH0sZGdfezB9KSJdLFswLDJdLFswLDEsIlxcd3IiLDJdXQ==
    \begin{tikzcd}
        {T_{0}S+ T_{0}S'} && {T_{0}S/(T_{0}S\cap T_{0}S')\oplus T_{0}S'/(T_{0}S\cap T_{0}S')} \\
        {T_{0}U} && {\RR^{m-\dim(S)}\oplus\RR^{m-\dim(S')}}
        \arrow[from=1-1, to=1-3]
        \arrow["\wr"', from=1-1, to=2-1]
        \arrow["{(df_{0},dg_{0})}", from=1-3, to=2-3]
        \arrow[from=2-1, to=2-3]
    \end{tikzcd}$$
    The top horizontal map is well-defined as if $v+w=v'+w'$ then $v-v'=w-w'$ is in $T_{0}S\cap T_{0}S'$. It remains to show the right vertical map is surjective, we show in particular it is an isomorphism. Observe the map is injective as as the kernels of $df_{0},dg_{0}$ in $T_{0}S,T_{0}S'$ is exactly the intersection. We then consider the short exact sequence 
    $$% https://q.uiver.app/#q=WzAsNSxbMCwwLCIwIl0sWzEsMCwiVF97MH1TXFxjYXAgVF97MH1TJyJdLFsyLDAsIlRfezB9U1xcb3BsdXMgVF97MH1TJyJdLFszLDAsIlRfezB9VSJdLFs0LDAsIjAiXSxbMCwxXSxbMSwyXSxbMiwzXSxbMyw0XV0=
    \begin{tikzcd}
        0 & {T_{0}S\cap T_{0}S'} & {T_{0}S\oplus T_{0}S'} & {T_{0}U} & 0
        \arrow[from=1-1, to=1-2]
        \arrow[from=1-2, to=1-3]
        \arrow[from=1-3, to=1-4]
        \arrow[from=1-4, to=1-5]
    \end{tikzcd}$$
    where the maps are $v\mapsto (v,v)$ and $(a,b)\mapsto a-b$. Computing the dimensions, $\dim(T_{0}S\cap T_{0}S')+m=\dim(S)+\dim(S')$ and thus $\dim(T_{0}S/(T_{0}S\cap T_{0}S'))=\dim(S')-(\dim(S)+\dim(S')-m)=m-\dim(S)$ and similarly $\dim(T_{0}S'/(T_{0}S\cap T_{0}S'))=\dim(S)-(\dim(S)+\dim(S')-m)=m-\dim(S')$ showing that the final map is surjective. The claim follows. 
\end{proof}
We can generalize transversality of manifolds to transversality of maps. This is motivated by the goal of producing fibered products in the category of smooth manifolds. 
\begin{definition}[Fibered Product]\label{def: fibered product}
    Let $f:X\to Z, g:Y\to Z$ be continuous maps of topological spaces. The fibered product $X\times_{Z}Y$ is given by 
    $$\{(x,y)\in X\times Y:f(x)=g(y)\in\ZZ\}\subseteq X\times Y$$
    with the subspace topology on the product. 
\end{definition}
\begin{remark}
    \Cref{def: fibered product} implies the universal property of fibered products. For all topological spaces $W$ making the solid diagram commute 
    $$% https://q.uiver.app/#q=WzAsNSxbMSwxLCJYXFx0aW1lc197Wn1ZIl0sWzEsMiwiWSJdLFszLDEsIlgiXSxbMywyLCJaIl0sWzAsMCwiVyJdLFsyLDMsImYiXSxbMSwzLCJnIiwyXSxbMCwxXSxbMCwyXSxbNCwxLCIiLDAseyJvZmZzZXQiOjEsImN1cnZlIjoyfV0sWzQsMiwiIiwwLHsiY3VydmUiOi0xfV0sWzQsMCwiXFxleGlzdHMhIiwxLHsic3R5bGUiOnsiYm9keSI6eyJuYW1lIjoiZGFzaGVkIn19fV1d
    \begin{tikzcd}
        W \\
        & {X\times_{Z}Y} && X \\
        & Y && Z
        \arrow["{\exists!}"{description}, dashed, from=1-1, to=2-2]
        \arrow[curve={height=-6pt}, from=1-1, to=2-4]
        \arrow[shift right, curve={height=12pt}, from=1-1, to=3-2]
        \arrow[from=2-2, to=2-4]
        \arrow[from=2-2, to=3-2]
        \arrow["f", from=2-4, to=3-4]
        \arrow["g"', from=3-2, to=3-4]
    \end{tikzcd}$$
    there is a unique map $W\to X\times_{Z}Y$ making the entire diagram commute. 
\end{remark}
Note that neither the categories $\Mfld$ of \Cref{def: category of topological manifolds} nor $\SmMfld$ of \Cref{def: category of smooth manifolds} admit fibered products. 
\begin{example}
    In the setup of \Cref{ex: graph of crossing lines}, the fibered product $S\times_{\RR^{3}}S'$ is the union of transversely intersecting lines in $\RR^{3}$ which is not a topological manifold even though $S,S'$ are topological and even smooth manifolds (cf. \Cref{ex: graphs are mflds}).
\end{example}
Extending \Cref{def: transverse submanifolds at a point,def: transverse submanifolds}, we can define transverse maps as follows. 
\begin{definition}[Transverse Maps at a Point]\label{def: transverse maps at a point}
    Let $M_{1},M_{2},N$ be smooth manifolds and $F:M_{1}\to N,G:M_{2}\to N$ be smooth maps. $F$ and $G$ are transverse at $F(p_{1})=F(p_{2})\in N$ if $\mathrm{im}(dF_{p_{1}})+\mathrm{im}(dG_{p_{2}})$ spans $T_{q}N$. 
\end{definition}
\begin{definition}[Transverse Maps]\label{def: transverse maps}
    Let $M_{1},M_{2},N$ be smooth manifolds and $F:M_{1}\to N,G:M_{2}\to N$ be smooth maps. $F$ and $G$ are transverse -- $F\pitchfork G$ -- if it is transverse at all $q\in Z$. 
\end{definition}
\begin{remark}
    Transversality of maps generalizes transversality of manifolds by taking $F:S\to M, G:S'\to M$ for smooth submanifolds $S,S'\subseteq M$. 
\end{remark}
\section{Lecture 12 -- 18th November 2024}\label{sec: lecture 12}
We continue with some formal properties of fibered products. 
\begin{proposition}\label{prop: triple fibered products are associative}
    Let $X\to S, Y\to S, W\to T, Y\to T$ be morphisms in a category admitting fibered products. Then there is a unique isomorphism $(X\times_{S}Y)\times_{T}W\cong X\times_{S}(Y\times_{T}W)$. 
\end{proposition}
\begin{proof}
    The ? of the diagram 
    $$% https://q.uiver.app/#q=WzAsOCxbMCwwLCI/Il0sWzAsMSwiWFxcdGltZXNfe1N9WSJdLFsyLDAsIllcXHRpbWVzX3tUfVciXSxbMiwxLCJZIl0sWzQsMCwiVyJdLFs0LDEsIlQiXSxbMiwyLCJTIl0sWzAsMiwiWCJdLFszLDVdLFsxLDNdLFszLDZdLFsxLDddLFs3LDZdLFs0LDVdLFsyLDRdLFsyLDNdLFswLDJdLFswLDFdXQ==
    \begin{tikzcd}
        {?} && {Y\times_{T}W} && W \\
        {X\times_{S}Y} && Y && T \\
        X && S
        \arrow[from=1-1, to=1-3]
        \arrow[from=1-1, to=2-1]
        \arrow[from=1-3, to=1-5]
        \arrow[from=1-3, to=2-3]
        \arrow[from=1-5, to=2-5]
        \arrow[from=2-1, to=2-3]
        \arrow[from=2-1, to=3-1]
        \arrow[from=2-3, to=2-5]
        \arrow[from=2-3, to=3-3]
        \arrow[from=3-1, to=3-3]
    \end{tikzcd}$$
    with all squares cartesian can be filled by $X\times_{S}(Y\times_{T}W)$ and $(X\times_{S}Y)\times_{T}W$ by considering the vertical and horizontal rectangles, respectively. But in any such diagram, the rectangles are Cartesian as well so both of the objects above satisfy the same universal property and are thus isomorphic. 
\end{proof}
We now show fibered products exist in general. We do this in a sequence of lemmas.\marginpar{We follow the presentation of \cite[\href{https://stacks.math.columbia.edu/tag/01JO}{Tag 01JO}]{stacks-project} in place of \cite[Thm. II.3.3]{Hartshorne} presented in class.}
\begin{lemma}
    Let $f:X\to S,g:Y\to S$ be morphisms of schemes and suppose $X\times_{S}Y$ exists. If $U\subseteq S, V\subseteq X, W\subseteq Y$ are open such that $f(V)\subseteq U$ and $g(W)\subseteq U$ then there is a unique morphism $V\times_{U}W\to X\times_{S}Y$ and $V\times_{U}W\subseteq X\times_{S}Y$ is open. 
\end{lemma}
\begin{proof}
    For any other scheme $Z$ admitting maps to $V,W$ whose compatible with $f|_{V},g|_{W}$, there is a unique morphism $Z\to X\times_{S}Y$ which is contained in the open subscheme $\pr_{X}^{-1}(V)\cap\pr_{Y}^{-1}(W)$ of $X\times_{S}Y$, giving an identification $V\times_{U}W\cong\pr_{X}^{-1}(V)\cap\pr_{Y}^{-1}(W)$ by the uniqueness of fibered products. 

    Uniqueness of the map follows from the diagram 
    $$% https://q.uiver.app/#q=WzAsOCxbMywxLCJYIl0sWzEsMiwiWSJdLFszLDIsIlMiXSxbMSwxLCJYXFx0aW1lc197U31ZIl0sWzQsMywiVSJdLFs0LDAsIlYiXSxbMCwzLCJXIl0sWzAsMCwiVlxcdGltZXNfe1V9VyJdLFszLDBdLFswLDJdLFszLDFdLFsxLDJdLFs1LDRdLFs0LDJdLFs1LDBdLFs2LDFdLFs2LDRdLFs3LDMsIlxcZXhpc3RzISIsMSx7InN0eWxlIjp7ImJvZHkiOnsibmFtZSI6ImRhc2hlZCJ9fX1dLFs3LDZdLFs3LDVdXQ==
    \begin{tikzcd}
        {V\times_{U}W} &&&& V \\
        & {X\times_{S}Y} && X \\
        & Y && S \\
        W &&&& U.
        \arrow[from=1-1, to=1-5]
        \arrow["{\exists!}"{description}, dashed, from=1-1, to=2-2]
        \arrow[from=1-1, to=4-1]
        \arrow[from=1-5, to=2-4]
        \arrow[from=1-5, to=4-5]
        \arrow[from=2-2, to=2-4]
        \arrow[from=2-2, to=3-2]
        \arrow[from=2-4, to=3-4]
        \arrow[from=3-2, to=3-4]
        \arrow[from=4-1, to=3-2]
        \arrow[from=4-1, to=4-5]
        \arrow[from=4-5, to=3-4]
    \end{tikzcd}$$
\end{proof}
\begin{proposition}\label{prop: fibered products exist}
    Let $X\to S, Y\to S$ be morphisms of schemes. Then the fibered product $X\times_{S}Y$ exists in the category of schemes. 
\end{proposition}
\begin{proof}
    Let $\{U_{i}\}_{i\in I}$ be an affine open cover of $S$, $\{V_{j}\}_{j\in J_{i}}$ an affine open cover of $f^{-1}(U_{i})$ for each $i$, and $\{W_{k}\}_{k\in K_{i}}$ an affine open cover of $g^{-1}(U_{i})$ for each $i$. By \Cref{prop: fibered products of affine schemes}, each $V_{j}\times_{U_{i}}W_{k}$ is affine and satisfies the universal property of the fibered product for morphisms factoring through $V_{j},W_{k}$ that agree on $U_{i}$ and any scheme $Z$ admitting maps to $X,Y$ that agree on $S$ is given by the data of maps to each such $V_{j}\times_{U_{i}}W_{k}$. Moreover these schemes satisfy the hypothesis of \Cref{prop: gluing schemes} so these schemes glue to give the fibered product $X\times_{S}Y$ which satisfies the expected universal property. 
\end{proof}
Note that fibered products of schemes can behave unexpectedly. 
\begin{example}
    Let $X=\spec(k[x_{1}]),Y=\spec(k[x_{2}]),S=\spec(k)$ for $k$ algebraically closed. $X\times_{S}Y=\spec(k[x_{1}]\otimes_{k}k[x_{2}])=\spec(k[x_{1},x_{2}])=\A^{2}_{k}$ but the underlying topological space $|\A^{2}_{k}|$ is not $|\A^{1}_{k}|\times|\A^{1}_{k}|$ -- the latter has points given by pairs of prime ideals in $k[x_{1}]\oplus k[x_{2}]$ but $(x_{1}-x_{2})$ is a point of $\A^{2}_{k}$ not of this form. 
\end{example}
\begin{example}
    Let $X=Y=\spec(\CC)$ and $S=\spec(\RR)$. $X\times_{S}Y=\spec(\RR[x]/(x^{2}-1)\otimes_{\RR}\CC)=\spec(\CC[x]/(x^{2}-1))$ which consists of two points given by the maximal ideals $(x-i),(x+i)$. But the product $|X|\times|Y|$ is just one point, so there is not even a natural map $|X\times Y|\to|X\times_{S}Y|$. 
\end{example}
One place fibered products are ubiquitous is in the computation of fibers of a morphism. 
\begin{definition}[Fiber]\label{def: fiber of morphism}
    Let $f:X\to Y$ be a morphism of schemes and $y\in Y$ with residue field $\kappa(y)$. The fiber $X_{y}$ of $f$ over $Y$ is the fibered product $X\times_{Y}\spec(\kappa(y))$. 
\end{definition}
More explicitly, the fiber is induced by the following diagram.
$$% https://q.uiver.app/#q=WzAsNCxbMCwwLCJYX3t5fT1YXFx0aW1lc197WX1cXHNwZWMoXFxrYXBwYSh5KSkiXSxbMCwxLCJcXHNwZWMoXFxrYXBwYSh5KSkiXSxbMiwwLCJYIl0sWzIsMSwiWSJdLFswLDJdLFsyLDNdLFsxLDNdLFswLDFdXQ==
\begin{tikzcd}
	{X_{y}=X\times_{Y}\spec(\kappa(y))} && X \\
	{\spec(\kappa(y))} && Y
	\arrow[from=1-1, to=1-3]
	\arrow[from=1-1, to=2-1]
	\arrow[from=1-3, to=2-3]
	\arrow[from=2-1, to=2-3]
\end{tikzcd}$$
In the case of $Y$ having a generic point, we can construct generic and closed fibers. 
\begin{definition}[Generic Fiber]\label{def: generic fiber}
    Let $f:X\to Y$ be a morphism of schemes and $\eta\in Y$ the unique generic point of $Y$ with residue field $\kappa(\eta)$. The generic fiber $X_{\eta}$ of $f$ over $Y$ is the fibered product $X\times_{Y}\spec(\kappa(\eta))$.
\end{definition}
\begin{definition}[Closed Fiber]\label{def: closed fiber}
    Let $f:X\to Y$ be a morphism of schemes and $y\in Y$ a closed point with residue field $\kappa(y)$. The closed fiber $X_{y}$ of $f$ over $Y$ is the fibered product $X\times_{Y}\spec(\kappa(y))$.
\end{definition}
\begin{example}
    Let $A$ be a discrete valuation ring. Then $\spec(A)=\{\eta,\pi\}$ where $\eta$ is the generic point and $\pi$ the prime ideal corresponding to the uniformizer. A scheme $X$ over $\spec(A)$ has two fibers: the generic fiber $X_{\eta}$ and the closed fiber $X_{\pi}$. 
\end{example}
Fibered products are also a key tool in working in Grothendieck's ``relative point of view.''
\begin{definition}[Category of $S$-Schemes]\label{def: category of S-schemes}
    The category of $S$-schemes $\Sch_{S}$ has objects morphisms of schemes $X\to S$ and morphisms commutative diagrams 
    $$% https://q.uiver.app/#q=WzAsMyxbMCwwLCJYIl0sWzIsMCwiWSJdLFsxLDEsIlMiXSxbMCwyXSxbMSwyXSxbMCwxXV0=
    \begin{tikzcd}
        X && Y \\
        & S.
        \arrow[from=1-1, to=1-3]
        \arrow[from=1-1, to=2-2]
        \arrow[from=1-3, to=2-2]
    \end{tikzcd}$$
\end{definition}
\begin{definition}[Category of $k$-Schemes]\label{def: category of k-schemes}
    Let $k$ be field. The category of $k$-schemes $\Sch_{k}$ has objects morphisms of schemes $X\to \spec(k)$ and morphisms commutative diagrams 
    $$% https://q.uiver.app/#q=WzAsMyxbMCwwLCJYIl0sWzIsMCwiWSJdLFsxLDEsIlxcc3BlYyhrKSJdLFswLDJdLFsxLDJdLFswLDFdXQ==
    \begin{tikzcd}
        X && Y \\
        & {\spec(k).}
        \arrow[from=1-1, to=1-3]
        \arrow[from=1-1, to=2-2]
        \arrow[from=1-3, to=2-2]
    \end{tikzcd}$$
\end{definition}
\begin{remark}
    When working in the setting of $k$-schemes and considering the fibered product of $X\to\spec(k),Y\to\spec(k)$ we will write $X\times_{k}Y$ in place of $X\times_{\spec(k)}Y$. 
\end{remark}
The fibered product gives us a way to ``extend scalars'' on schemes defined over a field. 
\begin{definition}[Base Change]\label{def: base change}
    Let $k$ be a field, $\Sch_{k}$ the category of $k$-schemes, and $L/k$ a field extension. The base change functor $(-)_{L}:\Sch_{k}\to\Sch_{L}$ is given by $X\mapsto X_{L}=X\times_{k}\spec(L)$ and morphisms those induced morphisms of $L$-schemes. 
\end{definition}
More precisely, for a morphism $X\to Y$ of $k$-schemes, we have a diagram 
$$% https://q.uiver.app/#q=WzAsNixbNSwyLCJZIl0sWzMsMiwiWCJdLFs0LDMsIlxcc3BlYyhrKSJdLFswLDAsIlhfe0x9Il0sWzIsMCwiWV97TH0iXSxbMSwxLCJcXHNwZWMoTCkiXSxbMywxXSxbNCwwXSxbNSwyXSxbMSwwXSxbMSwyXSxbMCwyXSxbMyw1XSxbNCw1XV0=
\begin{tikzcd}
	{X_{L}} && {Y_{L}} \\
	& {\spec(L)} \\
	&&& X && Y \\
	&&&& {\spec(k)}
	\arrow[from=1-1, to=2-2]
	\arrow[from=1-1, to=3-4]
	\arrow[from=1-3, to=2-2]
	\arrow[from=1-3, to=3-6]
	\arrow[from=2-2, to=4-5]
	\arrow[from=3-4, to=3-6]
	\arrow[from=3-4, to=4-5]
	\arrow[from=3-6, to=4-5]
\end{tikzcd}$$
with both rectangles Cartesian so there is a unique map $X_{L}\to Y_{L}$ making the diagram 
$$% https://q.uiver.app/#q=WzAsNixbMywyLCJcXHNwZWMoaykiXSxbMSwyLCJcXHNwZWMoTCkiXSxbMywxLCJZIl0sWzEsMSwiWV97TH0iXSxbMSwwLCJYIl0sWzAsMCwiWF97TH0iXSxbMSwwXSxbNSwxXSxbNSw0XSxbNCwyXSxbMiwwXSxbMywyXSxbMywxXSxbNSwzLCJcXGV4aXN0cyEiLDEseyJzdHlsZSI6eyJib2R5Ijp7Im5hbWUiOiJkYXNoZWQifX19XV0=
\begin{tikzcd}
	{X_{L}} & X \\
	& {Y_{L}} && Y \\
	& {\spec(L)} && {\spec(k)}
	\arrow[from=1-1, to=1-2]
	\arrow["{\exists!}"{description}, dashed, from=1-1, to=2-2]
	\arrow[from=1-1, to=3-2]
	\arrow[from=1-2, to=2-4]
	\arrow[from=2-2, to=2-4]
	\arrow[from=2-2, to=3-2]
	\arrow[from=2-4, to=3-4]
	\arrow[from=3-2, to=3-4]
\end{tikzcd}$$
commute. 

These constructions are especially important in arithmetic applications. 
\begin{definition}[Rational Points]\label{def: rational points}
    Let $X$ be a $k$-scheme and $L/k$ a field extension. The set of $L$-rational points of $X$ is the set $X(L)=\Mor_{\Sch_{k}}(\spec(L),X)$. 
\end{definition}
One can easily show that this is invariant under base change of the scheme to $L$. 
\begin{lemma}\label{lem: bijection of rational points}
    Let $X$ be a $k$-scheme and $L/k$ a field extension. Then $X(L)=X_{L}(L)$ as sets. 
\end{lemma}
\begin{proof}
    Any morphism $\spec(L)\to X$ factors over a morphism to $X_{L}$
    $$% https://q.uiver.app/#q=WzAsNSxbMSwyLCJcXHNwZWMoTCkiXSxbMywyLCJcXHNwZWMoaykiXSxbMywxLCJYIl0sWzEsMSwiWF97TH0iXSxbMCwwLCJcXHNwZWMoTCkiXSxbNCwyLCIiLDEseyJjdXJ2ZSI6LTF9XSxbNCwwLCJcXGlkX3tcXHNwZWMoTCl9IiwyLHsiY3VydmUiOjF9XSxbMywwXSxbMywyXSxbMiwxXSxbMCwxXSxbNCwzLCJcXGV4aXN0cyEiLDEseyJzdHlsZSI6eyJib2R5Ijp7Im5hbWUiOiJkYXNoZWQifX19XV0=
    \begin{tikzcd}
        {\spec(L)} \\
        & {X_{L}} && X \\
        & {\spec(L)} && {\spec(k)}
        \arrow["{\exists!}"{description}, dashed, from=1-1, to=2-2]
        \arrow[curve={height=-6pt}, from=1-1, to=2-4]
        \arrow["{\id_{\spec(L)}}"', curve={height=6pt}, from=1-1, to=3-2]
        \arrow[from=2-2, to=2-4]
        \arrow[from=2-2, to=3-2]
        \arrow[from=2-4, to=3-4]
        \arrow[from=3-2, to=3-4]
    \end{tikzcd}$$
    giving the claim. 
\end{proof}
The most ``absolute'' form of base change is the geometric fiber. 
\begin{definition}[Geometric Fiber]\label{def: geometric fiber}
    Let $f:X\to Y$ be a morphism of schemes,$y\in Y$ with residue field $\kappa(y)$, and $\overline{\kappa(y)}$ a choice of algebraic closure of $\kappa(y)$. The geometric fiber $X_{\overline{y}}$ is defined to be $X\times_{Y}\spec(\overline{\kappa(y)})$. 
\end{definition}
\begin{remark}
    The topology may change under passage to the geometric fiber. This often has better topological behavior as the underling topological spaces of schemes over algebraically closed fields $k$ are often identical to the set of $k$-rational points. 
\end{remark}
\begin{example}
    Let $X$ be a scheme over $\spec(\ZZ_{(p)})$, the localization of $\ZZ$ at the prime ideal $(p)$. $\ZZ_{(p)}$ consists of two points $\{\eta,\mfrak\}$ corresponding to the generic and maximal ideal. The generic fiber $X_{\eta}$ is a scheme over $\spec(\QQ)$ and the closed fiber $X_{\mfrak}$ is a scheme over $\spec(\FF_{p})$. On the other hand, the generic and closed fibers $X_{\overline{\eta}},X_{\overline{\mfrak}}$ are schemes over $\spec(\overline{\QQ}),\spec(\overline{\FF_{p}})$, respectively. 
\end{example}
Returning to a discussion of arithmetic, we consider conjugate $k$-schemes. 
\begin{definition}[Conjugate $k$-Schemes]\label{def: conjugate k-schemes}
    Let $k$ be a field, $X$ a $k$-scheme, and $\sigma\in\Aut(k)$. The conjugate $k$-scheme is defined as the fibered product 
    $$% https://q.uiver.app/#q=WzAsNCxbMCwwLCJYX3tcXHNpZ21hfSJdLFsyLDAsIlgiXSxbMiwxLCJcXHNwZWMoaykiXSxbMCwxLCJcXHNwZWMoaykiXSxbMywyLCJcXHNpZ21hIl0sWzEsMl0sWzAsMV0sWzAsM11d
    \begin{tikzcd}
        {X_{\sigma}} && X \\
        {\spec(k)} && {\spec(k).}
        \arrow[from=1-1, to=1-3]
        \arrow[from=1-1, to=2-1]
        \arrow[from=1-3, to=2-3]
        \arrow["\sigma", from=2-1, to=2-3]
    \end{tikzcd}$$
\end{definition}
Note that $X_{\sigma}\to X$ is an isomorphism of abstract schemes, but not necessarily as $k$-schemes, since $X_{\sigma}$ has a different structure map that commutes with the structure map of $X$ up to $\sigma$. 
\begin{example}
    Let $X=\spec(\RR[x_{1},x_{2}]/(x_{1}^{2}+x_{2}^{2}-\pi))$ Since $\pi$ is transcendental, there exists an automorphism $\sigma$ of $\RR$ taking $\pi$ to $-\pi$ and fixing all other elements. $X$ is nonempty but $X_{\sigma}=\spec(\RR[x_{1},x_{2}]/(x_{1}^{2}+x_{2}^{2}+\pi))$ has no $\RR$-rational points. 
\end{example}
\appendix
\begin{landscape}
\section{The Spectral Sequence Computation of Lecture \ref{sec: lecture 8}}\label{app: spectral sequence computations}
This appendix contains a number of diagrams for the spectral sequence computation presented in \Cref{sec: lecture 8}. 
\tiny
\begin{equation}\label{diag: sseq E1 page}
    % https://q.uiver.app/#q=WzAsMTAsWzMsMCwiSF97MH1cXGxlZnQoXFxHTF97Mn0oRiksXFxaWltcXFBQXnsxfShGKV1cXHJpZ2h0KVxcb3RpbWVzXFxRUSJdLFszLDEsIkhfezF9XFxsZWZ0KFxcR0xfezJ9KEYpLFxcWlpbXFxQUF57MX0oRildXFxyaWdodClcXG90aW1lc1xcUVEiXSxbMywyLCJIX3syfVxcbGVmdChcXEdMX3syfShGKSxcXFpaW1xcUFBeezF9KEYpXVxccmlnaHQpXFxvdGltZXNcXFFRIl0sWzMsMywiSF97M31cXGxlZnQoXFxHTF97Mn0oRiksXFxaWltcXFBQXnsxfShGKV1cXHJpZ2h0KVxcb3RpbWVzXFxRUSJdLFsyLDAsIkhfezB9XFxsZWZ0KFxcR0xfezJ9KEYpLFxcWlpbKFxcUFBeezF9KEYpKV57Mn1fe1xcbmVxfV1fe1xcU2lnbWFfezJ9fVxccmlnaHQpXFxvdGltZXNcXFFRIl0sWzIsMSwiSF97MX1cXGxlZnQoXFxHTF97Mn0oRiksXFxaWlsoXFxQUF57MX0oRikpXnsyfV97XFxuZXF9XV97XFxTaWdtYV97Mn19XFxyaWdodClcXG90aW1lc1xcUVEiXSxbMiwyLCJIX3syfVxcbGVmdChcXEdMX3syfShGKSxcXFpaWyhcXFBQXnsxfShGKSleezJ9X3tcXG5lcX1dX3tcXFNpZ21hX3syfX1cXHJpZ2h0KVxcb3RpbWVzXFxRUSJdLFsxLDAsIkhfezB9XFxsZWZ0KFxcR0xfezJ9KEYpLFxcWlpbKFxcUFBeezF9KEYpKV57M31fe1xcbmVxfV1fe1xcU2lnbWFfezN9fVxccmlnaHQpXFxvdGltZXNcXFFRIl0sWzEsMSwiSF97MX1cXGxlZnQoXFxHTF97Mn0oRiksXFxaWlsoXFxQUF57MX0oRikpXnszfV97XFxuZXF9XV97XFxTaWdtYV97M319XFxyaWdodClcXG90aW1lc1xcUVEiXSxbMCwwLCJIX3swfVxcbGVmdChcXEdMX3syfShGKSxcXFpaWyhcXFBQXnsxfShGKSleezR9X3tcXG5lcX1dX3tcXFNpZ21hX3s0fX1cXHJpZ2h0KVxcb3RpbWVzXFxRUSJdLFs5LDddLFs3LDRdLFs0LDBdLFs1LDFdLFs2LDJdLFs4LDVdXQ==
\begin{tikzcd}
	{H_{0}\left(\GL_{2}(F),\ZZ[(\PP^{1}(F))^{4}_{\neq}]_{\Sigma_{4}}\right)\otimes\QQ} & {H_{0}\left(\GL_{2}(F),\ZZ[(\PP^{1}(F))^{3}_{\neq}]_{\Sigma_{3}}\right)\otimes\QQ} & {H_{0}\left(\GL_{2}(F),\ZZ[(\PP^{1}(F))^{2}_{\neq}]_{\Sigma_{2}}\right)\otimes\QQ} & {H_{0}\left(\GL_{2}(F),\ZZ[\PP^{1}(F)]\right)\otimes\QQ} \\
	& {H_{1}\left(\GL_{2}(F),\ZZ[(\PP^{1}(F))^{3}_{\neq}]_{\Sigma_{3}}\right)\otimes\QQ} & {H_{1}\left(\GL_{2}(F),\ZZ[(\PP^{1}(F))^{2}_{\neq}]_{\Sigma_{2}}\right)\otimes\QQ} & {H_{1}\left(\GL_{2}(F),\ZZ[\PP^{1}(F)]\right)\otimes\QQ} \\
	&& {H_{2}\left(\GL_{2}(F),\ZZ[(\PP^{1}(F))^{2}_{\neq}]_{\Sigma_{2}}\right)\otimes\QQ} & {H_{2}\left(\GL_{2}(F),\ZZ[\PP^{1}(F)]\right)\otimes\QQ} \\
	&&& {H_{3}\left(\GL_{2}(F),\ZZ[\PP^{1}(F)]\right)\otimes\QQ}
	\arrow[from=1-1, to=1-2]
	\arrow[from=1-2, to=1-3]
	\arrow[from=1-3, to=1-4]
	\arrow[from=2-2, to=2-3]
	\arrow[from=2-3, to=2-4]
	\arrow[from=3-3, to=3-4]
\end{tikzcd}
\end{equation}
\normalsize
\begin{equation}\label{diag: sseq E1 page substituted}
    % https://q.uiver.app/#q=WzAsMTAsWzMsMCwiXFxRUSJdLFszLDEsIigoRl57XFx0aW1lc30pXnsyfVxcb3RpbWVzXFxRUSkiXSxbMywyLCJcXGJpZ3dlZGdlXnsyfSgoRl57XFx0aW1lc30pXnsyfVxcb3RpbWVzXFxRUSkiXSxbMywzLCJcXGJpZ3dlZGdlXnszfSgoRl57XFx0aW1lc30pXnsyfVxcb3RpbWVzXFxRUSkiXSxbMiwwLCIwIl0sWzIsMSwiRl57XFx0aW1lc31cXG90aW1lc1xcUVEiXSxbMiwyLCJcXGJpZ3dlZGdlXnsyfSgoRl57XFx0aW1lc30pXnsyfVxcb3RpbWVzXFxRUSlfe1xcU2lnbWFfezJ9fSJdLFsxLDAsIjAiXSxbMSwxLCIwIl0sWzAsMCwiXFxRUVtGXntcXHRpbWVzfVxcc2V0bWludXNcXHsxXFx9XV97XFxTaWdtYV97NH19Il0sWzksN10sWzcsNF0sWzQsMF0sWzUsMV0sWzYsMl0sWzgsNV1d
\begin{tikzcd}
	{\QQ[F^{\times}\setminus\{1\}]_{\Sigma_{4}}} & 0 & 0 & \QQ \\
	& 0 & {F^{\times}\otimes\QQ} & {((F^{\times})^{2}\otimes\QQ)} \\
	&& {\bigwedge^{2}((F^{\times})^{2}\otimes\QQ)_{\Sigma_{2}}} & {\bigwedge^{2}((F^{\times})^{2}\otimes\QQ)} \\
	&&& {\bigwedge^{3}((F^{\times})^{2}\otimes\QQ)}
	\arrow[from=1-1, to=1-2]
	\arrow[from=1-2, to=1-3]
	\arrow[from=1-3, to=1-4]
	\arrow[from=2-2, to=2-3]
	\arrow[from=2-3, to=2-4]
	\arrow[from=3-3, to=3-4]
\end{tikzcd}
\end{equation}



\end{landscape}


\printbibliography

\end{document}