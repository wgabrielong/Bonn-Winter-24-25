\section{Lecture 8 -- 13th December 2024}\label{sec: lecture 8}
We discuss the computation of algebraic $K$-theory groups of a field $F$ in degrees $\leq 3$. In fact, the first nontrivial case is $K_{3}$ as the 0th, 1st, and 2nd $K$-theory groups are given by $\ZZ$, $F^{\times}$, and $K^{\mathsf{M}}_{2}(F)$ the Milnor $K$-theory of $F$, respectively. 

In the case of interest, computing $K_{3}$ of a number field, it suffiecs to understand the homology of $\GL_{2}(F)$ in degrees at most 3. To do so, we consider the action of $\GL_{2}(F)$ on $\PP^{1}(F)$ by linear fractional transformations. The computation relies on the following lemma. 
\begin{proposition}\label{prop: resolutions by free Abelian groups}
    Let $(\PP^{1}(F))^{n}_{\neq}$ be the set of $n$ pairwise distinct points on $\PP^{1}(F)$. There is a functorial exact complex 
    $$\dots\to\ZZ[(\PP^{1}(F))^{2}_{\neq}]_{\Sigma_{2}}\to\ZZ[\PP^{1}(F)]\to\ZZ\to0$$
    where $(-)_{\Sigma_{n}}$ denotes the coinvariants of the natural action of the symmetric groups which is an exact complex of $\GL_{2}(F)$-modules. 
\end{proposition}
The resolution of \Cref{prop: resolutions by free Abelian groups} produces a spectral sequence converging to $H_{i+j}(*/\GL_{2}(F))$ with $E_{1}$-page given as in (\ref{diag: sseq E1 page}). We compute the homology in each case. We do so degree by degree in the complex of \Cref{prop: resolutions by free Abelian groups}
\begin{proposition}\label{prop: action on P1}
    The action of $\GL_{2}(F)$ on $\PP^{1}(F)$ by linear fractional transformations is transitive, and the stabilizer of $\infty$ is given by the Borel subgroup $\Bcal_{2}(F)$ of upper triangular matrices. Moreover, 
    $$H_{i}(*/\Bcal_{2}(F))\otimes\QQ\cong H_{i}(*/(F^{\times})^{2})\cong\bigwedge^{i}\left((F^{\times})^{2}\otimes\QQ\right).$$
\end{proposition}
In the case of $\ZZ[(\PP^{1}(F))^{2}_{\neq}]_{\Sigma_{2}}$, we have the following. 
\begin{proposition}\label{prop: action on P1 2}
    The homology of $H_{0}\left(\GL_{2}(F),\ZZ[(\PP^{1}(F))^{2}_{\neq}]_{\Sigma_{2}}\right)\otimes\QQ$ is given by 
    $$H_{0}\left(\GL_{2}(F),\ZZ[(\PP^{1}(F))^{2}_{\neq}]_{\Sigma_{2}}\right)\otimes\QQ=\begin{cases}
        0 & i = 0 \\ F^{\times}\otimes\QQ & i=1 \\ \bigwedge^{2}((F^{\times})^{2}\otimes\QQ)_{\Sigma_{2}} & i=2.
    \end{cases}$$
\end{proposition}
And in degrees $\ZZ[(\PP^{1}(F))^{3}_{\neq}]_{\Sigma_{3}}$ and $\ZZ[(\PP^{1}(F))^{4}_{\neq}]_{\Sigma_{4}}$, we have the following. 
\begin{proposition}\label{prop: action on P1 3}
    The action of $\GL_{2}(F)$ on $(\PP^{1}(F))^{3}_{\neq}$ factors over the simply transtive action of $\mathrm{PGL}_{2}(F)$ on $(\PP^{1}(F))^{3}_{\neq}$. In particular, the homology vanishes. 
\end{proposition}
\begin{proposition}\label{prop: action on P1 4}
    The 0th homology of $\ZZ[(\PP^{1}(F))^{4}_{\neq}]_{\Sigma_{4}}$ is given by 
    $$H_{0}\left(\GL_{2}(F),\ZZ[(\PP^{1}(F))^{4}_{\neq}]_{\Sigma_{4}}\right)\otimes\QQ=\QQ[F^{\times}\setminus\{1\}]_{\Sigma_{4}}.$$
\end{proposition}
This produces the necessary data for the $E_{1}$-page. For the $E_{2}$-page, we further need to understand $H_{0}\left(\GL_{2}(F),\ZZ[(\PP^{1}(F))^{5}_{\neq}]_{\Sigma_{5}}\right)\otimes\QQ$. 
\begin{proposition}\label{prop: action on P1 5}
    The 0th rational homology of $\ZZ[(\PP^{1}(F))^{5}_{\neq}]_{\Sigma_{5}}$ is given the $\QQ$-vector subspace of $\QQ[F^{\times}\setminus\{1\}]$ generated by the five term relations of the Bloch-Wigner dilogarithm. 
\end{proposition}
Substituting the results of \Cref{prop: action on P1,prop: action on P1 2,prop: action on P1 3,prop: action on P1 4} into (\ref{diag: sseq E1 page}) we get on the $E_{1}$-page (\ref{diag: sseq E1 page substituted}). In conjunction with \Cref{prop: action on P1 5}, we get on the $E_{2}$-page 
\begin{equation}\label{diag: sseq E2 page}
    % https://q.uiver.app/#q=WzAsMTAsWzMsMCwiXFxRUSJdLFszLDEsIigoRl57XFx0aW1lc30pXnsyfVxcb3RpbWVzXFxRUSkiXSxbMywyLCJcXGJpZ3dlZGdlXnsyfSgoRl57XFx0aW1lc30pXnsyfVxcb3RpbWVzXFxRUSkiXSxbMywzLCJcXGJpZ3dlZGdlXnszfSgoRl57XFx0aW1lc30pXnsyfVxcb3RpbWVzXFxRUSkiXSxbMiwwLCIwIl0sWzIsMSwiMCJdLFsyLDIsIjAiXSxbMSwwLCIwIl0sWzEsMSwiMCJdLFswLDAsIlxcd3AoRikiXV0=
\begin{tikzcd}
	{\wp(F)} & 0 & 0 & \QQ \\
	& 0 & 0 & {((F^{\times})^{2}\otimes\QQ)} \\
	&& 0 & {\bigwedge^{2}((F^{\times})^{2}\otimes\QQ)} \\
	&&& {\bigwedge^{3}((F^{\times})^{2}\otimes\QQ)}
\end{tikzcd}
\end{equation}
with no differentials. On the $E_{3}$-page, the map $\wp(F)\to\bigwedge^{2}((F^{\times})^{2}\otimes\QQ)$ recovers the second Bloch group \Cref{def: Bloch group} as the kernel and thus Bloch's result \Cref{thm: Bloch rational computation of K3} for $F=\KK$ a number field. 
\begin{remark}
    For this computation, one can also instead consider the action of the Picard groupoid $\sfPic(F)$ on the $K$-theory anima as in \Cref{def: K-theory of a ring} and consider the homotopy orbits $K(F)_{h(*/F^{\times})}\otimes\QQ$ which also recovers the pre-Bloch group as its third homotopy group. This exhibits $K(F)_{h(*/F^{\times})}$ as the group completion of $$\left(\coprod_{n\geq0}*/\GL_{n}(F)\right)_{h(*/F^{\times})}.$$
\end{remark}
Recalling \Cref{thm: Bloch rational computation of K3}, and the subsequent discussion we have regulator maps for each complex embedding $\tau:\KK\to\CC$ a Bloch regulator map $K_{3}(F)\cong B(F)\to\CC/(2\pi i)^{2}\QQ$ which can be extended to a map to $\RR$ by taking the imaginary part, with $\sum_{i}n_{i}[x_{i}]\mapsto\sum_{i}n_{i}D(\tau(x_{i}))$, here taking the Bloch-Wigner dilogarithm. 