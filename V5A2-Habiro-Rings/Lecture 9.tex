\section{Lecture 9 -- 20th December 2024}\label{sec: lecture 9}
We consider the notion of relative $K$-theory, a variant of $K$-theory in which $K$-theoretic classes are more easily described, and is as computable as $K$-theory. In this way, relative $K$-theory can be seen as more flexible than $K$-theory. 
\begin{definition}[Relative $K$-Theory]\label{def: relative K-theory}
    Let $R$ be a ring and $M$ an Abelian group with a map $M\to R^{\times}$. The relative $K$-theory 
    $$K(R/\ZZ[M])= K(R)\otimes_{\SSS[*/M]}\SSS$$
    where $K(R)$ is considered as a spectrum over $\SSS[*/M]$ by the map $*/M\to */R^{\times}\to R$. 
\end{definition}
\begin{remark}
    By the universal property of group rings, the existence of a map $M\to R^{\times}$ is equivalent to the existence of a map from the group ring $\ZZ[M]$ to $R$. 
\end{remark}
\begin{remark}
    One can generalize \Cref{def: relative K-theory} to the setting of $M$ a commutative monoid $M$ to $R$ which corresponds to a map $\ZZ[M]\to R$. One can then define 
    $$K(R/\ZZ[M])=K^{\log}(R,M)\otimes_{\SSS[*/M^{\mathsf{gp}}]}\SSS$$
    where $K^{\log}$ is logarithmic $K$-theory. 
\end{remark}
\begin{remark}
    The construction of \Cref{def: relative K-theory} above is equivalent to taking the homotopy orbits $K(R)_{h(*/M)}$ which in the case of $F$ a field produces $K(F)_{h(*/F^{\times})}=K(F/\ZZ[F^{\times}])$. 
\end{remark}
The upshot of this construction is that all constructions in $K$-theory generalize to the setting of relative $K$-theory. Of importance to us are regulators and polylogarithms. In particular, the dilogarithm is most naturally expressed in the setting of relative $K$-theory. 

Moreover, relative $K$-theory overcomes the rigidity of $K$-theory in high weights. For example, it is conjectured that the Bloch group of $\CC$ and of $\overline{\QQ}$ coincide. On the other hand, classes in relative $K$-theory is plentiful. As such, the regulators are now highly interesting special functions and not merely numbers and provides a coherent organizing principle for functions like dilogarithms. 

These constructions also arise in $p$-adic geometry. Let $R$ be a smooth algebra over $\CC_{p}$, the $p$-completion of the algebraic closure of $\QQ_{p}$. We often pick a system of coordinates $T_{1},\dots,T_{d}\in R$ invertible and consider an \'{e}tale map $\CC_{p}[T_{1}^{\pm},\dots,T_{d}^{\pm}]\to R$ and pass to
$$R_{\infty}=R\otimes_{\CC_{p}[T_{1}^{\pm},\dots,T_{d}^{\pm}]}\CC_{p}\left[T_{1}^{\pm\frac{1}{p^{\infty}}},\dots,T_{d}^{\pm\frac{1}{p^{\infty}}}\right]$$
which on further passage to an appropriate completion is a perfectoid algebra. This is similar to the data required for relative $K$-theory since the group generated by $T_{1}^{\pm},\dots,T_{d}^{\pm}$ naturally maps to the units of $R$. In this case, the \'{e}tale $p$-complete relative $K$-theory 
$$K_{\mathsf{\acute{e}t}}(R/\ZZ[T_{1}^{\pm},\dots,T_{d}^{\pm}])^{\wedge}_{p}$$
turns out to be the $p$-completed \'{e}tale $K$-theory $K_{\mathsf{\acute{e}t}}(R_{\infty})$. Morally, what happens here is the $p$-power roots vanish in the $K$-theory of the $p$-completion. See \cite{PrismaticCohDelta} for a further discussion on this topic, viewing syntomic cohomology as a form of $p$-adic $K$-theory. 

Going forward, we will take the perspective of polylogarithms being relative $K$-theory classes and functional equations of polylogarithms are already identities of relative $K$-theory. 

\begin{example}
    Let $R=\ZZ[t^{\pm},\frac{1}{1-t}]$ and $M=t^{\ZZ}$. There is a natural map $M\to R^{\times}$. We have 
    $$K(R/\ZZ[t^{\pm}])=K(R)\otimes_{\SSS[*/t^{\ZZ}]}\SSS$$
    inducing an exact triangle 
    $$K(R)\to K(R/\ZZ[t^{\pm}])\to K(R/\ZZ[t^{\pm}])[2]$$
    where the map $K(R/\ZZ[t^{\pm}])\to K(R/\ZZ[t^{\pm}])[2]$ can be thought of as a logarithmic $t$-derivative $\nabla^{\log}_{t}$ by the exact sequence of $\SSS[*/t^{\ZZ}]$-modules
    $$\SSS[*/t^{\ZZ}]\to\SSS\to\SSS[2].$$
    In particular, any $K$-theory class gives rise to a relative $K$-theory classes and $K$-theory classes can be recovered from those relative $K$-theory classes that vanish under the differential. 
\end{example}

Now recall the existence of a motivic filtration on $K$-theory \cite{MotivicFiltrationKTheory} and is still a topic of contemporary interest with T. Bouis' recent results in the mixed characteristic case \cite{BouisThesis}. Running this machinery on relative $K$-theory, this produces a relative motivic filtration and relative motivic cohomology of a ring relative to a group algebra taking $\ZZ[M]\to R$ to $\ZZ(n)(R/\ZZ[M])$. 

We have an exact triangle
$$\ZZ(n)(R)\to\ZZ(n)\left(R/\ZZ[t^{\pm}]\right)\to\ZZ(n-1)\left(R/\ZZ[t^{\pm}]\right)$$
which in weight $\leq 2$ computes $K$-theory in small degrees. Explicitly, we have $\ZZ(0)(R)=\ZZ$ and $\ZZ(1)(R)=R^{\times}[-1]$. By $\mathbb{A}^{1}$-invariance for regular rings, we have $K(\ZZ[t])\cong K(\ZZ)$ so we have $K(\ZZ[t^{\pm},\frac{1}{t-1}])\cong K(\ZZ)\oplus K(\ZZ)[1]\oplus K(\ZZ)[1]= K(\ZZ)\oplus K(\ZZ)[1]^{\oplus 2}$. In the relative setting, the exact triangle allow us to compute relative $K$-theory 
$$% https://q.uiver.app/#q=WzAsNCxbMCwwLCJcXFpaKDEpKFIpIl0sWzIsMCwiXFxaWigxKShSL1xcWlpbdF57XFxwbX1dKSJdLFs0LDAsIlxcWlooMCkoUi9cXFpaW3Ree1xccG19XSkiXSxbNiwwLCJSXntcXHRpbWVzfSJdLFswLDFdLFsxLDJdLFsyLDMsIjFcXG1hcHN0byB0Il1d
\begin{tikzcd}
	{\ZZ(1)(R)} && {\ZZ(1)(R/\ZZ[t^{\pm}])} && {\ZZ(0)(R/\ZZ[t^{\pm}])} && {R^{\times}}
	\arrow[from=1-1, to=1-3]
	\arrow[from=1-3, to=1-5]
	\arrow["{1\mapsto t}", from=1-5, to=1-7]
\end{tikzcd}$$
and where making the appropriate substitutions on rationalization gives 
$$0\longrightarrow \QQ(2)(R/\ZZ[t^{\pm}])\longrightarrow \QQ[-1]$$
so in fact there is an isomorphism $\QQ(2)(R/\ZZ[t^{\pm}])\to \QQ[-1]$ induced by the so-called universal dilogarithm that takes $\Li_{2}^{\mathrm{univ}}(t)$ to $(1-t)$ where we note that $\QQ[-1]$ is generated by $1-t$. 

We now want to observe that this so-called universal dilogarithm satisfies the expected functional equations. 
\begin{proposition}\label{prop: equality of the universal dilogarithm}
    There is an equality $\Li_{2}^{\mathrm{univ}}(t)=-\Li_{2}^{\mathrm{univ}}(1-t)$ in relative rational motivic cohomology 
    $$H^{1}\left(\QQ(2)\left(\ZZ\left[t^{\pm},\frac{1}{1-t}\right]/\ZZ[t^{\pm}, (1-t)^{\pm}]\right)\right).$$ 
\end{proposition}
\begin{proof}
    There is a Koszul-like complex computing $\ZZ(n)(R)$ as the limit of 
    $$% https://q.uiver.app/#q=WzAsMyxbMCwwLCJcXFpaKG4pXFxsZWZ0KFIvXFxaWlt0X3sxfV57XFxwbX0sdF97Mn1ee1xccG19XVxccmlnaHQpIl0sWzIsMCwiXFxaWihuLTEpXFxsZWZ0KFIvXFxaWlt0X3sxfV57XFxwbX0sdF97Mn1ee1xccG19XVxccmlnaHQpXntcXG9wbHVzIDJ9Il0sWzMsMCwiXFxaWihuLTIpXFxsZWZ0KFIvXFxaWlt0X3sxfV57XFxwbX0sdF97Mn1ee1xccG19XVxccmlnaHQpIl0sWzAsMSwiKFxcbmFibGFfe3RfezF9fV57XFxsb2d9LFxcbmFibGFfe3RfezJ9fV57XFxsb2d9KSJdLFsxLDJdXQ==
    \begin{tikzcd}
        {\ZZ(n)\left(R/\ZZ[t_{1}^{\pm},t_{2}^{\pm}]\right)} && {\ZZ(n-1)\left(R/\ZZ[t_{1}^{\pm},t_{2}^{\pm}]\right)^{\oplus 2}} & {\ZZ(n-2)\left(R/\ZZ[t_{1}^{\pm},t_{2}^{\pm}]\right)}
        \arrow["{(\nabla_{t_{1}}^{\log},\nabla_{t_{2}}^{\log})}", from=1-1, to=1-3]
        \arrow[from=1-3, to=1-4]
    \end{tikzcd}$$
    so noting that $\QQ(2)(R)\cong 0$ and 
    $$\QQ(i)\left(\ZZ\left[t^{\pm},\frac{1}{1-t}\right]/\ZZ[t^{\pm}, (1-t)^{\pm}]\right)$$
    vanishes for $i\in\{0,1\}$ we have that 
    $$\QQ(2)\left(\ZZ\left[t^{\pm},\frac{1}{1-t}\right]/\ZZ[t^{\pm}, (1-t)^{\pm}]\right)\cong\QQ[-1].$$
    So equality follows by considering the composite. 
\end{proof}
\begin{remark}
    In general if the dilogarithm on a ring $R$ the condition of $\sum_{i}\Li_{2}(f_{i}(t))$ being constant implies that $\sum_{i}f_{i}(t)\wedge (1-f_{i}(t))=0$ in $\bigwedge^{2}R^{\times}$. This holds for the five-term relation \Cref{eqn: BW dilogarithm five term relation}. 
\end{remark}
We now consider the Borel (complex) regulator on algebraic $K$-theory. There is a map 
$$K_{3}\left(\ZZ\left[t^{\pm},\frac{1}{1-t}\right]/\ZZ[t^{\pm}]\right)\to K_{3}(\CC/\ZZ[\CC^{\times}])\to K_{3}^{\cont}(\CC/\ZZ[\CC^{\times}])$$
which statisfies a functional equation so for $t\in\CC\setminus\{0,1\}$, its image in $K_{3}^{\cont}(\CC/\ZZ[\CC^{\times}])$ satisfies a functional equation as well. 

Rationally, we can consider rationalized relative motivic complexes $\QQ(n)^{\cont}(\CC/\ZZ[\CC^{\times}])$ where in the case of $\QQ(2)^{\cont}(\CC/\ZZ[\CC^{\times}])$ we can compute using $\QQ(0)^{\cont}(\CC/\ZZ[\CC^{\times}])\cong\QQ, \QQ(1)^{\cont}(\CC/\ZZ[\CC^{\times}])\cong(\CC^{\times}/\CC^{\times})[-1]\cong0$ so the fiber sequence computes $\QQ(2)^{\cont}(\CC)$ as $(\CC/(2\pi i)^{2}\QQ)[-1]$ and is given by an extension $E$ fitting into the short exact sequence 
$$% https://q.uiver.app/#q=WzAsNSxbMCwwLCIwIl0sWzEsMCwiXFxDQy8oMlxccGkgaSleezJ9XFxRUSJdLFsyLDAsIkUiXSxbMywwLCJcXFFRXFxvdGltZXNfe1xcWlp9XFxiaWd3ZWRnZV57Mn1cXENDXntcXHRpbWVzfSJdLFs0LDAsIjAiXSxbMCwxXSxbMSwyXSxbMiwzXSxbMyw0XV0=
\begin{tikzcd}
	0 & {\CC/(2\pi i)^{2}\QQ} & E & {\QQ\otimes_{\ZZ}\bigwedge^{2}\CC^{\times}} & 0
	\arrow[from=1-1, to=1-2]
	\arrow[from=1-2, to=1-3]
	\arrow[from=1-3, to=1-4]
	\arrow[from=1-4, to=1-5]
\end{tikzcd}$$
where the map $\CC\setminus\{0,1\}\to E$ by $t\mapsto (t)\wedge(1-t)$ satisfies the five-term relation. This extends to a map $\QQ[\CC\setminus\{0,1\}]$ to $E$ which necessarily factors over the pre-Bloch group $\wp_{2}(\CC)$ of \Cref{def: pre-Bloch group} since the map satisfies the five-term relation. This in turn induces a map $B_{2}(\CC)\to \CC/(2\pi i)^{2}\QQ$ as expected. More explicitly, $E$ is obtained as the cokernel of $\CC\otimes_{\ZZ}\CC\to \CC/(2\pi i)^{2}\QQ\oplus\CC^{\times}\otimes_{\ZZ}\CC$ by $x\otimes y\mapsto (xy, \exp(x)\otimes y+\exp(y)\otimes x)$. 